%\thispagestyle{fancy}   % 插入页眉页脚                                        %

%%%%---%%%%%%---- The Main Body OF THE PAPER ----%%%%%----%%%%%
\section{VASP软件的特点}
\textrm{VASP}软件\upcite{VASP_manual}是维也纳大学(Universit\"at Wien)\textrm{G. Kresse}等开发的第一原理模拟软件包。\textrm{VASP}采用的\textrm{PAW~(Projector Augmented-Wave)}方法\upcite{PRB50-17953_1994,PRB59-1758_1999},平衡了传统赝势方法和高精度全电子方法的优点,兼顾了计算的精度和效率。特别是实空间优化的投影函数\textrm{(Projector)},将主要的计算任务变换到实空间完成,大大节省了基组的维度,保证了计算精度和效率。在此基础上,\textrm{VASP}通过引入丰富多样的优化算法,提高自洽迭代过程中的矩阵对角化和电荷密度搜索的效率;~软件的\textrm{mpi}并行实现中,通过\textrm{FFT~(Fast Fourier Transformation)}计算网格与并行计算的计算格点的分配平衡,提升了软件的并行效率。相比于其他第一原理计算软件,\textrm{VASP}从物理思想与方法、优化算法和并行计算实现等多个方面都有更为出色的性能。有关\textrm{VASP}软件实现的文献,主要如下:~
\begin{enumerate}
	\item 物理思想与方法:~\textrm{PAW}赝势方法:~参阅文献\cite{PRB50-17953_1994,PRB59-1758_1999};~投影函数的实空间优化,参阅\cite{JPCM6-8245_1994,PRB44-13063_1991,PRB44-8503_1991}
	\item 优化算法:~参阅文献\cite{CMS6-15_1996,PRB54-11169_1996}
	\item 高效的并行计算:~ \textrm{FFT}网格与计算格点的分配:~参阅\textrm{VASP}源代码的\textbf{mgrid.F}
\end{enumerate}

\section{VASP的PAW方法实现}
\textrm{PAW}方法是\textrm{Bl\"ochl}于1994年独立提出来的一种计算方法\upcite{PRB50-17953_1994},该方法的基本思想与\textrm{OPW}很相似,但同时结合了赝势方法和\textrm{APW}方法的优点,达到平衡计算效率和精度的目的。\textrm{PAW}方法刚提出来的时候并未引起注意,直到1999年\textrm{Kresse}指明了\textrm{PAW}方法和超软赝势(\textrm{USPP})方法的密切关联,\textrm{USPP}方法的计算程序只要经过简单改造就能扩展为\textrm{PAW}方法,有力地推动了\textrm{PAW}方法的广泛应用\upcite{PRB59-1758_1999}。现在\textrm{PAW}方法已经成为最主要的可支持第一原理分子动力学\textrm{(Ab initio Molecular Dynamics, AIMD)}。

\subsection{PAW方法的基本思想}
与一般赝势方法不同,\textrm{PAW}方法的目标是全电子(\textrm{all-electron})波函数\footnote{注意,“全电子”在\textrm{Bl\"ochl}原始文献\cite{PRB50-17953_1994}中与“真实电子波函数”意义相近,强调电子在原子核附近的振荡行为,但并未严格区分价电子与芯电子;~而在\textrm{Kresse}的文献\cite{PRB59-1758_1999}中则是明确指价电子波函数。%强调价电子波函数因与芯层电子正交而在原子核附近振荡;
此外,与“全电子”概念密切关联的是“全势(\textrm{full-potential})”,两者在具体语境中有一定的区别,“全势”强调的是重现价电子感受到的势函数的效果。},体系中全部电子构成\textrm{Hilbert}空间,价电子与芯层态彼此正交,使得波函数在\textrm{Muffin-tin}球内振荡。
\textrm{Bl\"ochl}假设全电子波函数$|\Psi\rangle$与赝波函数$|\tilde\Psi\rangle$满足线性变换,即满足:
\begin{equation}
	|\Psi\rangle=\mathbf{\tau|}\tilde\Psi\rangle
	\label{eq:PAW-Blochl-01}
\end{equation}
%	$$\tau=\mathbf{1}+\sum_{\mathrm R}\hat\tau_{\mathrm R}$$
在原子核附近的$r_c$范围内\footnote{习惯上这个区域称为缀加区(\textrm{Augmentation region}).},除了平面波,还引入原子分波函数展开来表示波函数:
\begin{equation}
	|\Psi\rangle=|\tilde\Psi\rangle+\sum_i(|\phi_i\rangle-|\tilde\phi_i\rangle)\langle\tilde p_i|\tilde\Psi\rangle
	\label{eq:PAW-Blochl-02}
\end{equation}
在$r_c$外$|\tilde\Psi\rangle$与$|\Psi\rangle$变换前后保持不变,因此线性变换$\mathbf{\tau}$可表示为:
\begin{equation}
	\mathbf{\tau}=\mathbf{1}+\sum_i(|\phi_i\rangle-|\tilde\phi_i\rangle)\langle\tilde p_i|
	\label{eq:PAW-Blochl-03}
\end{equation}
其中$|\tilde p_i\rangle$是\textrm{MT}球内的投影函数,$i$表示原子位置$\vec R$、原子轨道($l,m$)和能级$\epsilon_k$的指标。\textrm{PAW}的波函数与赝波函数的关系,可以用图\ref{PAW_basic}表示。
\begin{figure}[h!]
\centering
\includegraphics[height=2.35in,width=4.1in,viewport=0 0 1280 745,clip]{PAW-baseset.png}
%\includegraphics[height=1.8in,width=4.in,viewport=30 210 570 440,clip]{PAW_projector.eps}
\caption{\small \textrm{The analysis of PAW basic function.}}%(与文献\upcite{EPJB33-47_2003}图1对比)
\label{PAW_basic}
\end{figure}

\subsubsection{PAW表示下的能量表示与补偿电荷}
\textrm{PAW}方法通过投影算符和赝波函数实现全电子波函数计算,因此完整的\textrm{PAW}方法下,一般算符的期望值计算为
\begin{equation}
	\langle A \rangle=\langle\Psi|\mathbf{A}|\Psi\rangle=\langle\tilde\Psi|\mathbf{\tau}^{\dag}\mathbf{A}\mathbf{\tau}|\tilde\Psi\rangle=\langle\tilde\Psi|\tilde{\mathrm{A}}|\tilde\Psi\rangle
	\label{eq:PAW-Blochl-04}
\end{equation}
式中将$\mathrm{\tau}$用式\eqref{eq:PAW-Blochl-03}展开,因此赝算符$\tilde A$可表示为
\begin{equation}
	\tilde A=\mathbf{A}+\sum_i|\tilde p_i\rangle(\langle\phi_i|\mathbf{A}|\phi_i\rangle-\langle\tilde\phi_i|\mathbf{A}|\tilde\phi_i\rangle)\langle\tilde p_i|
	\label{eq:PAW-Blochl-05}
\end{equation}
%不难看出,赝重叠算符$\tilde O$可展开为
%\begin{equation}
%	\tilde O=\mathbf{1}+\sum_i|\tilde p_i\rangle(\langle\phi_i|\phi_i\rangle-\langle\tilde\phi_i|\tilde\phi_i\rangle)\langle\tilde p_i|
%	\label{eq:PAW-Blochl-06}
%\end{equation}
实空间中,电荷密度的算符为$|r\rangle\langle r|$,根据式\eqref{eq:PAW-Blochl-03},则电荷密度的计算
\begin{equation}
	n(\vec r)=\tilde n(\vec r)+n^1(\vec r)-\tilde n^1(\vec r)
	\label{eq:PAW-Blochl-07}
\end{equation}
这里
\begin{displaymath}
	\begin{aligned}
		\tilde n(\vec r)=&\sum_nf_n\langle\tilde\Psi_n|\vec r\rangle\langle\vec r|\tilde\Psi_n\rangle \\
n^1(\vec r)=&\sum_{n,(i,j)}f_n\langle\tilde\Psi_n|\tilde p_i\rangle\langle\phi_i|\vec r\rangle\langle\vec r|\phi_j\rangle\langle\tilde p_j|\tilde\Psi_n\rangle \\
\tilde n^1(\vec r)=&\sum_{n,(i,j)}f_n\langle\tilde\Psi_n|\tilde p_i\rangle\langle\tilde\phi_i|\vec r\rangle\langle\vec r|\tilde\phi_j\rangle\langle\tilde p_j|\tilde\Psi_n\rangle
	\end{aligned}
\end{displaymath}
$f_n$是占据态的电子数。注意,这里的$n^1$和$\tilde n^1$中包含了芯层电荷的贡献,即$\sum_n\langle\phi_n^c|\vec r\rangle\langle\vec r|\phi_n^c\rangle$,$\sum_n\langle\tilde\phi_n^c|\vec r\rangle\langle\vec r|\tilde\phi_n^c\rangle$和$\sum_n\langle\tilde\Psi_n^c|\vec r\rangle\langle\vec r|\tilde\Psi_n^c\rangle$。但是实际应用中,一般都是直接构造芯层态电荷密度,并不考虑芯层态波函数。

类似地,根据\textrm{DFT}理论,体系总能量泛函可以表示为
\begin{equation}
	\begin{aligned}
		E&=\sum_nf_n\langle\Psi_n|-\dfrac12\nabla^2|\Psi_n\rangle\\
		 &+\dfrac12\int\mathrm{d}\vec r\int\mathrm{d}\vec r^{\prime}\dfrac{(n+n^Z)(n+n^Z)}{|\vec r-\vec r^{\prime}|}+\int\mathrm{d}\vec r n\epsilon_{\mathrm{XC}}(n)
	\end{aligned}
	\label{eq:PAW-Blochl-08}
\end{equation}
在\textrm{PAW}框架下,总能可分解为$E=\tilde E+E^1-\tilde E^1$,每一项分别表示为:
\begin{equation}
	\begin{aligned}
		\tilde E&=\sum_nf_n\langle\tilde\Psi_n|-\dfrac12\nabla^2|\tilde\Psi_n\rangle\\
		 &+\dfrac12\int\mathrm{d}\vec r\int\mathrm{d}\vec r^{\prime}\dfrac{(\tilde n+\hat n)(\tilde n+\hat n)}{|\vec r-\vec r^{\prime}|}+\int\mathrm{d}\vec r \tilde n\bar v+\int\mathrm{d}\vec r \tilde n\epsilon_{\mathrm{XC}}(\tilde n)
 	\end{aligned}
	\label{eq:PAW-Blochl-09}
\end{equation}
\begin{equation}
	\begin{aligned}
		E^1&=\sum_{n,(i,j)}f_n\langle\tilde\Psi_n|\tilde p_i\rangle\langle\phi_i|-\dfrac12\nabla^2|\phi_j\rangle\langle\tilde p_j|\tilde\Psi_n\rangle\\
		 &+\dfrac12\int\mathrm{d}\vec r\int\mathrm{d}\vec r^{\prime}\dfrac{(n^1+n^Z)(n^1+n^Z)}{|\vec r-\vec r^{\prime}|}+\int\mathrm{d}\vec r n^1\epsilon_{\mathrm{XC}}(n^1)
 	\end{aligned}
	\label{eq:PAW-Blochl-10}
\end{equation}
\begin{equation}
	\begin{aligned}
		\tilde E^1&=\sum_{n,(i,j)}f_n\langle\tilde\Psi_n|\tilde p_i\rangle\langle\tilde\phi_i|-\dfrac12\nabla^2|\tilde\phi_j\rangle\langle\tilde p_j|\tilde\Psi_n\rangle\\
		 &+\dfrac12\int\mathrm{d}\vec r\int\mathrm{d}\vec r^{\prime}\dfrac{(\tilde n^1+\hat n)(\tilde n^1+\hat n)}{|\vec r-\vec r^{\prime}|}+\int\mathrm{d}\vec r \tilde n^1\bar v+\int\mathrm{d}\vec r \tilde n^1\epsilon_{\mathrm{XC}}(\tilde n^1)
 	\end{aligned}
	\label{eq:PAW-Blochl-11}
\end{equation}
式\eqref{eq:PAW-Blochl-09}和\eqref{eq:PAW-Blochl-11}中$\bar v$是缀加区内的任意局域函数,只要该区域内满足条件$\tilde n=\tilde n^1$,则$\bar v$对总能量的贡献为0。与赝势方法类似,$\bar v$取去屏蔽局域赝势。

类似地,式\eqref{eq:PAW-Blochl-09}和\eqref{eq:PAW-Blochl-11}中$\hat n$为补偿电荷,取值范围限于缀加区。与超软赝势中的补偿电荷的要求类似,\textrm{PAW}方法中的$\hat n$存在使得电荷密度$(n^1+n^Z)$和$(\tilde n^1+\hat n)$在缀加区外的多极矩贡献相等,因此不必再考虑电荷在缀加区外的相互作用,只需考虑$\hat n$在$\tilde E$中的贡献。$\hat n$可表示为单个原子截断区间的补偿电荷之和,即$\hat n=\sum_R\hat n_R$,并且$\hat n_R(r)$可用\textrm{Gaussian}函数展开,即
\begin{equation}
	\hat n_R(r)=\sum_{L=(l,m)}g_{RL}(r)Q_{RL}
	\label{eq:PAW-Blochl-12}
\end{equation}
式中$g_{RL}(r)$是广义的\textrm{Gaussian}函数,表示如下
	$$g_{RL}(r)=C_l|r-R|^lY_L(r-R)\mathrm{e}^{-(|r-R|/r_c)^2}$$
其系数$C_l$是归一化系数,由条件
	$\int\mathrm{d}rr^lY_L(r)g_L(r)=1$
确定。
式\eqref{eq:PAW-Blochl-12}的$Q_{RL}$由构造补偿电荷所须满足的多极矩条件确定
	$$Q_{RL}=\int\mathrm{d}r|r-R|^l\big[n_R^1(r)+n_R^Z(r)-\tilde n_R^1(r)\big]Y_L^{\ast}(r-R)$$

一般来说,原子的补偿电荷的\textrm{Gaussian}展开在缀加区会快速衰减,这意味着最终需要很高的平面波来展开。在\textrm{Bl\"ochl}的方案中,建议引入新的补偿电荷$\hat n^{\prime}$,满足条件:~\footnote{这是\textrm{Bl\"ochl}方案与后来\textrm{Kresse}方案处理补偿电荷的主要区别。}
	\begin{itemize}
		\item $\hat n^{\prime}$与$\hat n$具有相同的多极矩(保留原来的补偿电荷的基本要求)
		\item $\hat n^{\prime}$的\textrm{Gaussian}函数展开的衰减半径$r_c^{\prime}$比$r_c$大得多,可以用很少的平面波展开
	\end{itemize}
因此,能量$\tilde E$中的静电相互作用可以表示为
	\begin{equation}
		\begin{aligned}
			&\dfrac12\int\mathrm{d}r\int\mathrm{d}r^{\prime}\dfrac{(\tilde n+\hat n)(\tilde n+\hat n)}{|r-r^{\prime}|}\\
			=&\underline{\dfrac12\int\mathrm{d}r\int\mathrm{d}r^{\prime}\dfrac{(\tilde n+\hat n^{\prime})(\tilde n+\hat n^{\prime})}{|r-r^{\prime}|}}
			+\underline{\int\mathrm{d}r\tilde n(r)\hat v(r)}+\underline{\sum_{R,R^{\prime}}U_{R,R^{\prime}}}
		\end{aligned}
		\label{eq:PAW-Blochl-13}
	\end{equation}
其中第一项是平滑函数,可以在\textrm{Fourier}空间计算
	$$2\pi V\sum_G\dfrac{|\tilde n(G)+\hat n^{\prime}(G)|^2}{G^2}$$
第二项的$\hat v(r)$表示为
	$$\hat v(r)=\int\mathrm{d}r^{\prime}\dfrac{\hat n(r^{\prime})-\hat n^{\prime}(r^{\prime})}{|r-r^{\prime}|}$$
虽然$\hat v(r)$和$n(r)$一样有高\textrm{Fourier}截断,但因为$\tilde n(G)$只需要少量平面波展开,所以$\hat v(r)$的高阶部分不会对$\int\mathrm{d}r\tilde n(r)\hat v(r)$有贡献。最后一项中$U_{R,R^{\prime}}$是原子间的短程成对势
	$$U_{R,R^{\prime}}=\dfrac12\int\mathrm{d}r\int\mathrm{d}r^{\prime}\dfrac{\hat n_R(r)\hat n_{R^{\prime}}(r^{\prime})-\hat n_R^{\prime}(r)\hat n_{R^{\prime}}^{\prime}(r^{\prime})}{|r-r^{\prime}|}$$
这一项可以通过\textrm{Ewald}求和方法计算。

\subsubsection{PAW表示下的Kohn-Sham方程与原子分波、投影函数}
在\textrm{DFT}框架下,应用\textrm{PAW}方法,除了总能量的表示外,还要考虑\textrm{Kohn-Sham}方程。根据式\eqref{eq:PAW-Blochl-05},平面波基表示的重叠算符可以写成:
\begin{equation}
	\tilde O=\mathbf{1}+\sum_{i,j}|\tilde p_i\rangle\bigg[\langle\phi_i|\phi_j\rangle-\langle\tilde\phi_i|\tilde\phi_j\rangle\bigg]\langle\tilde p_j|
	\label{eq:PAW-Blochl-14}
\end{equation}
类似地,可以得到根据定义,经典的\textrm{DFT}中\textrm{Hamilitonian}算符:%\upcite{PRB50-17953_1994}
\begin{equation}
	\begin{aligned}
		\dfrac{\mathrm{d}E}{\mathrm{d}\rho}=&\dfrac{\partial\mathrm{Tr}[-\frac12\nabla^2\rho]}{\partial\rho}+\int\mathrm{d}r\dfrac{\partial E}{\partial n(r)}\dfrac{\mathrm{Tr}[|r\rangle\langle r|\rho]}{\partial\rho}\\
		=&-\dfrac12\nabla^2+v
	\end{aligned}
	\label{eq:PAW-DFT-H}
\end{equation}
这里$v(r)=|r\rangle\dfrac{\partial E}{\partial n(r)}\langle r|$。\textrm{PAW}方法中,\textrm{Hamiltonian}算符是总能量对赝电荷密度$\tilde\rho=\sum\limits_i|\tilde\Psi_n\rangle f_n\langle\tilde\Psi_n|$的变分(约束条件是$\langle\tilde\Psi_n|\tilde O|\tilde\Psi_m\rangle=\delta_{nm}$):
\begin{equation}
	\begin{aligned}
		\dfrac{\mathrm{d}E}{\mathrm{d}v\tilde\rho}=&\dfrac{\partial\mathrm{Tr}[-\frac12\nabla^2\rho]}{\partial\rho}+\int\mathrm{d}r\dfrac{\partial E}{\partial n(r)}\dfrac{\mathrm{Tr}[|r\rangle\langle r|\rho]}{\partial\rho}\\
		=&-\dfrac12\nabla^2+v
	\end{aligned}
	\label{eq:PAW-Blochl-H}
\end{equation}
注意,这里势能是$\tilde n$、$\tilde n^1$、$\tilde n^1$和多极矩$Q_{\mathrm{RL}}$的函数,因此可得
\begin{equation}
	\begin{aligned}
		\dfrac{\mathrm{d}E}{\mathrm{d}\tilde\rho}=&\dfrac{\partial\mathrm{Tr}[\tilde\rho\tilde T]}{\partial\rho}+\int\mathrm{d}r\dfrac{\partial E}{\partial n(r)}\dfrac{\partial\tilde n}{\partial\rho}\\
		&+\int\mathrm{d}r\bigg(\dfrac{\partial E}{\partial n^1}+\sum_{\mathrm{R,L}}\dfrac{\partial E}{\partial Q_{\mathrm{RL}}}\dfrac{\partial Q_{\mathrm{RL}}}{\partial n^1}\bigg)\dfrac{\partial n^1}{\partial\rho}\\
		&+\int\mathrm{d}r\bigg(\dfrac{\partial E}{\partial\tilde n^1}+\sum_{\mathrm{R,L}}\dfrac{\partial E}{\partial Q_{\mathrm{RL}}}\dfrac{\partial Q_{\mathrm{RL}}}{\partial\tilde n^1}\bigg)\dfrac{\partial\tilde n^1}{\partial\rho} 
	\end{aligned}
	\label{eq:PAW-Blochl-H-n}
\end{equation}
式\eqref{eq:PAW-Blochl-H-n}中动能算符$\tilde T$的定义为:~
		\begin{equation}
			\tilde T=-\dfrac12\nabla^2+\sum_{i,j}|\tilde p_i\rangle[\langle\phi_i|-\dfrac12\nabla^2|\phi_j\rangle-\langle\tilde\phi_i|-\dfrac12\nabla^2|\tilde\phi_j\rangle]\langle\tilde p_j|
			\label{eq:PAW-Blochl-T}
		\end{equation}
式\eqref{eq:PAW-Blochl-H-n}中对赝电荷密度求导的项为:~
\begin{equation}
	\begin{aligned}
		\tilde v(r)=\dfrac{\partial E}{\partial\tilde n(r)}=&\int\mathrm{d}r^{\prime}\dfrac{\tilde n(r^{\prime})+\hat n^{\prime}(r^{\prime})}{|r-r^{\prime}|}\\
		&+\hat v(r)+\bar v(r)+\mu_{\mathrm{XC}}[\tilde n(r)]
	\end{aligned}
	\label{eq:PAW-Blochl-v}
\end{equation}
式\eqref{eq:PAW-Blochl-H-n}中对多极矩的求导,注意到多极矩是通过补偿电荷$\hat n$和$\hat n^{\prime}$进入总能量的表达式,因此有:~
\begin{displaymath}
	\begin{aligned}
		\dfrac{\partial E}{\partial Q_{\mathrm{RL}}}=&\int\mathrm{d}r\dfrac{\partial E}{\partial \hat n(r)}\dfrac{\partial\hat n(r)}{\partial Q_{\mathrm{RL}}}+\int\mathrm{d}r\dfrac{\partial E}{\partial \hat n^{\prime}(r)}\dfrac{\partial\hat n^{\prime}(r)}{\partial Q_{\mathrm{RL}}}\\
		=&\int_{\mathrm M}\mathrm{d}r\int_{\mathrm M}\mathrm{d}r^{\prime}\dfrac{g_{\mathrm{RL}}(r)\tilde n(r^{\prime})+g_{\mathrm{RL}}^{\prime}(r)\hat n^{\prime}(r^{\prime})}{|r-r^{\prime}|}\\
		=&\int_{\mathrm A}\mathrm{d}r\int_{\mathrm A}\mathrm{d}r^{\prime}\dfrac{g_{\mathrm{RL}}(r)\tilde n(r^{\prime})+g_{\mathrm{RL}}^{\prime}(r)\hat n^{\prime}(r^{\prime})}{|r-r^{\prime}|}\\
		=&\int_{\mathrm{RG}}\mathrm{d}r\int_{\mathrm{RG}}\mathrm{d}r^{\prime}\dfrac{g_{\mathrm{RL}}(r)\tilde n(r^{\prime})+g_{\mathrm{RL}}^{\prime}(r)\hat n^{\prime}(r^{\prime})}{|r-r^{\prime}|}
	\end{aligned}
\end{displaymath}
这里积分域$\mathrm{M}$表示平面波网格点(\textrm{Fourier}网格点),$\mathrm{RG}$表示径向网格点,$\mathrm{A}$表示解析积分(或\textrm{Ewald}求和),因此可有
\begin{equation}
	\begin{aligned}
		v_{\mathrm R}^0(r)=&\sum_{\mathrm L}\dfrac{\partial E}{\partial Q_{\mathrm{RL}}}\dfrac{\partial Q_{\mathrm{RL}}}{\partial n^1(r)}=-\sum_{\mathrm L}\dfrac{\partial E}{\partial Q_{\mathrm{RL}}}\dfrac{\partial Q_{\mathrm{RL}}}{\partial\tilde n^1(r)}\\
		=&\sum_{\mathrm L}(r-R)^lY_L^{\ast}(|r-R|)\dfrac{\partial E}{\partial Q_{\mathrm{RL}}}
	\end{aligned}
	\label{eq:PAW-Blochl-v_R}
\end{equation}
\begin{equation}
	\begin{aligned}
		v_{\mathrm R}^1(r)=&\dfrac{\partial E}{\partial n^1(r)}+\sum_{\mathrm L}\dfrac{\partial E}{\partial Q_{\mathrm{RL}}}\dfrac{\partial Q_{\mathrm{RL}}}{\partial n^1(r)}\\
		=&\int_{\mathrm R}\mathrm{d}r^{\prime}\dfrac{n_{\mathrm R}^1(r^{\prime})+n_{\mathrm R}^Z(r^{\prime})}{|r-r^{\prime}|}+\mu_{\mathrm{XC}}[n_{\mathrm R}^1(r)]+v_{\mathrm R}^0(r)
	\end{aligned}
	\label{eq:PAW-Blochl-v1_R}
\end{equation}
\begin{equation}
	\begin{aligned}
		\tilde v_{\mathrm R}^1(r)=&\bigg(\dfrac{\partial E}{\partial\tilde n^1(r)}+\sum_{\mathrm L}\dfrac{\partial E}{\partial Q_{\mathrm{RL}}}\dfrac{\partial Q_{\mathrm{RL}}}{\tilde n^1(r)})\\
		=&\int_{\mathrm R}\mathrm{d}r^{\prime}\dfrac{\tilde n_{\mathrm R}^1(r^{\prime})+\hat n_{\mathrm R}(r^{\prime})}{|r-r^{\prime}|}+\mu_{\mathrm{XC}}[\tilde n_{\mathrm R}^1(r)]+v_{\mathrm R}^0(r)
	\end{aligned}
	\label{eq:PAW-Blochl-tv1_R}
\end{equation}
综上,\textrm{Hamilton}算符:
\begin{equation}
	\begin{aligned}
		\tilde H=&-\dfrac12\nabla^2+\tilde v+\sum_{i,j}|\tilde p_i\rangle\bigg[\langle\phi_i|-\dfrac12\nabla^2+v^1|\phi_j\rangle\\
			&-\langle\tilde\phi_i|-\dfrac12\nabla^2+\tilde v^1|\tilde\phi_j\rangle\bigg]\langle\tilde p_j| 
	\end{aligned}
	\label{eq:PAW-Blochl-15}
\end{equation}
在\textrm{PAW}方法中,完整的势函数\textrm{(full potential)}算符表示为:
\begin{equation}
	v(\vec r)=\tilde v(\vec r)+v^1(\vec r)-\tilde v^1(\vec r)
	\label{eq:PAW-Blochl-16}
\end{equation}
与电荷密度的表示类似,势函数是由平滑的平面波表示赝势函数$\tilde v$和位于每个原子缀加区的单中心局域的原子势$v^1$和$\tilde v^1$叠加而成。平滑势可用平面波展开,而原子势则表示成径向函数和角度部分乘积。

总能量泛函对赝波函数求导,可有
\begin{equation}
	\left.\dfrac{\partial E[\tilde\Psi, R]}{\partial\langle\tilde\Psi_n|}\right|_R=\tilde H|\tilde\Psi_n\rangle f_n
	\label{eq:PAW-Blochl-17}
\end{equation}

\subsection{PAW方法与USPP的内在联系}
%\begin{figure}[h!]
%\centering
%\includegraphics[height=2.3in,width=4.0in,viewport=0 0 1280 745,clip]{PAW-baseset.png}
%\caption{\small \textrm{The Augmentation of PAW.}}%(与文献\upcite{EPJB33-47_2003}图1对比)
%\label{PAW_baiseset}
%\end{figure}
%
%\frame
%{
%	\frametitle{\textrm{PAW Augmentation}}
%\begin{figure}[h!]
%\centering
%\includegraphics[height=2.3in,width=4.0in,viewport=0 0 1100 745,clip]{Figures/PAW-projector.png}
%\caption{\small \textrm{The projector of PAW.}}%(与文献\upcite{EPJB33-47_2003}图1对比)
%\label{PAW_projector}
%\end{figure}
%}
\textrm{PAW}方法在提出后的很长一段时间内都没有得到足够重视,直到\textrm{G. Kresse}等\upcite{PRB59-1758_1999}将\textrm{Bl\"ochl}的原始方案中电荷密度计算方法作出调整,明确了\textrm{PAW}方法与\textrm{USPP}方法的内在联系后,特别是伴随着\textrm{Kresse}等将\textrm{PAW}方法引入第一原理分子动力学模拟软件包\textrm{VASP~(Vienna Ab-initio Simulation Package)}\upcite{VASP_manual}中,有力地推动了\textrm{PAW}方法的广泛应用。

\textrm{Kresse}等注意到了\textrm{PAW}方法与\textrm{USPP}方法的密切关系,指出如果投影函数$\tilde p_i$相同,\textrm{PAW}方法和\textrm{USPP}方法计算得到的总电荷密度(式\eqref{eq:PAW-Blochl-07})是完全等价的,只是实际计算时,\textrm{USPP}方法直接赝化补偿电荷。\footnote{严格地说,\textrm{Kresse}等提出的\textrm{PAW}方法是一种冻芯近似的全电子方法。}为了更清晰地阐明\textrm{PAW}方法和\textrm{USPP}方法的关系,\textrm{Kresse}等引入冻芯近似(\textrm{frozen core approximation}),明确指出$\tilde n$、$\tilde n^1$和$n^1$仅限于描述价电子电荷密度。对于芯电荷与核电荷,引入$n_c$、$\tilde n_c$和$n_{\mathrm{Z}c}$和$\tilde n_{\mathrm{Z}c}$,其中$n_c$和$\tilde n_c$是动芯近似下的芯层电荷密度,$n_{\mathrm{Z}c}$是核电荷(点核电$n_{\mathrm Z}$)和冻芯电荷$n_c$的和
\begin{displaymath}
	n_{\mathrm{Z}c}=n_{\mathrm{Z}}+n_c
\end{displaymath}
赝芯电荷$n_{\mathrm{Z}c}$的构造须满足条件
\begin{equation}
	\int_{\Omega_r}n_{\mathrm{Z}c}(\vec r)\mathrm{d}^3\vec r=\int_{\Omega_r}\tilde n_{\mathrm{Z}c}(\vec r)\mathrm{d}^3\vec r
	\label{eq:PAW_Kresse_01}
\end{equation}
这里积分$\int_{\Omega_r}$表示对缀加区径向积分;~对$n_{\mathrm{Z}c}$和$\tilde n_{\mathrm{Z}c}$的积分满足电中性要求,即积分区的总电荷为$-Z{\mathrm{ion}}$。在具体计算中,在平面波表示区,电荷密度是所有原子电荷密度的叠加,局域原子附近的缀加区,只考虑当前原子的电荷密度贡献。
%\begin{itemize}
%	\item 芯层电荷与核电荷构成离子实电荷:$n_{Zc}=n_Z+n_c$
%	\item 赝离子实电荷的构造$$\int_{\Omega_c}n_{Zc}(\vec r)\mathrm{d}^3\vec r=\int_{\Omega_c}\tilde n_{Zc}(\vec r)\mathrm{d}^3\vec r$$
%\end{itemize}

为了揭示\textrm{PAW}方法与\textrm{USPP}的关联,\textrm{Kresse}将\textrm{Bl\"ochl}方案中的电荷密度分解方式由“原子核+电子”改变为“离子实+价电子”形式:~
\begin{equation}
	\begin{aligned}
		n_T=n+n_{Zc}\equiv&\underbrace{(\tilde n+\hat n+\tilde n_{Zc})}\\
				 		&\quad\qquad\tilde n_T\\
				  &+\underbrace{(n^1+\hat n+n_{Zc})}-\underbrace{(\tilde n^1+\hat n+\tilde n_{Zc})}\\
				                  &\quad\qquad n_T^1\qquad\qquad\qquad\tilde n_T^1
	\end{aligned}
	\label{eq:PAW_Kresse_02}
\end{equation}
\textrm{Kresse}方案中补偿电荷$\hat n$的构造,没有用\textrm{Gaussian}函数展开,而是与\textrm{LAPW}方法相似。
局域在每个缀加球内。

因为$\tilde n_T$在缀加区外的电荷与真实电荷相同,不难证明,用$\tilde n_T$计算不同缀加区之间、间隙区和缀加区之间静电相互作用时是精确的,只是在计算每个缀加区内部在位相互作用(on site interaction)时,才会引入误差。因此可有
\begin{equation}
	\begin{aligned}
		\dfrac12(n_T)(n_T)=&\dfrac12(\tilde n_T)(\tilde n_T)+(n_T^1-\tilde n_T^1)(\tilde n_T)\\
				&+\dfrac12(n_T^1-\tilde n_T^1)(n_T^1-\tilde n_T)
	\end{aligned}
	\label{eq:PAW_Kresse_03}
\end{equation}
这里记号$(a)(b)$表示$$(a)(b)=\int\mathrm{d}\vec r\mathrm{d}\vec r^{\prime}\dfrac{a(\vec r)b(\vec r\,^{\prime})}{|\vec r-\vec r\,^{\prime}|}$$
注意到因为构造的$\hat n$使得$(n_T^1-\tilde n_T)$在每个缀加区内的多极矩为零,因此式\eqref{eq:PAW_Kresse_03}第二、第三项的贡献只要考虑每个缀加区内部分。不过第二项积分是平面波网格点上的$\tilde n_T$和缀加区网格点上的$n_T^1-\tilde n_T^1$的积分。这里沿用\textrm{Bl\"ochl}近似,用$\tilde n_T^1$代替$\tilde n_T$(当投影函数是完备基时,这一处理是精确的)。\footnote{该近似引入的误差为
$(n_T^1-\tilde n_T^1)(\tilde n_T-\tilde n_T^1)$。}
因此,电子的静电相互作用近似为
\begin{equation}
	\dfrac12(n_T)(n_T)=\dfrac12(\tilde n_T)(\tilde n_T)-\dfrac12\overline{(\tilde n_T^1)(\tilde n_T^1)}+\dfrac12\overline{(n_T^1)(n_T^1)}
	\label{eq:PAW_Kresse_04}
\end{equation}
这里记号$\overline{(a)(b)}$表示只考虑缀加区的径向网格积分,最终式\eqref{eq:PAW_Kresse_03}不再有不同网格积分的贡献。用\textrm{Kresse}的电荷密度分解,式\eqref{eq:PAW_Kresse_04}的表达形式更接近传统赝势,其中第一项可写成:
\begin{equation}
	\begin{aligned}
		&\dfrac12(\tilde n+\hat n)(\tilde n+\hat n)+(\tilde n_{\mathrm{Z}c})(\tilde n+\hat n)+\dfrac12(\tilde n_{\mathrm{Z}c})(\tilde n_{\mathrm{Z}c})\\
		=&\dfrac12(\tilde n+\hat n)(\tilde n+\hat n)+(\tilde n_{\mathrm{Z}c})(\tilde n+\hat n)+\dfrac12\overline{(\tilde n_{\mathrm{Z}c})(\tilde n_{\mathrm{Z}c})}+U(\vec R,Z_{\mathrm{ion}})
	\end{aligned}
	\label{eq:PAW_Kresse_05}
\end{equation}
这里等式中假设芯层电荷并不重叠,其中$\dfrac12(\tilde n+\hat n)(\tilde n+\hat n)$描述价电子在平面波网格上的静电相互作用;~$(\tilde n_{\mathrm{Z}c})(\tilde n+\hat n)$描述冻芯赝电荷与价电子的相互作用;~$U(\vec R,Z_{\mathrm{ion}})$是点电荷$Z_{\mathrm{ion}}$相互作用,一般采用\textrm{Ewald}求和计算。

式\eqref{eq:PAW_Kresse_04}的第二项是
\begin{equation}
	-\dfrac12\overline{(\tilde n^1+\hat n)(\tilde n^1+\hat n)}-\overline{(\tilde n_{\mathrm{Z}c})(\tilde n^1+\hat n)}-\dfrac12\overline{(\tilde n_{\mathrm{Z}c})(\tilde n_{\mathrm{Z}c})}
	\label{eq:PAW_Kresse_06}
\end{equation}
显然式中$\dfrac12\overline{(\tilde n_{\mathrm{Z}c})(\tilde n_{\mathrm{Z}c})}$将与式\eqref{eq:PAW_Kresse_05}中对应部分抵消。

式\eqref{eq:PAW_Kresse_04}的第三项可以写成
\begin{equation}
	\dfrac12\overline{(n^1)(n^1)}+\overline{(n_{\mathrm{Z}c})(n^1)}+\dfrac12\overline{(n_{\mathrm{Z}c})(n_{\mathrm{Z}c})}
	\label{eq:PAW_Kresse_07}
\end{equation}
注意,式\eqref{eq:PAW_Kresse_05}-\eqref{eq:PAW_Kresse_07}确定体系电子的经典\textrm{Hartree}相互作用,但是在最终的总能量表达式中,并不包括赝芯电荷的自相互作用$\dfrac12\overline{(\tilde n_{\mathrm{Z}c})(\tilde n_{\mathrm{Z}c})}$,因为这一项只是会影响能量零点位置的定义。

交换-相关能泛函计算时,要包括全部电子密度的贡献,\textrm{G. Kresse}方案中电子密度的分解方式为:~
\begin{equation}
	n_c+n=(\tilde n+\hat n+\tilde n_c)+(n^1+n_c)-(\tilde n^1+\hat n+\tilde n_c)
	\label{eq:PAW_Kresse_08}
\end{equation}
与\textrm{Bl\"ochl}方案中电荷分解(式\eqref{eq:PAW-Blochl-07})异趣。另一方面,注意到交换-相关能泛函是非线性的,因此交换-相关能的计算公式写成:~
\begin{equation}
	E_{\mathrm{XC}}[\tilde n+\hat n+\tilde n_c]+\overline{E_{\mathrm{XC}}[n^1+n_c]}-\overline{E_{\mathrm{XC}}[\tilde n^1+\hat n+\tilde n_c]}
	\label{eq:PAW_Kresse_09}
\end{equation}
这里$\overline{E}$表示来自缀加区径向积分贡献。按照\textrm{Kresse}的电荷密度分解方案,$\tilde n^1+\hat n+\tilde n_c$与$n^1+n_c$在缀加区及很大范围内接近,这比\textrm{Bl\"ochl}方案大大降低了分波不完备引起的误差,对于芯层态扩展到缀加区边缘的体系,该电荷密度分解的优势更显著。
%\textcolor{blue}{两种不同的电荷密度分解方案根源}:\\\textrm{G. Kresse}方案中赝离子实电荷$\tilde n_{Zc}$与\textrm{Bl\"ochl}方案中$\tilde n_c$的约束条件不同!

与\textrm{Bl\"ochl}方案类似,\textrm{Kresse}方案的体系总能量表达式可以写成:
$$E=\tilde E+E^1-\tilde E^1$$其中
	\begin{equation}
		\begin{aligned}
			\tilde E=&\sum_nf_n\langle\tilde\Psi_n|-\frac12\nabla^2|\tilde\Psi_n\rangle+E_{\mathrm{XC}}[\tilde n+\hat n+\tilde n_c]+E_H[\tilde n+\hat n]\\
			&+\int v_H[\tilde n_{Zc}][\tilde n(\vec r)+\hat n(\vec r)]\mathrm{d}\vec r+U(\vec R,Z_{\mathrm{ion}})\\
		\end{aligned}
		\label{eq:PAW_Kresse_10-1}
	\end{equation}
	\begin{equation}
		\begin{aligned}
			\tilde E^1=&\sum_{(i,j)}\rho_{ij}\langle\tilde\phi_i|-\frac12\nabla^2|\tilde\phi_j\rangle+\overline{E_{\mathrm{XC}}[\tilde n^1+\hat n+\tilde n_c]}+\overline{E_H[\tilde n^1+\hat n]}\\
			&+\int_{\Omega_r}v_H[\tilde n_{Zc}][\tilde n^1(\vec r)+\hat n(\vec r)]\mathrm{d}\vec r
		\end{aligned}
		\label{eq:PAW_Kresse_10-2}
	\end{equation}
	\begin{equation}
		\begin{aligned}
			E^1=&\sum_{(i,j)}\rho_{ij}\langle\phi_i|-\frac12\nabla^2|\phi_j\rangle+\overline{E_{\mathrm{XC}}[n^1+n_c]}+\overline{E_H[n^1]}\\
			&+\int_{\Omega_r}v_H[n_{Zc}]n^1(\vec r)\mathrm{d}\vec r
		\end{aligned}
		\label{eq:PAW_Kresse_10-3}
	\end{equation}
这里$\rho_{ij}$是轨道电子的占据数密度矩阵:
\begin{displaymath}
	\rho_{ij}=\sum_nf_n\langle\tilde\Psi_n|\tilde p_i\rangle\langle\tilde p_j|\tilde\Psi_n\rangle
\end{displaymath}
$v_{\mathrm{H}}$是电荷密度$n$的\textrm{Coulomb}势:
\begin{displaymath}
	v_H[n](\vec r)=\int\dfrac{n(\vec r\,^{\prime})}{|\vec r-\vec r\,^{\prime}|}\mathrm{d}\vec r\,^{\prime}
\end{displaymath}
$E_{\mathrm H}[n]$是对应的经典\textrm{Hartree}能
\begin{displaymath}
	E_{\mathrm H}[n]=\dfrac12(n)(n)=\dfrac12\int\mathrm{d}\vec r\mathrm{d}\vec r\,^{\prime}\dfrac{n(\vec r)n(\vec r\,^{\prime})}{|\vec r-\vec r\,^{\prime}|}
\end{displaymath}
$\tilde E$中的积分在平面波网格点上计算,而$\tilde E^1$和$E^1$中的积分则是在每个缀加区的径向网格点计算,并且只考虑价电子的贡献。与\textrm{Bl\"ochl}方法相比:~两者的区别主要是
\begin{enumerate}
	\item \textrm{Hartree}能计算处理不同:~\textrm{Kresse}方案中芯电荷不重叠,因此不考虑$\dfrac12\overline{(\tilde n_{\mathrm{Z}c})(\tilde n_{\mathrm{Z}c})}$的贡献。其根源则在于两者的补偿电荷构造不同:~\textrm{Bl\"ochl}方案的补偿电荷与总电荷密度差($n^1-\tilde n^1+n_{\mathrm{Z}c}$)有相同的多极矩,而\textrm{Kresse}方案中则是价电子电荷密度差($n^1-\tilde n^1$),因此\textrm{Bl\"ochl}方案中,芯层电荷相互作用包括在$E_{\mathrm H}[\tilde n+\hat n]$中,而\textrm{Kresse}方案中,芯电荷的相互作用,完全通过式\eqref{eq:PAW_Kresse_10-1}的$U(\vec R,Z_{\mathrm{ion}})$计算;
	\item 交换-相关能的计算方式不同:~对于平面波基组的积分贡献,\textrm{Bl\"ochl}方案中电荷密度是$\tilde n$,而\textrm{Kresse}方案则考虑了$\tilde n+\hat n$和$\tilde n_c$对泛函的贡献。这两种方案本质上是等价的,严格地说,形式上\textrm{Bl\"ochl}方案的交换-相关能计算更严格,但实际应用中\textrm{Kresse}方案更有优势。
\end{enumerate}

%	$U(\vec R,Z_{\mathrm{ion}})$\textcolor{blue}{由\textrm{Ewald}求和计算}
根据\textrm{Kresse}方案,补偿电荷$\hat n$要求满足$\tilde n^1+\hat n$与$n^1$在缀加区有相同的多极矩,即约束条件满足 
\begin{equation}
	\int_{\Omega_c}(n^1-\tilde n^1-\hat n)|\vec r-\vec R|^lY_{lm}^{\ast}(\widehat{\vec r-\vec R})\mathrm{d}\vec r=0
	\label{eq:PAW_Kresse_11}
\end{equation}
参照\textrm{USPP}的基本思想,定义电荷密度差
\begin{equation}
	Q_{ij}(\vec r)=\phi_i^{\ast}(\vec r)\phi_j(\vec r)-\tilde\phi_i^{\ast}(\vec r)\tilde\phi_j(\vec r)
	\label{eq:PAW_Kresse_12}
\end{equation}
$Q_{ij}(\vec r)$对应的多极矩为
\begin{equation}
	q_{ij}^L(\vec r)=\int_{\Omega_r}Q_{ij}(\vec r)|\vec r-\vec R|^lY_{lm}^{\ast}(\widehat{\vec r-\vec R})\mathrm{d}\vec r
	\label{eq:PAW_Kresse_13}
\end{equation}
参照\textrm{LAPW}方法的赝电荷密度构造的思想,满足约束条件式\eqref{eq:PAW_Kresse_11}的补充电荷的计算形式为:~
\begin{equation}
	\begin{aligned}
		\hat n=\sum_{(i,j),L}\sum_n f_n\langle\tilde\Psi_n|\tilde p_i\rangle\langle\tilde p_j|\Psi_n\rangle\hat Q_{ij}^L(\vec r)\\
		\hat Q_{ij}^L(\vec r)=q_{ij}^Lg_l(|\vec r-\vec R|)Y_{lm}(\widehat{\vec r-\vec R})
	\end{aligned}
	\label{eq:PAW_Kresse_14}
\end{equation}
与\textrm{LAPW}方法的主要区别是,式\eqref{eq:PAW_Kresse_14}中$g(r)$的具体形式,将留待在下一节“\textrm{PAW}的原子数据集”中讨论。

除了上述基本区别,从总能量的角度综合考虑,更能表明\textrm{USPP}方法是\textrm{PAW}方法(\textrm{Kresse}“冻芯近似”方案)的近似:~将式\eqref{eq:PAW_Kresse_10-1}-\eqref{eq:PAW_Kresse_10-3}中原子缀加区的贡献按原子价电子占据数比例$\rho_{ij}^\mathrm{a}$作线性近似,类似地,对应的电荷密度记作$n_{\mathrm{a}}^1$、$\tilde n_{\mathrm{a}}^1$、$\hat n_{\mathrm{a}}$,则在$n_{\mathrm{a}}^1$附近,$E^1$中的交换-相关能和\textrm{Hartree}能的贡献展开到一阶的表达式为
\begin{displaymath}
	\begin{aligned}
		&E_{\mathrm{XC}}(n^1_{\mathrm{a}}+n_c)+E_{\mathrm{H}}(n_{\mathrm{a}}^1)\\
		&+\int(v_{\mathrm{XC}}[n_{\mathrm{a}}^1+n_c]+v_{\mathrm{H}}[n_a^1])[n^1(\vec r)-n^1_{\mathrm{a}}(\vec r)]\mathrm{d}\vec r\\
		=&C+\sum_{(i,j)}\rho_{ij}\langle\phi_i|v_{\mathrm{XC}}[n_{\mathrm{a}}^1+n_c]+v_{\mathrm{H}}[n_{\mathrm{a}}^1]|\phi_j\rangle
	\end{aligned}
\end{displaymath}
最后的等式中$C$是常数。另外注意到动能、离子实-价电子相互作用已经随$\rho_{ij}$的线性化展开到一阶,因此总能量对$\rho_{ij}$展开到一阶的表达式为
\begin{equation}
		E^1\approx C+\sum_{i,j}\rho_{ij}\langle\phi_i|-\dfrac12\nabla^2+v_{\mathrm{eff}}^{\mathrm a}|\phi_j\rangle
	\label{eq:PAW_Kresse_E1}
\end{equation}
其中局域势$v_{\mathrm{eff}}^{\mathrm a}$是相应原子的全电子势:~
\begin{displaymath}
	v_{\mathrm{eff}}^{\mathrm a}=v_{\mathrm{H}}[n_a^1+n_{\mathrm{Z}c}]+v_{\mathrm{XC}}[n_{\mathrm{a}}^1+n_c]
\end{displaymath}
类似地,可以将$\tilde E^1$也作近似展开,需要指出的是$\tilde n^1$和$\hat n$都是占据数$\rho_{ij}$的函数,有
\begin{equation}
	\tilde E^1\approx\tilde C+\sum_{i,j}\rho_{ij}\bigg[\langle\tilde\phi_i|-\dfrac12\nabla^2+\tilde v_{\mathrm{eff}}^{\mathrm a}|\phi_j\rangle+\int\hat Q_{ij}^L(\vec r)\tilde v_{\mathrm{eff}}^{\mathrm a}(\vec r)\mathrm{d}\vec r \bigg]
	\label{eq:PAW_Kresse_tE1}
\end{equation}
其中局域势$i\tilde v_{\mathrm{eff}}^{\mathrm a}$是相应原子的赝势:~
\begin{displaymath}
	\tilde v_{\mathrm{eff}}^{\mathrm a}=v_{\mathrm{H}}[\tilde n_a^1+\hat n_{\mathrm{a}}+\tilde n_{\mathrm{Z}c}]+v_{\mathrm{XC}}[n_{\mathrm{a}}^1+\hat n_{\mathrm{a}}+\tilde n_c]
\end{displaymath}
最后得到体系的总能量的近似表达式为
\begin{equation}
	\begin{aligned}
		E=&\sum_nf_n\langle\tilde\Psi_n|-\dfrac12\nabla^2+\sum_{(i,j)}|\tilde p_i\rangle\langle\tilde p_j|G_{ij}^{\mathrm{US}}|\tilde\Psi_n\rangle\\
		&+E_{\mathrm{XC}}[\tilde n+\hat n+\tilde n_c]+E_{\mathrm{H}}[\tilde n+\hat n]\\
		&+\int v_{\mathrm{H}}[\tilde n_{\mathrm{Z}c}][\tilde n(\vec r)+\hat n(\vec r)]\mathrm{d}\vec r+U(\vec R,Z_{\mathrm{ion}})
	\end{aligned}
	\label{eq:PAW_Kresse_E}
\end{equation}
其中
\begin{displaymath}
	\begin{aligned}
		G_{ij}^{\mathrm{US}}=&\underline{\langle\phi_ii\bigg|-\dfrac12\nabla^2+v_{\mathrm{eff}}^{\mathrm{a}}|\phi_j\rangle-\langle\tilde\phi_i|-\dfrac12\nabla^2+\tilde v_{\mathrm{eff}}^{\mathrm{a}}\bigg|\tilde\phi_j\rangle}\\
&-\int\hat Q_{ij}^L(\vec r)\tilde v_{\mathrm{eff}}^{\mathrm a}(\vec r)\mathrm{d}\vec r
	\end{aligned}
\end{displaymath}
\textrm{Kresse}注意到,如果补偿电荷$\hat n$取\textrm{USPP}中的形式,则式\eqref{eq:PAW_Kresse_E}与\textrm{USPP}中总能表达式\eqref{eq:uspp_5}相同。此外$G_{ij}^{\mathrm{US}}$的前两项和第三项分别对应赝势的强度指数(式\eqref{eq:uspp_6}中的$\mathbf{D}_{s,s^{\prime}}$)和其去屏蔽部分($\mathbf{D}_{s,s^{\prime}}^{\mathrm{ion}}$)。

另一方面,从\textrm{PAW}方法的角度考虑,假设$\hat n=n^1-\tilde n^1$,并且$\tilde n_{\mathrm{Z}c}=n_{\mathrm{Z}c}$、$\tilde n_c=n_c$,则式\eqref{eq:PAW_Kresse_10-2}-\eqref{eq:PAW_Kresse_10-3}将只有
\begin{displaymath}
	E^1-\tilde E^1=\sum_{(i,j)}\rho_{ij}(\langle\phi_i|-\dfrac12\nabla^2|\phi_j\rangle-\langle\tilde\phi_i|-\dfrac12\nabla^2|\tilde\phi_i\rangle)
\end{displaymath}
会对总能量有贡献。这也正说明\textrm{USPP}方法是\textrm{PAW}方法的极端情况,特别是当冻芯近似下,两者趋于等价,并且在该极端条件下,补偿电荷满足:~
\begin{displaymath}
	\hat Q_{ij}^L(\vec r)=Q_{ij}(\vec r)=\phi_i^{\ast}(\vec r)\phi_j(\vec r)-\tilde\phi_i^{\ast}(\vec r)\tilde\phi_j(\vec r)
\end{displaymath}
上述简单推导证明,如果\textrm{USPP}方法中提高补偿电荷的赝化函数的构造方式,将有可能系统地提升\textrm{USPP}的计算精度。但是,即使\textrm{USPP}的补偿电荷达到极限($\hat Q_{ij}^L(\vec r)=Q_{ij}(\vec r)$),式\eqref{eq:PAW_Kresse_E1}-\eqref{eq:PAW_Kresse_E}的推导也表明,\textrm{USPP}方法只能对赝化原子电荷密度的修正精确到一阶。因为在实际的不同计算体系中,\textrm{USPP}方法中的赝化补偿电荷引起的赝势移植性误差,并不相同,特别是体系伴有电荷转移(如强极化的共价键和离子键)、原子轨道中电子占据数发生变化(如轨道杂化或电子发生受激跃迁)、受到强极化(如原子电荷分布受偶极矩或四极矩诱导)或受到强的局域磁矩作用时,这种赝势移植性误差都会比较大。

综上所述,\textrm{PAW}方法和\textrm{USPP}的核心差别是对补偿电荷的处理不同:~\textrm{PAW}方法中,补偿电荷是在缀加区的径向网格布点上完成的,特别是如果采用\textrm{Kresse}方案建议的构造方式,补偿电荷可以更平缓;~\textrm{USPP}方法中,为了提升赝势的可靠性,则构造的补偿电荷更收缩(类比图\ref{Norm-US-wave}),这将大大增加计算成本。

在\textrm{DFT}理论中,与总能量关系最密切的是\textrm{Kohn-Sham}方程,\textrm{PAW}方法的赝波函数$\tilde\Psi_n$满足正交条件:~
	\begin{displaymath}
		\langle\tilde\Psi_n|\mathbf{S}|\tilde\Psi_m\rangle=\delta_{nm}
	\end{displaymath}
这里重叠矩阵为
\begin{equation}
	\tilde S[\{\mathbf{R}\}]=\mathbf{1}+\sum_i|\tilde p_i\rangle q_{ij}\langle\tilde p_j|
	\label{eq:PAW_Kresse_15}
\end{equation}
并且$$q_{ij}=\langle\phi_i|\phi_j\rangle-\langle\tilde\phi_i|\tilde\phi_j\rangle$$
在讨论\textrm{Hamiltonian}矩阵时,\textrm{Kresse}方案也特别注意了\textrm{PAW}方法与\textrm{USPP}的关联:~与\textrm{Bl\"ochl}方案类似,
\begin{equation}
	\begin{aligned}
		\tilde H=\dfrac{\mathrm{d}E}{\mathrm{d}\tilde\rho}=\dfrac{\partial E}{\partial\tilde\rho}+\int\dfrac{\delta E}{\delta\tilde n(\vec r)}&\underbrace{\dfrac{\partial\tilde n(\vec r)}{\partial\tilde\rho}}\mathrm{d}\vec r+\sum_{(i,j)}\dfrac{\partial E}{\partial\rho_{ij}}&\underbrace{\dfrac{\partial\rho_{ij}}{\partial\tilde\rho}}\\
		&|\vec r\rangle\langle\vec r| &|\tilde p_i\rangle\langle\tilde p_j|
	\end{aligned}
	\label{eq:PAW_Kresse_16}
\end{equation}
与式\eqref{eq:PAW-Blochl-15}导出不同,\textrm{Kresse}方案根据能量表达式\eqref{eq:PAW_Kresse_10-1}-\eqref{eq:PAW_Kresse_10-3},详细推导了式\eqref{eq:PAW_Kresse_16}的细节,注意到$\dfrac{\partial\tilde E}{\partial\tilde\rho}=-\dfrac12\nabla^2$,并且
\begin{displaymath}
	\tilde v_{\mathrm{eff}}=\dfrac{\delta E}{\delta\tilde n(\vec r)}=v_{\mathrm{H}}[\tilde n+\hat n+\tilde n_{\mathrm{Z}c}]+v_{\mathrm{XC}}[\tilde n+\hat n+\tilde n_c]
\end{displaymath}
\begin{displaymath}
	\hat D_{ij}=\dfrac{\delta E}{\delta\hat n(\vec r)}\dfrac{\partial\hat n(\vec r)}{\partial\rho_{ij}}\mathrm{d}\vec r=\sum_L\int\tilde v_{\mathrm{eff}}(\vec r)\hat Q_{ij}^L(\vec r)\mathrm{d}\vec r
\end{displaymath}
$\hat D_{ij}$的表达式是对赝波函数$\tilde\Psi_n$在长程静电行为的修正(与全电子波函数$\Psi_n$相比)。

类似地,对$E^1$和$\tilde E^1$类似处理可有
$$D_{ij}^1=\dfrac{\partial E^1}{\partial\rho_{ij}}=\langle\phi_i|-\dfrac12\nabla^2+v_{\mathrm{eff}}^1|\phi_j\rangle$$
其中$$v_{\mathrm{eff}}^1[n^1]=v_{\mathrm{H}}[n^1+n_{\mathrm{Z}c}]+v_{\mathrm{XC}}[n^1+n_c]$$
	$$\tilde D_{ij}^1=\dfrac{\partial\tilde E^1}{\partial\rho_{ij}}=\langle\tilde\phi_i|-\dfrac12\nabla^2+\tilde v_{\mathrm{eff}}^1|\tilde\phi_j\rangle+\sum_L\int_{\Omega_r}\mathrm{d}\vec r\tilde v_{\mathrm{eff}}^1(\vec r)\hat Q_{ij}^L(\vec r)$$
其中$$\tilde v_{\mathrm{eff}}^1[\tilde n^1]=v_{\mathrm{H}}[\tilde n^1+\hat n+\tilde n_{Zc}]+v_{\mathrm{XC}}[\tilde n^1+\hat n+\tilde n_c]$$
最后,\textrm{Hamiltonian}矩阵表示为
\begin{equation}
	H[\rho,\{\mathbf{R}\}]=-\dfrac12\nabla^2+\tilde v_{\mathrm{eff}}+\sum_{(i,j)}|\tilde p_i\rangle(\hat D_{ij}+D_{ij}^1-\tilde D_{ij}^1)\langle\tilde p_j|
		\label{eq:PAW_Kresse_17}
\end{equation}
注意,形式上看\eqref{eq:PAW_Kresse_17}与\textrm{Bl\"ochl}方案的式\eqref{eq:PAW-Blochl-15}很相似,但是如果逐项对比,却又似乎很难看出彼此的对应关系,一方面是因为两种方案的交换-相关势的计算方式有差别,也是两者补偿电荷构造方式不同引起的,\textrm{Bl\"ochl}方案的补偿电荷彼此允许重叠,而\textrm{Kresse}方案中引入冻芯近似,补偿电荷局域在缀加区内。只要将式\eqref{eq:PAW-Blochl-v_R}-\eqref{eq:PAW-Blochl-tv1_R}的$v_{\mathrm R}^0$的表达式展开,可以看出这是$\hat D_{ij}$和$\tilde D_{ij}^1$的第二项重新排列组合,表面上看这几项并未出现在式\eqref{eq:PAW-Blochl-15}中,但是从\textrm{Kresse}方案式\eqref{eq:PAW_Kresse_17}更直观地看出两者的关系:
\begin{itemize}
	\item $\bigg(-\dfrac12\nabla^2+\tilde v_{\mathrm{eff}}\bigg)$是一般\textrm{Kohn-Sham}方程都有的项
	\item $D_{ij}^1$和$\tilde D_{ij}^1$描述缀加区的在位项相互作用:~表明$\tilde v_{\mathrm{eff}}$与赝波函数$\tilde\Psi_n$在离子实附近并未出现剧烈变化
	\item $\hat D_{ij}$有关的项
		$$\sum_{(i,j),L}\langle\tilde\Psi_n|\tilde p_i\rangle\langle\tilde p_j|\tilde\Psi_n\rangle\int\tilde v_{\mathrm{eff}}(\vec r)\hat Q_{ij}^L(\vec r)\mathrm{d}\vec r$$
		描述的是每个电子的补偿电荷与有效单电子势(长程静电效应)的相互作用
	\item 在缀加区$\tilde D_{ij}^1$与$-\dfrac12\nabla^2+\tilde v_{\mathrm{eff}}+\sum\limits_{(i,j)}|\tilde p_i\rangle\hat D_{ij}\langle\tilde p_j|$彼此抵消(当分波函数是完备的,这种抵消是精确的)
\end{itemize}
所以式\eqref{eq:PAW-Blochl-15}与式\eqref{eq:PAW_Kresse_17}是本质上是等价的。

\textrm{Kresse}等还注意到,如果将式\eqref{eq:PAW_Kresse_17}中$D_{ij}^1$和$\tilde D_{ij}^1$中的$v_{\mathrm{eff}}^1$和$\tilde v_{\mathrm{eff}}^1$替换成原子势$v_{\mathrm{eff}}^a$和$\tilde v_{\mathrm{eff}}^a$,则可以得到\textrm{USPP}的\textrm{Hamiltonian},从这个角度看,\textrm{USPP}方法可以近似成\textrm{PAW}方法中的$D_{\mathrm{eff}}^1$和$\tilde D_{\mathrm{eff}}^1$在迭代过程保持不变的计算。
%	$$\tilde v_{eff}=v_H[\tilde n+\hat n+\tilde n_{Zc}]+v_{\mathrm{XC}}[\tilde n+\hat n+\tilde n_{Zc}]$$
%
%	$$\hat D_{ij}=\dfrac{\partial\tilde E}{\partial\rho_{ij}}=\int\dfrac{\delta\tilde E}{\delta\hat n(\vec  r)}\dfrac{\partial\hat n(\vec r)}{\partial\rho_{ij}}\mathrm{d}\vec r=\sum_{L}\int\tilde v_{eff}\hat Q_{ij}^L(\vec r)\mathrm{d}\vec r$$

实际的计算中,由于体系动能计算比较复杂,一般总能量都通过式\eqref{eq:PP_TOT_R}更方便,\textrm{Kohn-Sham}本征值求和再扣除\textrm{Double counting},回避直接的动能计算。在\textrm{PAW}方法中,修正项为
	\begin{equation}
		\begin{aligned}
			\tilde E_{dc}=&-E_H[\tilde n+\hat n]+E_{\mathrm{XC}}[\tilde n+\hat n+\tilde n_c]\\
			&-\int v_{\mathrm{XC}}[\tilde n+\hat n+\tilde n_c](\tilde n+\hat n)\mathrm{d}\vec r\\
			E_{dc}^1=-\overline{E_H[n^1]}&+\overline{E_{\mathrm{XC}}[n^1+n_c]}-\int_{\Omega_r}v_{\mathrm{XC}}[n^1+n_c]n^1\mathrm{d}\vec r\\
			\tilde E_{dc}^1=&-\overline{E_H[\tilde n^1+\hat n]}+\overline{E_{\mathrm{XC}}[\tilde n^1+\hat n+\tilde n_c]}\\
			&-\int v_{\mathrm{XC}}[\tilde n^1+\hat n+\tilde n_c](\tilde n^1+\hat n)\mathrm{d}\vec r
		\end{aligned}
	\end{equation}
因此总能量的计算表达式是
	$$E=\sum_nf_n\langle\tilde\Psi_n|H|\tilde\Psi_n\rangle+\tilde E_{dc}+E_{dc}^1-\tilde E_{dc}^1+U(\vec R,Z_{\mathrm{ion}})$$
\textrm{USPP}计算中,$\tilde E_{dc}^1$和$E_{dc}^1$是常数,只需要在赝势生成时计算一次即可。

\subsection{PAW~中原子核附近的原子轨道基函数}
\textrm{PAW}方法除了采用平面波作为基函数,在每个原子核附近$r_c$范围内的缀加区,还保留了原子波函数和以平面波表示的赝波函数,它们构成\textrm{PAW}方法的基函数的一部分,%在\textrm{PAW}方法中,将与原子分波展开所有相关的数据统称原子数据集(data set),
也是\textrm{PAW}计算的基础。

\textrm{PAW}与原子分波展开所有相关的数据主要包括:~
	\begin{itemize}
		\item 原子分波信息:~原子分波函数$\phi_i$、赝分波函数$\tilde\phi_i$和投影波函数$p_i$
		\item 电荷密度信息:~$r_c$内的电荷密度$n^1$、赝电荷密度$\tilde n^1$和补充电荷$\hat n$
		\item 赝势信息:~原子局域赝势$\tilde v_{\mathrm{loc}}(\vec r)$
	\end{itemize}
\textrm{PAW}方法的原子分波与赝势方法相似,一套原子数据集将用于各种化学环境下的\textrm{PAW}计算,即要求原子数据集有良好的可移植性;~与赝势方法不同之处在于\textrm{PAW}原子集中除了赝原子的信息,还包含了真实原子的信息。

原子分波函数由原子\textrm{Schr\"odinger}方程确定
\begin{equation}
	\bigg(-\dfrac12\nabla^2+v_{at}-\varepsilon_i^1\bigg)|\phi_i\rangle=0
	\label{eq:PAW-Blochl-18}
\end{equation}
根据\textrm{Bl\"ochl}的建议,为确定分波函数$\phi_i$,一般通过适当选择$\varepsilon_i^1$(原子价电子能量附近),并要求在缀加区与芯层分波函数正交。实际应用中,还可以类似\textrm{LAPW}方法,对每个分波引入多个(一般是两个)分波函数。

对于赝原子分波,\textrm{Bl\"ochl}建议的赝化方案与传统赝势构造类似\upcite{PRL43-1494_1979,PRB26-4199_1982,PRB40-2089_1989}:
通过引入局域赝势:%$$w_i(r)=\tilde v_{at}(r)+c_ik(r)=\tilde v_{at}(0)\mathrm{e}^{-(r/r_k)^{\lambda}}+[1-k(r)]v_{at}(r)+c_i\mathrm{e}^{-(r/r_k)^{\lambda}}$$
\begin{equation}
	w_i(r)=\tilde v_{\mathrm{at}}(0)\mathrm{e}^{-(r/r_k)^{\lambda}}+[1-\mathrm{e}^{-(r/r_k)^{\lambda}}]v_{\mathrm{at}}(r)+c_i\mathrm{e}^{-(r/r_k)^{\lambda}}
	\label{eq:PAW-Blochl-19}
\end{equation}
其中$\tilde v_{\mathrm{at}}$是用多项式近似的原子赝势,参数$r_k$和$\lambda$由条件在$r_c$外赝势与原子势相等确定。赝分波函数由方程
\begin{equation}
	\bigg(-\dfrac12\nabla^2+w_i(r)-\varepsilon_i^1\bigg)|\tilde\phi_i\rangle=0
	\label{eq:PAW-Blochl-20}
\end{equation}
确定。式\eqref{eq:PAW-Blochl-19}的系数$c_i$根据赝分波函数$\tilde\phi_i$与分波函数$\phi_i$在$r_c$外相等确定。
\begin{figure}[h!]
\centering
\includegraphics[height=2.6in,width=3.2in,viewport=0 0 570 545,clip]{PAW-partical.png}
\caption{\small \textrm{The partical-wave of Fe atom.}}%(与文献\upcite{EPJB33-47_2003}图1对比)
\label{PAW_partical_Fe}
\end{figure}

按照\textrm{Bl\"ochl}建议的投影函数的构造方案,初始形式与超软赝势的辅助函数计算类似:
\begin{equation}
	|\tilde p_i\rangle=\bigg(-\dfrac12\nabla^2+\tilde v_{at}-\varepsilon_i^1\bigg)|\tilde\phi_i\rangle
	\label{eq:PAW-Blochl-21}
\end{equation}
考虑到投影函数与赝分波函数的正交性$\langle\tilde p_i|\tilde\phi_j\rangle=\delta_{ij}$,对于给定下标的投影函数$\tilde p_i$,要求与所有$j<i$的赝分波函数正交,因此\textrm{Bl\"ochl}采用\textrm{Gram-Schmidt}正交化方法,得到正交的投影函数
$$|\tilde p_i\rangle=|\tilde p_i\rangle-\sum_{j=1}^{i-1}|\tilde p_j\rangle\langle\tilde\phi_j|\tilde p_i\rangle$$
%	$$|\phi_i\rangle=|\phi_i\rangle-\sum_{j=1}^{i-1}|\phi_j\rangle\langle\tilde p_j|\tilde\phi_i\rangle$$
%	$$|\tilde\phi_i\rangle=|\tilde\phi_i\rangle-\sum_{j=1}^{i-1}|\tilde\phi_j\rangle\langle\tilde p_j|\tilde\phi_i\rangle$$
在此基础上,分波函数与赝分波函数也分别与投影函数正交
$$|\phi_i\rangle=|\phi_i\rangle-\sum\limits_{j=1}^{i-1}|\phi_j\rangle\langle\tilde p_j|\tilde\phi_i\rangle$$
$$|\tilde\phi_i\rangle=|\tilde\phi_i\rangle-\sum\limits_{j=1}^{i-1}|\tilde\phi_j\rangle\langle\tilde p_j|\tilde\phi_i\rangle$$
最终作为基函数的$\phi_i$、$\tilde\phi_i$、$\tilde p_i$是通过迭代得到的。
%\begin{figure}[h!]
%\centering
%\vspace*{-0.4in}
%\includegraphics[height=1.5in,width=2.3in,viewport=0 0 1100 745,clip]{PAW_projector-2.png}
%\caption{\small \textrm{The projector of PAW.}}%(与文献\upcite{EPJB33-47_2003}图1对比)
%\label{PAW_projector}
%\end{figure}
%\begin{itemize}
%	\item 与分波具有相同的角动量$l$
%	\item 局域在缀加区域(Augmentation region)
%	\item 节点依次增加
%\end{itemize}

与赝势的去屏蔽过程类似,%在得到赝势$\tilde v_{\mathrm{at}}$的基础上,可以计算
缀加区的局域势表示形式为:
\begin{equation}
	\bar v(r)=\tilde v_{at}(r)-\int\mathrm{d}r^{\prime}\dfrac{\tilde n(r^{\prime})+\hat n(r^{\prime})}{|r-r^{\prime}|}-\mu_{xc}[\tilde n(r)]
	\label{eq:PAW-Blochl-22}
\end{equation}
注意到在缀加区,式\eqref{eq:PAW-Blochl-22}由束缚态计算得到,计算电子的\textrm{Coulomb}相互作用、交换-相关势的贡献时,赝电荷密度$\tilde n(r)$包括赝芯电荷密度$\tilde n^c$和赝价电子密度。其中赝芯电荷的构造与赝势方法相同,价电子密度来自束缚态赝分波函数$|\tilde\Psi_j\rangle$,它由方程\upcite{PRB50-17953_1994}
%中的赝电荷密度$\tilde n(r)$由赝分波波函数$|\tilde\Psi_j\rangle$确定。对于束缚态,
\begin{equation}
	\bigg[-\dfrac12\nabla^2+\tilde v_{at}-\varepsilon+\sum_{(i,j)}|\tilde p_i\rangle\big(\mathrm{d}H_{ij}-\varepsilon\mathrm{d}O_{ij}\big)\langle\tilde p_j|\bigg]|\tilde\Psi_j\rangle=0
	\label{eq:PAW-Blochl-23}
\end{equation}
确定,式\eqref{eq:PAW-Blochl-23}中$\mathrm{d}H_{ij}$和$\mathrm{d}O_{ij}$分别是
$$\mathrm{d}H_{ij}=\langle\phi_i|-\dfrac12\nabla^2+v_{at}|\phi_j\rangle-\langle\tilde\phi_i|-\dfrac12\nabla^2+\tilde v_{at}|\tilde\phi_j\rangle$$
$$\mathrm{d}O_{ij}=\langle\phi_i|\phi_j\rangle-\langle\tilde\phi_i|\tilde\phi_j\rangle$$
式\eqref{eq:PAW-Blochl-23}的求解,参见文献\cite{PRB50-17953_1994}。

上述讨论不难看出,\textrm{PAW}方法的原子信息计算与赝势方法很相似,其中包括
\begin{itemize}
	\item 缀加局域区截断半径$r_c$
	\item 芯电荷密度$n^c$和赝芯电荷密度$\tilde n^c$
	\item (价)电子分波函数$|\phi_i\rangle$和赝分波函数$|\tilde\phi_i\rangle$
	\item 矩阵元$\langle\phi_i|-\dfrac12\nabla^2|\phi_j\rangle-\langle\tilde\phi_i|-\dfrac12\nabla^2|\tilde\phi_j\rangle$和$\langle\phi_i|\phi_j\rangle-\langle\tilde\phi_i|\tilde\phi_j\rangle$
\end{itemize}
%如果解$$|\tilde\Psi\rangle=|u\rangle+\sum_i|w_i\rangle c_i$$
%其中$|u\rangle$和$|w_i\rangle$的定义分别为
%$$\big(-\frac12\nabla^2+\tilde v-\varepsilon\big)|u\rangle=0$$
%$$\big(-\frac12\nabla^2+\tilde v-\varepsilon\big)|w_i\rangle=|\tilde p_i\rangle$$
%因此可以得到
%$$c_i=-\sum_{j,l}\bigg[\delta_{ij}+\sum_k\mathrm{d}H_{ik}-\varepsilon\mathrm{d}O_{ik}\langle\tilde p_k|w_j\rangle\bigg]^{-1}\big(\mathrm{d}H_{jl}-\varepsilon\mathrm{d}O_{jl}\big)\langle p_l|u\rangle$$

%\frame
%{
%%	\frametitle{\textrm{PAW}原子数据集}
%	\frametitle{\textrm{PAW Augmentation}}
%\begin{figure}[h!]
%\centering
%\includegraphics[height=2.3in,width=4.0in,viewport=0 0 1280 745,clip]{Figures/PAW-baseset.png}
%\caption{\small \textrm{The Augmentation of PAW.}}%(与文献\upcite{EPJB33-47_2003}图1对比)
%\label{PAW_baiseset}
%\end{figure}
%}

%\frame
%{
%	\frametitle{\textrm{PAW Augmentation}}
%\begin{figure}[h!]
%\centering
%\includegraphics[height=2.3in,width=4.0in,viewport=0 0 1100 745,clip]{Figures/PAW-projector.png}
%\caption{\small \textrm{The projector of PAW.}}%(与文献\upcite{EPJB33-47_2003}图1对比)
%\label{PAW_projector}
%\end{figure}
%}

\subsubsection{VASP~中的原子数据集}
\textrm{Kresse}方案中将与原子分波有关的数据称为原子数据集(\textrm{PAW Datasets}),这是\textrm{VASP}的主要计算文件\textrm{POTCAR}的数据构建方式,主要包括:~
\begin{enumerate}
	\item 全电子分波函数$\phi_i$和赝分波函数$\tilde\phi_i$
	\item 投影函数$\tilde p_i$
	\item 芯电荷密度$n_c$ 、局域离子赝势$v_{\mathrm H}[\tilde n_{Zc}]$(赝化离子实电荷密度$\tilde n_{Zc}$仅出现在$v_{\mathrm H}[\tilde n_{Zc}]$中,因此直接构造$v_{\mathrm H}[\tilde n_{Zc}]$)和赝芯电荷密度$\tilde n_c$
	\item 补偿电荷构造函数$g_l(r)$
\end{enumerate}

其赝原子分波函数和投影函数的构造,主要参考文献\cite{JPCM6-8245_1994}:~首先计算原子的全电子分波函数$\phi_i(\vec r)$,为构造形如
	\begin{equation}
		\tilde\phi_{i=Lk}(\vec r)=Y_L(\widehat{\vec r-\vec R}~)\tilde\phi_{lk}(|\vec r-\vec R|)
	\end{equation}
	的赝分波函数,应用\textrm{RRKJ}赝波函数方法\upcite{PRB41-1227_1990},径向赝分波函数由球\textrm{Bessel}函数线性组合
	\begin{equation}
		\tilde\phi_{lk}(r)=\left\{
		\begin{aligned}
			&\sum_{i=1}^2\alpha_ij_l(q_ir)\quad &r<r_c^l\\
			&\phi_{lk}(r)\quad&r>r_c^l
		\end{aligned}
		\right.
	\end{equation}
调节系数$\alpha_i$和$q_i$赝分波函数$\phi_{lk}(r)$在截断半径$r_c^l$处两阶连续可微。

投影波函数$\tilde p_i$由前面介绍的\textrm{Vanderbilt}超软赝势中投影子构造方法\upcite{PRB41-7892_1990}或\textrm{Bl\"ochl}方案的\textrm{Gram-Schmidt}正交化方法\upcite{PRB50-17953_1994}得到,这两种方法得到的\textrm{PAW}投影函数完全相同。
%$\langle\tilde p_i|\tilde\phi_j\rangle=\delta_{ij}$确定

\textrm{Kresse}方案中的局域离子赝势$v_{\mathrm H}[\tilde n_{Zc}]$,只要求其在缀加区外与真实离子势$v_{\mathrm H}[n_{Zc}]$相同,先构造局域原子赝势$\tilde v_{\mathrm{eff}}^a$,再去按有效离子势去屏蔽方式得到$v_{\mathrm H}[\tilde n_{Zc}]$。在截断半径$r_{loc}$内定义原子局域赝势$\tilde v_{eff}^a$为
$$\tilde v_{eff}^a=A\dfrac{\sin(q_{loc}r)}r\quad r<r_{loc}$$
这里$q_{loc}$和$A$的取值要求是使得局域赝势在截断半径$r_{loc}$处连续到一阶。

\textrm{Kresse}方案中赝芯电荷密度$\tilde n_c$的定义为:~在截断半径$r_{pc}$内,用两个\textrm{Bessel}函数$j_0$展开$\tilde n_c$
$$\sum_{i=1,2}B_i\dfrac{\sin(q_ir)}r\quad r<r_{pc}$$
类似地,要求截断半径$r_{pc}$外,$\tilde n_c$与全电子芯电荷密度$n_c$相同,系数$q_i$和$B_i$可使得赝芯电荷密度$\tilde n_c(r)$在$r_{pc}$处连续到两阶。

局域离子赝势$v_H[\tilde n_{Zc}]$可由原子局域赝势$\tilde v_{\mathrm{eff}}^a$去屏蔽得到
$$v_{\mathrm H}[\tilde n_{Zc}]=\tilde v_{\mathrm{eff}}^a-v_{\mathrm H}[\tilde n_a^1+\hat n_a]-v_{\mathrm{XC}}[\tilde n_a^1+\hat n_a+\tilde n_c]$$
%	在\textrm{VASP}的\textrm{POTCAR}生成过程中,
\textrm{Kresse}建议的各截断半径的参考条件:~$r_{\mathrm{rad}}=\max({r_c^l})$,$r_{pc}\approx r_{\mathrm{rad}}/1.2$,$r_{\mathrm{loc}}<r_{rad}/1.2$

最后,介绍\textrm{Kresse}方案中每个原子球内用两个球\textrm{Bessel}函数展开的补偿电荷构造函数$g_l(r)$
$$g_l(r)=\sum_{i=1}^2\alpha_i^lj_l(q_i^lr)$$
调节系数$q_i^l$和$\alpha_i^l$使得补偿电荷构造函数$g_l(r)$在截断半径$r_{\mathrm{comp}}$处的数值和前两阶导数值都是0,因此可以选择$q_i^l$使得多极矩满足:~
$$\int_0^{r_{\mathrm{comp}}}g_l(r)r^{l+2}\mathrm{d}r=1$$
实际上这一条件并不难实现,只要选择$q_i^l$满足约束条件
$$\dfrac{\mathrm{d}}{\mathrm{d}r}j_l(q_i^lr)\bigg|_{r_{\mathrm{comp}}}=0$$
并且要求$\alpha_i^l$可使得$g_l(r_{\mathrm{comp}})=0$,即可实现。在此,\textrm{Kresse}建议的截断半径取值参考条件是:~$r_{\mathrm{comp}}=r_{\mathrm{rad}}/1.3\sim r_{\mathrm{rad}}/1.2$,主要考虑尽可能让补偿电荷局域在缀加区,防止出现\textrm{Bl\"ochl}方案中芯层区重叠。

近年来,\textrm{Marsman}等突破了\textrm{Kresse}方案的“冻芯近似”,在自洽迭代计算中允许芯层电荷参与。迭代过程中要求确保价电子和芯电子的正交化,这其中的关键是(1)迭代过程中保持投影函数不变,(2)在每个自洽步中更新赝分波函数。具体可参阅文献\cite{JCP125-104101_2006}。

\subsubsection{投影函数的实空间表示}
%\textrm{PAW}方法是\textrm{Bl\"ochl}于1994年独立提出来的一种计算方法\upcite{PRB50-17953_1994},该方法同时结合了赝势方法和\textrm{APW}方法的优点,达到平衡计算效率和精度的目的。\textrm{PAW}方法刚提出来的时候并未引起注意,直到1999年\textrm{Kresse}讨论了\textrm{PAW}方法和超软赝势\textrm{(Ultra-Soft Pseudo-Potential, USPP)}方法的密切关联,指出\textrm{USPP}方法的计算程序简单改造就能引入\textrm{PAW}方法,才推动了\textrm{PAW}方法的广泛应用。\upcite{PRB59-1758_1999}现在\textrm{PAW}方法已经成为最主要的可支持第一原理分子动力学\textrm{(Ab initio Molecular Dynamics, AIMD)}的计算方法。
由\textrm{PAW}的基本定义式\eqref{eq:PAW-Blochl-02}不难看出,投影函数$\tilde p_i$是关联在位原子分波($\phi_i$和$\tilde\phi_i$)和倒空间表示的赝波函数$\tilde\Psi$的枢纽。因为原子分波局域在离子实附近,显然,如果投影函数可以在实空间表示,将有效地提高计算效率。回顾赝势理论,因为构造赝波函数时,截断半径$R_l$和能量参数$E_l$都是可调的,因此可以通过参数调节,优化赝波函数、投影函数和赝势。

当平面波截断$G_{\mathrm{max}}$确定条件下,投影函数在实空间的表示,实际上也就是投影函数的优化:~
\begin{itemize}
	\item 确定平面波截断参数$G_{\mathrm{max}}$
	\item 投影函数正空间表示与倒空间表示满足条件
		\begin{equation}
			\tilde p_l(q)=\int_0^{\infty}r^2\tilde p_l(r)j_l(qr)\mathrm{d}r
			\label{eq:projector_G_R}
		\end{equation}
		这里$j_l$是球\textrm{Bessel}函数。
	\item 调节$\tilde p_l(q)$,以最小化积分
		\begin{equation}
			I=\int_{G_{\mathrm{max}}}^{\infty}[q\tilde p_l(q)]^2\mathrm{d}q
			\label{eq:projector_Int}
		\end{equation}
		由此可确定$\tilde p_l(r)$在实空间优化的函数表示。
\end{itemize}
\textrm{VASP}软件中正是利用这一思想,得到优化的正空间投影函数,将物理量的计算尽可能地限制在实空间内完成,必要时再通过\textrm{FFT}变换到倒空间。这样既保留了计算的精度,又有效地降低了计算量。
%上述讨论中主要围绕\textrm{PAW}方法对于电子结构计算的方法,没有从第一原理分子动力学\textrm{(Ab initio Molecular Dynamics, AIMD)}角度讨论体系中原子和离子的受力运动,基于平面波基组的\textrm{AIMD}计算方法的一般概念和\textrm{PAW}方法中有关原子受力问题的讨论,可参阅文献\cite{PRB47-10142_1993,PRB50-17953_1994,PRB59-1758_1999,Comput_Phys}。

\begin{figure}[h!]
\centering
\includegraphics[height=4.5in,width=3.6in,viewport=0 0 480 630,clip]{VASP_procedure.png}
%\includegraphics[height=1.8in,width=4.in,viewport=30 210 570 440,clip]{PAW_projector.eps}
\caption{\small \textrm{The Flow of calculation for KS-ground state in VASP.}}%(与文献\cite{EPJB33-47_2003}图1对比)
\label{PAW_procedure}
\end{figure}

\subsection{VASP~中的优化算法}
\textrm{VASP}的主要任务是根据\textrm{DFT}理论迭代求解\textrm{Kohn-Sham}方程,获得电子的本征态波函数,进而获得电子基态能量。主要的计算流程如图\ref{PAW_procedure}所示。\textrm{VASP}中的迭代求解基态电子密度,本质上是采用数值优化的过程。根据计算对象的不同,主要是两类优化问题:
\begin{itemize}
	\item 能量泛函变分在基态密度时取极小:~$\mathrm{Min}\{E[\rho(\vec r)]\}$
	\item 迭代对角化求解\textrm{Kohn-Sham}方程本征值
\end{itemize}
表观上,这两类计算差别比较大,但是从数值计算的角度考虑,这两类优化本质上都可以归结为“不动点问题”\upcite{Numerical-Analysis},因此\textrm{VASP}中应用的计算算法类似,包括最陡下降\textrm{(Steepest Descent, ST)}、共轭梯度\textrm{(Conjugate Gradient, CG)}和\textrm{RMM-DIIS~(Residual Minimization Method-Direct Inversion in the Iterative Subspace)}等。其中最有特色的是\textrm{RMM-DIIS}方法。

方程迭代求解过程中,定义残量
	\begin{displaymath}
		(\mathbf{H}-\varepsilon^n\mathbf{S})|\psi^n\rangle=|R[\psi^n]\rangle
	\end{displaymath}
近似地,如果体系本征态的逼近量与残量满足线性关系,有
\begin{displaymath}
	|\delta\psi^{n+1}\rangle=\mathbf{K}|R[\psi^n]\rangle
\end{displaymath}
这里$\mathbf{K}$代表了函数优化的方向。可定义$\mathbf{K}$
\begin{equation}
	\mathbf{K}=\sum_{\vec q}\dfrac{|\vec q\rangle\langle\vec q|}{\langle\vec q|\mathbf{H}-\varepsilon\mathbf{S}|\vec q\rangle}
	\label{eq:Predict}
\end{equation}
在\textrm{VASP}中,$\mathbf{K}$的取值为:
\begin{displaymath}
	\mathbf{K}=-\sum_q\dfrac{2|\vec q\rangle\langle\vec q|}{E^{\mathrm{kin}}(R)}\times\dfrac{27+18x+12x^2+8x^3}{27+18x+12x^2+8x^3+16x^4}
\end{displaymath}
这里$x=\dfrac{\hbar^2}{2m_e}\dfrac{q^2}{\frac32E^{\mathrm{kin}}(R)}$,$E^{\mathrm{kin}}(R)$表示残矢动能。

\subsection{FFT~并行实现}
\textrm{VASP}计算中要处理大量的\textrm{FFT}变换。一般\textrm{FFT}计算网格数较多,为了提高计算效率,\textrm{VASP}实现了\textrm{FFT}网格计算的并行化。主要通过\textrm{VASP}的子程序\textbf{mgrid.F}实现。子程序\textbf{mgrid.F}建立了\textrm{FFT}网格和并行计算网格的映射。假设有8个计算节点,按照$2\times4$排列,则8个节点依次编号为$0~(0,1)/1~(0,2)/2~(0,3)\cdots7~(3,1)$,如果有四个平面波函数,则可以按下表分配:~
\begin{table}[h!]
%\tabcolsep 0pt \vspace*{-12pt}
\caption{The topology of the Wave and the nodes.}
\label{Table-Gpoint-Nodes}
\begin{minipage}{\textwidth}
%\begin{center}
\centering
\def\temptablewidth{0.54\textwidth}
\rule{\temptablewidth}{1pt}
\begin{tabular*} {\temptablewidth}{@{\extracolsep{\fill}}c@{\extracolsep{\fill}}c@{\extracolsep{\fill}}c}
%-------------------------------------------------------------------------------------------------------------------------
\textrm{Plane-Wave}  & \multicolumn{2}{c}{\textrm{multi-cores}}   \\ \hline
%-------------------------------------------------------------------------------------------------------------------------
\textrm{Wave~1} &0~(0,0) &1~(0,1)   \\% \cline{3-7}
\textrm{Wave~2} &2~(1,0) &3~(1,1)   \\% \cline{3-7}
\textrm{Wave~3} &4~(2,0) &5~(2,1)   \\% \cline{3-7}
\textrm{Wave~4} &6~(3,0) &7~(3,1)   \\% \cline{3-7}
%-------------------------------------------------------------------------------------------------------------------------
\end{tabular*}
\rule{\temptablewidth}{1pt}
\end{minipage}%{\textwidth}
\end{table}
通过建立这样的映射关系,可建立平面波与计算节点的对应关系,可以方便地实施\textrm{FFT}变换,提高计算效率。

综上所述,\textrm{VASP}通过对物理思想和方法、计算算法和编程过程的综合考虑,均衡计算精度和效率,成为第一原理计算高效能软件的典型代表。对\textrm{VASP~}代码的梳理和分析,也使得我们更深入地理解了\textrm{PAW~}方法的第一原理计算的方法和算法,为开发和拓展有关软件的功能提供了必要的支持。

\subsection{小结}
基于\textrm{VASP}软件,我们针对材料电子计算的基本问题如晶体结构空间群表示理论、电子基态总能计算和\textrm{Fermi~}能等展开一些讨论,重点分析了
\begin{itemize}
	\item 空间群与点群和平移/滑移对应关系,开发出适用于\textrm{VASP}软件的空间群分析模块;
	\item 针对电子基态总能计算中的奇点排除问题,讨论了在高阶奇点下的\textrm{Fermi~}能表示(可与分子、原子体系对应);~
	\item 通过对\textrm{VASP~}基本理论和程序实现的分析,指出了\textrm{VASP}软件能在第一原理计算软件中以优异的效能脱颖而出的原因。
\end{itemize}
上述工作,加深了我们对相关软件的理论、算法和程序代码的理解,加强了对复杂材料开展研究的能力。


近年来,\textrm{Marsman}等突破了\textrm{Kresse}方案的“冻芯近似”,在自洽迭代计算中允许芯层电荷参与。迭代过程中要求确保价电子和芯电子的正交化,这其中的关键是(1)迭代过程中保持投影函数不变,(2)在每个自洽步中更新赝分波函数。具体可参阅文献\cite{JCP125-104101_2006}。

上述讨论中主要围绕\textrm{PAW}方法对于电子结构计算的方法,没有从第一原理分子动力学\textrm{(Ab initio Molecular Dynamics, AIMD)}角度讨论体系中原子和离子的受力运动,基于平面波基组的\textrm{AIMD}计算方法的一般概念和\textrm{PAW}方法中有关原子受力问题的讨论,可参阅文献\cite{PRB47-10142_1993,PRB50-17953_1994,PRB59-1758_1999,Comput-Phys}。


\section{Introduction}
体系的对称性会对物理性质有决定性作用。在材料计算与模拟中,充分利用材料的对称性不仅可以有效提高计算效率,对各种尺度的物性分析也有很重要的帮助,因此一般的材料模拟软件中都包含对称性分析模块。材料电子结构的表示,与体系的对称性密切关联,在表示周期材料的电子能带时,路径的选择总是沿着$\vec k$空间中高对称性方向。这些高对称性$\vec k$点路径方向的选择主要依靠研究者的经验和习惯,并没有统一的规则,有着很大的随意性。\textrm{Setyawan}等\upcite{CMS49-299_2010}建议了,能带表示的标准化$\vec k$-\textrm{path}选择方案。我们在研究中注意到,材料电子计算领域著名软件\textrm{VASP}在对称性分析部分只提供了点群对称性分析,并未提供晶体空间群分析,其能带表示的$\vec k$-\textrm{path}也完全依赖人工选择,因此开发了\textbf{KPATH}软件,扩展了\textrm{VASP}软件的对称性分析功能并集成了能带表示标准化模块。

我们的软件主要结构分为四个功能模块,如图\ref{Call_graph:basic}所示:
\begin{itemize}
	\item 模型晶胞参数(\textrm{RD\_POSCAR\_HEAD})和晶胞中原子坐标(\textrm{RD\_POSCAR})的读入
	\item 输入晶胞的晶体学标准化和初基原胞的点群对称性分析与判断(\textrm{INISYM})
	\item 空间群对称性分析与判断(SGROUP)
	\item 标准化$\vec k$-\textrm{path}方案(\textrm{KPATH})
\end{itemize}
上述模块中,除了\textrm{SGROUP}是用\textbf{C}语言编写的,其余部分都是采用\textbf{FORTRAN90}编写的;\textrm{KPATH}是根据文献\cite{CMS49-299_2010}的思想实现的;点群对称性分析是将\textrm{VASP}软件的有关部分抽取出来独立而成的。
\begin{figure}[h!]
\centering
\includegraphics[height=2.35in]{CallsGraph-K_Points_Path_Basic.png}
\caption{\small The basic call graph of KPATH.}%(与文献\cite{EPJB33-47_2003}图1对比)
\label{Call_graph:basic}
\end{figure}

\section{晶胞参数和原子坐标的读入}
子程序\textbf{RD\_POSCAR\_HEAD}和\textbf{RD\_POSCAR}\\
\textbf{\textcolor{blue}{输入~}}\textrm{VASP}软件的晶体建模文件\textrm{POSCAR}\\
\textbf{\textcolor{red}{输出~}}晶胞体积、晶胞参数矩阵、倒格子参数矩阵,数组形式存储的原子坐标
%\begin{figure}[h!]
%\centering
%\includegraphics[height=0.95\textheight]{ControlFlowGraph-RD_POSCAR_HEAD.png}
%%\includegraphics[width=0.95\textwidth,viewport=0 0 400 475,clip]{ControlFlowGraph-RD_POSCAR_HEAD.png}
%\caption{\small The contralflow graph of RD\_POSCAR\_HEAD.}%(与文献\cite{EPJB33-47_2003}图1对比)
%\label{ContralFlow_graph:POSCAR-HEAD}
%\end{figure}
晶胞参数读入程序的流程关系见图\ref{ContralFlow_graph:POSCAR-HEAD},任务是读入\textrm{VASP}软件指定的晶体建模文件\textrm{POSCAR}格式的前7行,包括晶胞参数(以$\mathbf{A}$($\vec A_1,\vec A_2,\vec A_3$)三维矢量方式给定)和原子类型$\mathit{NTYP}$及对应原子数目$\mathit{NIONTP}$。通过三维晶胞参数矢量计算晶胞体积$\Omega$,倒格矢晶胞参数三维矢量$\mathbf{B}$,此外还统计了晶胞模型中全部原子数目$\mathit{NIOND}$。

原子坐标读入程序的流程关系见图\ref{ContralFlow_graph:POSCAR},任务是继续读入\textrm{POSCAR}文件的后续各行原子坐标,并根据原子坐标方式(用字符$\mathrm{d/D}$标志分数坐标,$\mathrm{C/K/c/k}$标志\textrm{Cartisian}坐标),分别计算实际原子坐标的数值,并将坐标统一写入数组$\mathit{POSITION}(1:3,1:\mathit{NIOND})$中。
%\begin{figure}[h!]
%\centering
%\includegraphics[height=0.95\textheight]{ControlFlowGraph-RD_POSCAR.png}
%%\includegraphics[width=0.95\textwidth,viewport=0 0 400 475,clip]{ControlFlowGraph-RD_POSCAR_HEAD.png}
%\caption{\small The contralflow graph of RD\_POSCAR.}%(与文献\cite{EPJB33-47_2003}图1对比)
%\label{ContralFlow_graph:POSCAR}
%\end{figure}
至此,我们完成了对晶体建模文件的完整读入和最初步的处理,以下进入程序的对称性分析具体实施阶段。

\section{晶胞参数的晶体学标准化和初基原胞的对称性分析}
该部分功能主要由模块\textbf{INISYM}完成
\subsection{晶胞参数的晶体学标准化}
%\begin{figure}[!ht]
%\centering
%\includegraphics[height=1.05\textheight]{ControlFlowGraph-LATTYP.png}
%%\includegraphics[width=0.95\textwidth,viewport=0 0 400 475,clip]{ControlFlowGraph-RD_POSCAR_HEAD.png}
%\caption{\small The contralflow graph of LATTYP.}%(与文献\cite{EPJB33-47_2003}图1对比)
%\label{ContralFlow_graph:LATTYP}
%\end{figure}

子程序\textbf{LATTYP}\\
\textbf{\textcolor{blue}{输入~}}初始晶胞矢量$\vec A_1$、$\vec A_2$、$\vec A_3$\\
\textbf{\textcolor{red}{输出~}}标准化晶胞的\textrm{Bravais}格子类型、晶胞参数$\mathit{CELDIM}(1:6)$和晶格矢量$\vec A_1$、$\vec A_2$、$\vec A_3$

\textcolor{magenta}{找到最小的}晶格矢量(\textrm{shortest lattice vector})确定\textcolor{red}{标准晶格的}基(\textrm{primitive basis})算法:~

\begin{itemize}
	\item 根据输入的晶胞矢量,确定对应的晶胞参数$a$,\,$b$,\,$c$,\,$\cos\alpha$,\,$\cos\beta$,\,$\cos\gamma$\\
		计算方案
		\begin{displaymath}
			\begin{aligned}
			a=&|\vec A_1|\\
			b=&|\vec A_2|/|\vec A_1|\cdot a\\
			c=&|\vec A_3|/|\vec A_1|\cdot a\\
			\cos\alpha=&\frac{|\vec A_2\cdot\vec A_3|}{|\vec A_2|\times|\vec A_3|}\\
			\cos\beta=&\frac{|\vec A_1\cdot\vec A_3|}{|\vec A_1|\times|\vec A_3|}\\
			\cos\gamma=&\frac{|\vec A_1\cdot\vec A_2|}{|\vec A_1|\times|\vec A_2|}
			\end{aligned}
			\label{eq:Cell_DM}
		\end{displaymath}
		初步判断晶体所属\textrm{Bravais~}格子类型($\mathit{IBRAV}$),共14类,判据如下(以简单立方、体心立方为例),并将标准晶胞参数存入$\mathit{CELLDIM}(1:6)$:~
		\begin{enumerate}
			\item 简单立方\textrm{cell}\\
				如果 $|\vec A_1|=|\vec A_2|=|\vec A_3|$且$\cos\alpha=\cos\beta=\cos\gamma=0.0$则\\
				$\mathit{IBRAV}=1$\\
				$\mathit{CELLDIM}(1)=|\vec A_1|$
			\item 体心立方\textrm{cell}\\
				如果 $|\vec A_1|=|\vec A_2|=|\vec A_3|$且$\cos\alpha=-\frac13$则\\
				$\mathit{IBRAV}=2$\\
				$\mathit{CELLDIM}(1)=|\vec A_1|\cdot\frac2{\sqrt3}$
			\item 面心立方\textrm{cell}
			\item 六方\textrm{cell}
			\item 简单四方\textrm{cell}
			\item 体心四方\textrm{cell}
			\item 三方\textrm{cell}
			\item 简单正交\textrm{cell}
			\item 体心正交\textrm{cell}
			\item 面心正交\textrm{cell}
			\item 底心正交\textrm{cell}
			\item 简单单斜\textrm{cell}
			\item 底心单斜\textrm{cell}
			\item 三斜\textrm{cell}
		特别地,针对面心立方,如果将(111)面选为基准面,则$\mathit{ITYP}=15$
		\end{enumerate}
	\item 检查“病态”晶胞,当晶胞差别比较大时,如$c/a$,$b/a$特别大或特别小,以及晶胞夹角接近$0^{\circ}$或$180^{\circ}$
	\item \textcolor{red}{搜索原始晶胞中的最小晶格矢量}:~依次针对矢量$\vec A_1$,$\vec A_2$,$\vec A_3$,用迭代方式检查
		\begin{enumerate}
			\item 用\textcolor{green}{六组循环}分别检查矢量
				\begin{displaymath}
					\begin{aligned}
						\vec A_1&=\vec A_1-\mathit{ICOUNT}\cdot\vec A_2\\
						\vec A_1&=\vec A_1-\mathit{ICOUNT}\cdot\vec A_3\\
						\vec A_2&=\vec A_2-\mathit{ICOUNT}\cdot\vec A_1\\
						\vec A_2&=\vec A_2-\mathit{ICOUNT}\cdot\vec A_3\\
						\vec A_3&=\vec A_3-\mathit{ICOUNT}\cdot\vec A_1\\
						\vec A_3&=\vec A_3-\mathit{ICOUNT}\cdot\vec A_2
					\end{aligned}
				\end{displaymath}
				并检查可能的
				\begin{displaymath}
					\begin{aligned}
						\vec A_1&=\vec A_1+\mathit{ICOUNT}\star\vec A_2\\
						\vec A_1&=\vec A_1+\mathit{ICOUNT}\star\vec A_3\\
						\vec A_2&=\vec A_2+\mathit{ICOUNT}\star\vec A_1\\
						\vec A_2&=\vec A_2+\mathit{ICOUNT}\star\vec A_3\\
						\vec A_3&=\vec A_3+\mathit{ICOUNT}\star\vec A_1\\
						\vec A_3&=\vec A_3+\mathit{ICOUNT}\star\vec A_2
					\end{aligned}
				\end{displaymath}
				直到找到各个方向最小的矢量$\vec A_1^{\mathrm{min}}$,$\vec A_2^{\mathrm{min}}$,$\vec A_3^{\mathrm{min}}$
			\item 检查标准晶胞是否存在“病态”晶胞($a$、$b$、$c$存在特别大或特别小的值,导致$\alpha$、$\beta$、$\gamma$接近$0^{\circ}$或$180^{\circ}$)
			\item 将找到的矢量$\vec A_1$、$\vec A_2$、$\vec A_3$线性组合(按特定标准组合)得到标准晶格的矢量,确保标准晶胞与初始晶胞的体积不变,算法如下:\\
				为确定标准胞
				\begin{displaymath}
					\begin{pmatrix}
						X_1\\
						X_2\\
						X_3
					\end{pmatrix}=
\begin{pmatrix}
						N_1 N_2 N_3\\
						N_4 N_5 N_6\\
						N_7 N_8 N_9
\end{pmatrix}
\begin{pmatrix}
	\vec A_1^{\mathrm{min}} \\
	\vec A_2^{\mathrm{min}} \\
	\vec A_3^{\mathrm{min}}
\end{pmatrix}
				\end{displaymath}
				要求对变换矩阵(矩阵元可取整数$N_i=-2,-1,0,1,2$),其行列式满足
				\begin{displaymath}
					\begin{vmatrix}
						N_1 N_2 N_3\\
						N_4 N_5 N_6\\
						N_7 N_8 N_9
					\end{vmatrix}=\mathbf{1}
				\end{displaymath}
		\end{enumerate}
			\item 根据得到的标准晶胞参数,判断标准晶格所属\textrm{Bravais~}格子类型($\mathit{ITYP}$),将标准晶胞参数存入$\mathit{CELLDIM}(1:6)$(算法与初始晶格判断算法相同)\\
				数组$\vec A_1$、$\vec A_2$、$\vec A_3$保存标准化晶格矢量\\
				\textcolor{red}{注意}:~对于简单单斜晶系,要求$\cos\gamma<0$,并指定特定轴方向为$b$~轴,有$|\vec A_1|<|\vec A_3|$;对于这种情况,程序会自动调整矢量的顺序
			\item 对比初始晶格类型$\mathit{IBRAV}$和标准晶格类型$\mathit{ITYP}$,如果两者不一致,给出警告信息
			\item 输出晶格所属\textrm{Bravais~}格子类型和对应的晶胞参数
%	\item 检查标准晶格矢量构造的晶格的特征,所属\textrm{Bravais~}格子和对应的晶胞参数$a$,\,$b$,\,$c$,\,$\alpha$,\,$\beta$,\,$\gamma$
\end{itemize}

\subsection{确定初基原胞}
子程序\textbf{PRICEL}\\
\textbf{\textcolor{blue}{输入~}}初始晶胞矢量$A^0_1$、$A^0_2$、$A^0_3$、\textrm{Bravais}格子类型$\mathit{IBRAV}$、标准化晶胞参数$\mathit{CELDIM}(6)$和初始晶胞中全部原子数、每类原子数及全部原子位置$\mathit{TAU}(N,3)$\\
\textbf{\textcolor{red}{输出~}}初基原胞矢量$\vec P_1$、$\vec P_2$、$\vec P_3$、初基原胞的\textrm{Bravais}类型$\mathit{IPTYP}$、初基原胞参数$\mathit{PDIM}(6)$、初始晶胞包含初基原胞数$\mathit{NCELL}$
\begin{itemize}
	\item 在初始晶胞中,将坐标原点置于原始晶胞中心,即将原子全部坐标$\mathit{TAU}$变换到$[-0.5,0.5)$区间内),算法如下:
			\begin{displaymath}
				\begin{aligned}
					&TAU(I,i)=TAU(I,i)-NINT(TAU(I,i)) \\
					&TAU(I,i)=\mathbf{MOD}(TAU(I,i)+100.5\_q)-0.5\_q \\
					&TAU(I,i)=\mathbf{MOD}(TAU(I,i)+100.5\_q)-0.5\_q \\
					&IF (\mathbf{ABS}(TAU(I,i)-0.5\_q)<T_{inty})\quad TAU(I,i)=-0.5\_q \\
					&IF (\mathbf{ABS}(TAU(I,i)+0.5\_q)<T_{inty})\quad TAU(I,i)=-0.5\_q
				\end{aligned}
			\end{displaymath}
	\item 将得到的每一类原子按坐标升序排列,存于$\mathit{TAU}$数组。采用堆排序\textrm{(heapsort algorithm)}算法,排升序原则:~先按$x$坐标排序,再对$y$坐标排序,最后按$z$坐标排序
	\item 确定晶胞中\textcolor{magenta}{原子数最少的原子类型}$\mathit{IMINST}$和\textcolor{magenta}{原子数目},并记录其\textcolor{magenta}{第一个原子}的坐标$\mathit{TAUSAV}$
	\item 确定初始晶胞中的许可平移矢量:~
		\begin{enumerate}
			\item 选定原子数最少的一类原子,以其第一个原子坐标$\mathit{TAUSAV}$为参考,\textcolor{red}{分别确定其他各原子$\mathit{TAU}(IMINST,3)$与第一个原子的“平移矢量”关系}并存于数组$\mathit{TRA}$(这里只需要考虑同一类原子坐标平移)
			\item 确定每一类原子的每个原子坐标$\mathit{TAU}$,经矢量$\mathit{TRA}$平移作用后的原子坐标$\mathit{TAU}+\mathit{TRA}$,置于$\mathit{TAUROT}$数组
			\item 再次将原子坐标$\mathit{TAUROT}$变换到$[-0.5,0.5)$区间内,并按升序排列
			\item 如果$\mathit{TAUROT}$数组的坐标与$\mathit{TAU}$数组中坐标重合,则由此确定一个许可平移,平移矢量存入$\mathit{PTRANS}(N_I,3)$。
		\end{enumerate}
	\item 将许可平移矢量$\mathit{PTRANS}$也按升序排列,\textcolor{red}{确定所有平移矢量中最小的三个不共面的矢量为初基原胞矢量}$\vec P_1$、$P_2$、$P_3$
	\item 按子程序\textbf{LATTYP}确定初基原胞的\textrm{Bravais}格子类型$\mathit{IPTYP}$和原胞参数$\mathit{PDIM}$
        \item 根据初始晶胞矢量和初基原胞矢量分别计算的体积,判定复晶胞的数目$\mathit{NCELL}$
\begin{displaymath}
	\mathrm{NCELL}=\dfrac{\Omega_{latt}}{\Omega_{prim}}
\end{displaymath}
	\item 检查初始晶胞中包含的初基原胞数目,算法如下:~
		\begin{enumerate}
			\item 计算初始晶胞矢量在每个方向上所含初基原胞数目:
		\begin{displaymath}
			\dfrac1{\vec p_i\cdot\vec B_j}=\dfrac1{\vec p_i\cdot\vec A_j^{-1}}=\dfrac{A_j}{\vec p_i}=\mathrm{N_{i}}
		\end{displaymath}
	\item 检查全部通过平移矢量$\mathit{PTRANS}$作用后仍在初始晶胞内的初基原胞,并统计许可平移数目$\mathit{ICOUNT}$
		\end{enumerate}
	\item 要求$\mathit{ICOUNT}=\mathit{NCELL}$
\end{itemize}

\subsection{原子坐标表示的标准化}
将初始晶胞的原子坐标,由初始晶胞矢量构成的坐标系变换到标准晶胞矢量构成的坐标系下。算法如下:~
\begin{itemize}
	\item 标准晶胞矢量构成变换矩阵$\mathbf B$,求逆阵$\mathbf{B}^{-1}$
	\item 记每个原子在原始晶胞中的坐标为矢量$\vec C$,计算原子实际的位置矢量$\vec R=\mathbf{A}\vec C$
	\item 标准晶胞下原子坐标矢量为$\mathbf{B}^{-1}\vec R$
\end{itemize}

\subsection{确定体系的点群对称操作}
体系的点群对称性(含平移操作)分析主要通过两层接口子程序\textbf{SETGRP}和\textbf{GETGRP}完成。

\textbf{SETGRP}主要是根据初始晶胞所属\textrm{Bravais}格子类型$\mathit{IBRAV}$,生成全部点群对称操作矩阵,该点群操作矩阵标志的是原子放置位置,与所选坐标系无关,因此是整数表示的。对14种\textrm{Bravais}格子,列举了全部各类格子的生成元素。
如立方晶系的3种初始生成元素分别为
\begin{itemize}
	\item 简单立方
\begin{displaymath}
	\mathbf{INV}=
	\begin{pmatrix}
		-1, 0, 0 \\ 
		0,-1, 0 \\
		0, 0, -1
	\end{pmatrix}\quad
	\mathbf{R3D}=
	\begin{pmatrix}
		0, 0, 1 \\ 
		1, 0, 0 \\
		0, 1, 0
	\end{pmatrix}\quad
	\mathbf{R4ZP}=
	\begin{pmatrix}
		0, -1, 0 \\ 
		1, 0, 0 \\
		0, 0, 1
	\end{pmatrix}
\end{displaymath}
	\item 体心立方
\begin{displaymath}
	\mathbf{INV}=
	\begin{pmatrix}
		-1, 0, 0 \\ 
		0,-1, 0 \\
		0, 0, -1
	\end{pmatrix}\quad
	\mathbf{R3D}=
	\begin{pmatrix}
		0, 0, 1 \\ 
		1, 0, 0 \\
		0, 1, 0
	\end{pmatrix}\quad
	\mathbf{R4ZBC}=
	\begin{pmatrix}
		0, 1, 0 \\ 
		0, 1, -1 \\
		-1, 1, 0
	\end{pmatrix}
\end{displaymath}
	\item 面心立方
\begin{displaymath}
	\mathbf{INV}=
	\begin{pmatrix}
		-1, 0, 0 \\ 
		0,-1, 0 \\
		0, 0, -1
	\end{pmatrix}\quad
	\mathbf{R3D}=
	\begin{pmatrix}
		0, 0, 1 \\ 
		1, 0, 0 \\
		0, 1, 0
	\end{pmatrix}\quad
	\mathbf{R4ZFC}=
	\begin{pmatrix}
		1, 1, 1 \\ 
		0, 0, -1 \\
		-1, 0, 0
	\end{pmatrix}
\end{displaymath}
\end{itemize}
根据群论理论,每一类点群,由生成元素矩阵,可以得到体系的全部许可的点群对称操作矩阵。

子程序\textbf{GETGRP}由\textbf{SETGRP}完成点群对称操作生成后调用,主要是检查根据当前初始晶胞原子位置,哪些对称操作可以保留,哪些必须丢弃,从而确定初始晶胞实际最后最后许可的点群。
\subsubsection{确定初始晶胞的实际许可的对称操作矩阵}
子程序\textbf{CHKSYM~}是实际许可对称操作检查的核心部分,基本思想非常清晰
\begin{itemize}
	\item 对初始原胞中的每个原子坐标,依次用找到的点群元素依次作用后变换坐标,将变换后的原胞原子位置与初始的原子位置对比,存在\textbf{三种可能性}:
\begin{enumerate}
	\item 所有的原子位置可重合(纯粹的点群操作)
	\item 除了点群对称操作,须外加\textbf{滑移}对称性检查(空间群操作)
	\item 原子位置无法重合(不允许的对称操作)
\end{enumerate}
	\item 关于\textcolor{red}{滑移}对称性的判断,说明如下
		\begin{enumerate}
			\item 针对初始晶胞中的第一个原子,对比全部点群元素操作前后原子坐标差,将构成一组矢量。这组矢量中其中必定包含许可的滑移操作(如果存在滑移的话)。并且该滑移矢量适合全部原子。
			\item 对点群元素作用后的原子坐标,扣除滑移矢量部分的贡献,将会回到初始晶胞的原子状态。
			\item 原子坐标位置对比,与子程序\textbf{PRICEL}中的算法类似:~
				\begin{description}
					\item[-] 将原子坐标根据堆排序\textrm{(heapsort algorithm)}算法排序,分别得到两组原子位置序列
					\item[-] 依次分别对比每个序列中的原子位置
					\item[-] 只有当两个序列原子位置完全一致,才是许可的滑移(原子位置的数值对比,精度对结果的影响很大)
				\end{description}
特别需要注意:~初始原胞很可能是\textrm{super cell~},因此是非\textrm{primitive~}的。因此会存在\textrm{non-primitive primitive~}滑移(\textrm{trivial translations of generating cell~}),为了避免对滑移定义的不唯一性,规定取所有许可滑移矢量中取模量最小的作为滑移矢量。
		\end{enumerate}
\end{itemize}
具体程序执行流程
\begin{itemize}
	\item 对初始晶胞中每个原子坐标$\mathit{TAU}$,通过将坐标原点置于晶胞中心位置,将原子坐标变换到$[-0.5,0,5)$,将标准化后的原子坐标$\mathit{TAU}$按升序排列
	\item 每个点群操作元素依次作用于原子坐标,得到$\mathit{TAUROT}$,它也将变换到$[-0.5,0,5)$,并按升序排列
	\item 找到原子数最少类型的原子,记其对称操作后的坐标为$\mathit{TAUSAV}$
		\begin{enumerate}
			\item 依次计算全部$\mathit{TAU}(i,I)-\mathit{TAUSAV(i)}$,并记作$\mathit{GTRANS}$(测试滑移矢量)
			\item 如果该矢量是整个晶格的平移(\textrm{trivial translation}),则排除
		\end{enumerate}
	\item 对原胞中的每一类原子坐标$TAUROT$,依次用找到的尝试矢量$\mathit{GTRANS}$作用后,变换到$[-0.5,0.5)$,再按升序排列
	\item 依次对比每一类原子的每个坐标,只有当两者完全重合才确认找到许可的滑移矢量,并存入数组$\mathit{TRA}$
	\item 对比完毕后,要将原子坐标$TAUROT$中的尝试矢量$\mathit{GTRANS}$扣除,为后一个对称操作准备
	\item 最后只保留每一组点群元素许可的最小滑移操作,存入$\mathit{GTRANS}$,并统计对称操作数目$\mathit{NROT}$
\end{itemize}
最后有必要指出的是:~该流程必须分别对每一类的每个原子都执行,结果(许可的对称及nontrivial平移)必须对全部原子都适用,否则不能构成有效的对称操作
\subsubsection{点群矩阵的最后确定}
根据子程序\textbf{CHKSYM}返回的许可对称操作和滑移操作,将最终确定的对称操作矩阵矩阵和滑移操作矢量,分别存入$\mathit{S}(3,3,I)$和$\mathit{GTRANS(3,I)}$,此外全部矩阵元均清零。($\mathit{S}$矩阵和$\mathit{GTRANS}$的I上限是48)
\subsection{确定体系的点群名}
子程序\textbf{PGROUP}根据实际确定的点群对称元素,确定初始晶胞所属点群。
\begin{itemize}
	\item 对于对称操作元素数目唯一确定的(如$\mathit{NROT}=1/3/16/48$),分别快速确定点群$\mathbf{C}_1$、$\mathbf{C}_3$、$\mathbf{D}_{4h}$、$\mathbf{O}_h$
	\item 对于剩余的对称操作数和对称操作元,\textcolor{purple}{可能构成的不可约子群元素}分别为$\mathbf{E}$、$\mathbf{I}$、$\mathbf{C}_2$、$\mathbf{C}_3$、$\mathbf{C}_4$、$\mathbf{C}_6$、$\mathbf{S}_2$=$\mathbf{m}$、$\mathbf{S}_6$、$\mathbf{S}_4$、$\mathbf{S}_3$,采用枚举的方式
		\begin{enumerate}
			\item 分别计算对称元素矩阵全部的迹(\textrm{Trace})和行列式值(\textrm{Determinant}),根据迹和行列式值统计相应的不可约子群元素数目
			\item 结合对称元素数目分别给出所属点群名
		\end{enumerate}
\end{itemize}

在此基础上,通过矩阵变换,完成对称操作元素(含滑移操作)在初始晶胞中的表示;
\subsection{点群对称操作前后原子位置的关联}
子程序\textbf{POSMAP}\\
\begin{itemize}
	\item 对初始晶胞中每个原子坐标重新存入数组$\mathit{TAU}$,通过将坐标原点置于晶胞中心位置,将原子坐标变换到$[-0.5,0,5)$,得到标准化后的原子坐标$\mathit{TAU}$
		\item 点群(含滑移)操作依次作用于$\mathit{TR}$后,同样变换到$[-0.5,0.5)$
		\item 依次计算每一类原子的变换前后的坐标关系,满足$|\mathit{TR}(i,I)-\mathit{TAU}(i,J)|<\mathit{TINY}\;(i=1,2,3)$,则建立两者的原子序号、对称操作(含平移)关系,存入$\mathit{ROTMAP}$
\end{itemize}

\section{确定体系的空间群}
空间群判断子程序\textbf{SGROUP}是在点群判断的基础上,根据体系所属的点群对称性结合许可的平移操作数目检查,确定所属的空间群。这部分代码是用\textbf{C}语言编写的。
\subsection{确定体系的空间群}
子程序\textbf{SGROUP}主要通过调用核心代码子程序\textbf{find\_space\_group}完成。这部分代码的算法流程如下:
\begin{itemize}
	\item 采用枚举方式,给出判断的点群所关联的空间群所对应的平移矢量:~
		以$\mathbf{O}_h$点群为例:其对应的十个空间群为
		\begin{enumerate}
			\item 221(P\;m\;-3\;m)
			\item 222(P\;n\;-3\;n) \;[origin choice 2]
			\item 223(P\;m\;-3\;n)
			\item 224(P\;n\;-3\;m) \;[origin choice 2]
			\item 225(F\;m\;-3\;m)
			\item 226(F\;m\;-3\;c)
			\item 227(F\;d\;-3\;m) \;[origin choice 2]
			\item 228(F\;d\;-3\;c) \;[origin choice 2]
			\item 229(I\;m\;-3\;m)
			\item 230(I\;a\;-3\;m)\\
				每个空间群都是由不同的平移操作与48个点群操作相关联。其中第230号空间群,与48个点群操作关联的平移操作为
				\begin{displaymath}
				\begin{matrix}
     0.0000, & 0.0000, & 0.0000, /*    1  */ \quad 0.5000, & 0.0000, & 0.5000, /*    2  */ \quad 0.0000, & 0.5000, & 0.5000, /*    3  */\\
     0.5000, & 0.5000, & 0.0000, /*    4  */ \quad 0.0000, & 0.0000, & 0.0000, /*    5  */ \quad 0.5000, & 0.5000, & 0.0000, /*    6  */ \\
     0.5000, & 0.0000, & 0.5000, /*    7  */ \quad 0.0000, & 0.5000, & 0.5000, /*    8  */ \quad 0.0000, & 0.0000, & 0.0000, /*    9  */ \\
     0.0000, & 0.5000, & 0.5000, /*    10  */\quad 0.5000, & 0.5000, & 0.0000, /*    11  */\quad 0.5000, & 0.0000, & 0.5000, /*    12  */\\
     0.7500, & 0.2500, & 0.2500, /*    13  */\quad 0.7500, & 0.7500, & 0.7500, /*    14  */\quad 0.2500, & 0.2500, & 0.7500, /*    15  */\\
     0.2500, & 0.7500, & 0.2500, /*    16  */\quad 0.7500, & 0.2500, & 0.2500, /*    17  */\quad 0.2500, & 0.7500, & 0.2500, /*    18  */\\
     0.7500, & 0.7500, & 0.7500, /*    19  */\quad 0.2500, & 0.2500, & 0.7500, /*    20  */\quad 0.7500, & 0.2500, & 0.2500, /*    21  */\\
     0.2500, & 0.2500, & 0.7500, /*    22  */\quad 0.2500, & 0.7500, & 0.2500, /*    23  */\quad 0.7500, & 0.7500, & 0.7500, /*    24  */\\
     0.0000, & 0.0000, & 0.0000, /*    25  */\quad 0.5000, & 0.0000, & 0.5000, /*    26  */\quad 0.0000, & 0.5000, & 0.5000, /*    27  */\\
     0.5000, & 0.5000, & 0.0000, /*    28  */\quad 0.0000, & 0.0000, & 0.0000, /*    29  */\quad 0.5000, & 0.5000, & 0.0000, /*    30  */\\
     0.5000, & 0.0000, & 0.5000, /*    31  */\quad 0.0000, & 0.5000, & 0.5000, /*    32  */\quad 0.0000, & 0.0000, & 0.0000, /*    33  */\\
     0.0000, & 0.5000, & 0.5000, /*    34  */\quad 0.5000, & 0.5000, & 0.0000, /*    35  */\quad 0.5000, & 0.0000, & 0.5000, /*    36  */\\
     0.2500, & 0.7500, & 0.7500, /*    37  */\quad 0.2500, & 0.2500, & 0.2500, /*    38  */\quad 0.7500, & 0.7500, & 0.2500, /*    39  */\\
     0.7500, & 0.2500, & 0.7500, /*    40  */\quad 0.2500, & 0.7500, & 0.7500, /*    41  */\quad 0.7500, & 0.2500, & 0.7500, /*    42  */\\
     0.2500, & 0.2500, & 0.2500, /*    43  */\quad 0.7500, & 0.7500, & 0.2500, /*    44  */\quad 0.2500, & 0.7500, & 0.7500, /*    45  */\\
     0.7500, & 0.7500, & 0.2500, /*    46  */\quad 0.7500, & 0.2500, & 0.7500, /*    47  */\quad 0.2500, & 0.2500, & 0.2500  /*    48  */\\
					\end{matrix}
				\end{displaymath}
		\end{enumerate}
	\item 将每个点群的矩阵表示元素约化到初基原胞中,并得到全部点群操作。
		\begin{description}
			\item[-] 枚举所属点群的生成元素,并得到矩阵$(\mathbf{g}-\mathbf{E})$(对于$x$、$y$、$z$方向不固定的情形,枚举特别的处理)
			\item[-] 引入初始的平移量$\mathbf{R}_{sh}$,对于\textcolor{blue}{三方和六方晶系},$\mathbf{R}_{sh}$的三个分量取值可是$0,1/2,1/3,2/3$任意组合;对其余晶系$\mathbf{R}_{sh}$的三个分量可是$0,1/4,1/2,3/4$任意组合
			\item[-]采用枚举方式检查可能的初始平移量中$\mathbf{R}_{sh}$中实际允许的平移矢量\\
				由等式
				\begin{displaymath}
					(\mathbf{g}-\mathbf{E})\mathbf{r}=\mathbf{R}_{op}
				\end{displaymath}
				确定变换矩阵$(\mathbf{g}-\mathbf{E})^{-1}$和全部可能的平移矢量$\mathbf{r}$。这里$\mathbf{g}$是点群生成元素,$\mathbf{R}_{op}$是\textcolor{magenta}{由平移量$\mathbf{r}$导致的点群对称元素的完全平移矢量}(因此只有经约化的平移量$\mathit{rop}<\mathit{TOL}$才有可能是合理的合理的),由此确定\textcolor{red}{晶胞许可的}$\mathbf{r}$(即$\mathbf{R}_{sh}$)
			\item[-] \textcolor{blue}{遍历点群的对称元素,依次用晶胞许可的坐标系表示},由此确定\textcolor{red}{晶胞所在的坐标系下的实际完全平移量}$\mathbf{r}^{\prime}$,并点群元素和许可的平移量按空间群中的顺序排列
			\item[-] \textcolor{blue}{遍历点群可关联的空间群许可的平移量$\mathbf{r}$},由等式
		\begin{displaymath}
			\mathbf{r}^{\prime}=\mathbf{r}+(\mathbf{g}-\mathbf{E})\mathbf{R}_{sh}
		\end{displaymath}
		依次计算并确定\textcolor{blue}{由当前坐标系表示引起的平移量为}
		\begin{displaymath}
			\mathbf{R}_{sh}^{\mathrm{coor}}=(\mathbf{g}-\mathbf{E})^{-1}(\mathbf{r}-\mathbf{r}^{\prime})
		\end{displaymath}
		\end{description}
		校正坐标系贡献后,点群对称操作的平移矢量$\mathbf{R}^{\prime}$为
		\begin{displaymath}
			\mathbf{R}^{\prime}=\mathbf{r}^{\prime}+\mathbf{R}_{sh}^{\mathrm{coor}}=\mathbf{r}^{\prime}+(\mathbf{g}-\mathbf{E})^{-1}(\mathbf{r}-\mathbf{r}^{\prime})
		\end{displaymath}

	\item 符合空间群的平移(滑移)操作的检验
		\begin{description}
			\item[-] 变换成点群对称操作的平移矢量$\mathbf{R}^{\prime}$,经约化最后得到$\mathbf{r}_{op}^{\prime}$
			\item[-] 枚举点群可对应的空间群许可平移矢量$\mathbf{r}$,如与点群对称元素得到平移矢量$\mathbf{r}_{op}^{\prime}$(已经按照空间群中有关顺序排列)与一致,则确定为平移为空间群许可的平移量。
		\end{description}
	\item 根据当前确定的平移矢量,统计(要遍历全部点群元素$\mathit{nop}$和许可平移量$\mathit{nsh}$)
		\begin{enumerate}
			\item 检查初基原胞的许可平移$Rshft[1:3]$(上限为8组)
			\item 总的许可的平移矢量数$\mathit{nshft}$
		\end{enumerate}
	\item 遍历点群关联的全部空间群,根据确定的平移矢量数目$\mathit{nshft}$,记录对应的空间群与点群关联顺序$\mathit{isgrp}$,得到空间群名(空间群记号)$\mathit{sgrp\_name}$
	\item 将全部空间群对称元素($\mathit{sym\_op}[1:4][1:3]$,含对称元素的平移操作)的表示约化到初基原胞中
\end{itemize}
\textcolor{red}{注意:~}由于有些\textrm{Bravais}格子因为坐标系选择,对称操作矩阵表示会有差别($\mathit{Nb}>0$),程序在考虑这个问题的时候是通过引入转换矩阵$\mathbf{A}=\mathbf{Rot}$来实现不同坐标系下的表示和转换关系的。
\subsection{输出空间群}
子程序\textbf{SGROUP}枚举了230个空间群的全部记号,根据空间群名$\mathit{sgrp\_name}$,即可以得到标准晶胞的空间群记号,最后由子程序\textbf{WRITE\_RES}输出空间群记号、全部点群和空间群的对称操作元素矩阵表示。

\section{确定体系的标准化$\vec k$点路径}
根据体系所属的\textrm{Bravais~}格子,枚举全部标准化$\vec k$-\textrm{path},结果写到\textrm{KPOINTS\_BAND}文件中。以立方晶系为例,对应的标准化$\vec k$-\textrm{path}设置为:~
\begin{itemize}
	\item 简单立方\quad $\Gamma$–$X$–$M$–$\Gamma$–$R$–$X$|$M$–$R$
	\item 面心立方\quad $\Gamma$–$X$–$W$–$K$–$\Gamma$–$L$–$U$–$W$–$L$–$K$|$U$–$X$
	\item 体心立方\quad $\Gamma$–$H$–$N$–$\Gamma$–$P$-$H$|$P$-$N$
\end{itemize}


%Introduction
\section{周期体系计算中的能量零点的移动与\textrm{Fermi}能}
\subsection{问题的提出}
在电子结构计算中,对分子、原子等有限尺度体系,习惯上将能量零点取在无穷远,即无穷远处的静止电子的能量为零。这样选择的能量参考点,束缚态的电子能量都是负值,并且基态最高占据态的电子能级与第一电离能的负值对应。但是对于理想的周期体系来说,“无穷远”因为引入周期性而消失,所以必须另外选择能量零点。\upcite{JPC-SSP12-4409_1979,XIE-LU}

\subsection{晶体总能量计算与能量零点选择}
一般地,晶体中的基态总能量$E_T$可以表示成晶格中的电子能量$E_{e-e}$与离子实排斥能$E_{N-N}$之和:~
	\begin{equation}
		E_T=E_{e-e}+E_{N-N}=T[\rho]+E_{ext}+E_{\mathrm{Coul}}+E_{\mathrm{XC}}+E_{N-N}
		\label{eq:Crystal_ENE_R}
	\end{equation}
根据密度泛函理论(\textrm{Density-Functional Theory, DFT})和\textrm{Kohn-Sham}方程\upcite{PRB136-864_1964,PRA140-1133_1965},电子本征态方程为:~
\begin{equation}
	\bigg[\dfrac12\nabla^2+V_{ext}(\vec r)+V_{\mathrm{Coul}}(\vec r)+V_{\mathrm{XC}}[\rho(\vec r)]\bigg]|\psi_i(\vec r)\rangle=\varepsilon_i|\psi_i(\vec r)\rangle
	\label{eq:DFT}
\end{equation}
动能泛函用单电子能量表示为
\begin{equation}
	T[{\rho}]=\sum_in_i\langle\psi_i|\varepsilon_i-V_{\mathrm{KS}}|\psi_i\rangle
	\label{eq:DFT_Kin}
\end{equation}
$n_i$是$\psi_i$上的电子占据数,$\varepsilon_i$是其能量本征值,因此总能量的泛函表示为:
\begin{equation}
	E_T=\sum_in_i\varepsilon_i-\dfrac12\int\int\mathrm{d}\vec r\mathrm{d}\vec r\dfrac{\rho(\vec r)\rho(\vec r^{\prime})}{|\vec r-\vec r^{\prime}|}+\int\mathrm{d}\vec r\rho(\vec r)[\epsilon_{\mathrm{XC}}(\vec r)-V_{\mathrm{XC}}(\vec r)]+E_{N-N}
	\label{eq:DFT_ENE_R}
	\end{equation}

对于周期体系来说,因为电子的能量本征态是与动量空间($\vec K$空间)相关联,即\textrm{Kohn-Sham}方程表示为:
\begin{equation}
	\bigg[\dfrac12\vec k^2+V_{ext}(\vec k)+V_{\mathrm{Coul}}(\vec k)+V_{\mathrm{XC}}[\rho(\vec k)]\bigg]|\psi_i^{\vec k}(\vec r)\rangle=\varepsilon_i^{\vec k}|\psi_i^{\vec k}(\vec r)\rangle
	\label{eq:DFT-k}
\end{equation}
显然,总能量在动量空间中计算更方便:~
\begin{equation}
	E_T=\sum_{i,\vec k}n_i\varepsilon_i^{\vec k}-\dfrac{\Omega}2\sum_{\vec k}\rho^{\ast}(\vec k)V_{\mathrm{Coul}}(\vec k)+\Omega\sum_{\vec k}\rho^{\ast}(\vec k)[\epsilon_{\mathrm{XC}}(\vec k)-V_{\mathrm{XC}}(\vec k)]+E_{N-N}
	\label{eq:DFT_ENE_G}
\end{equation}
其中$V_{\mathrm{Coul}}(\vec k)$、$\epsilon_{\mathrm{XC}}(\vec k)$与$\rho^{\ast}(\vec k)$分别是\textrm{Coulomb}相互作用、单个电子的交换-相关能、交换-相关势和电子密度的\textrm{Fourier}分量。

实际计算中需要作一些数学处理:~
\begin{itemize}
	\item 交换-相关势和交换-相关能的计算一般先在实空间计算$\epsilon_{\mathrm{XC}}(\vec r)$和$V_{\mathrm{XC}}(\vec r)$后,再通过\textrm{Fourier~}变换到动量空间,得到$\epsilon_{\mathrm{XC}}(\vec k)$和$V_{\mathrm{XC}}(\vec k)$。
	\item 由\textrm{Poisson}方程
\begin{equation}
	\nabla^2V_{\mathrm{Coul}}(\vec r)=-4\pi\rho(\vec r)
	\label{eq:Poisson}
\end{equation}
的\textrm{Fourier}展开有
\begin{equation}
	V_{\mathrm{Coul}}(\vec k)=\dfrac{4\pi\rho^{\ast}(\vec k)}{|\vec k|^2}
	\label{eq:FFT_Poisson}
\end{equation}
显然$V_{\mathrm{Coul}}(\vec k=0)$是发散的;
	\item 考虑离子间\textrm{Coulomb}相互作用能之和
	\begin{equation}
		E_{N-N}=\dfrac12\sum_{\vec R,s}\sideset{}{^{\prime}}\sum_{\vec R^{\prime},\vec s^{\prime}}\dfrac{Z_sZ_{s^{\prime}}}{|\vec R+\vec r_s-\vec R^{\prime}-\vec r_s^{\prime}|}
		\label{eq:Ion_Coulomb_ENE}
	\end{equation}
这里$Z_s$是离子实的电荷数,$\vec R$表示晶格点的位矢,$\vec r_s$代表元胞内原子的相对位矢。因为$E_{N-N}$求和包含无穷多项,是发散的;
	\item 用于求解能量本征态的式\eqref{eq:DFT-k}中$V_{ext}$的\textrm{Fourier}分量在$\vec k=0$处也是发散的。
\end{itemize}
因此总能量泛函中,$E_{N-N}$、$V_{\mathrm{Coul}}(\vec k=0)$和$V_{ext}(\vec k=0)$这三项单独都是发散的,但因为整个体系出于电中性,所以这些发散项相互抵消,应是一个常数。

因此实际的总能计算中,首先在求解\textrm{Kohn-Sham}方程时,先将$V_{\mathrm{Coul}}(\vec k=0)$和$V_{ext}(\vec k=0)$同时置为零,这相当于势能作一平移,或者说重新定义势能零点。由此得到的总能泛函为:~
\begin{equation}
	E_T=\sum_{i,\textcolor{red}{\vec k\neq0}}n_i\varepsilon_i^{\vec k}-\dfrac{\Omega}2\sum_{\textcolor{red}{\vec k\neq 0}}\rho^{\ast}(\vec k)V_{\mathrm{Coul}}(\vec k)+\Omega\sum_{\vec k}\rho^{\ast}(\vec k)[\epsilon_{\mathrm{XC}}(\vec k)-V_{\mathrm{XC}}(\vec k)]+E_{N-N}
	\label{eq:DFT_ENE_G-2}
\end{equation}
最后在总能量计算中,考虑补偿势能零点的这一平移。

\subsection{发散项的处理}
根据上面的讨论,总能量中发散项之和为:~
	\begin{equation}
		\begin{aligned}
			\lim_{\vec k\rightarrow0}\Omega&\bigg[\dfrac12V_{\mathrm{Coul}}(\vec k)+\sum_sv_{ext}^s(\vec k)\bigg]\rho^{\ast}(\vec k)+\dfrac12\sum_{\vec R,s}\sideset{}{^{\prime}}\sum_{\vec R^{\prime},\vec s^{\prime}}\dfrac{Z_sZ_{s^{\prime}}}{|\vec R+\vec r_s-\vec R^{\prime}-\vec r_s^{\prime}|}\\
			=&\sum_s\alpha_s\sum_sZ_s+E_{\mathrm{Ewald}}
		\end{aligned}
		\label{eq:diver-term}
	\end{equation}
	
$V_{ext}$在不存在其他外场时,一般只考虑离子-电子的\textrm{Coulomb}相互作用,
	\begin{equation}
		\begin{aligned}
			V_{ext}(\vec r)&=\sum_{\vec R,s}\dfrac{-Z_s}{|\vec r-\vec R-\vec r_s|}\\
			&\equiv\sum_{\vec R,s}v_{ext}^s(\vec r-\vec R-\vec r_s)
		\end{aligned}
		\label{eq:Ion-ele_Coulomb}
	\end{equation}

对于形如$Z_s/r$的外场,其\textrm{Fourier}分量在$\vec k=0$附近展开
	\begin{equation}
		v_{ext}^s(\vec k)=-\dfrac{4\pi Z_s}{\Omega|\vec k|^2}+\alpha_s+O(\vec k)
		\label{eq:V_ext}
	\end{equation}
展开$\rho^{\ast}(\vec k)$,有
	\begin{equation}
		\lim_{\vec k\rightarrow 0}\rho^{\ast}(\vec k)=\dfrac{\sum_sZ_s}{\Omega}+\beta|\vec k|^2+O(\vec k)
		\label{eq:rho_ext}
	\end{equation}
去掉高次项,有
\begin{equation}
	\begin{aligned}
		\lim_{\vec k\rightarrow 0}&\bigg[\boxed{\textcolor{blue}{\dfrac{\Omega}2\dfrac{4\pi[\rho^{\ast}(\vec k)]^2}{|\vec k|^2}}}+\boxed{\Omega}\bigg(\boxed{\textcolor{blue}{-\dfrac{4\pi\sum_sZ_s}{\Omega|\vec k|^2}}}+\sum_s\alpha_s\bigg)\boxed{\rho^{\ast}(\vec k)}+\boxed{\textcolor{red}{\dfrac12\dfrac{4\pi(\sum_sZ_s)^2}{\Omega|\vec k|^2}}}\bigg]\\
		&+\boxed{\dfrac12\sum_{\vec R,s}\sideset{}{^{\prime}}\sum_{\vec R^{\prime},\vec s^{\prime}}\dfrac{Z_sZ_{s^{\prime}}}{|\vec R+\vec r_s-\vec R^{\prime}-\vec r_{s^{\prime}}|}-\lim_{\vec k\rightarrow0}\textcolor{red}{\dfrac12\dfrac{4\pi(\sum_sZ_s)^2}{\Omega|\vec k|^2}}}\\
		=&\sum_s\alpha_s\sum_sZ_s+\textcolor{magenta}{E_{\mathrm{Ewald}}}
	\end{aligned}
	\label{eq:V_ext_exp2}
\end{equation}
其中离子间排斥势采用\textrm{Ewald~}方法得到\upcite{Born-Huang,R.Martin}:~对于形如点电荷形式的静电势$\dfrac{e^2}r$,可引入\textrm{Gauss~}误差函数\upcite{Grosso-Parravicini}
\begin{equation}
	\begin{aligned}
		&\mathrm{erf}(x)=\dfrac2{\sqrt{\pi}}\int_0^{x}\mathrm{e}^{-t^2}\mathrm{d}t\\
		&\mathrm{erfc}(x)=\dfrac2{\sqrt{\pi}}\int_x^{\infty}\mathrm{e}^{-t^2}\mathrm{d}t\\
		\mbox{满足}\quad&\mathrm{erf}(x)+\mathrm{erfc}(x)=1
	\end{aligned}
	\label{eq:err_fun}
\end{equation}
得到恒等式(见图\ref{Error_Function}):
\begin{equation}
	\dfrac{e^2}r\equiv\dfrac{e^2}r\mathrm{erf}(\sqrt{\eta}r)+\dfrac{e^2}r\mathrm{erfc}(\sqrt{\eta}r)
	\label{eq:err_fun_comp}
\end{equation}
\begin{figure}[h!]
\centering
\vspace*{-0.10in}
\includegraphics[height=2.55in,width=5.8in,viewport=0 0 1100 455,clip]{Ewald_method.png}
\caption{\small \textrm{Decomposition of the potential $-e^2/r$ (singular at the origin and of long-range nature) into a contribution $-(e^2/r)\mathrm{erf}(\sqrt{\eta}r)$(regular at the origin of long-range) and a contribution $-(e^2/r)\mathrm{erfc}(\sqrt{\eta}r)$ (singular at the origin and of short-range nature). Here $\sqrt{\eta}=1 (\mathrm{Bohr radius unit})$ is chosen.}\upcite{Grosso-Parravicini}}%(与文献\cite{EPJB33-47_2003}图1对比)
\label{Error_Function}
\end{figure}

	\begin{equation}
		\begin{aligned}
			E_{\textrm{Ewald}}=&\dfrac12\sum_{\vec R,s}\sideset{}{^{\prime}}\sum_{\vec R^{\prime},\vec s^{\prime}}\dfrac{Z_sZ_{s^{\prime}}}{|\vec R+\vec r_s-\vec R^{\prime}-\vec r_{s^{\prime}}|}-\lim_{\vec k\rightarrow0}\dfrac12\times\dfrac{4\pi(\sum_sZ_s)^2}{\Omega|\vec k|^2}\\
			=&\dfrac12\sum_{\vec R,s}\sideset{}{^{\prime}}\sum_{\vec R^{\prime},\vec s^{\prime}}\dfrac{Z_sZ_{s^{\prime}}}{|\vec R+\vec r_s-\vec R^{\prime}-\vec r_{s^{\prime}}|}-\dfrac1{2\Omega}\sum_{s,s^{\prime}}\int\mathrm{d}\vec r\dfrac{Z_sZ_{s^{\prime}}}r\\
			=&\sum_{s,s^{\prime}}Z_sZ_{s^{\prime}}\bigg\{\dfrac{2\pi}{\Omega}\sum_{\vec k\neq 0}\cos[\vec k\cdot(\vec r_s-\vec r_{s^{\prime}})]\dfrac{\mathrm{e}^{-|\vec k|^2/4\eta}}{|\vec k|^2}\\
			&-\dfrac{\pi}{2\eta\Omega}+\dfrac14\sum_{\vec R}\dfrac{\mathrm{erf}(\sqrt{\eta}x)}x\bigg|_{\vec R+\vec r_s-\vec r_s^{\prime}\neq0}-\sqrt{\dfrac{\eta}{\pi}}\delta_{s,s^{\prime}}\bigg\}
		\end{aligned}
		\label{eq:Ewald_ENE}
	\end{equation}
	$\mathrm{erf}(x)$是误差函数,$\sqrt{\eta}$原则上是任意参数。$\alpha_s$由$v_{ext}^s(\vec r)$确定:~
	\begin{equation}
		\alpha_s=\lim_{\vec k\rightarrow0}\bigg[v_{ext}^s(\vec k)+\dfrac{4\pi Z_s}{\Omega|\vec k|^2}\bigg]=\dfrac1{\Omega}\int\mathrm{d}\vec r\bigg[v_{ext}^s(\vec r)+\dfrac{Z_s}r\bigg]
		\label{eq:alpha_s}
	\end{equation}
由此得到的总能量表达式是:
\begin{equation}
	\begin{aligned}
		E_T=&\sum_i\varepsilon_i-\dfrac{\Omega}2\sum_{\vec k\neq0}\rho^{\ast}(\vec k)V_{\mathrm{Coul}}(\vec k)\\
		&+\Omega\sum_{\vec k}\rho^{\ast}(\vec k)[\epsilon_{\mathrm{XC}}(\vec k)-V_{\mathrm{XC}}(\vec k)]\\
		&+\sum_s\alpha_s\sum_sZ_s+E_{\mathrm{Ewald}}
	\end{aligned}
	\label{eq:TOT_ENE_Finial}
\end{equation}

\textrm{VASP~}软件的总能量计算即遵照式\eqref{eq:TOT_ENE_Finial}计算的。图\ref{TOTEN_VASP}给出就是\textrm{VASP~}总能计算的输出形式:~
\begin{figure}[h!]
\centering
\vspace*{-0.12in}
\includegraphics[height=3.85in,width=4.2in,viewport=0 0 600 495,clip]{VASP_Total_ENE.png}
\caption{\small \textrm{The Total-E calculated by VASP.}}%(与文献\cite{EPJB33-47_2003}图1对比)
\label{TOTEN_VASP}
\end{figure}

%根据\textrm{Ewald}的势能计算方法,$\boxed{\dfrac12\dfrac{4\pi(\sum_sZ_s)^2}{\Omega|\vec k|^2}}$表示的电子势能在$\vec k=0$处的贡献,可分为
%	\begin{itemize}
%		\item \textcolor{purple}{对应于实空间电子势能的长程可收敛部分:~}式\eqref{eq:Ewald_ENE}中第三项
%			\begin{displaymath}
%				-(\sum\limits_{s,s^{\prime}}Z_sZ_{s^{\prime}})\dfrac14\sum_{\vec R}\dfrac{\mathrm{erf}(\sqrt{\eta}x)}x\bigg|_{\vec R+\vec r_s-\vec r_s^{\prime}\neq0}
%			\end{displaymath}
%		\item \textcolor{purple}{对应于实空间电子势能的短程发散部分:~}式\eqref{eq:Ewald_ENE}中第二项
%			\begin{displaymath}
%				(\sum_{s,s^{\prime}}Z_sZ_{s^{\prime}})\dfrac{\pi}{2\eta\Omega}
%			\end{displaymath}
%		\textcolor{red}{注意:~实际计算中,因为误差函数的参数$\sqrt{\eta}$不为零,因此该发散部分表示为一个大数而不是$\infty$}。
%	\end{itemize}
%	类似地,不难看出,式\eqref{eq:Ewald_ENE}中\textcolor{blue}{第一项}和\textcolor{magenta}{第四项}分别对应离子-电子的\textrm{Coulomb}相互作用
%	\begin{displaymath}
%		\boxed{\dfrac12\sum_{\vec R,s}\sideset{}{^{\prime}}\sum_{\vec R^{\prime},\vec s^{\prime}}\dfrac{Z_sZ_{s^{\prime}}}{|\vec R+\vec r_s-\vec R^{\prime}-\vec r_{s^{\prime}}|}}
%	\end{displaymath}
%	的\textcolor{blue}{长程收敛}和\textcolor{magenta}{短程发散}部分。
%
%以\textrm{FCC-Al}为例,采用\textrm{VASP~}计算得到有关数值如下:~
%\begin{displaymath}
%	\begin{aligned}
%	&\mathrm{E-fermi}:~&7.4406\\
%	&\sum\alpha_iZ_i:~&-0.1949\\
%	&\mathrm{Ewald-Energy}:~&-72.4621\\
%	&\mathrm{XC(G=0)}:~&-10.00040 \\
%	\end{aligned}
%\end{displaymath}
%将\textrm{VASP~}计算中的\textrm{Ewald-Energy}按式\eqref{eq:Ewald_ENE}分解,各部分对应的数值为:~
%\begin{displaymath}
%	\begin{aligned}
%	&\mathrm{Part-1}:~&1.320907 \\
%	&\mathrm{Part-2}:~&-50.176618 \\ 
%	&\mathrm{Part-3}:~&1.481905 \\
%	&\mathrm{Part-4}:~&-25.088309 \\
%	\end{aligned}
%\end{displaymath}
%\textcolor{red}{不难看出,这里第二项和第四项分别代表两部分势能在$\vec k=0$的发散项贡献,因此数值的绝对值比另外两项大得多。}

\subsection{能量零点移动对能量本征态的影响}
根据上述讨论,因为能量零点的平移,周期体系计算的能量本征值$\varepsilon_i$的数值一般不绝对为负值。习惯上在能带和态密度表示时,常常将\textrm{Fermi~}能设置成零。

参照总能计算中能量零点移动的讨论,可以计算\textrm{Kohn-Sham~}方程中势能零点引起的能量本征值的移动,%根据检索\textrm{VASP~}的代码发现:~\textcolor{red}{\textrm{VASP~}程序在构造\textrm{Fock~}矩阵的时候,已经包括了$V_{\mathrm{XC}}(\vec k=0)$的贡献~},即\textcolor{purple}{$\mathrm{XC(G=0)}$}对应的数值(见图\ref{TOTEN_VASP});~
$V_{\mathrm{Coul}}(\vec k=0)$和$V_{ext}[\rho(\vec k=0)]$。
即
\begin{equation}
	\lim_{\vec k\rightarrow0}\bigg[V_{\mathrm{Coul}}(\vec k)+\sum_sv_{ext}^s(\vec k)\bigg]
	\label{eq:part_diver-term}
\end{equation}
式\eqref{eq:part_diver-term}\textbf{势的移动}并不简单对应式\eqref{eq:diver-term}中\textbf{能量移动}的发散项求和。

在$\vec k=0$附近,将式\eqref{eq:V_ext}和\eqref{eq:rho_ext}代入式\eqref{eq:part_diver-term},去掉高次项,可有
\begin{equation}
	\begin{aligned}
		&\lim_{\vec k\rightarrow 0}\bigg[\dfrac{4\pi\rho^{\ast}(\vec k)}{|\vec k|^2}+\bigg(-\dfrac{4\pi\sum_sZ_s}{\Omega|\vec k|^2}+\sum_s\alpha_s\bigg)\bigg]\\
		=&\lim_{\vec k\rightarrow 0}\bigg[\boxed{\dfrac{4\pi}{\Omega}\dfrac{\sum_sZ_s}{|\vec k|^2}}+4\pi\beta\boxed{-\dfrac{4\pi\sum_sZs}{\Omega|\vec k^2|}}+\sum_s\alpha_s\bigg]\\
		=&\sum_s\alpha_s+4\pi\beta
	\end{aligned}
	\label{eq:V_shift-term}
\end{equation}
式\eqref{eq:V_shift-term}对应\textrm{VASP~}软件中给出的\textcolor{purple}{$\mathrm{alpha+bet}$}的数值,见图\ref{TOTEN_VASP}(在\textrm{VASP~}中,\textcolor{purple}{bet}项对应式\eqref{eq:V_shift-term}的$4\pi\beta$)。

上述推导也验证了赝势理论的基本思想:~对于电中性的周期体系,在倒空间中,电子\textrm{Coulomb~}势与原子核的吸引势相互抵消后,净的作用可近似为高阶奇点和一个平缓的势函数。\textrm{VASP~}软件中,$\alpha_s$和$\beta$的数值计算方式如下:~
\begin{itemize}
	\item \textcolor{blue}{$\alpha_s$取原子赝势在径向的第一个点(即离$\vec k=0$最近)的数值}
	\item \textcolor{blue}{$\beta$由各原子赝电荷密度前5个点(即离$\vec k=0$足够近)的数值两阶差分后求和得到}
\end{itemize}

传统的求解\textrm{Kohn-Sham~}方程计算能量本征值时,将势函数中包括核-电子吸引和电子间排斥排斥势的全部$\vec k=0$部分的贡献扣除。如果考虑补偿函数式\eqref{eq:V_shift-term}的贡献,则利用了上述两项的奇点能量部分抵消的特性,保留了势能零点在无穷远时的部分特征(\textcolor{red}{注意:~采用该能量补偿方案,高阶奇点仍然存在!})。

综上所述,在\textrm{VASP~}中,如果考虑周期体系的势能零点移动,则\textrm{Fermi~}的数值可取为\textbf{两项之和}:~
\begin{displaymath}
	\mathrm{E_{fermi}}=\textcolor{blue}{\mathrm{E-fermi}}+\textcolor{purple}{\mathrm{alpha+bet}}
\end{displaymath}
这就是排除高阶奇点后的\textrm{Fermi~}能,与传统的分子、原子中能量计算结果相近。
%\subsection{一点讨论}
%我们对\textrm{FCC-Al}的计算表明,
%\begin{displaymath}
%	\begin{aligned}
%	&\mathrm{E-fermi}:~&7.4406\\
%	&\sum\alpha_iZ_i:~&-0.1949\\
%	&\mathrm{Ewald-Energy}:~&-72.4621\\
%	&\mathrm{XC(G=0)}:~&-10.00040 \\
%	&\mathrm{alpha+bet}:~-&14.2459\\
%	\end{aligned}
%\end{displaymath}
%因此考虑势能零点移动修正的\textrm{Fermi}能应为(\textcolor{blue}{包含\textbf{OUTCAR}中$\mathrm{alpha+bet}$项}):
%\begin{displaymath}
%	7.4406-14.2459=-6.80\;\mathrm{eV}
%\end{displaymath}
%$\ast$注:上述计算验证~
%\begin{itemize}
%	\item 在\textrm{VASP~}计算中,$\mathrm{alpha+bet\mbox{项}}<0$恒成立
%	\item \textrm{VASP~}中$|\mathrm{E-Fermi}|<|\mathrm{alpha+bet}|$成立
%\end{itemize}

\subsection{\rm{VASP}计算结果对上述推导的检验}
考虑周期体系,如果原子间距离足够远,则周期体系计算结果应该逼近原子分子体系的计算结果,基于该思想,可以用\textrm{VASP}软件对假想的\textrm{Si}和\textrm{Fe}孤立原子(分别在足够大的晶胞中),然后逐渐减少原子间距离,考察\textrm{OUTCAR}文件中\textcolor{purple}{\textrm{alpha+bet}}的变化,结果列于图\ref{Fig:VASP_alpha+bet}。
\begin{figure}[h!]
\centering
\hspace*{-0.7in}
\vspace*{-0.2in}
\includegraphics[width=1.2\textwidth, viewport=0 0 1520 780, clip]{VASP_Fermi_alpha-bet.pdf}
\caption{\small Compare the \textcolor{purple}{\textrm{alpha+bet}} with the distance of atom.}%(与文献\cite{EPJB33-47_2003}图1对比)
\label{Fig:VASP_alpha+bet}
\end{figure}

不难看出,随着原子间距离的逐步增大,\textcolor{purple}{\textrm{alpha+bet}}的数值逐渐趋向于零,相应的\textrm{Fermi}数值逐渐逼近原子能级(数据未在此列出)。一般地认为原子间距离超过10~\AA ,可以原子间没有相互作用,考虑本次极端模型中原子采用\textrm{FCC}密堆积,原子间距离为10~\AA~时对应的晶胞参数为15.874~\AA ,当晶胞参数大于15.874~\AA ,\textcolor{purple}{\textrm{alpha+bet}}的数值将趋向于0,这一结论与图\ref{Fig:VASP_alpha+bet}中曲线趋向零的起始位置吻合得很好。因此,趋于极端情况的假想模型计算结果也验证了\textcolor{purple}{\textrm{alpha+bet}}表示的就是排除高阶奇点后的能量本征值修正项,与前一节讨论的结论一致。

\section{高效并行与并发式自动流程算法实现} \label{chap:parallel_Concurrent}
高性能计算(\textrm{High~performance~computing, HPC})是计算机科学的一个分支,最早起源可追溯到20世纪70年代,其经历了从早期的巨型计算机到向量机,再到大规模并行处理的\textrm{MPP}~计算机和集群处理的超级计算机的发展历程。高性能计算机通过各种互联技术将多个标准计算机系统或者多个高度专用的硬件连接在一起,构成一个计算机集群系统(\textrm{computer cluster}),使用大量处理器、内存和存储设备结合大规模专业软件组成的综合计算能力,主要用于解决大型计算问题。

%完整的计算集群从基本层次上可划分为:~计算单元、管理系统、网络系统、共享存储系统、集群软件和应用软件等部分,其中计算单元包含:双路和多路计算单元;管理系统包含:管理结点和登陆结点;网络系统包含:管理网络、计算网络;共享存储系统包含:\textrm{I/O}~节点、存储设备、集群文件系统;集群软件包含:操作系统、集群管理软件、作业调度、编译环境、并行计算中间件、数学函数库等;
随着信息化社会和大数据的快速发展,%人类对科技创新、信息处理能力的要求越来越高,不仅在生物、工程、石油勘探、气象预报、航天国防、科学研究等领域得到了广泛的使用,而且在金额、政府信息化、教育、企业、网络游戏等领域的应用也取得了迅速发展。
高性能计算技术作为工具,为各学科、行业提供越来越重要的支持,大大促进了社会的生产和创新能力。从材料科学的分子动力学模拟、气象科学的数值模拟、生命科学中的基因分析、分子成像到汽车碰撞安全、电子家电乃至航空航天等领域都越来越倚重高性能计算;石油天然气与地球科学、建筑工程、媒体娱乐等行业,都需要专业的图形渲染能力,有强烈的计算可视化需求。

%随着模拟精度的提高、处理问题的深入和处理对象复杂程度的增大,对计算资源和计算机性能也提出越来越高的需求。基于模拟仿真的工程科学结合传统工程领域的知识技术与高性能计算,提供经济高效地设计与实践方法,例如新材料的研发制备过程需要多重复杂流动物理现象模拟、纳米技术领域的复合材料结构分析和功能预测、新材料的发明都需要进行高性能计算进行实验模拟验证。

进入20世纪90年代,伴随着理论化学、计算物理方法的快速发展以及计算机软硬件技术不断升级和更新,计算材料科学获得了空前发展,它与物理、化学、工程力学以及应用数学等诸多基础和应用学科日益交叉并融合,逐渐成为一门新兴学科,在材料研究中发挥越来越重要的作用\upcite{NatMat3-429_2004,App-CataA254-5_2003,JACS125-4306_2003,JCombChem5-472_2003,Meas_Sci-Tech16-1_2005,Nature392-694_1998}。尤其值得注意的是,近年来,得益于高精度的多尺度计算方法和高性能并行计算技术的突破\upcite{PRL88-255506_2002,Nano-Lett3-1183_2003},高通量(\textrm{high~throughput})材料计算\footnote{材料研究领域的高通量\upcite{Nature410-643_2001},最先借鉴自药物合成中的组合合成与筛选\textrm{(combinatorial synthesis and screening)}的思想而出现的,组合与筛选研究兴起于1990年代中期\upcite{Science268-1738_1995},在21世纪最初十年,逐渐扩展到计算材料研究领域,形成“高通量材料计算”的理念。}在创新发展新材料、发现新现象方面显现出强大的能力,借助机器学习技术进行材料性能预测,加速材料属性改善、优化和提升,是近年来的研究热点,前景广阔。

针对材料计算的特点,为适应材料科学软件的运行特点而搭建的分布式计算环境称为计算材料集群,其拓扑结构(\textrm{topological structure})如图\ref{Fig:HPC_Cluster}所示。
\begin{figure}[h!]
\centering
\includegraphics[height=2.1in,width=4.3in,viewport=0 0 400 210,clip]{HPC_Material-Cluster.png}%}
\caption{计算材料集群的基本拓扑结构.}%
\label{Fig:HPC_Cluster}
\end{figure}
和通用的高性能计算集群相比,计算材料集群有其自身的一些特色:~比如集群主要面向不同应用场景下的材料模拟和计算任务,因此将集成更多的并行编程工具和应用中间件,以支撑不同尺度下的材料计算软件的运行环境;当前材料模拟的主要计算需求来自微观尺度的第一原理和分子动力学部分,因此需要更大的并行规模和更灵活的调度策略;计算材料软件运行过程中需要处理大量的微分方程迭代求解问题,要求集群支持更多多路计算设备和更大内存计算设备;近年来,随着\textrm{GPU}加速计算的流行,对集群支持异构计算\textrm{(Heterogeneous Computing)}的需求日益强烈,因此具备专业的\textrm{GPU}协处理器加速,并可支持\textrm{CUDA}的开发环境的计算集群也越来越流行;计算材料科学领域的研究软件有相当部分是开源的自由软件,要求集群提供更为自由、开放和灵活的运行环境等等。\upcite{NatMat12-191_2013}

%因为材料计算学的数据特征,导致材料计算需要高性能计算来实现。传统的材料设计主要有以下几个模块。优化模块接收模块;知识获取模块以及计算模块。在高性能计算模式逐渐转化成为网络系统之后,这些模块的识别和人工智能的神经网络为基础的专家系统成为了主流,目前在处理材料科学记录中,越来越多的应用在处理规律不明显,组分变量多,非线性方法问题复杂的特殊性能的情况下,这种系统的优越性凸显出来,并且可以通过对数学模型以及计算结果的验证进行分析和处理。

进入21世纪以后,欧洲、日本、美国等发达国家先后启动了以多尺度模拟、研发和科学设计为手段,提升材料设计成功率,缩短材料开发周期为目的的多种类型的国家级科研专项,其中以美国的“材料基因组计划\textrm{(Materials Genome Initiative, MGI)}”\upcite{MGI_USA}最为著名。该计划的根本目的是通过增效集成各个尺度的材料模拟工具、高效实验手段和数据库,把材料研发从传统经验式试错提升到有科学理论指导的理性设计,以期大幅缩短材料的研发周期。我国也于2016年起启动了国家重点研发计划“材料基因工程关键技术与支撑平台”,推动我国在相关领域的研究和跟进。

实现材料物性设计与模拟计算的高通量自动运行是“材料基因组”计划的核心内涵之一, 已在能源材料预测\upcite{PRL108-068701_2012}、拓扑绝缘体发现\upcite{RMP82-3045_2010}、热电材料\upcite{JACS128-12140_2006}、催化材料\upcite{ACIE46-6016_2007}、轻质镁合金研究\upcite{PRB84-084101_2011}、超导材料\upcite{PRL105-217003_2010}、磁性材料\upcite{NatMat10-158_2011,JPD40-R337_2007},复杂多组元化合物表面设计\upcite{Science316-732_2007,ACSNano5-247_2011},对二元或三元化合物结构稳定性判断\upcite{PRB85-144116_2012},以及高强高温合金等体系中有广泛的成功应用和尝试。因为不同尺度下的材料组成基本单元服从不同的物理规律,使用的模拟与计算软件也千差万别,因此从应用软件组织的角度说,高通量自动流程主要面向材料模拟和计算过程的文件组织、软件提交和数据分析过程的自动实现,重点围绕以下问题:~
\begin{itemize}
	\item 材料计算和模拟软件\textrm{I/O}数据生成、传递的自动化;
	\item 计算和模拟作业提交、控制及执行的自动化;
	\item 计算中间结果和最终结果的解析和可视化展示的自动化。
\end{itemize}
伴随着高通量的新材料计算、设计和优化实现过程,一大批开放型材料物性数据库也应运而生。随着材料数据的不断积累、丰富和充实,辅之以数据挖掘(\textrm{data mining})技术,有望加快材料设计、优选的进程。

\subsection{高效并行与并发环境下的实时自动流程}
在讨论材料计算的自动流程之前,首先明确一下并行(\textrm{parallel})和并发(\textrm{concurrent})的区别:~计算机学科中,并行和并发是两个既关联又有区别的概念,并行与串行(\textrm{serial})相对,侧重于强调某一任务执行过程中有可能被分解成多个独立的部分,分别提交到多个处理器上同时执行;~并发与顺序(\textrm{sequential})相对,更多是从用户角度描述,可同时向处理器提交多个任务,但对处理器而言,实际是通过分时(\textrm{time-sharing})技术分别来完成这些并发任务的,并非像并行那样真正做到“同时”处理。\textrm{ Joe Armstrong}曾用排队接咖啡的漫画(图\ref{Fig:Current_and_Parallel})形象地说明了并发和并行的区别。
\begin{figure}[h!]
\centering
\includegraphics[height=3.1in,width=4.3in,viewport=0 0 600 450,clip]{concurrent_and_parallel.jpg}%}
\caption{\textrm{Joe Armstrong}用排队接咖啡的漫画形象地说明了并发和并行的区别.}%
\label{Fig:Current_and_Parallel}
\end{figure}

通用的计算机集群都配备管理调度系统,负责对硬件、用户和计算资源进行有效分配、组织和管理,其中作业管理系统支持用户队列产生的作业提交、调度和监控,这些都是并发处理的。具体到材料模拟计算过程,自动流程也会对计算作业的提交、计算软件的组织和运行产生一定的影响。与通用的作业调度最重要的区别在于,前者主要通过提供、组织材料计算核心软件所需的文件,调度材料计算核心软件的运行所要求的资源,确定计算优先级和运行模式,确保完整的材料模拟计算过程在集群上稳定执行。换言之,作业调度更多侧重于计算机软硬件层面的资源管理,计算流程更多负责提供和调控材料模拟所需物理信息。

对一个材料计算作业来说,在完成计算核心软件所需控制文件和参数传递后,一旦作业被提交,计算进程就由作业管理系统接管,自动流程一般不会对进程进行干预,但对计算过程的产生的文件和输出会保持监控,直到核心计算软件进程结束,自动流程再从作业管理系统处接管作业的控制权,通过必要判断——如检测计算进程是否属于正常结束、后续计算控制参数是否正确产生——然后决定是否继续启动后续作业进程。材料计算核心软件一般都可以并行执行,如流行的第一原理计算软件\textrm{VASP}\upcite{VASP_manual}、\textrm{ABINIT}\upcite{CMS25-478_2002,CPC180-2175_2009}、\textrm{Siesta}\upcite{JPCM14-2745_2002,JCP152-204108_2020}以及量子化学软件\textrm{Gaussian}\upcite{Gaussian-UG_2004}、\textrm{ADF}\upcite{ADF-UG_1995}等,都有较好的\textrm{MPI}并行能力;分子动力学软件如\textrm{LAMMPS}\upcite{JCP117-1_1995}、\textrm{Gromacs}\upcite{JCC26-1701_2005}等不仅可以具备\textrm{MPI}并行能力,近年来利用\textrm{GPU}加速,计算能力得到进一步的提升。

当前材料计算的自动流程主要以支持原子、分子尺度等微观模拟计算为主,目标是实现材料电子结构、分子动力学和热力学物理量计算流程的自动化,并建立相应的数据库。一般材料计算的自动流程和数据库的示意简图如\ref{Fig:MP_data_flow}所示。
\begin{figure}[h!]
\centering
\includegraphics[height=3.1in,width=4.3in,viewport=0 0 800 650,clip]{MP_data_flow.png}%}
\caption{一般材料计算自动流程中的数据流示意(引自文献\upcite{CMS49-299_2010}).}%
\label{Fig:MP_data_flow}
\end{figure}
%\section{材料基因组的基本思想}

近年来国际上涌现的知名高效材料计算自动流程软件主要有:~\textrm{Automatic Flow(AFLOW)}\upcite{CMS49-299_2010},\textrm{Materials Project(MP)}\upcite{CMS50-2295_2011},\textrm{Quantum Materials Informatics Project (QMIP)}\upcite{QMIP_URL},\textrm{Clean Energy Project (CEP)}\upcite{JPCL2-2241_2011},\textrm{Atomic Simulation Environment (ASE)}\upcite{JPCM29-273002_2017}以及我国中科院计算机网络信息中心杨小渝等开发的\textrm{Matcloud}\upcite{CMS146-319_2018}等。这些软件主要通过集成密度泛函理论\textrm{(DFT)}为基础的第一原理和分子动力学程序,如\textrm{VASP}、\textrm{Quantum ESPRESSO (QE)}\upcite{JPCM21-395502_2009}、\textrm{ABINIT}\upcite{CPC180-2582_2009}、\textrm{Gaussian}、\textrm{LAMMPS}、\textrm{CP2k}\upcite{JCP152-194103_2020}等,完成材料的微观电子结构和相关物性计算。与传统的材料计算不同,高通量自动流程的最终目标之一是建立系统的材料数据库,高通量计算的结果在自动流程的终端将存储到对应的材料数据库中。

%\section{主要高通量材料计算自动流程软件与实现}
材料计算的自动流程实现了对传统材料计算与模拟过程的完全自动化与程序化,而高通量计算主要解决材料计算大规模数据生成和筛选的问题。目前的高通量材料计算以支持第一原理\textrm{DFT}计算为主,少数可支持跨尺度/多尺度\textrm{(DFT-MD)}计算。考虑材料计算过程的一般特点,不难发现,自动流程主体结构主要包括: 
\begin{enumerate}
	\item 结构建模自动化、计算流程参数控制;
	\item 计算任务生成与提交,计算进程监控与修正;
	\item 计算结果数据分析、可视化;
	\item 材料数据的传输与数据库管理。
\end{enumerate}
不同软件的主要区别在于,程序主体结构实现采用的编程语言和支持框架不同,主体结构之间的耦合程度不同。以下将简要介绍各种高通量材料模拟自动流程软件的结构和程序实现。

\subsubsection{\rm{AFLOW}}
\begin{figure}[h!]
\centering
\includegraphics[height=2.4in,width=3.2in,viewport=0 0 720 660,clip]{AFLOW_database.png}%}
%\includegraphics[height=1.2in,width=1.7in,viewport=0 0 670 530,clip]{MP_comp_infrastructure.png}%}
%\includegraphics[height=1.2in,width=1.7in,viewport=0 0 670 420,clip]{QMIP_shame.png}%}
%\includegraphics[height=1.2in,width=1.6in,viewport=0 0 1020 730,clip]{CEP_structure_flow.png}%}
\caption{\textrm{AFLOW}的执行界面.}%
\label{Auto_Flow_Platform-1}
\end{figure}
\textrm{AFLOW}是由美国\textrm{Duke}大学材料系~\textrm{S. Curtarolo}~等开发的高通量计算流程框架\upcite{CMS49-299_2010,NatMat12-191_2013},程序主体部分用\textrm{C++}编写,代码量超过\textrm{150,000}行。\textrm{AFLOW}主要用于无机材料、无机化合物、合金材 料设计等研究领域。\textrm{AFLOW}转变了传统电子结构和物性计算的模式,通过设计自动流程来完成材料物性系列计算。\textrm{AFLOW}框架中集成的第一原理计算程序包为\textrm{VASP}。除此之外,\textrm{AFLOW}主体程序分为两部分,前处理部分主要是计算对象的结构文件转换和对称性分析模块,后处理程序\textrm{APENNSY}是数据分析与可视化模块。这些功能模块结构清晰,既可以整合使用,也可拆分独立工作,非常灵活。\textrm{AFLOW}的最重要的成果之一是其庞大的数据库\textrm{AFLOWLIB},涵盖二元合金数据库、电子结构数据库、\textrm{Heusler}金属间化合物数据库和元素数据库等。同时\textrm{AFLOWLIB}也成为高通量自动流程的重要支撑,提升了软件的建模和分析能力。

\subsubsection{\rm{MP}}
\textrm{MP}最初是美国\textrm{MIT}材料科学与工程系\textrm{G. Ceder}等为加速$\mathrm{Li}^+$、$\mathrm{Na}^+$电池研发进程而开发的无机材料化合物的物性数据库生成工具\upcite{ECC12-427_2010,JECS158-A309_2011,CMS97-209_2015}。%目前其数据库已经建立涵盖~80,000~多种无机化合物的物性。 
\textrm{MP}是用\textrm{Python}开发的,具有很好的可移植性、通用性和扩展性。主要功能模块包括:~前后处理模块\textrm{Python Materials Genomics (Pymatgen)}\upcite{CMS68-314_2013}、进程管理模块\textrm{FireWorks}和自纠错模块\textrm{Custodian}。

与\textrm{AFLOW}不同, \textrm{MP}通过\textrm{FireWorks}将高通量计算全部流程也纳入\textrm{MongoDB}数据库统一管理,相比于其他流程控制模式,\textrm{FireWorks}的流程管理要灵活、方便得多。此外,\textrm{MP}中的\textrm{Custodian}模块提供了软件的容错机制,允许用户定义简单的自动判断和纠错规则。
\begin{figure}[h!]
\centering
%\includegraphics[height=1.2in,width=1.6in,viewport=0 0 720 660,clip]{AFLOW_database.png}%}
\includegraphics[height=2.4in,width=3.4in,viewport=0 0 670 530,clip]{MP_comp_infrastructure.png}%}
%\includegraphics[height=1.2in,width=1.7in,viewport=0 0 670 420,clip]{QMIP_shame.png}%}
%\includegraphics[height=1.2in,width=1.6in,viewport=0 0 1020 730,clip]{CEP_structure_flow.png}%}
\caption{\textrm{Materials Project (MP)}的基本构架(引自文献\upcite{CMS97-209_2015}).}%
\label{Auto_Flow_Platform-2}
\end{figure}

\subsubsection{\rm{QMIP}}
\textrm{QMIP}是丹麦技术大学、美国\textrm{Argonne}国家实验室和\textrm{Stanford}大学合作开发中的高通量催化材料数据库生成工具\upcite{QMIP_URL}。\textrm{QMIP}主要由\textrm{Computational Materials Repository (CMR)}和\textrm{CatApp}两大功能模块构成。其中\textrm{CMR}由丹麦技术大学用\textrm{Python}开发的催化材料计算与数据库高通量计算与管理模块\upcite{CMR_URL},用于计算钙钛矿吸收谱及其能带电子性质。\textrm{CatApp}是\textrm{Stanford}大学化工系\textrm{J.~K.~N{\o}rskov}等开发应用于表面化学和非均相催化的网页版高通量应用程序\upcite{ACIE51-272_2012},主要针对高通量框架下表面化学反应活化能数据计算,用\textrm{JavaScript}、\textrm{SVG}和\textrm{html}语言实现。\textrm{QMIP}的数据库是以催化材料为特色,目前计算并提供的各类异质结构催化数据库, 包括:\textrm{2D}材料、\textrm{Van der Waals}异质结构、有机金属卤化物钙钛矿、基于卟啉染料、新型捕光材料、裂解水钙钛矿材料、低对称钙钛矿等; 此外数据库还为各种不同软件和方法提供用于“标定结果”的参照算例,这是其他高通量计算软件不具备的特色,标定对象主要面向\textrm{DFT}软件(如\textrm{FHI-aims}\upcite{CPC180-2175_2009}、\textrm{VASP}、\textrm{QE}、\textrm{Atomistix ToolKit~(ATK)}\upcite{JPCM32-015901_2020}、\textrm{ABINIT}、\textrm{CASTEP}\upcite{ZFK220-567_2005}、GPAW\upcite{JPCM22-253202_2010}等)、局域和杂化泛对含对3\,\textit{d} 电子处理效果、原子\textrm{O}和\textrm{C}在~\textit{fcc}~结构的[111]面上的 吸附能、高通量计算的赝势等。不过目前\textrm{QMIP}的数据库容量还比较小,仍在开发和扩充中。
\begin{figure}[h!]
\centering
%\includegraphics[height=1.2in,width=1.6in,viewport=0 0 720 660,clip]{AFLOW_database.png}%}
%\includegraphics[height=2.4in,width=3.4in,viewport=0 0 670 530,clip]{MP_comp_infrastructure.png}%}
\includegraphics[height=2.4in,width=3.4in,viewport=0 0 670 420,clip]{QMIP_shame.png}%}
%\includegraphics[height=1.2in,width=1.6in,viegwport=0 0 1020 730,clip]{CEP_structure_flow.png}%}
\caption{\textrm{QMIP}的基本构架.}%
\label{Auto_Flow_Platform-3}
\end{figure}

\subsubsection{\rm{CEP}}
\textrm{CEP}是\textrm{Harvard}大学化学与化学生物系\textrm{Al{\'a}n Aspuru-Guzik}等最初为寻找高效有机光电材料设计的高通量计算化学软件\upcite{JPCL2-2241_2011},也是用\textrm{Python}开发的,其软件结构见图\ref{Auto_Flow_Platform-4}。\textrm{CEP}目前广泛应用于多种有机高分子材料的研发应用,如筛选有机太阳能电池、有机半导体、燃料电池中的高分子膜材料等。\textrm{CEP}的计算方案中,结合了传统材料模拟方案和现代药物研发模式,通过集成\textrm{MOPAC}\upcite{JCAMD4-1_1990}, \textrm{Q-Chem}\upcite{PCCP8-3172_2006}等量子化学软件,系统计算并建立了覆盖$\sim1,000,000$种电子结构的候选光电材料数据库(\textrm{MySOL},\textrm{Django}形式)。在此基础上,\textrm{CEP}希望借助化学信息学和数据挖掘技术,更高效、快速地筛选合适的光电材料。化学信息学辅助的机器学习是\textrm{CEP}的显著技术特色。 但\textrm{CEP}的大部分功能模块都只是部分完成或还在研发中,其代码也未开源,因此\textrm{CEP}的完善程度远不如其他几类软件平台,目前主要在完成数据库的积累。 

\begin{figure}[h!]
\centering
%\includegraphics[height=1.2in,width=1.6in,viewport=0 0 720 660,clip]{AFLOW_database.png}%}
%\includegraphics[height=2.4in,width=3.4in,viewport=0 0 670 530,clip]{MP_comp_infrastructure.png}%}
%\includegraphics[height=2.4in,width=3.4in,viewport=0 0 670 420,clip]{QMIP_shame.png}%}
\includegraphics[height=2.4in,width=3.2in,viewport=0 0 1020 730,clip]{CEP_structure_flow.png}%}
\caption{\textrm{Clean Energy Project (CEP)}的结构与流程图(引自文献\upcite{JPCL2-2241_2011}).}
\label{Auto_Flow_Platform-4}
\end{figure}

\subsubsection{\rm{ASE}}
\textrm{ASE}是由丹麦技术大学物理系\textrm{K.~W.~Jacobsen}等用\textrm{Python}开发的多尺度计算自动流程框架\upcite{JPCM29-273002_2017}。与上述自动流程相比,\textrm{ASE}的功能模块最大的优势是计算模块提供了足够多的软件接口,可以支持第一原理-分子动力学跨尺度计算; \textrm{ASE}支持的结构建模与分析模块,允许用户根据需要任意地组装原子、分子、界面和块体等多相结构,大大提升了软件对建模的自主性和灵便性的支持。另一方面,\textrm{ASE}对多任务作业的支持和管理与\textrm{AFLOW}和\textrm{MP}不同,主要依赖\textrm{Python}对脚本和接口软件的支持。
\begin{figure}[h!]
\centering
%\includegraphics[height=1.2in,width=1.6in,viewport=0 0 720 660,clip]{AFLOW_database.png}%}
%\includegraphics[height=2.4in,width=3.4in,viewport=0 0 670 530,clip]{MP_comp_infrastructure.png}%}
%\includegraphics[height=2.4in,width=3.4in,viewport=0 0 670 420,clip]{QMIP_shame.png}%}
\includegraphics[height=2.4in]{ASE_Python_lib.png}%}
\caption{\textrm{ASE}的\textrm{Python}模块关系(引自文献\upcite{ASE_URL}).}
\label{Auto_Flow_Platform-5}
\end{figure}

\subsubsection{\rm{MatCloud}}
\textrm{MatCloud}是由中国科学院计算机网络信息中心材料基因实验室杨小渝等开发的,可使用\textrm{VASP}开展计算的高通量计算和数据管理平台\upcite{CMS146-319_2018},前端用户界面采用\textrm{JavaScript}开发,作业生成、运行与管理采用\textrm{.NETCore}框架开发,后台数据管理采用\textrm{MogonDB}。相比于国外的各 类高通量自动流程软件,\textrm{MatCloud}的数据集成度高。\textrm{MatCloud}的界面友好,方便使用和作业提交,数据管理和集成的机器学习功能也有一定的特色,但目前提供的核心软件接口主要支持\textrm{VASP}。

\begin{figure}[h!]
\centering
%\includegraphics[height=1.2in,width=1.6in,viewport=0 0 720 660,clip]{AFLOW_database.png}%}
%\includegraphics[height=2.4in,width=3.4in,viewport=0 0 670 530,clip]{MP_comp_infrastructure.png}%}
%\includegraphics[height=2.4in,width=3.4in,viewport=0 0 670 420,clip]{QMIP_shame.png}%}
\includegraphics[height=2.4in]{Matcloud.jpg}%}
\caption{\textrm{MatCloud}的结构与流程图(引自文献\upcite{CMS146-319_2018}).}
\label{Auto_Flow_Platform-6}
\end{figure}

各种高通量材料计算自动流程软件功能的对比列于表\ref{Table-Cost}。
\begin{table}[!h]
\tabcolsep 0pt \vspace*{-5pt}
\begin{minipage}{0.95\textwidth}
%\begin{center}
\centering
\caption{各种高通量材料计算自动流程软件概况}\label{Table-Cost}
\def\temptablewidth{0.92\textwidth}
\renewcommand\arraystretch{0.8} %表格宽度控制(普通表格宽度的两倍)
\rule{\temptablewidth}{1pt}
\begin{tabular*} {\temptablewidth}{@{\extracolsep{\fill}}c@{\extracolsep{\fill}}c@{\extracolsep{\fill}}c@{\extracolsep{\fill}}c@{\extracolsep{\fill}}c@{\extracolsep{\fill}}c@{\extracolsep{\fill}}c}
%-------------------------------------------------------------------------------------------------------------------------
	&\multirow{2}{*}{\fontsize{9.2pt}{7.2pt}\selectfont{编程语言}}	&\fontsize{9.2pt}{7.2pt}\selectfont{建模} &\multicolumn{2}{|c|}{\fontsize{9.2pt}{7.2pt}\selectfont{任务提交与管理}} &\multirow{2}{*}{\fontsize{9.2pt}{7.2pt}\selectfont{后处理}} &\multirow{2}{*}{\fontsize{9.2pt}{7.2pt}\selectfont{数据组织管理}} \\\cline{4-5}
	&	&\fontsize{9.2pt}{7.2pt}\selectfont{功能} &\multicolumn{1}{|c|}{\fontsize{9.2pt}{7.2pt}\selectfont{~~软件接口~~}} &\multicolumn{1}{c|}{\fontsize{9.2pt}{7.2pt}\selectfont{运行容错~~~}} & & \\\hline
	\fontsize{9.2pt}{7.2pt}\selectfont{{\textrm{AFLOW}}} &\fontsize{9.2pt}{7.2pt}\selectfont{\textrm{C++}} &\checkmark &$\triangle$ &$\star$ &$\star$ &\fontsize{9.2pt}{7.2pt}\selectfont{{\textrm{Django}}} \\
	\fontsize{9.2pt}{7.2pt}\selectfont{{\textrm{MP}}} &\fontsize{9.2pt}{7.2pt}\selectfont{\textrm{Python}} &\checkmark &\checkmark &$\star$ &$\star$ &\fontsize{9.2pt}{7.2pt}\selectfont{{\textrm{MongoDB}}} \\
	\multirow{2}{*}{\fontsize{9.2pt}{7.2pt}\selectfont{{\textrm{QMIP}}}} &\fontsize{9.2pt}{7.2pt}\selectfont{\textrm{JavaScript/SVG}} &\multirow{2}{*}{\checkmark} &\multirow{2}{*}{\checkmark} &\multirow{2}{*}{--} &\multirow{2}{*}{\checkmark} &\multirow{2}{*}{--} \\
	&\fontsize{9.2pt}{7.2pt}\selectfont{\textrm{+html/Python}} & & & & & \\
	\fontsize{9.2pt}{7.2pt}\selectfont{{\textrm{CEP}}} &\fontsize{9.2pt}{7.2pt}\selectfont{\textrm{Python}} &\checkmark &\checkmark &-- &\checkmark &\fontsize{9.2pt}{7.2pt}\selectfont{{\textrm{Django/MySQL}}} \\
	\fontsize{9.2pt}{7.2pt}\selectfont{{\textrm{ASE}}} &\fontsize{9.2pt}{7.2pt}\selectfont{\textrm{Python}} &$\star$ &$\star$ &-- &$\triangle$ &-- \\
	\multirow{2}{*}{\fontsize{9.2pt}{7.2pt}\selectfont{{\textrm{MatCloud}}}} &\fontsize{9.2pt}{7.2pt}\selectfont{\textrm{JavaScript}} &\multirow{2}{*}{\checkmark} &\multirow{2}{*}{$\triangle$} &\multirow{2}{*}{\checkmark} &\multirow{2}{*}{\checkmark} &\multirow{2}{*}{\fontsize{9.2pt}{7.2pt}\selectfont{{\textrm{MongoDB}}}} \\
	&\fontsize{9.2pt}{7.2pt}\selectfont{\textrm{+.NETCore}} & & & & &
\end{tabular*}
\rule{\temptablewidth}{1pt}
\fontsize{8.2pt}{5.2pt}\selectfont{
%\begin{description}
%	\item[$\star$]~表示该功能较突出
%	\item[\checkmark]~表示该功能基本满足需求
%	\item[$\triangle$]~表示该功能存在不足
%\end{description}
	\begin{itemize}
		\item $\star$~表示该功能较突出;~\checkmark~表示该功能基本满足需求;~$\triangle$~表示该功能存在不足
	\end{itemize}}
\end{minipage}
%\end{center}
\end{table}

对比上述各类流程软件,当前主要支持电子结构第一原理计算和分子动力学计算过程,电子结构计算求解\textrm{Kohn-Sham}方程,主要以矩阵对角化和密度迭代优化为主,这类计算会涉及大量的矩阵-向量乘和矩阵变换,特别需要强大的并行计算支持;~分子动力学计算的任务是模拟大量遵守\textrm{Newton}方程的粒子的运动轨迹,适合采用并发式计算,也正因此\textrm{GPU}支持的异质架构越来越成为处理分子动力学问题的有力武器。计算流程的主要任务包括:
\begin{itemize}
	\item 指定所需计算资源
	\item 为核心计算过程准备输入文件
	\item 传递计算中间数据
	\item 提取和处理计算结果
\end{itemize}
在算法层面上,将上述任务以适当的方式排列组合,就构成了完整的计算流程。在上述流程软件中,实现计算流程主要有两种思路:~一种是将分解的计算任务用程序语言(\textrm{C++}、\textrm{Python}或\textrm{JavaScript})直接写成代码,依次执行。这样实现的计算流程简洁、直观,执行高效,但是一般不能支持复杂的计算流程。\textrm{AFLOW}、\textrm{QMIP}、\textrm{CEP}和\textrm{ASE}等都是采用这一方案,这些流程软件主要支持的是第一原理计算中“弛豫-基态物性”计算模式或分子动力学计算。另一种是利用数据库技术,将计算流程中每一步执行所需的数据或输入文件、核心计算软件和计算资源都以数据库中的元素形式存放,这样做的好处是可以灵活地设计、组织复杂的计算流程,包括完整的跨尺度计算,而且通过数据库保存计算流程,实现流程与核心计算软件的剥离,计算流程成为半独立的通用模块,大大方便了流程开发,这对新材料设计有着很重要的意义,但是也因为计算流程本身需要数据库的支持,流程的执行效率将有所降低,而运行过程中一旦出现错误,对错误出现位置的确定也会更艰难,目前只有\textrm{MP}采用了这种数据库支持的计算流程。不过从长远看,两类数据库支持的计算流程设计思想各有所长。在可以预见的未来,这两类计算流程在新材料研究和设计领域,有着各自深远的应用场景。
%都无法给出完善的解决方案,因此有必要在充分调研计算流程软件研发状况的基础上,针对催化燃烧计算任务的特点,开发面向适合$\mathrm{CH}_4$催化材料研究的高通量计算流程软件。

%\section{批量作业生成、并发式自动流程计算和工作流程设计}\label{chap:workflow} 
\subsection{高通量第一原理计算数据的自动处理}\label{chap:database} 
第一原理计算,特别是基于高通量的\textrm{DFT}计算已经成为电子结构和原子尺度材料设计最重要的方法。由高性能计算支持的第一原理计算产生了成千上万的材料数据记录。随着计算材料数据指数级别的增长,有必要开发功能强大的材料数据库来支持数据的管理、存储、检索、展示和操作。本节讨论材料设计所需的最主要的第一原理数据库并比较其优劣。\footnote{本节的数据库部分主要参考了文献\upcite{MPC4-148_2015}}

%\subsection{高通量第一原理自动流程}
高通量的概念最初出现在实验领域,早期材料研究和制药研究,一般通过大量备选材料试错,最终才得到合适的材料或药物主要功能成分,这就可以视为是一种高通量筛选。需要明确的是,在文献中常会提到“高通量”和“组合方法”(\textrm{Combinational approach})的概念,但很少区分着两者的区别,这里明确一下,所谓“高通量”是用户产生或处理的数据量极大,没有计算机自动处理是无法完成的;~而“组合方法”是针对影响研究对象的各种可能自由度的分门别类研究。换言之,高通量考虑的是利用计算机“一视同仁”地自动化式处理海量数据,而组合方法更强调对特定影响因素的筛查和组合研究,这是这两个概念内涵的主要区别。

高通量计算流程主要包括三方面的任务:
\begin{itemize}
	\item 增加材料模拟的计算数据:~包括采用热力学、电子结构计算获得的材料数据;
	\item 存储合理的材料数据:~将材料数据系统地保存起来,用以构建材料数据库;
	\item 表征和筛选材料数据:~为了获得更好的材料或提升材料的特定性能,可实现对材料数据的检索和分析。
\end{itemize}
第一原理高通量计算流程就是具体实现上述三个流程开发的计算任务,高通量计算流程的主要目标之一是构建材料数据库(可以是一般的通用材料数据库或特定目标的材料数据库),所以高通量计算软件与相应的数据库有着密不可分的关联。如前所述,著名的高通量计算流程都有相应的数据库,如\textrm{AFLOW}的数据库为\textrm{AFLOWLIB}\upcite{CMS58-227_2012,AFLOWORG_URL},\textrm{MP}的数据库为同名的\textrm{Materials Project}\upcite{CMS50-2295_2011,MP_URL},\textrm{ASE}的数据库为\textrm{CMR~(The Computational Materials Repository)}\upcite{CSE14-51_2012,CMR_URL}。中科院物理所也拥有一个通用的材料科学数据库\textrm{Atomly}\upcite{ATOMLY_URL}。

虽然高通量计算流程包括三个方面,但是由于第一原理材料计算的成本较高,构建一个相对完善的材料数据库需要耗费相当的计算资源和人力。因此当前发展的各种自动处理流程的重点主要集中在收集材料模拟的计算数据和实现计算结果自动入库为主。自动流程面向的对象是数据,鉴于能完成材料第一原理计算的软件有很多,相应的输出文件的格式也是千差万别,因此要实现材料计算过程的自动处理,首先需要解决的是数据格式的规范化问题。习惯的数据处理方案有两种:~一种是面向各类第一原理软件数据的规范化,通过程序语言实现材料计算过程的自动化,不妨称为数据标准化型自动流程;~一种是面向典型的材料计算软件,开发数据库支持的材料自动化计算流程,不妨称为流程标准化型自动流程。这两种策略各有利弊:~前者可以利用数据的规范化,根据计算软件的特色组织丰富多样的材料物性计算,灵活性较好,但一般只支持相对简单的计算流程;~后者可利用数据库技术将自动流程组织得更复杂多样,并作为数据库条目存储下来,稳定性较好,但计算的材料物性受软件能力的限制较多。因为\textrm{Python}\upcite{Python_URL}的模块化组织和灵活性,主要的自动流程都采用\textrm{Python}来实现。

数据规范化型的自动处理以\textrm{ASE/CMR}为典型代表,通过构建各类\textrm{Python}模块,支持多种成熟的\textrm{DFT}计算软件的\textrm{Kohn-Sham}方程求解过程。根据图\ref{Auto_Flow_Platform-5}可知,\textrm{ASE}的核心模块主要包括\textrm{Atoms}(原子、分子建模)、\textrm{Calculator}(各类计算软件运行支持与控制)和物性计算、功能分析和结果可视化模块。

\textrm{Atoms}是材料计算建模的主要模块,功能包括:
\begin{itemize}
	\item 生成各种计算软件所需的结构模型,包括原子、分子、晶体、表面和界面等;
	\item 读入各种格式的结构模型文件,包括\texttt{xyz}、\textrm{POSCAR}、\texttt{cif}等65种格式;
	\item 可将各类结构模型统一以\texttt{traj}或\texttt{json}格式写入数据库,实现计算模型数据的标准化。
\end{itemize}

\textrm{Calculator}是支持各类\textrm{DFT}计算软件运行的主要模块,封装了\textrm{DFT}计算的过程,依次执行以下步骤:
\begin{enumerate}
	\item 生成\textrm{DFT}计算软件所需的输入(控制)文件;
	\item 启动\textrm{DFT}软件,以子进程方式开始计算过程;
	\item 模块守候直至\textrm{DFT}计算子进程结束;
	\item 根据要求解析\textrm{DFT}计算软件生成文件,并可将计算结果以\texttt{json}格式写入数据库。
\end{enumerate}
类似地,物性计算、功能分析模块通过集成全局结构搜索算法(\textrm{Base-hopping}和\textrm{minima-hopping}算法)、反应动力学模拟\textrm{NEB}算法和势能面鞍点搜索算法、分子动力学模拟算法、几何结构优化算法和分子振动与声子振动分析算法,可完成材料物性的自动化计算,得到的材料物性数据仍将以标准化形式存入数据库。

这样设计的自动处理结构最大的好处是,\textrm{Python}模块与\textrm{DFT}计算软件的交互简单,无须改变\textrm{DFT}计算的任何代码;~只是\textrm{Python}执行过程中会有较多的\textrm{I/O}处理,脚本运行效率不高。

为了增加自动流程的可控和灵活性,\textrm{ASE}引入\textrm{checkpointing}模块,可以协助用户排查计算中的错误定位和重启计算。\textrm{checkpointing}大大增加了\textrm{ASE}对计算流程的控制能力。

和\textrm{ASE}相应的材料数据库\textrm{CMR}采用关系型数据库管理系统\textrm{MySQL},延续了\textrm{ASE}的思路,先将标准化数据转成数据库文件(称为\textit{cmr}-文件),要求数据库中的文件名尽可能与原始文件名保持一致,由此用户可以不进入数据库即可对材料数据进行检验。\textrm{CMR}在存储材料数据时也有了更大的兼容性。

流程标准化型的自动处理以\textrm{MP}的流程控制\textrm{FireWorks}为代表。\textrm{FireWorks}是一款开源的通用工作流定义、管理和执行软件,可以支持\textrm{Python}实现与执行,复杂的工作流可以数据形式保存到\textrm{MongoDB}数据库中,用\textrm{FireWorks}设计的工作流具有较高的稳定性。\textrm{FireWorks}的架构如图\ref{FireWorks_FW}左所示,包括流程发布(称为\textrm{LachPad})和流程执行(称为\textrm{FireWorkers}),换言之\textrm{FireWorks}的自动流程是中心化的“发布-执行”模式。\textrm{LaunchPad}是工作流的主管者,主要负责自动流程的定义、分发、排队、增删和对工作流的反馈与响应;~\textrm{FireWorkers}是工作流的执行者,包括一个或多个计算资源(个人计算机、小型工作站、超级计算机等)。\textrm{FireWorkers}从\textrm{LaunchPad}处获得计算任务,执行完毕后再将计算结果返回到\textrm{LaunchPad}。

\textrm{FireWorks}的这种“发布-执行”结构使得计算任务与软件、硬件高度解耦,用户可根据需要随时向\textrm{LaunchPad}添加新的工作流,承担计算任务的\textrm{FireWorkers}彼此也可以是完全异构的,具有很好的机动性。
\begin{figure}[h!]
\centering
\vspace*{-0.1in}
\includegraphics[height=1.6in]{MP_fireworks.png}
\hskip 5pt
\includegraphics[height=1.6in]{MP_multiple_fw.png}
\hskip 5pt
\includegraphics[height=1.6in]{MP_Fireworks_fwactions.png}
\caption{\textrm{FireWorks}的架构(左)、工作流组成(中)和单元组间数据传递与处理(右)示意.图片引自文献\upcite{FireWorks_URL}.}%
\label{FireWorks_FW}
\end{figure} 
\textrm{FireWorks}发布的工作流成如图\ref{FireWorks_FW}中所示,由三层嵌套结构组成:
\begin{itemize}
	\item \textrm{Firetask}:~基本执行单元,是执行计算的最基本脚本命令或\textrm{Python}命令。
	\item \textrm{Firework}:~组织基本执行单元构成任务单元组,并指定各基本执行单元所需的参数。
	\item \textrm{Workflow}:~彼此相关联的任务单元组构成完整的工作流程:\\
		\textrm{FireWork}之间的数据传递、任务执行序列等由\textrm{FWAction}完成。
\end{itemize}
对于材料第一原理计算自动流程而言,一个\textrm{DFT}计算过程就是一个\textrm{Firework},可以分解为:
\begin{enumerate}
	\item 指定计算控制参数(参数在数据库中\texttt{Json}存储,由\textrm{Spec}传入)
	\item 计算控制文件生成(每个\textrm{Firetask}生成一个控制文件)
	\item \textrm{DFT}计算作业提交(一个\textrm{Firetask})
\end{enumerate}
在此基础上,可以通过\textrm{FWAction}修改控制参数,将\textrm{DFT}计算单元组组织成完整的材料第一原理计算流程,并将最终结果直接导入材料计算数据库。

可见,\textrm{FireWorks}是以任务单元组为基本组成的来实现工作流程的,任务单元组之间依靠数据传递相关联,流程执行完毕也将返回数据,因此\textrm{FWAction}模块主要负责任务单元组之间的数据传递和任务分配。图\ref{FireWorks_FW}右示意了工作流中\textrm{FWAction}的工作模式。不难看出,\textrm{FWAction}允许用户根据需要设计和更改流程参数、增添、删减和改变流程(子)单元组,这一模块大大增加了\textrm{FireWorks}工作流的灵活性。

数据库技术支持的\textrm{FireWorks}解耦了\textrm{DFT}计算软件与计算流程,允许用户根据需要设计出稳定、复杂的材料物性模拟与设计流程。由于不同\textrm{DFT}计算软件的计算控制文件格式存在较大的差别,目前\textrm{FireWorks}流程设计只对特定计算软件(如\textrm{VASP})设计了最常用的计算流程模块(如结构弛豫、基态总能计算、能带和态密度计算等)。更丰富的计算流程有待应用研究中不断开发。



%------------%%%%%%%%%%%%%%%%%%%%%%%%%%%%%%%%%%%%%------------%

%---%%%%%%%%%%%-----The Figure Of The Paper-----%%%%%%%%%----
%\begin{figure}[h!]
%\centering
%\includegraphics[height=3.35in,width=2.85in,viewport=0 0 400 475,clip]{PbTe_Band_SO.eps}
%\hspace{0.5in}
%\includegraphics[height=3.35in,width=2.85in,viewport=0 0 400 475,clip]{EuTe_Band_SO.eps}
%\caption{\small Band Structure of PbTe (a) and EuTe (b).}%(与文献\cite{EPJB33-47_2003}图1对比)
%\label{Pb:EuTe-Band_struct}
%\end{figure}
%------------%%%%%%%%%%%%%%%%%%%%%%%%%%%%%%%%%%%%%------------%

%--%%%%%%%%%%%------The Equation Of The Paper-----%%%%%%%%%---%
%\begin{equation}
%\varepsilon_1(\omega)=1+\frac2{\pi}\mathscr P\int_0^{+\infty}\frac{\omega'\varepsilon_2(\omega')}{\omega'^2-\omega^2}d\omega'
%\label{eq:magno-1}
%\end{equation}

%\begin{equation} 
%\begin{split}
%\varepsilon_2(\omega)&=\frac{e^2}{2\pi m^2\omega^2}\sum_{c,v}\int_{BZ}d{\vec k}\left|\vec e\cdot\vec M_{cv}(\vec k)\right|^2\delta [E_{cv}(\vec k)-\hbar\omega] \\
% &= \frac{e^2}{2\pi m^2\omega^2}\sum_{c,v}\int_{E_{cv}(\vec k=\hbar\omega)}\left|\vec e\cdot\vec M_{cv}(\vec k)\right|^2\dfrac{dS}{\nabla_{\vec k}E_{cv}(\vec k)}
% \end{split}
%\label{eq:magno-2}
%\end{equation}
%------------%%%%%%%%%%%%%%%%%%%%%%%%%%%%%%%%%%%%%------------%

%---%%%%%%%%%%%-----The Table Of The Paper----%%%%%%%%%%%%%---%
%\begin{table}[!h]
%\tabcolsep 0pt \vspace*{-12pt}
%%\caption{The representative $\vec k$ points contributing to $\sigma_2^{xy}$ of interband transition in EuTe around 2.5 eV.}
%\label{Table-EuTe_Sigma}
%\begin{minipage}{\textwidth}
%%\begin{center}
%\centering
%\def\temptablewidth{0.84\textwidth}
%\rule{\temptablewidth}{1pt}
%\begin{tabular*} {\temptablewidth}{|@{\extracolsep{\fill}}c|@{\extracolsep{\fill}}c|@{\extracolsep{\fill}}l|}

%-------------------------------------------------------------------------------------------------------------------------
%&Peak (eV)  & {$\vec k$}-point            &Band{$_v$} to Band{$_c$}  &Transition Orbital
%Components\footnote{波函数主要成分后的括号中,$5s$、$5p$和$5p$、$4f$、$5d$分别指碲和铕的原子轨道。} &Gap (eV)   \\ \hline
%-------------------------------------------------------------------------------------------------------------------------
%&2.35       &(0,0,0)         &33$\rightarrow$34    &$4f$(31.58)$5p$(38.69)$\rightarrow$$5p$      &2.142   \\% \cline{3-7}
%&       &(0,0,0)         &33$\rightarrow$34    &$4f$(31.58)$5p$(38.69)$\rightarrow$$5p$      &2.142   \\% \cline{3-7}
%-------------------------------------------------------------------------------------------------------------------------
%\end{tabular*}
%\rule{\temptablewidth}{1pt}
%\end{minipage}{\textwidth}
%\end{table}
%------------%%%%%%%%%%%%%%%%%%%%%%%%%%%%%%%%%%%%%------------%

%---%%%%%%%%%%%-----The Long Table Of The Paper---%%%%%%%%%%%%----%
%\begin{small}
%%\begin{minipage}{\textwidth}
%%\begin{longtable}[l]{|c|c|cc|c|c|} %[c]指定长表格对齐方式
%\begin{longtable}[c]{|c|c|p{1.9cm}p{4.6cm}|c|c|}
%\caption{Assignment for the peaks of EuB$_6$}
%\label{tab:EuB6-1}\\ %\\长表格的caption中换行不可少
%\hline
%%
%--------------------------------------------------------------------------------------------------------------------------------
%\multicolumn{2}{|c|}{\bfseries$\sigma_1(\omega)$谱峰}&\multicolumn{4}{c|}{\bfseries部分重要能带间电子跃迁\footnotemark}\\ \hline
%\endfirsthead
%--------------------------------------------------------------------------------------------------------------------------------
%%
%\multicolumn{6}{r}{\it 续表}\\
%\hline
%--------------------------------------------------------------------------------------------------------------------------------
%标记 &峰位(eV) &\multicolumn{2}{c|}{有关电子跃迁} &gap(eV)  &\multicolumn{1}{c|}{经验指认} \\ \hline
%\endhead
%--------------------------------------------------------------------------------------------------------------------------------
%%
%\multicolumn{6}{r}{\it 续下页}\\
%\endfoot
%\hline
%--------------------------------------------------------------------------------------------------------------------------------
%%
%%\hlinewd{0.5$p$t}
%\endlastfoot
%--------------------------------------------------------------------------------------------------------------------------------
%%
%% Stuff from here to \endlastfoot goes at bottom of last page.
%%
%--------------------------------------------------------------------------------------------------------------------------------
%标记 &峰位(eV)\footnotetext{见正文说明。} &\multicolumn{2}{c|}{有关电子跃迁\footnotemark} &gap(eV) &\multicolumn{1}{c|}{经验指认\upcite{PRB46-12196_1992}}\\ \hline
%--------------------------------------------------------------------------------------------------------------------------------
%
%     &0.07 &\multicolumn{2}{c|}{电子群体激发$\uparrow$} &--- &电子群\\ \cline{2-5}
%\raisebox{2.3ex}[0pt]{$\omega_f$} &0.1 &\multicolumn{2}{c|}{电子群体激发$\downarrow$} &--- &体激发\\ \hline
%--------------------------------------------------------------------------------------------------------------------------------
%
%     &1.50 &\raisebox{-2ex}[0pt][0pt]{20-22(0,1,4)} &2$p$(10.4)4$f$(74.9)$\rightarrow$ &\raisebox{-2ex}[0pt][0pt]{1.47} &\\%\cline{3-5}
%     &1.50$^\ast$ & &2$p$(17.5)5$d_{\mathrm E}$(14.0)$\uparrow$ & &4$f$$\rightarrow$5$d_{\mathrm E}$\\ \cline{3-5}
%     \raisebox{2.3ex}[0pt][0pt]{$a$} &(1.0$^\dagger$) &\raisebox{-2ex}[0pt][0pt]{20-22(1,2,6)} &\raisebox{-2ex}[0pt][0pt]{4$f$(89.9)$\rightarrow$2$p$(18.7)5$d_{\mathrm E}$(13.9)$\uparrow$}\footnotetext{波函数主要成分后的括号中,2$s$、2$p$和5$p$、4$f$、5$d$、6$s$分别指硼和铕的原子轨道;5$d_{\mathrm E}$、5$d_{\mathrm T}$分别指铕的(5$d_{z^2}$,5$d_{x^2-y^2}$和5$d_{xy}$,5$d_{xz}$,5$d_{yz}$)轨道,5$d_{\mathrm{ET}}$(或5$d_{\mathrm{TE}}$)则指5个5$d$轨道成分都有,成分大的用脚标的第一个字母标示;2$ps$(或2$sp$)表示同时含有硼2$s$、2$p$轨道成分,成分大的用第一个字母标示。$\uparrow$和$\downarrow$分别标示$\alpha$和$\beta$自旋电子跃迁。} &\raisebox{-2ex}[0pt][0pt]{1.56} &激子跃迁。 \\%\cline{3-5}
%     &(1.3$^\dagger$) & & & &\\ \hline
%--------------------------------------------------------------------------------------------------------------------------------

%     & &\raisebox{-2ex}[0pt][0pt]{19-22(0,0,1)} &2$p$(37.6)5$d_{\mathrm T}$(4.5)4$f$(6.7)$\rightarrow$ & & \\\nopagebreak %\cline{3-5}
%     & & &2$p$(24.2)5$d_{\mathrm E}$(10.8)4$f$(5.1)$\uparrow$ &\raisebox{2ex}[0pt][0pt]{2.78} &a、b、c峰可能 \\ \cline{3-5}
%     & &\raisebox{-2ex}[0pt][0pt]{20-29(0,1,1)} &2$p$(35.7)5$d_{\mathrm T}$(4.8)4$f$(10.0)$\rightarrow$ & &包含有复杂的\\ \nopagebreak%\cline{3-5}
%     &2.90 & &2$p$(23.2)5$d_{\mathrm E}$(13.2)4$f$(3.8)$\uparrow$ &\raisebox{2ex}[0pt][0pt]{2.92} &强激子峰。$^{\ast\ast}$\\ \cline{3-5}
%$b$  &2.90$^\ast$ &\raisebox{-2ex}[0pt][0pt]{19-22(0,1,1)} &2$p$(33.9)4$f$(15.5)$\rightarrow$ & &B2$s$-2$p$的价带 \\ \nopagebreak%\cline{3-5}
%     &3.0 & &2$p$(23.2)5$d_{\mathrm E}$(13.2)4$f$(4.8)$\uparrow$ &\raisebox{2ex}[0pt][0pt]{2.94} &顶$\rightarrow$B2$s$-2$p$导\\ \cline{3-5}
%     & &12-15(0,1,2) &2$p$(39.3)$\rightarrow$2$p$(25.2)5$d_{\mathrm E}$(8.6)$\downarrow$ &2.83 &带底跃迁。\\ \cline{3-5}
%     & &14-15(1,1,1) &2$p$(42.5)$\rightarrow$2$p$(29.1)5$d_{\mathrm E}$(7.0)$\downarrow$ &2.96 & \\\cline{3-5}
%     & &13-15(0,1,1) &2$p$(40.4)$\rightarrow$2$p$(28.9)5$d_{\mathrm E}$(6.6)$\downarrow$ &2.98 & \\ \hline
%--------------------------------------------------------------------------------------------------------------------------------
%%\hline
%%\hlinewd{0.5$p$t}
%\end{longtable}
%%\end{minipage}{\textwidth}
%%\setlength{\unitlength}{1cm}
%%\begin{picture}(0.5,2.0)
%%  \put(-0.02,1.93){$^{1)}$}
%%  \put(-0.02,1.43){$^{2)}$}
%%\put(0.25,1.0){\parbox[h]{14.2cm}{\small{\\}}
%%\put(-0.25,2.3){\line(1,0){15}}
%%\end{picture}
%\end{small}
%-------------%%%%%%%%%%%%%%%%%%%%%%%%%%%%%%%%%%%%%-------------%

%-----%%%%%%%%%%%%%%%%%%%%%%%%%%%%%%%%%%%%%%%%%%%%-----直-接-插-入-文-件-----%%%%%%%%%%%%%%%%%%%%%%%%%%%%%%%%%%%%%%%%%%%%%%%%%%%%------%
%\textcolor{red}{\textbf{直接插入文件}}:\verbatiminput{/home/jun_jiang/Documents/Latex_art_beamer/Daily_WORKS/Report-2020_model.tex} %为保险:~选用文件名绝对路径
%\textcolor{red}{\textbf{备忘录}}:\verbatiminput{/home/jun_jiang/Documents/备忘录.txt}
%--------------------------------------%%%%%%%%%%%%%%%%%%%%%%%%%%%%%%%%%%%%%%%%%%%%%%%%%%%%%%%%%%%%%%----------------------------------%

%-----------------------------------------------The Bibliography of The Papart------------------------------------%
%%%%%% 没有 \chapterbib 时仍然可以每个章节都出现参考文献~(\thebibliography 模式)~,但意义不大

%\phantomsection\addcontentsline{toc}{subsection}{Bibliography} %直接调用\addcontentsline命令可能导致超链指向不准确,一般需要在之前调用一次\phantomsection命令加以修正%
%\begin{thebibliography}{99}								      %
%%----%%%%%%%%%%%%%%%%%%%%%--------The Biblography of The Paper--------%%%%%%%%%%%%%%%%%%%-----%
%\newpage										      %
\phantomsection\addcontentsline{toc}{section}{Bibliography} %直接调用\addcontentsline命令可能导致超链指向不准确,一般需要在之前调用一次\phantomsection命令加以修正%
%\phantomsection\addcontentsline{toc}{subsection}{\CJKfamily{hei} 主要参考资料}               % 

%---------------------------------------------------------------------------------------------%
%\begin{thebibliography}{99}								      %
%											      %
%											      %
%--------%%%%%%%%%%%%%%%%%%%%%%%%%%%%%%%%%%%%%%%%%%%%%--------%
%%\bibitem{PRL58-65_1987}H.Feil, C. Haas, {\it Phys. Rev. Lett.} {\bf 58}, 65 (1987).         %
%\bibitem{kp-method} \textrm{Zhenxi Pan, Yong Pan, Jun Jiang$^{\ast}$, Liutao Zhao}, \textrm{High-Throughput Electronic Band Structure Calculations for Hexaborides}, \textit{Intelligent Computing}, \textbf{Springer}, \textbf{P.386-395}, (2019).              %
%\bibitem{QCQC_2014} \textrm{姜骏},\textrm{PAW原子数据集的构造与检验}, \textit{中国化学会第十二届全国量子化学会议论文摘要集},\textbf{太原},(2014).
%											      %
%\end{thebibliography}								              %

%
\bibliography{../../ref/Myref}                         %
%%\bibliography{../ref/Myref_from_2013}             %
\bibliographystyle{../../ref/mybib}                   %%   接近ieeert样式

%%%%%%%%%%%%%%%%%%%%%%%%%%%%      \bibliographystyle         %%%%%%%%%%%%%%%%%%%%%%%%%%%%%%%%%%
%%%%%%      LaTeX 参考文献标准选项及其样式共有以下8种:                                %%%%%%%%
% plain,按字母的顺序排列,比较次序为作者、年度和标题.                                        %
% unsrt,样式同plain,只是按照引用的先后排序.                                                 %
% alpha,用作者名首字母+年份后两位作标号,以字母顺序排序.                                     %
% abbrv,类似plain,将月份全拼改为缩写,更显紧凑.                                             %
% ieeetr,国际电气电子工程师协会期刊样式.                                                     %
% acm,美国计算机学会期刊样式.                                                                %
% siam,美国工业和应用数学学会期刊样式.                                                       %
% apalike,美国心理学学会期刊样式.                                                            %
%%%%%%%%%%%%%%%%%%%%%%%%%%%%%%%%%%%%%%%%%%%%%%%%%%%%%%%%%%%%%%%%%%%%%%%%%%%%%%%%%%%%%%%%%%%%%%%

%\nocite{*}                                                                                   %
%---------------------------------------------------------------------------------------------%
%
%---%%%%%%%%%%%%%%%%%%%%%-------%%%%%%%%%%%%%%%%%%%%%%%%%%%%%--------%%%%%%%%%%%%%%%%%%%------%

                                                      %
%\end{thebibliography}								              %
%%%%%%%%%%%%%%%%%%%%%%%%%%%%%%%%%%%%%%%%%%%%%%%

%---------------------------------------------------------------------------------------------%

%%----%%%%%%%%%%%%%%%%%%%%%--------The Biblography of The Paper--------%%%%%%%%%%%%%%%%%%%-----%
%\newpage										      %
\phantomsection\addcontentsline{toc}{section}{Bibliography} %直接调用\addcontentsline命令可能导致超链指向不准确,一般需要在之前调用一次\phantomsection命令加以修正%
%\phantomsection\addcontentsline{toc}{subsection}{\CJKfamily{hei} 主要参考资料}               % 

%---------------------------------------------------------------------------------------------%
%\begin{thebibliography}{99}								      %
%											      %
%											      %
%--------%%%%%%%%%%%%%%%%%%%%%%%%%%%%%%%%%%%%%%%%%%%%%--------%
%%\bibitem{PRL58-65_1987}H.Feil, C. Haas, {\it Phys. Rev. Lett.} {\bf 58}, 65 (1987).         %
%\bibitem{kp-method} \textrm{Zhenxi Pan, Yong Pan, Jun Jiang$^{\ast}$, Liutao Zhao}, \textrm{High-Throughput Electronic Band Structure Calculations for Hexaborides}, \textit{Intelligent Computing}, \textbf{Springer}, \textbf{P.386-395}, (2019).              %
%\bibitem{QCQC_2014} \textrm{姜骏},\textrm{PAW原子数据集的构造与检验}, \textit{中国化学会第十二届全国量子化学会议论文摘要集},\textbf{太原},(2014).
%											      %
%\end{thebibliography}								              %

%
\bibliography{../../ref/Myref}                         %
%%\bibliography{../ref/Myref_from_2013}             %
\bibliographystyle{../../ref/mybib}                   %%   接近ieeert样式

%%%%%%%%%%%%%%%%%%%%%%%%%%%%      \bibliographystyle         %%%%%%%%%%%%%%%%%%%%%%%%%%%%%%%%%%
%%%%%%      LaTeX 参考文献标准选项及其样式共有以下8种:                                %%%%%%%%
% plain,按字母的顺序排列,比较次序为作者、年度和标题.                                        %
% unsrt,样式同plain,只是按照引用的先后排序.                                                 %
% alpha,用作者名首字母+年份后两位作标号,以字母顺序排序.                                     %
% abbrv,类似plain,将月份全拼改为缩写,更显紧凑.                                             %
% ieeetr,国际电气电子工程师协会期刊样式.                                                     %
% acm,美国计算机学会期刊样式.                                                                %
% siam,美国工业和应用数学学会期刊样式.                                                       %
% apalike,美国心理学学会期刊样式.                                                            %
%%%%%%%%%%%%%%%%%%%%%%%%%%%%%%%%%%%%%%%%%%%%%%%%%%%%%%%%%%%%%%%%%%%%%%%%%%%%%%%%%%%%%%%%%%%%%%%

%\nocite{*}                                                                                   %
%---------------------------------------------------------------------------------------------%
%
%---%%%%%%%%%%%%%%%%%%%%%-------%%%%%%%%%%%%%%%%%%%%%%%%%%%%%--------%%%%%%%%%%%%%%%%%%%------%


%%%%%%%%%%%%%%%%%%%%%%%%%%%%%%%%%%%%%%%%%%%%%%%
%----------------------------------------------------------------------------------------------------------------------------------------------------%

%%%%%%%%%%%%%%%%%%%%%%%%%%%%%%%%%%%%%%%%%%%%%%%%%%%%%%%%%%%%%%%%%%%%%%%%%%%%%%%%%%%%%%%%%%%%%%%%%%%%%%%%%%%%%%%%%%%%%%%%%%%%%%%%%%%%%%%%%%%%%%%
