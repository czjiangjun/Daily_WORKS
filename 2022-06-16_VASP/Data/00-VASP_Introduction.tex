\section{\rm{VASP}软件的特点}
\textrm{VASP}软件\upcite{VASP_manual}是维也纳大学(Universit\"at Wien)\textrm{G. Kresse}等开发的第一原理模拟软件包。\textrm{VASP}采用的\textrm{PAW~(Projector Augmented-Wave)}方法\upcite{PRB50-17953_1994,PRB59-1758_1999},平衡了传统赝势方法和全电子计算优点,兼顾了计算的精度和效率。特别是实空间优化的投影函数\textrm{(Projector)}的思想,将主要的计算任务变换到实空间完成,大大节省了基组的维度,保证了计算精度和效率。在此基础上,\textrm{VASP}通过引入多样的优化算法,提高了矩阵对角化和电荷密度搜索的效率;~在程序并行计算中,通过\textrm{FFT~(Fast Fourier Transformation)}计算网格与并行计算的计算格点的分配平衡,提升了软件的并行效率。相比于其他第一原理计算软件,\textrm{VASP}从物理思想与方法、优化算法和并行计算实现等多个方面都有更为出色的性能。有关\textrm{VASP}软件实现的文献,主要如下:~
\begin{enumerate}
	\item 物理思想与方法:~\textrm{PAW}赝势方法:~参阅文献\cite{PRB50-17953_1994,PRB59-1758_1999};~投影函数的实空间优化,参阅\cite{JPCM6-8245_1994,PRB44-13063_1991,PRB44-8503_1991}
	\item 优化算法:~参阅文献\cite{CMS6-15_1996,PRB54-11169_1996}
	\item 高效的并行计算:~ \textrm{FFT}网格与计算格点的分配:~参阅\textrm{VASP}源代码的\textbf{mgrid.F}
\end{enumerate}

\subsection{\rm{PAW}方法与实空间投影}
\textrm{PAW}方法是\textrm{Bl\"ochl}于1994年独立提出来的一种计算方法\upcite{PRB50-17953_1994},该方法同时结合了赝势方法和APW方法的优点,达到平衡计算效率和精度的目的。\textrm{PAW}方法刚提出来的时候并未引起注意,直到1999年\textrm{Kresse}讨论了\textrm{PAW}方法和超软赝势\textrm{(Ultra-Soft Pseudo-Potential, USPP)}方法的密切关联,指出\textrm{USPP}方法的计算程序简单改造就能引入\textrm{PAW}方法,才推动了\textrm{PAW}方法的广泛应用。\upcite{PRB59-1758_1999}现在\textrm{PAW}方法已经成为最主要的可支持第一原理分子动力学\textrm{(Ab initio Molecular Dynamics, AIMD)}的计算方法。

\subsubsection{PAW方法的基本思想}
与一般赝势方法不同,\textrm{PAW}方法的目标是全电子\textrm{(all-electron)}波函数\footnote{注意,“全电子”在\textrm{Bl\"ochl}原始文献\cite{PRB50-17953_1994}中与“真实电子波函数”意义相近,强调电子在原子核附近的振荡行为,但并未严格区分价电子与芯电子;~而在\textrm{Kresse}的文献\cite{PRB59-1758_1999}中则是明确指价电子波函数。%强调价电子波函数因与芯层电子正交而在原子核附近振荡;
此外,与“全电子”概念密切关联的是“全势(full-potential)”,两者在具体语境中有一定的区别,“全势”强调的是重现价电子感受到的势函数的效果。},体系中全部电子构成\textrm{Hilbert}空间,价电子与芯层态彼此正交,使得波函数在\textrm{Muffin-tin}球内振荡。
\textrm{Bl\"ochl}假设全电子波函数$|\Psi\rangle$与赝波函数$|\tilde\Psi\rangle$满足线性变换,即满足:
\begin{equation}
	|\Psi\rangle=\mathbf{\tau|}\tilde\Psi\rangle
	\label{eq:PAW-Blochl-01}
\end{equation}
%	$$\tau=\mathbf{1}+\sum_{\mathrm R}\hat\tau_{\mathrm R}$$
在原子核附近的$r_c$范围内\footnote{习惯上这个区域称为缀加区(\textrm{Augmentation region}).},除了平面波,还引入原子分波函数展开来表示波函数:
\begin{equation}
	|\Psi\rangle=|\tilde\Psi\rangle+\sum_i(|\phi_i\rangle-|\tilde\phi_i\rangle)\langle\tilde p_i|\tilde\Psi\rangle
	\label{eq:PAW-Blochl-02}
\end{equation}
在$r_c$外$|\tilde\Psi\rangle$与$|\Psi\rangle$变换前后保持不变,因此线性变换$\mathbf{\tau}$可表示为:
\begin{equation}
	\mathbf{\tau}=\mathbf{1}+\sum_i(|\phi_i\rangle-|\tilde\phi_i\rangle)\langle\tilde p_i|
	\label{eq:PAW-Blochl-03}
\end{equation}
其中$|\tilde p_i\rangle$是\textrm{MT}球内的投影函数,$i$表示原子位置$\vec R$、原子轨道($l,m$)和能级$\epsilon_k$的指标。\textrm{PAW}的波函数与赝波函数的关系,可以用图\ref{PAW_basic}表示。
\begin{figure}[h!]
\centering
\includegraphics[height=2.35in,width=4.1in,viewport=0 0 1280 745,clip]{PAW-baseset.png}
%\includegraphics[height=1.8in,width=4.in,viewport=30 210 570 440,clip]{PAW_projector.eps}
\caption{\small \textrm{The analysis of PAW basic function.}}%(与文献\cite{EPJB33-47_2003}图1对比)
\label{PAW_basic}
\end{figure}

\textrm{Kresse}等注意到了\textrm{PAW}方法与\textrm{USPP}方法的密切关系,指出如果投影函数$\tilde p_i$相同,\textrm{PAW}方法和\textrm{USPP}方法计算得到的总电荷密度完全等价,只是实际计算时,\textrm{USPP}方法直接赝化补偿电荷。\footnote{严格地说,\textrm{Kresse}等提出的\textrm{PAW}方法是一种冻芯近似的全电子方法。}

为了揭示\textrm{PAW}方法与\textrm{USPP}的关联,\textrm{Kresse}将\textrm{Bl\"ochl}方案中的电荷密度分解方式由“原子核+电子”改变为“离子实+价电子”形式(冻芯近似(\textrm{frozen core approximation})):~
\begin{equation}
	\begin{aligned}
		n_T=n+n_{Zc}\equiv&\underbrace{(\tilde n+\hat n+\tilde n_{Zc})}\\
				 		&\quad\qquad\tilde n_T\\
				  &+\underbrace{(n^1+\hat n+n_{Zc})}-\underbrace{(\tilde n^1+\hat n+\tilde n_{Zc})}\\
				                  &\quad\qquad n_T^1\qquad\qquad\qquad\tilde n_T^1
	\end{aligned}
	\label{eq:PAW_Kresse_02}
\end{equation}
这里$\tilde n$、$\tilde n^1$和$n^1$仅限于描述价电子电荷密度。对于芯电荷与核电荷,引入$n_c$、$\tilde n_c$和$n_{\mathrm{Z}c}$和$\tilde n_{\mathrm{Z}c}$,其中$n_c$和$\tilde n_c$是动芯近似下的芯层电荷密度,$n_{\mathrm{Z}c}$是核电荷(点核电$n_{\mathrm Z}$)和冻芯电荷$n_c$的和
\begin{displaymath}
	n_{\mathrm{Z}c}=n_{\mathrm{Z}}+n_c
\end{displaymath}
构造的赝芯电荷$n_{\mathrm{Z}c}$满足
\begin{equation}
	\int_{\Omega_r}n_{\mathrm{Z}c}(\vec r)\mathrm{d}^3\vec r=\int_{\Omega_r}\tilde n_{\mathrm{Z}c}(\vec r)\mathrm{d}^3\vec r
	\label{eq:PAW_Kresse_01}
\end{equation}
这里积分$\int_{\Omega_r}$表示对缀加区径向积分;~对$n_{\mathrm{Z}c}$和$\tilde n_{\mathrm{Z}c}$的积分满足电中性要求,即积分区的总电荷为$-Z_{\mathrm{ion}}$。%在具体计算中,在平面波表示区,电荷密度是所有原子电荷密度的叠加,局域原子附近的缀加区,只考虑当前原子的电荷密度贡献。

\textrm{Kresse}方案中,补偿电荷$\hat n$要求满足$\tilde n^1+\hat n$与$n^1$在缀加区有相同的多极矩,即约束条件满足 
\begin{equation}
	\int_{\Omega_c}(n^1-\tilde n^1-\hat n)|\vec r-\vec R|^lY_{lm}^{\ast}(\widehat{\vec r-\vec R})\mathrm{d}\vec r=0
	\label{eq:PAW_Kresse_11}
\end{equation}
定义电荷密度差
\begin{equation}
	Q_{ij}(\vec r)=\phi_i^{\ast}(\vec r)\phi_j(\vec r)-\tilde\phi_i^{\ast}(\vec r)\tilde\phi_j(\vec r)
	\label{eq:PAW_Kresse_12}
\end{equation}
$Q_{ij}(\vec r)$对应的多极矩为
\begin{equation}
	q_{ij}^L(\vec r)=\int_{\Omega_r}Q_{ij}(\vec r)|\vec r-\vec R|^lY_{lm}^{\ast}(\widehat{\vec r-\vec R})\mathrm{d}\vec r
	\label{eq:PAW_Kresse_13}
\end{equation}
因此满足约束条件式\eqref{eq:PAW_Kresse_11}的补充电荷的计算形式为:~
\begin{equation}
	\begin{aligned}
		\hat n=\sum_{(i,j),L}\sum_n f_n\langle\tilde\Psi_n|\tilde p_i\rangle\langle\tilde p_j|\Psi_n\rangle\hat Q_{ij}^L(\vec r)\\
		\hat Q_{ij}^L(\vec r)=q_{ij}^Lg_l(|\vec r-\vec R|)Y_{lm}(\widehat{\vec r-\vec R})
	\end{aligned}
	\label{eq:PAW_Kresse_14}
\end{equation}
%与\textrm{LAPW}方法的主要区别是,式\eqref{eq:PAW_Kresse_14}中$g(r)$的具体形式,将留待在下一节“\textrm{PAW}的原子数据集”中讨论。

不难看出,\textrm{Kresse}方案通过电荷密度的分解和补偿电荷的构造,沟通了\textrm{USPP}和\textrm{Blochl}的经典\textrm{PAW}方法:
\begin{itemize}
	\item \textrm{Kresse}方案和\textrm{USPP}方法的核心差别是对补偿电荷的处理不同:~\textrm{Kresse}方案的补偿电荷,不再局限于原子轨道本身(式\eqref{eq:PAW_Kresse_14}),因此比\textrm{USPP}原子电荷密度补偿电荷可以更平缓;~反之,如果\textrm{USPP}方法中提高补偿电荷的赝化函数的构造方式,将有可能系统地提升\textrm{USPP}的计算精度。
	\item \textrm{Kresse}方案从电荷密度分解的思想(式\eqref{eq:PAW_Kresse_02})出发,在\textrm{USPP}近似基础上,引入各原子的在位(\textrm{on-site})修正(主要来自补偿电荷部分),得到\textrm{PAW}方法下的计算方案。
\end{itemize}

\subsubsection{\rm{PAW~}原子数据集}
\textrm{Kresse}方案中将与原子分波有关的数据称为原子数据集(\textrm{PAW Datasets}),这是\textrm{VASP}的主要计算文件\textrm{POTCAR}的数据构建方式,主要包括:~
\begin{enumerate}
	\item 全电子分波函数$\phi_i$和赝分波函数$\tilde\phi_i$
	\item 投影函数$\tilde p_i$
	\item 芯电荷密度$n_c$ 、局域离子赝势$v_{\mathrm H}[\tilde n_{Zc}]$(赝化离子实电荷密度$\tilde n_{Zc}$仅出现在$v_{\mathrm H}[\tilde n_{Zc}]$中,因此直接构造$v_{\mathrm H}[\tilde n_{Zc}]$)和赝芯电荷密度$\tilde n_c$
	\item 补偿电荷构造函数$g_l(r)$
\end{enumerate}

其赝原子分波函数和投影函数的构造,主要参考文献\cite{JPCM6-8245_1994}:~首先计算原子的全电子分波函数$\phi_i(\vec r)$,为构造形如
	\begin{equation}
		\tilde\phi_{i=Lk}(\vec r)=Y_L(\widehat{\vec r-\vec R}~)\tilde\phi_{lk}(|\vec r-\vec R|)
	\end{equation}
	的赝分波函数,应用\textrm{RRKJ}赝波函数方法\upcite{PRB41-1227_1990},径向赝分波函数由球\textrm{Bessel}函数线性组合
	\begin{equation}
		\tilde\phi_{lk}(r)=\left\{
		\begin{aligned}
			&\sum_{i=1}^2\alpha_ij_l(q_ir)\quad &r<r_c^l\\
			&\phi_{lk}(r)\quad&r>r_c^l
		\end{aligned}
		\right.
	\end{equation}
调节系数$\alpha_i$和$q_i$赝分波函数$\phi_{lk}(r)$在截断半径$r_c^l$处两阶连续可微。

投影波函数$\tilde p_i$由前面介绍的\textrm{Vanderbilt}超软赝势中投影子构造方法\upcite{PRB41-7892_1990}或\textrm{Bl\"ochl}方案的\textrm{Gram-Schmidt}正交化方法\upcite{PRB50-17953_1994}得到,这两种方法得到的\textrm{PAW}投影函数完全相同。
%$\langle\tilde p_i|\tilde\phi_j\rangle=\delta_{ij}$确定

\textrm{Kresse}方案中的局域离子赝势$v_{\mathrm H}[\tilde n_{Zc}]$,只要求其在缀加区外与真实离子势$v_{\mathrm H}[n_{Zc}]$相同,先构造局域原子赝势$\tilde v_{\mathrm{eff}}^a$,再去按有效离子势去屏蔽方式得到$v_{\mathrm H}[\tilde n_{Zc}]$。在截断半径$r_{loc}$内定义原子局域赝势$\tilde v_{eff}^a$为
$$\tilde v_{eff}^a=A\dfrac{\sin(q_{loc}r)}r\quad r<r_{loc}$$
这里$q_{loc}$和$A$的取值要求是使得局域赝势在截断半径$r_{loc}$处连续到一阶。

\textrm{Kresse}方案中赝芯电荷密度$\tilde n_c$的定义为:~在截断半径$r_{pc}$内,用两个\textrm{Bessel}函数$j_0$展开$\tilde n_c$
$$\sum_{i=1,2}B_i\dfrac{\sin(q_ir)}r\quad r<r_{pc}$$
类似地,要求截断半径$r_{pc}$外,$\tilde n_c$与全电子芯电荷密度$n_c$相同,系数$q_i$和$B_i$可使得赝芯电荷密度$\tilde n_c(r)$在$r_{pc}$处连续到两阶。

局域离子赝势$v_H[\tilde n_{Zc}]$可由原子局域赝势$\tilde v_{\mathrm{eff}}^a$去屏蔽得到
$$v_{\mathrm H}[\tilde n_{Zc}]=\tilde v_{\mathrm{eff}}^a-v_{\mathrm H}[\tilde n_a^1+\hat n_a]-v_{\mathrm{XC}}[\tilde n_a^1+\hat n_a+\tilde n_c]$$
%	在\textrm{VASP}的\textrm{POTCAR}生成过程中,
\textrm{Kresse}建议的各截断半径的参考条件:~$r_{\mathrm{rad}}=\max({r_c^l})$,$r_{pc}\approx r_{\mathrm{rad}}/1.2$,$r_{\mathrm{loc}}<r_{rad}/1.2$

最后,介绍\textrm{Kresse}方案中每个原子球内用两个球\textrm{Bessel}函数展开的补偿电荷构造函数$g_l(r)$
$$g_l(r)=\sum_{i=1}^2\alpha_i^lj_l(q_i^lr)$$
调节系数$q_i^l$和$\alpha_i^l$使得补偿电荷构造函数$g_l(r)$在截断半径$r_{\mathrm{comp}}$处的数值和前两阶导数值都是0,因此可以选择$q_i^l$使得多极矩满足:~
$$\int_0^{r_{\mathrm{comp}}}g_l(r)r^{l+2}\mathrm{d}r=1$$
实际上这一条件并不难实现,只要选择$q_i^l$满足约束条件
$$\dfrac{\mathrm{d}}{\mathrm{d}r}j_l(q_i^lr)\bigg|_{r_{\mathrm{comp}}}=0$$
并且要求$\alpha_i^l$可使得$g_l(r_{\mathrm{comp}})=0$,即可实现。在此,\textrm{Kresse}建议的截断半径取值参考条件是:~$r_{\mathrm{comp}}=r_{\mathrm{rad}}/1.3\sim r_{\mathrm{rad}}/1.2$,主要考虑尽可能让补偿电荷局域在缀加区,防止出现\textrm{Bl\"ochl}方案中芯层区重叠。

近年来,\textrm{Marsman}等突破了\textrm{Kresse}方案的“冻芯近似”,在自洽迭代计算中允许芯层电荷参与。迭代过程中要求确保价电子和芯电子的正交化,这其中的关键是(1)迭代过程中保持投影函数不变,(2)在每个自洽步中更新赝分波函数。具体可参阅文献\cite{JCP125-104101_2006}。

\subsubsection{投影函数的实空间表示}
由\textrm{PAW}的基本定义式\eqref{eq:PAW-Blochl-02}不难看出,投影函数$\tilde p_i$是关联在位原子分波($\phi_i$和$\tilde\phi_i$)和倒空间表示的赝波函数$\tilde\Psi$的枢纽。因为原子分波局域在离子实附近,显然,如果投影函数可以在实空间表示,将有效地提高计算效率。回顾赝势理论,因为构造赝波函数时,截断半径$R_l$和能量参数$E_l$都是可调的,因此可以通过参数调节,优化赝波函数、投影函数和赝势。

当平面波截断$G_{\mathrm{max}}$确定条件下,投影函数在实空间的表示,实际上也就是投影函数的优化:~
\begin{itemize}
	\item 确定平面波截断参数$G_{\mathrm{max}}$
	\item 投影函数正空间表示与倒空间表示满足条件
		\begin{equation}
			\tilde p_l(q)=\int_0^{\infty}r^2\tilde p_l(r)j_l(qr)\mathrm{d}r
			\label{eq:projector_G_R}
		\end{equation}
		这里$j_l$是球\textrm{Bessel}函数。
	\item 调节$\tilde p_l(q)$,以最小化积分
		\begin{equation}
			I=\int_{G_{\mathrm{max}}}^{\infty}[q\tilde p_l(q)]^2\mathrm{d}q
			\label{eq:projector_Int}
		\end{equation}
		由此可确定$\tilde p_l(r)$在实空间优化的函数表示。
\end{itemize}
\textrm{VASP}软件中正是利用这一思想,得到优化的正空间投影函数,将物理量的计算尽可能地限制在实空间内完成,必要时再通过\textrm{FFT}变换到倒空间。这样既保留了计算的精度,又有效地降低了计算量。
%上述讨论中主要围绕\textrm{PAW}方法对于电子结构计算的方法,没有从第一原理分子动力学\textrm{(Ab initio Molecular Dynamics, AIMD)}角度讨论体系中原子和离子的受力运动,基于平面波基组的\textrm{AIMD}计算方法的一般概念和\textrm{PAW}方法中有关原子受力问题的讨论,可参阅文献\cite{PRB47-10142_1993,PRB50-17953_1994,PRB59-1758_1999,Comput_Phys}。

\begin{figure}[h!]
\centering
\includegraphics[height=4.5in,width=3.6in,viewport=0 0 480 630,clip]{VASP_procedure.png}
%\includegraphics[height=1.8in,width=4.in,viewport=30 210 570 440,clip]{PAW_projector.eps}
\caption{\small \textrm{The Flow of calculation for KS-ground state in VASP.}}%(与文献\cite{EPJB33-47_2003}图1对比)
\label{PAW_procedure}
\end{figure}
\subsection{\rm{VASP}中的优化算法}
\textrm{VASP}的主要任务是根据\textrm{DFT}理论迭代求解\textrm{Kohn-Sham}方程,获得电子的本征态波函数,进而获得电子基态能量。主要的计算流程如图\ref{PAW_procedure}所示。\textrm{VASP}中的迭代求解基态电子密度,本质上是采用数值优化的过程。根据计算对象的不同,主要是两类优化问题:
\begin{itemize}
	\item 能量泛函变分在基态密度时取极小:~$\mathrm{Min}\{E[\rho(\vec r)]\}$
	\item 迭代对角化求解\textrm{Kohn-Sham}方程本征值
\end{itemize}
表观上,这两类计算差别比较大,但是从数值计算的角度考虑,这两类优化本质上都可以归结为“不动点问题”\upcite{Numerical-Analysis},因此\textrm{VASP}中应用的计算算法类似,包括最陡下降\textrm{(Steepest Descent, ST)}、共轭梯度\textrm{(Conjugate Gradient, CG)}和\textrm{RMM-DIIS~(Residual Minimization Method-Direct Inversion in the Iterative Subspace)}等。其中最有特色的是\textrm{RMM-DIIS}方法。

方程迭代求解过程中,定义残量
	\begin{displaymath}
		(\mathbf{H}-\varepsilon^n\mathbf{S})|\psi^n\rangle=|R[\psi^n]\rangle
	\end{displaymath}
近似地,如果体系本征态的逼近量与残量满足线性关系,有
\begin{displaymath}
	|\delta\psi^{n+1}\rangle=\mathbf{K}|R[\psi^n]\rangle
\end{displaymath}
这里$\mathbf{K}$代表了函数优化的方向。可定义$\mathbf{K}$
\begin{equation}
	\mathbf{K}=\sum_{\vec q}\dfrac{|\vec q\rangle\langle\vec q|}{\langle\vec q|\mathbf{H}-\varepsilon\mathbf{S}|\vec q\rangle}
	\label{eq:Predict}
\end{equation}
在\textrm{VASP}中,$\mathbf{K}$的取值为:
\begin{displaymath}
	\mathbf{K}=-\sum_q\dfrac{2|\vec q\rangle\langle\vec q|}{E^{\mathrm{kin}}(R)}\times\dfrac{27+18x+12x^2+8x^3}{27+18x+12x^2+8x^3+16x^4}
\end{displaymath}
这里$x=\dfrac{\hbar^2}{2m_e}\dfrac{q^2}{\frac32E^{\mathrm{kin}}(R)}$,$E^{\mathrm{kin}}(R)$表示残矢动能。

\subsection{\rm{FFT}的并行实现}
\textrm{VASP}计算中要处理大量的\textrm{FFT}变换。一般\textrm{FFT}计算网格数较多,为了提高计算效率,\textrm{VASP}实现了\textrm{FFT}网格计算的并行化。主要通过\textrm{VASP}的子程序\textbf{mgrid.F}实现。子程序\textbf{mgrid.F}建立了\textrm{FFT}网格和并行计算网格的映射。假设有8个计算节点,按照$2\times4$排列,则8个节点依次编号为$0~(0,1)/1~(0,2)/2~(0,3)\cdots7~(3,1)$,如果有四个平面波函数,则可以按下表分配:~
\begin{table}[h!]
%\tabcolsep 0pt \vspace*{-12pt}
\caption{The topology of the Wave and the nodes.}
\label{Table-Gpoint-Nodes}
\begin{minipage}{\textwidth}
%\begin{center}
\centering
\def\temptablewidth{0.54\textwidth}
\rule{\temptablewidth}{1pt}
\begin{tabular*} {\temptablewidth}{@{\extracolsep{\fill}}c@{\extracolsep{\fill}}c@{\extracolsep{\fill}}c}
%-------------------------------------------------------------------------------------------------------------------------
\textrm{Plane-Wave}  & \multicolumn{2}{c}{\textrm{multi-cores}}   \\ \hline
%-------------------------------------------------------------------------------------------------------------------------
\textrm{Wave~1} &0~(0,0) &1~(0,1)   \\% \cline{3-7}
\textrm{Wave~2} &2~(1,0) &3~(1,1)   \\% \cline{3-7}
\textrm{Wave~3} &4~(2,0) &5~(2,1)   \\% \cline{3-7}
\textrm{Wave~4} &6~(3,0) &7~(3,1)   \\% \cline{3-7}
%-------------------------------------------------------------------------------------------------------------------------
\end{tabular*}
\rule{\temptablewidth}{1pt}
\end{minipage}%{\textwidth}
\end{table}
通过建立这样的映射关系,可建立平面波与计算节点的对应关系,可以方便地实施\textrm{FFT}变换,提高计算效率。

综上所述,\textrm{VASP}通过对物理思想和方法、计算算法和编程过程的综合考虑,均衡计算精度和效率,成为第一原理计算高效能软件的典型代表。对\textrm{VASP~}代码的梳理和分析,也使得我们更深入地理解了\textrm{PAW~}方法的第一原理计算的方法和算法,为开发和拓展有关软件的功能提供了必要的支持。
\section{小结}
基于\textrm{VASP}软件,我们针对材料电子计算的基本问题如晶体结构空间群表示理论、电子基态总能计算和\textrm{Fermi~}能等展开一些讨论,重点分析了
\begin{itemize}
	\item 空间群与点群和平移/滑移对应关系,开发出适用于\textrm{VASP}软件的空间群分析模块;
	\item 针对电子基态总能计算中的奇点排除问题,讨论了在高阶奇点下的\textrm{Fermi~}能表示(可与分子、原子体系对应);~
	\item 通过对\textrm{VASP~}基本理论和程序实现的分析,指出了\textrm{VASP}软件能在第一原理计算软件中以优异的效能脱颖而出的原因。
\end{itemize}
上述工作,加深了我们对相关软件的理论、算法和程序代码的理解,加强了对复杂材料开展研究的能力。

