\section{VASP软件的特点}
\textrm{VASP}软件\upcite{VASP_manual}是维也纳大学(Universit\"at Wien)\textrm{G. Kresse}等开发的第一原理模拟软件包。\textrm{VASP}采用的\textrm{PAW~(Projector Augmented-Wave)}方法\upcite{PRB50-17953_1994,PRB59-1758_1999},平衡了传统赝势方法和高精度全电子方法的优点,兼顾了计算的精度和效率。特别是实空间优化的投影函数\textrm{(Projector)},将主要的计算任务变换到实空间完成,大大节省了基组的维度,保证了计算精度和效率。在此基础上,\textrm{VASP}通过引入丰富多样的优化算法,提高自洽迭代过程中的矩阵对角化和电荷密度搜索的效率;~软件的\textrm{mpi}并行实现中,通过\textrm{FFT~(Fast Fourier Transformation)}计算网格与并行计算的计算格点的分配平衡,提升了软件的并行效率。相比于其他第一原理计算软件,\textrm{VASP}从物理思想与方法、优化算法和并行计算实现等多个方面都有更为出色的性能。有关\textrm{VASP}软件实现的文献,主要如下:~
\begin{enumerate}
	\item 物理思想与方法:~\textrm{PAW}赝势方法:~参阅文献\cite{PRB50-17953_1994,PRB59-1758_1999};~投影函数的实空间优化,参阅\cite{JPCM6-8245_1994,PRB44-13063_1991,PRB44-8503_1991}
	\item 优化算法:~参阅文献\cite{CMS6-15_1996,PRB54-11169_1996}
	\item 高效的并行计算:~ \textrm{FFT}网格与计算格点的分配:~参阅\textrm{VASP}源代码的\textbf{mgrid.F}
\end{enumerate}
