%Introduction
\section{周期体系计算中的能量零点的移动与\textrm{Fermi}能}
\subsection{问题的提出}
在电子结构计算中,对分子、原子等有限尺度体系,习惯上将能量零点取在无穷远,即无穷远处的静止电子的能量为零。这样选择的能量参考点,束缚态的电子能量都是负值,并且基态最高占据态的电子能级与第一电离能的负值对应。但是对于理想的周期体系来说,“无穷远”因为引入周期性而消失,所以必须另外选择能量零点。\upcite{JPC-SSP12-4409_1979,XIE-LU}

\subsection{晶体总能量计算与能量零点选择}
一般地,晶体中的基态总能量$E_T$可以表示成晶格中的电子能量$E_{e-e}$与离子实排斥能$E_{N-N}$之和:~
	\begin{equation}
		E_T=E_{e-e}+E_{N-N}=T[\rho]+E_{ext}+E_{\mathrm{Coul}}+E_{\mathrm{XC}}+E_{N-N}
		\label{eq:Crystal_ENE_R}
	\end{equation}
根据密度泛函理论(\textrm{Density-Functional Theory, DFT})和\textrm{Kohn-Sham}方程\upcite{PRB136-864_1964,PRA140-1133_1965},电子本征态方程为:~
\begin{equation}
	\bigg[\dfrac12\nabla^2+V_{ext}(\vec r)+V_{\mathrm{Coul}}(\vec r)+V_{\mathrm{XC}}[\rho(\vec r)]\bigg]|\psi_i(\vec r)\rangle=\varepsilon_i|\psi_i(\vec r)\rangle
	\label{eq:DFT}
\end{equation}
动能泛函用单电子能量表示为
\begin{equation}
	T[{\rho}]=\sum_in_i\langle\psi_i|\varepsilon_i-V_{\mathrm{KS}}|\psi_i\rangle
	\label{eq:DFT_Kin}
\end{equation}
$n_i$是$\psi_i$上的电子占据数,$\varepsilon_i$是其能量本征值,因此总能量的泛函表示为:
\begin{equation}
	E_T=\sum_in_i\varepsilon_i-\dfrac12\int\int\mathrm{d}\vec r\mathrm{d}\vec r\dfrac{\rho(\vec r)\rho(\vec r^{\prime})}{|\vec r-\vec r^{\prime}|}+\int\mathrm{d}\vec r\rho(\vec r)[\epsilon_{\mathrm{XC}}(\vec r)-V_{\mathrm{XC}}(\vec r)]+E_{N-N}
	\label{eq:DFT_ENE_R}
	\end{equation}

对于周期体系来说,因为电子的能量本征态是与动量空间($\vec K$空间)相关联,即\textrm{Kohn-Sham}方程表示为:
\begin{equation}
	\bigg[\dfrac12\vec k^2+V_{ext}(\vec k)+V_{\mathrm{Coul}}(\vec k)+V_{\mathrm{XC}}[\rho(\vec k)]\bigg]|\psi_i^{\vec k}(\vec r)\rangle=\varepsilon_i^{\vec k}|\psi_i^{\vec k}(\vec r)\rangle
	\label{eq:DFT-k}
\end{equation}
显然,总能量在动量空间中计算更方便:~
\begin{equation}
	E_T=\sum_{i,\vec k}n_i\varepsilon_i^{\vec k}-\dfrac{\Omega}2\sum_{\vec k}\rho^{\ast}(\vec k)V_{\mathrm{Coul}}(\vec k)+\Omega\sum_{\vec k}\rho^{\ast}(\vec k)[\epsilon_{\mathrm{XC}}(\vec k)-V_{\mathrm{XC}}(\vec k)]+E_{N-N}
	\label{eq:DFT_ENE_G}
\end{equation}
其中$V_{\mathrm{Coul}}(\vec k)$、$\epsilon_{\mathrm{XC}}(\vec k)$与$\rho^{\ast}(\vec k)$分别是\textrm{Coulomb}相互作用、单个电子的交换-相关能、交换-相关势和电子密度的\textrm{Fourier}分量。

实际计算中需要作一些数学处理:~
\begin{itemize}
	\item 交换-相关势和交换-相关能的计算一般先在实空间计算$\epsilon_{\mathrm{XC}}(\vec r)$和$V_{\mathrm{XC}}(\vec r)$后,再通过\textrm{Fourier~}变换到动量空间,得到$\epsilon_{\mathrm{XC}}(\vec k)$和$V_{\mathrm{XC}}(\vec k)$。
	\item 由\textrm{Poisson}方程
\begin{equation}
	\nabla^2V_{\mathrm{Coul}}(\vec r)=-4\pi\rho(\vec r)
	\label{eq:Poisson}
\end{equation}
的\textrm{Fourier}展开有
\begin{equation}
	V_{\mathrm{Coul}}(\vec k)=\dfrac{4\pi\rho^{\ast}(\vec k)}{|\vec k|^2}
	\label{eq:FFT_Poisson}
\end{equation}
显然$V_{\mathrm{Coul}}(\vec k=0)$是发散的;
	\item 考虑离子间\textrm{Coulomb}相互作用能之和
	\begin{equation}
		E_{N-N}=\dfrac12\sum_{\vec R,s}\sideset{}{^{\prime}}\sum_{\vec R^{\prime},\vec s^{\prime}}\dfrac{Z_sZ_{s^{\prime}}}{|\vec R+\vec r_s-\vec R^{\prime}-\vec r_s^{\prime}|}
		\label{eq:Ion_Coulomb_ENE}
	\end{equation}
这里$Z_s$是离子实的电荷数,$\vec R$表示晶格点的位矢,$\vec r_s$代表元胞内原子的相对位矢。因为$E_{N-N}$求和包含无穷多项,是发散的;
	\item 用于求解能量本征态的式\eqref{eq:DFT-k}中$V_{ext}$的\textrm{Fourier}分量在$\vec k=0$处也是发散的。
\end{itemize}
因此总能量泛函中,$E_{N-N}$、$V_{\mathrm{Coul}}(\vec k=0)$和$V_{ext}(\vec k=0)$这三项单独都是发散的,但因为整个体系出于电中性,所以这些发散项相互抵消,应是一个常数。

因此实际的总能计算中,首先在求解\textrm{Kohn-Sham}方程时,先将$V_{\mathrm{Coul}}(\vec k=0)$和$V_{ext}(\vec k=0)$同时置为零,这相当于势能作一平移,或者说重新定义势能零点。由此得到的总能泛函为:~
\begin{equation}
	E_T=\sum_{i,\textcolor{red}{\vec k\neq0}}n_i\varepsilon_i^{\vec k}-\dfrac{\Omega}2\sum_{\textcolor{red}{\vec k\neq 0}}\rho^{\ast}(\vec k)V_{\mathrm{Coul}}(\vec k)+\Omega\sum_{\vec k}\rho^{\ast}(\vec k)[\epsilon_{\mathrm{XC}}(\vec k)-V_{\mathrm{XC}}(\vec k)]+E_{N-N}
	\label{eq:DFT_ENE_G-2}
\end{equation}
最后在总能量计算中,考虑补偿势能零点的这一平移。

\subsection{发散项的处理}
根据上面的讨论,总能量中发散项之和为:~
	\begin{equation}
		\begin{aligned}
			\lim_{\vec k\rightarrow0}\Omega&\bigg[\dfrac12V_{\mathrm{Coul}}(\vec k)+\sum_sv_{ext}^s(\vec k)\bigg]\rho^{\ast}(\vec k)+\dfrac12\sum_{\vec R,s}\sideset{}{^{\prime}}\sum_{\vec R^{\prime},\vec s^{\prime}}\dfrac{Z_sZ_{s^{\prime}}}{|\vec R+\vec r_s-\vec R^{\prime}-\vec r_s^{\prime}|}\\
			=&\sum_s\alpha_s\sum_sZ_s+E_{\mathrm{Ewald}}
		\end{aligned}
		\label{eq:diver-term}
	\end{equation}
	
$V_{ext}$在不存在其他外场时,一般只考虑离子-电子的\textrm{Coulomb}相互作用,
	\begin{equation}
		\begin{aligned}
			V_{ext}(\vec r)&=\sum_{\vec R,s}\dfrac{-Z_s}{|\vec r-\vec R-\vec r_s|}\\
			&\equiv\sum_{\vec R,s}v_{ext}^s(\vec r-\vec R-\vec r_s)
		\end{aligned}
		\label{eq:Ion-ele_Coulomb}
	\end{equation}

对于形如$Z_s/r$的外场,其\textrm{Fourier}分量在$\vec k=0$附近展开
	\begin{equation}
		v_{ext}^s(\vec k)=-\dfrac{4\pi Z_s}{\Omega|\vec k|^2}+\alpha_s+O(\vec k)
		\label{eq:V_ext}
	\end{equation}
展开$\rho^{\ast}(\vec k)$,有
	\begin{equation}
		\lim_{\vec k\rightarrow 0}\rho^{\ast}(\vec k)=\dfrac{\sum_sZ_s}{\Omega}+\beta|\vec k|^2+O(\vec k)
		\label{eq:rho_ext}
	\end{equation}
去掉高次项,有
\begin{equation}
	\begin{aligned}
		\lim_{\vec k\rightarrow 0}&\bigg[\boxed{\textcolor{blue}{\dfrac{\Omega}2\dfrac{4\pi[\rho^{\ast}(\vec k)]^2}{|\vec k|^2}}}+\boxed{\Omega}\bigg(\boxed{\textcolor{blue}{-\dfrac{4\pi\sum_sZ_s}{\Omega|\vec k|^2}}}+\sum_s\alpha_s\bigg)\boxed{\rho^{\ast}(\vec k)}+\boxed{\textcolor{red}{\dfrac12\dfrac{4\pi(\sum_sZ_s)^2}{\Omega|\vec k|^2}}}\bigg]\\
		&+\boxed{\dfrac12\sum_{\vec R,s}\sideset{}{^{\prime}}\sum_{\vec R^{\prime},\vec s^{\prime}}\dfrac{Z_sZ_{s^{\prime}}}{|\vec R+\vec r_s-\vec R^{\prime}-\vec r_{s^{\prime}}|}-\lim_{\vec k\rightarrow0}\textcolor{red}{\dfrac12\dfrac{4\pi(\sum_sZ_s)^2}{\Omega|\vec k|^2}}}\\
		=&\sum_s\alpha_s\sum_sZ_s+\textcolor{magenta}{E_{\mathrm{Ewald}}}
	\end{aligned}
	\label{eq:V_ext_exp2}
\end{equation}
其中离子间排斥势采用\textrm{Ewald~}方法得到\upcite{Born-Huang,R.Martin}:~对于形如点电荷形式的静电势$\dfrac{e^2}r$,可引入\textrm{Gauss~}误差函数\upcite{Grosso-Parravicini}
\begin{equation}
	\begin{aligned}
		&\mathrm{erf}(x)=\dfrac2{\sqrt{\pi}}\int_0^{x}\mathrm{e}^{-t^2}\mathrm{d}t\\
		&\mathrm{erfc}(x)=\dfrac2{\sqrt{\pi}}\int_x^{\infty}\mathrm{e}^{-t^2}\mathrm{d}t\\
		\mbox{满足}\quad&\mathrm{erf}(x)+\mathrm{erfc}(x)=1
	\end{aligned}
	\label{eq:err_fun}
\end{equation}
得到恒等式(见图\ref{Error_Function}):
\begin{equation}
	\dfrac{e^2}r\equiv\dfrac{e^2}r\mathrm{erf}(\sqrt{\eta}r)+\dfrac{e^2}r\mathrm{erfc}(\sqrt{\eta}r)
	\label{eq:err_fun_comp}
\end{equation}
\begin{figure}[h!]
\centering
\vspace*{-0.10in}
\includegraphics[height=2.55in,width=5.8in,viewport=0 0 1100 455,clip]{Ewald_method.png}
\caption{\small \textrm{Decomposition of the potential $-e^2/r$ (singular at the origin and of long-range nature) into a contribution $-(e^2/r)\mathrm{erf}(\sqrt{\eta}r)$(regular at the origin of long-range) and a contribution $-(e^2/r)\mathrm{erfc}(\sqrt{\eta}r)$ (singular at the origin and of short-range nature). Here $\sqrt{\eta}=1 (\mathrm{Bohr radius unit})$ is chosen.}\upcite{Grosso-Parravicini}}%(与文献\cite{EPJB33-47_2003}图1对比)
\label{Error_Function}
\end{figure}

	\begin{equation}
		\begin{aligned}
			E_{\textrm{Ewald}}=&\dfrac12\sum_{\vec R,s}\sideset{}{^{\prime}}\sum_{\vec R^{\prime},\vec s^{\prime}}\dfrac{Z_sZ_{s^{\prime}}}{|\vec R+\vec r_s-\vec R^{\prime}-\vec r_{s^{\prime}}|}-\lim_{\vec k\rightarrow0}\dfrac12\times\dfrac{4\pi(\sum_sZ_s)^2}{\Omega|\vec k|^2}\\
			=&\dfrac12\sum_{\vec R,s}\sideset{}{^{\prime}}\sum_{\vec R^{\prime},\vec s^{\prime}}\dfrac{Z_sZ_{s^{\prime}}}{|\vec R+\vec r_s-\vec R^{\prime}-\vec r_{s^{\prime}}|}-\dfrac1{2\Omega}\sum_{s,s^{\prime}}\int\mathrm{d}\vec r\dfrac{Z_sZ_{s^{\prime}}}r\\
			=&\sum_{s,s^{\prime}}Z_sZ_{s^{\prime}}\bigg\{\dfrac{2\pi}{\Omega}\sum_{\vec k\neq 0}\cos[\vec k\cdot(\vec r_s-\vec r_{s^{\prime}})]\dfrac{\mathrm{e}^{-|\vec k|^2/4\eta}}{|\vec k|^2}\\
			&-\dfrac{\pi}{2\eta\Omega}+\dfrac14\sum_{\vec R}\dfrac{\mathrm{erf}(\sqrt{\eta}x)}x\bigg|_{\vec R+\vec r_s-\vec r_s^{\prime}\neq0}-\sqrt{\dfrac{\eta}{\pi}}\delta_{s,s^{\prime}}\bigg\}
		\end{aligned}
		\label{eq:Ewald_ENE}
	\end{equation}
	$\mathrm{erf}(x)$是误差函数,$\sqrt{\eta}$原则上是任意参数。$\alpha_s$由$v_{ext}^s(\vec r)$确定:~
	\begin{equation}
		\alpha_s=\lim_{\vec k\rightarrow0}\bigg[v_{ext}^s(\vec k)+\dfrac{4\pi Z_s}{\Omega|\vec k|^2}\bigg]=\dfrac1{\Omega}\int\mathrm{d}\vec r\bigg[v_{ext}^s(\vec r)+\dfrac{Z_s}r\bigg]
		\label{eq:alpha_s}
	\end{equation}
由此得到的总能量表达式是:
\begin{equation}
	\begin{aligned}
		E_T=&\sum_i\varepsilon_i-\dfrac{\Omega}2\sum_{\vec k\neq0}\rho^{\ast}(\vec k)V_{\mathrm{Coul}}(\vec k)\\
		&+\Omega\sum_{\vec k}\rho^{\ast}(\vec k)[\epsilon_{\mathrm{XC}}(\vec k)-V_{\mathrm{XC}}(\vec k)]\\
		&+\sum_s\alpha_s\sum_sZ_s+E_{\mathrm{Ewald}}
	\end{aligned}
	\label{eq:TOT_ENE_Finial}
\end{equation}

\textrm{VASP~}软件的总能量计算即遵照式\eqref{eq:TOT_ENE_Finial}计算的。图\ref{TOTEN_VASP}给出就是\textrm{VASP~}总能计算的输出形式:~
\begin{figure}[h!]
\centering
\vspace*{-0.12in}
\includegraphics[height=3.85in,width=4.2in,viewport=0 0 600 495,clip]{VASP_Total_ENE.png}
\caption{\small \textrm{The Total-E calculated by VASP.}}%(与文献\cite{EPJB33-47_2003}图1对比)
\label{TOTEN_VASP}
\end{figure}

%根据\textrm{Ewald}的势能计算方法,$\boxed{\dfrac12\dfrac{4\pi(\sum_sZ_s)^2}{\Omega|\vec k|^2}}$表示的电子势能在$\vec k=0$处的贡献,可分为
%	\begin{itemize}
%		\item \textcolor{purple}{对应于实空间电子势能的长程可收敛部分:~}式\eqref{eq:Ewald_ENE}中第三项
%			\begin{displaymath}
%				-(\sum\limits_{s,s^{\prime}}Z_sZ_{s^{\prime}})\dfrac14\sum_{\vec R}\dfrac{\mathrm{erf}(\sqrt{\eta}x)}x\bigg|_{\vec R+\vec r_s-\vec r_s^{\prime}\neq0}
%			\end{displaymath}
%		\item \textcolor{purple}{对应于实空间电子势能的短程发散部分:~}式\eqref{eq:Ewald_ENE}中第二项
%			\begin{displaymath}
%				(\sum_{s,s^{\prime}}Z_sZ_{s^{\prime}})\dfrac{\pi}{2\eta\Omega}
%			\end{displaymath}
%		\textcolor{red}{注意:~实际计算中,因为误差函数的参数$\sqrt{\eta}$不为零,因此该发散部分表示为一个大数而不是$\infty$}。
%	\end{itemize}
%	类似地,不难看出,式\eqref{eq:Ewald_ENE}中\textcolor{blue}{第一项}和\textcolor{magenta}{第四项}分别对应离子-电子的\textrm{Coulomb}相互作用
%	\begin{displaymath}
%		\boxed{\dfrac12\sum_{\vec R,s}\sideset{}{^{\prime}}\sum_{\vec R^{\prime},\vec s^{\prime}}\dfrac{Z_sZ_{s^{\prime}}}{|\vec R+\vec r_s-\vec R^{\prime}-\vec r_{s^{\prime}}|}}
%	\end{displaymath}
%	的\textcolor{blue}{长程收敛}和\textcolor{magenta}{短程发散}部分。
%
%以\textrm{FCC-Al}为例,采用\textrm{VASP~}计算得到有关数值如下:~
%\begin{displaymath}
%	\begin{aligned}
%	&\mathrm{E-fermi}:~&7.4406\\
%	&\sum\alpha_iZ_i:~&-0.1949\\
%	&\mathrm{Ewald-Energy}:~&-72.4621\\
%	&\mathrm{XC(G=0)}:~&-10.00040 \\
%	\end{aligned}
%\end{displaymath}
%将\textrm{VASP~}计算中的\textrm{Ewald-Energy}按式\eqref{eq:Ewald_ENE}分解,各部分对应的数值为:~
%\begin{displaymath}
%	\begin{aligned}
%	&\mathrm{Part-1}:~&1.320907 \\
%	&\mathrm{Part-2}:~&-50.176618 \\ 
%	&\mathrm{Part-3}:~&1.481905 \\
%	&\mathrm{Part-4}:~&-25.088309 \\
%	\end{aligned}
%\end{displaymath}
%\textcolor{red}{不难看出,这里第二项和第四项分别代表两部分势能在$\vec k=0$的发散项贡献,因此数值的绝对值比另外两项大得多。}

\section{能量零点移动对能量本征态的影响}
根据上述讨论,因为能量零点的平移,周期体系计算的能量本征值$\varepsilon_i$的数值一般不绝对为负值。习惯上在能带和态密度表示时,常常将\textrm{Fermi~}能设置成零。

参照总能计算中能量零点移动的讨论,可以计算\textrm{Kohn-Sham~}方程中势能零点引起的能量本征值的移动,%根据检索\textrm{VASP~}的代码发现:~\textcolor{red}{\textrm{VASP~}程序在构造\textrm{Fock~}矩阵的时候,已经包括了$V_{\mathrm{XC}}(\vec k=0)$的贡献~},即\textcolor{purple}{$\mathrm{XC(G=0)}$}对应的数值(见图\ref{TOTEN_VASP});~
$V_{\mathrm{Coul}}(\vec k=0)$和$V_{ext}[\rho(\vec k=0)]$。
即
\begin{equation}
	\lim_{\vec k\rightarrow0}\bigg[V_{\mathrm{Coul}}(\vec k)+\sum_sv_{ext}^s(\vec k)\bigg]
	\label{eq:part_diver-term}
\end{equation}
式\eqref{eq:part_diver-term}\textbf{势的移动}并不简单对应式\eqref{eq:diver-term}中\textbf{能量移动}的发散项求和。

在$\vec k=0$附近,将式\eqref{eq:V_ext}和\eqref{eq:rho_ext}代入式\eqref{eq:part_diver-term},去掉高次项,可有
\begin{equation}
	\begin{aligned}
		&\lim_{\vec k\rightarrow 0}\bigg[\dfrac{4\pi\rho^{\ast}(\vec k)}{|\vec k|^2}+\bigg(-\dfrac{4\pi\sum_sZ_s}{\Omega|\vec k|^2}+\sum_s\alpha_s\bigg)\bigg]\\
		=&\lim_{\vec k\rightarrow 0}\bigg[\boxed{\dfrac{4\pi}{\Omega}\dfrac{\sum_sZ_s}{|\vec k|^2}}+4\pi\beta\boxed{-\dfrac{4\pi\sum_sZs}{\Omega|\vec k^2|}}+\sum_s\alpha_s\bigg]\\
		=&\sum_s\alpha_s+4\pi\beta
	\end{aligned}
	\label{eq:V_shift-term}
\end{equation}
式\eqref{eq:V_shift-term}对应\textrm{VASP~}软件中给出的\textcolor{purple}{$\mathrm{alpha+bet}$}的数值,见图\ref{TOTEN_VASP}(在\textrm{VASP~}中,\textcolor{purple}{bet}项对应式\eqref{eq:V_shift-term}的$4\pi\beta$)。

上述推导也验证了赝势理论的基本思想:~对于电中性的周期体系,在倒空间中,电子\textrm{Coulomb~}势与原子核的吸引势相互抵消后,净的作用可近似为高阶奇点和一个平缓的势函数。\textrm{VASP~}软件中,$\alpha_s$和$\beta$的数值计算方式如下:~
\begin{itemize}
	\item \textcolor{blue}{$\alpha_s$取原子赝势在径向的第一个点(即离$\vec k=0$最近)的数值}
	\item \textcolor{blue}{$\beta$由各原子赝电荷密度前5个点(即离$\vec k=0$足够近)的数值两阶差分后求和得到}
\end{itemize}

传统的求解\textrm{Kohn-Sham~}方程计算能量本征值时,将势函数中包括核-电子吸引和电子间排斥排斥势的全部$\vec k=0$部分的贡献扣除。如果考虑补偿函数式\eqref{eq:V_shift-term}的贡献,则利用了上述两项的奇点能量部分抵消的特性,保留了势能零点在无穷远时的部分特征(\textcolor{red}{注意:~采用该能量补偿方案,高阶奇点仍然存在!})。

综上所述,在\textrm{VASP~}中,如果考虑周期体系的势能零点移动,则\textrm{Fermi~}的数值可取为\textbf{两项之和}:~
\begin{displaymath}
	\mathrm{E_{fermi}}=\textcolor{blue}{\mathrm{E-fermi}}+\textcolor{purple}{\mathrm{alpha+bet}}
\end{displaymath}
这就是排除高阶奇点后的\textrm{Fermi~}能,与传统的分子、原子中能量计算结果相近。
%\subsection{一点讨论}
%我们对\textrm{FCC-Al}的计算表明,
%\begin{displaymath}
%	\begin{aligned}
%	&\mathrm{E-fermi}:~&7.4406\\
%	&\sum\alpha_iZ_i:~&-0.1949\\
%	&\mathrm{Ewald-Energy}:~&-72.4621\\
%	&\mathrm{XC(G=0)}:~&-10.00040 \\
%	&\mathrm{alpha+bet}:~-&14.2459\\
%	\end{aligned}
%\end{displaymath}
%因此考虑势能零点移动修正的\textrm{Fermi}能应为(\textcolor{blue}{包含\textbf{OUTCAR}中$\mathrm{alpha+bet}$项}):
%\begin{displaymath}
%	7.4406-14.2459=-6.80\;\mathrm{eV}
%\end{displaymath}
%$\ast$注:上述计算验证~
%\begin{itemize}
%	\item 在\textrm{VASP~}计算中,$\mathrm{alpha+bet\mbox{项}}<0$恒成立
%	\item \textrm{VASP~}中$|\mathrm{E-Fermi}|<|\mathrm{alpha+bet}|$成立
%\end{itemize}

\section{\rm{VASP}计算结果对上述推导的检验}
考虑周期体系,如果原子间距离足够远,则周期体系计算结果应该逼近原子分子体系的计算结果,基于该思想,可以用\textrm{VASP}软件对假想的\textrm{Si}和\textrm{Fe}孤立原子(分别在足够大的晶胞中),然后逐渐减少原子间距离,考察\textrm{OUTCAR}文件中\textcolor{purple}{\textrm{alpha+bet}}的变化,结果列于图\ref{Fig:VASP_alpha+bet}。
\begin{figure}[h!]
\centering
\hspace*{-0.7in}
\vspace*{-0.2in}
\includegraphics[width=1.2\textwidth, viewport=0 0 1520 780, clip]{VASP_Fermi_alpha-bet.pdf}
\caption{\small Compare the \textcolor{purple}{\textrm{alpha+bet}} with the distance of atom.}%(与文献\cite{EPJB33-47_2003}图1对比)
\label{Fig:VASP_alpha+bet}
\end{figure}

不难看出,随着原子间距离的逐步增大,\textcolor{purple}{\textrm{alpha+bet}}的数值逐渐趋向于零,相应的\textrm{Fermi}数值逐渐逼近原子能级(数据未在此列出)。一般地认为原子间距离超过10~\AA ,可以原子间没有相互作用,考虑本次极端模型中原子采用\textrm{FCC}密堆积,原子间距离为10~\AA~时对应的晶胞参数为15.874~\AA ,当晶胞参数大于15.874~\AA ,\textcolor{purple}{\textrm{alpha+bet}}的数值将趋向于0,这一结论与图\ref{Fig:VASP_alpha+bet}中曲线趋向零的起始位置吻合得很好。因此,趋于极端情况的假想模型计算结果也验证了\textcolor{purple}{\textrm{alpha+bet}}表示的就是排除高阶奇点后的能量本征值修正项,与前一节讨论的结论一致。
