\usepackage{ctex}
%%%%%%%%%%%%%%%%% CJK 中文版面控制  %%%%%%%%%%%%%%%%%%%%%%%%%%%%%%
%\usepackage{CJK} % CTEX-CJK 中文支持                            %
\usepackage{xeCJK} % seperate the english and chinese		 %
\usepackage{CJKutf8} % Texlive 中文支持                         %
\usepackage{CJKnumb} %中文序号                                   %
\usepackage{indentfirst} % 中文段落首行缩进                      %
%\setlength\parindent{22pt}       % 段落起始缩进量               %
\renewcommand{\baselinestretch}{1.2} % 中文行间距调整            %
\setlength{\textwidth}{16cm}                                     %
\setlength{\textheight}{24cm}                                    %
\setlength{\topmargin}{-1cm}                                     %
\setlength{\oddsidemargin}{0.1cm}                                %
\setlength{\evensidemargin}{\oddsidemargin}                      %
%%%%%%%%%%%%%%%%%%%%%%%%%%%%%%%%%%%%%%%%%%%%%%%%%%%%%%%%%%%%%%%%%%

\usepackage{authblk}					 %作者地址和E-mail
\usepackage{amsmath,amsthm,amsfonts,amssymb,bm}          %数学公式
\usepackage{mathrsfs}                                    %英文花体
\usepackage{tikz}					 %绘制平面图形
\usepackage{pifont}
%\usepackage[dvipdfmx]{movie15_dvipdfmx} %插入视频
\usepackage{xcolor}                                        %使用默认允许使用颜色
%\usepackage{hyperref} 
\usepackage{graphicx}
\usepackage{subfigure}           %图片跨页
\usepackage{animate}		 %插入动画
\usepackage{caption}
\captionsetup{font=footnotesize}

%\usepackage[version=3]{mhchem}		%化学公式
\usepackage{chemformula}
\usepackage{chemfig}		%化学公式

\usepackage{longtable}                                   %使用长表格
\usepackage{multirow}
\usepackage{makecell}		%允许单元格内换行

\usepackage{arydshln}
\newcommand{\adots}{\mathinner{\mkern2mu%
\newcommand{\hollowstar}{\mathbin{\text{\FiveStarOpen}}}
\raisebox{0.1em}{.}\mkern2mu\raisebox{0.4em}{.}%
\mkern2mu\raisebox{0.7em}{.}\mkern1mu}}

\usepackage{newclude}	       % 用\include*{} 实现文件插入时(两个之间)不分页
% newclude 要放在natbib之前,否则引用文献会报错

%%%%%%%%%%%%%%%%%%%%%%%%%  参考文献引用 %%%%%%%%%%%%%%%%%%%%%%%%%%%
%%尽量使用 BibTeX(含有超链接,数据库的条目URL即可)                %
%%%%%%%%%%%%%%%%%%%%%%%%%%%%%%%%%%%%%%%%%%%%%%%%%%%%%%%%%%%%%%%%%%%

\usepackage[numbers,sort&compress]{natbib} %紧密排列              %
%\usepackage[square,super,numbers,sort&compress]{natbib} %紧密排列%
%\usepackage{footbib}			   %脚注列出参考文献     %
%                                                                 %
% !! chapterbib 只有当 Book & chapter 状态下有效/ Article 状态下全文不会有错,但分割正文会报错 !! %
%\usepackage[sectionbib]{chapterbib}        %每章节单独参考文献   %
% !!!!!!!!!!!!!!!!!!!!!!!!!!!!!!!!!!!!!!!!!!!!!!!!!!!!!!!!!!!!!!!!!!!!!!!!!!!!!!!!!!!!!!!!!!!! %
\usepackage{hypernat}                                                                          %
%\usepackage[dvipdfm,bookmarksopen=true,pdfstartview=FitH,CJKbookmarks]{hyperref}              %
\usepackage[bookmarksopen=true,pdfstartview=FitH,CJKbookmarks]{hyperref}                       %
\hypersetup{bookmarksnumbered,colorlinks,linkcolor=purple,citecolor=blue,urlcolor=red}         %
%参考文献含有超链接引用时需要下列宏包,注意与natbib有冲突        %
%\usepackage[dvipdfm]{hyperref}                                  %
%\usepackage{hypernat}                                           %
\newcommand{\upcite}[1]{\hspace{0ex}\textsuperscript{\cite{#1}}} %
%%%%%%%%%%%%%%%%%%%%%%%%%%%%%%%%%%%%%%%%%%%%%%%%%%%%%%%%%%%%%%%%%%%%%%%%%%%%%%%%%%%%%%%%%%%%%%%
%\AtBeginDvi{\special{pdf:tounicode GBK-EUC-UCS2}} %CTEX用dvipdfmx的话,用该命令可以解决      %
%						   %pdf书签的中文乱码问题		      %
%%%%%%%%%%%%%%%%%%%%%%%%%%%%%%%%%%%%%%%%%%%%%%%%%%%%%%%%%%%%%%%%%%%%%%%%%%%%%%%%%%%%%%%%%%%%%%%
%%%%%%%%%%%%%%%%-- Biblatex --%%%%%%%%%%%%%%%%%%%%%%%%%%%%%%%%%%%%%%%%%%%%%%%%%%%%%%%%%%%%%%%%%
%%%%            新版本的BibLaTeX默认使用Biber而非BibTeX进行处理参考文献。                     %
%\usepackage[backend=bibtex]{biblatex}                                                        %
%\usepackage[style=authoryear,backend=biber]{biblatex}                                        %
%%%%%%%%%%%%%%%%%%%%%%%%%%%%%%%%%%%%%%%%%%%%%%%%%%%%%%%%%%%%%%%%%%%%%%%%%%%%%%%%%%%%%%%%%%%%%%%

\usepackage{verbatim}			%Verbatim 宏包重新实现了 Verbatim 环境,并且提供一个命令可以导入一个 ASCII 文件到文档中
%\verbatiminput{filename}

%在beamer里面使用verbatim环境,可以通过在frame的参数里面添加 containsverbatim / fragile来解决,不过 containsverbatim 会导致pause失效
%\begin{frame}[containsverbatim] %也可以用 \begin{frame}[fragile]
%	\begin{verbatim}
%	\usepackage{xcolor}
%	TEST
%	\end{verbatim}
%\end{frame}

%%%%%%%%%%%%%%%%%%%%%%%%%  % 段落缩进   %%%%%%%%%%%%%%%%%%%%%%%%%%%
%%%%%%       首行缩进   %%%%%%%%%%%
%\setlength{\parindent}{2em}{     %
%首行缩进2em的段落(可跨段落)      %
%}                                %
%%\CJKindent    %中文首行缩进     %
%---------------------------------%
%%%%%%%%%%% 取消多段首行缩进%%%%%%%%%%%%%%%
%\setlength{\parindent}{0pt}{             %
% 取消缩进段落(可跨段落)                  %
%}                                        %
%-----------------------------------------------------------------%
%%%%%%%%%%% 悬挂缩进 %%%%%%%%%%%%%%%%%%%%%%%%%%%%%%%%%%%%%%%%%%%%%% 
%\noindent                      %1. 取消⾸⾏缩进                  %
%\hangafter=1                   %2. 设置从第1⾏之后开始悬挂缩进  %
%\setlength{\hangindent}{2em}   %3. 设置悬挂缩进量                %
%{ 悬挂缩进段落(不可跨段落)                                       %
%}                                                                %
%-----------------------------------------------------------------%
%%%%%%%%%%%%%%%%%%%%%%%%%%%%%%%%%%%%%%%%%%%%%%%%%%%%%%%%%%%%%%%%%%%

