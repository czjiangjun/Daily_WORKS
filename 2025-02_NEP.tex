%---------------------- TEMPLATE FOR REPORT ------------------------------------------------------------------------------------------------------%

%\thispagestyle{fancy}   % 插入页眉页脚                                        %
%%%%%%%%%%%%%%%%%%%%%%%%%%%%% 用 authblk 包 支持作者和E-mail %%%%%%%%%%%%%%%%%%%%%%%%%%%%%%%%%
%\title{More than one Author with different Affiliations}				     %
%\title{\rm{VASP}的电荷密度存储文件\rm{CHGCAR}}
%\title{面向高温合金材料设计的计算模拟软件中的几个主要问题}
\title{如何快速做NEP势函数(合金篇)}
\author[ ]{根据文本重新排版}   %
%\author[ ]{姜~骏\thanks{jiangjun@bcc.ac.cn}}   %
%\affil[ ]{北京市计算中心}    %
%\author[a]{Author A}									     %
%\author[a]{Author B}									     %
%\author[a]{Author C \thanks{Corresponding author: email@mail.com}}			     %
%%\author[a]{Author/通讯作者 C \thanks{Corresponding author: cores-email@mail.com}}     	     %
%\author[b]{Author D}									     %
%\author[b]{Author/作者 D}								     %
%\author[b]{Author E}									     %
%\affil[a]{Department of Computer Science, \LaTeX\ University}				     %
%\affil[b]{Department of Mechanical Engineering, \LaTeX\ University}			     %
%\affil[b]{作者单位-2}			    						     %
											     %
%%% 使用 \thanks 定义通讯作者								     %
											     %
\renewcommand*{\Authfont}{\small\rm} % 修改作者的字体与大小				     %
\renewcommand*{\Affilfont}{\small\it} % 修改机构名称的字体与大小			     %
\renewcommand\Authands{ and } % 去掉 and 前的逗号					     %
\renewcommand\Authands{ , } % 将 and 换成逗号					     %
\date{} % 去掉日期									     %
%\date{2020-12-30}									     %

%%%%%%%%%%%%%%%%%%%%%%%%%%%%%%%%%%%%%%%%%%  不使用 authblk 包制作标题  %%%%%%%%%%%%%%%%%%%%%%%%%%%%%%%%%%%%%%%%%%%%%%
%-------------------------------The Title of The Report-----------------------------------------%
%\title{报告标题:~}   %
%-----------------------------------------------------------------------------

%----------------------The Authors and the address of The Paper--------------------------------%
%\author{
%\small
%Author1, Author2, Author3\footnote{Communication author's E-mail} \\    %Authors' Names	       %
%\small
%(The Address,City Post code)						%Address	       %
%}
%\affil[$\dagger$]{清华大学~材料加工研究所~A213}
%\affil{清华大学~材料加工研究所~A213}
%\date{}					%if necessary					       %
%----------------------------------------------------------------------------------------------%
%%%%%%%%%%%%%%%%%%%%%%%%%%%%%%%%%%%%%%%%%%%%%%%%%%%%%%%%%%%%%%%%%%%%%%%%%%%%%%%%%%%%%%%%%%%%%%%%%%%%%%%%%%%%%%%%%%%%%
\maketitle
%\thispagestyle{fancy}   % 首页插入页眉页脚 

%\#-----------------------------------------------------------------------------
\section{流程概述}
%\#-----------------------------------------------------------------------------
\subsection{用势函数研究什么问题}
\begin{itemize}
	\item 低温、室温、中温、高温:\\ 
		如果是固态(低温、室温甚至中温,多数体系都是固态),可以考虑加一部分\textrm{AIMD+ML (VASP~6.3.0以上版本)},速度是普通\textrm{AIMD}的30倍左右,适合采集构型。如果是高温(多数已经是液态),这时\textrm{AIMD+ML}可能不太合适,因为内存需求可能是普通\textrm{AIMD}的几十倍。
	\item 常压、高压:\\
		高压往往需要更高的单点能计算精度(即$\vec k$-\textrm{mesh}密度更高),因为原子间距可能更小 ,力可能更大。
	\item 通用势函数、专注于某种性质的势函数、核辐照势函数
		通用势函数,需要非常丰富的结构,计算量大。专注于某种性质的势函数,仅需覆盖目标研究问题,结构需求少,计算量少。核辐照势函数,需更高单点能计算精度(例如:~\textrm{k-spacing=0.15}),和更多小团簇结构的各种扭曲晶格,以充分覆盖原子局域环境的剧烈变化,\textcolor{blue}{且必须加\textrm{zbl},或计算\textrm{dimer}的压缩、拉伸}。
\end{itemize}
%    \#-----------------------
\subsection{结构设计}
\begin{enumerate}
	\item 找平衡结构,然后缩放、微扰\\
		先找各种情况下的平衡结构,例如不同温度、压强下的平衡晶格常数,目的是有个参考点,可以更明确、更具控制性的进行坐标、盒子的缩放、微扰,从而让训练集的力、压强关于参考点(例如原点)更对称,后续训练势函数的\textrm{NEP\_vs\_DFT}图更对称、均匀,可以更好的覆盖后续用势函数跑\textrm{MD}的力、压强。当然,如果不想找平衡点,也没有原则性问题。
	\item \textrm{VASP}跑\textrm{AIMD}\\
		训练集里,尽量多进行``1.''的缩放、微扰,而从\textrm{AIMD}中抽取的构型,不要占太大比例,个人觉得,可能不要超过\textrm{40\%}(这个阈值仅供参考,需要自己摸索),主要原因是\textrm{AIMD}构型之间关联性(或相似度)更高,加太多\textrm{AIMD}结构,会导致训练集内结构的同质化严重,构型的覆盖范围不充分,影响势函数的描述能力。
	\item 建议用\textrm{PyNEP}进行结构选择\\
		\textrm{NEP}势函数的特点是不需大量结构,只需结构丰富、覆盖范围广,也就是少、精、广。基于这个特点,可以用各种方式对结构进行挑选(这也属于主动学习),推荐\textrm{PyNEP}。
\end{enumerate}
%    \#-----------------------
\subsection{流程总结(目前所知可能的最优流程)}
\begin{enumerate}
	\item 充分理解你研究问题涉及到的范围,例如温度范围、压强范围,从而确定力的范围(例如\textrm{-30~eV/\AA}到\textrm{30~eV/\AA})、最小原子间距(例如任一结构里,任一对原子间距最好大于\textrm{1.6~\AA},原子间距往往决定了力的范围)等。
	\item 基于``1''寻找各种情况下的平衡晶格常数(或平衡密度),由此展开坐标、盒子的微扰和缩放,\textrm{dpdata}可以负责缩放、微扰;~再跑一点\textrm{AIMD}(如果体系有现成的经验势函数,强烈建议抛弃\textrm{AIMD},选择用经验势函数跑\textrm{MD},以采样各种结构)。获得一个覆盖广、结构少而精的初始结构集,然后全部统一精度计算单点能(根据文献、自己计算资源选择精度),得到训练集。然后,可以把训练集分成\textrm{9:1}的两部分,\textrm{9/10}作为真正的训练集\textrm{train\_1.in},\textrm{1/10}作为测试集\textrm{test\_1.in},注意测试集的结构,要尽可能覆盖训练集的所有类型。\textrm{AIMD}时间步长统一用\textrm{3~fs},因为\textrm{AIMD}目的是采样,不需要精确,只要体系不崩溃就行。
	\item 从\textrm{OUTCAR}得到\textrm{NEP}的训练集、测试集\\
		\begin{enumerate}
			\item 第一步,用``\textrm{outcar2deep.py}''(\textrm{QQ}群\textrm{887975816}群文件里,后续可能加入\textrm{GPUMD}包的\textrm{GPUMD-master}\textbackslash\textrm{tools}\textbackslash\textrm{nep\_related}里),把所有\textrm{OUTCAR}转换为\textrm{DP}势函数的训练集;
			\item 第二步,用\textrm{GPUMD-master}\textbackslash\textrm{tools}\textbackslash\textrm{nep\_related}里的``\textrm{deep2nep}''把\textrm{DP}训练集转换为\textrm{NEP}训练集。
		\end{enumerate}
	\item 第一轮训练,用\textrm{train\_1.in}开始训练,快速得到势函数\textrm{nep\_1.txt},基于\textrm{nep\_1.txt},用\textrm{PyNEP}计算所选结构是否符合``少、精、广''的原则,删掉不需要的结构,然后用\textrm{nep\_1.txt}跑\textrm{MD}以获得想要的更多结构,\textrm{VASP}计算单点能,同样按\textrm{9:1}分别加入\textrm{train\_1.in}、\textrm{test\_1.in},形成新的\textrm{train\_2.in}、\textrm{test\_2.in},继续训练获得势函数\textrm{nep\_2.txt},此时势函数一般已经比较精确了,如果验证之后发现还不够,可重复上述步骤得到\textrm{train\_3.in}、\textrm{test\_3.in}、\textrm{nep\_3.txt},直到满意为止。
\end{enumerate}
%\#-----------------------------------------------------------------------------
\section{流程详解}
%\#-----------------------------------------------------------------------------
\subsection{获得训练集、测试集}
\begin{itemize}
	\item 单原子\\
		所有元素单原子的单点能计算,一般看作非金属,按非金属设置,如果计算不收敛或不正常,可以不做,因为单原子能量是作为能量的参考点,而\textrm{NEP}势函数自身可以自动确定参考点。
	\item \textrm{dimer}\\
		可以不做,除非势函数是面对核辐照,因为\textrm{dimer}单点能可能难以收敛或者不正常。
    \item 单元素块体\\
	    可以做一系列典型结构的块体,例如\textrm{fcc}、\textrm{bcc}、\textrm{hcp}等你体系可能遇到的结构,个人觉得,$2\times2\times2$的超胞就可以了。然后,可以用\textrm{dpdata}做一系列平衡晶格常数附近的缩放、微扰,这样,从一个初始结构出发,可以得到数个新结构。

	    每个元素可以另外做一些大一点的超胞,例如$3\times3\times3$的\textrm{fcc}结构,用\textrm{VASP}跑一条\textrm{AIMD}轨迹,即熔点以上某个温度的\textrm{NVT},抽样(间隔\textrm{200}或\textrm{400}或\textrm{500}步)获得一些单元素液态结构。
    \item 二元合金\\
	    元素对的合金,例如三元合金就有\textrm{AB}、\textrm{AC}、\textrm{BC}三种元素对合金,对每种元素对合金,例如~\textrm{AB},可以做一系列组分的构型,$2\times2\times2$或$3\times3\times3$超胞,例如每种组分,从\textrm{5\%}到\textrm{95\%},间隔\textrm{5\%},即~$\mathrm{A}_{0.05}\mathrm{B}_{0.95}$,$\mathrm{A}_{0.10}\mathrm{B}_{0.90}$,$\mathrm{A}_{0.15}\mathrm{B}_{0.85}$,等等,每种组分,都可以正负缩放+微扰构造出几个新构型。同时,还可以分别按\textrm{fcc}、\textrm{bcc}、\textrm{hcp}等构建,这样就有更多构型了。
    \item 三元合金\\
    类似上述``二元合金''的方式做,具体做多少自己决定,但是要覆盖后续研究目标的典型组分。更多元合金以此类推。
    \item 从\textrm{Materials~Project}网站,把你体系包含的所有元素相关的结构都下载下来,以每个结构为基础,都可以进行微扰、缩放\\
	    注意,这些结构中部分原子可能距离很近了,所以微扰要注意,否则很容易让原子进一步靠近导致单点能计算时,力非常大,不太合适。
    \item 空位\\
    以前述的一些典型结构为基础,随机删除指定个数原子,获得空位结构。
    \item 表面\\
	    例如正常的$3\times3\times2$的\textrm{fcc}结构,把$z$的正方向盒子边长增大\textrm{8\AA},就是$z$正方向的表面构型了,以此类推可以做单表面、双表面等。
	    \textrm{AIMD}\\
	    例如要做低温到高温都适应的势函数,可以跑一系列温度点的\textrm{AIMD},例如\textrm{300~K}、\textrm{700~K}、\textrm{1100~K}、\textrm{1500~K}、\textrm{1900~K}、\textrm{2300~K}等等,温度间隔不需要太小,因为不同温度得到的构型,有部分重叠。值得注意的是,某些体系在某些特定温度会有特定结构或相,则对应温度附近温度点可稍密。\\
	    \textcolor{red}{跑\textrm{AIMD}的原则是,体系不崩溃的前提下,精度尽可能粗糙,步数尽可能多。}
	    \begin{itemize}
		    \item \textrm{NVT}精度参考:~\textrm{ENCUT=450},$k$点设置\textrm{111}~(或\textrm{k-spacing}为大于~1.5~的某个数),时间步长\textrm{3~fs}。
			    \item \textrm{NPT}精度参考:~\textrm{ENCUT=600}以保证\textrm{Pullay stress}为零(详见官网说明),$k$点\textrm{111},步长\textrm{3~fs}。
		    \item \textrm{NVT(NPT)+ML}:~精度如上,主要注意\textrm{ML\_MB}参数设置,还有\textrm{ML\_LOGFILE}的\textrm{LCONF}值,它与\textrm{ML\_MB}有一定对应关系,\textrm{ML\_MB}需要看这个值的变化设置,详见官网,需要测试。
	    \end{itemize}
    \item 单点能\\
    结合自己研究目标和文献,统一设置精度进行所有结构的单点能计算,形成训练集、测试集。
\end{itemize}
%    \#-----------------------
\subsection{训练势函数}
\begin{itemize}
	\item 设置\textrm{nep.in}文件的超参数,请咨询樊老师。
	\item 训练过程,请咨询樊老师。
	\item 验证势函数,请咨询樊老师。
\end{itemize}
%\#-----------------------------------------------------------------------------
%    III. 参考文献、资料
%\#-----------------------------------------------------------------------------
\begin{thebibliography}{99}
		\bibitem{Neuroevolution} Neuroevolution machine learning potentials: Combining high accuracy and low cost in atomistic simulations and application to heat transport. ~(\url{https://journals.aps.org/prb/abstract/10.1103/PhysRevB.104.104309})
		\bibitem{Development} Development of a deep machine learning interatomic potential for metalloid-containing Pd-Si compounds.~(\url{https://journals.aps.org/prb/abstract/10.1103/PhysRevB.100.174101})
		\bibitem{Modeling} Modeling refractory high-entropy alloys with efficient machine-learned interatomic potentials: Defects and segregation.~(见补充材料)~(\url{https://journals.aps.org/prb/abstract/10.1103/PhysRevB.104.104101})
		\bibitem{Machine} Machine learning based interatomic potential for amorphous carbon.~(\url{https://journals.aps.org/prb/abstract/10.1103/PhysRevB.95.094203)}
		\bibitem{GPUMD-NEP} GPUMD的NEP~(\url{https://gpumd.zheyongfan.org/index.php/Main\_Page#Inputs\_for\_the\_src.2Fnep\_executable})
		\bibitem{PyNEP} PyNEP~(\url{https://github.com/bigd4/PyNEP})
		\bibitem{VASP} \textrm{VASP} ~官网~(\url{https://www.vasp.at/wiki/index.php/The\_VASP\_Manual})
\end{thebibliography}
%\#-----------------------------------------------------------------------------

