%---------------------- TEMPLATE FOR REPORT ------------------------------------------------------------------------------------------------------%

%\thispagestyle{fancy}   % 插入页眉页脚                                        %
%%%%%%%%%%%%%%%%%%%%%%%%%%%%% 用 authblk 包 支持作者和E-mail %%%%%%%%%%%%%%%%%%%%%%%%%%%%%%%%%
%\title{More than one Author with different Affiliations}				     %
%\title{\rm{VASP}的电荷密度存储文件\rm{CHGCAR}}
%\title{面向高温合金材料设计的计算模拟软件中的几个主要问题}
%\title{适用于异质界面的高通量材料计算自动流程软件的结构与实现}
%\author[ ]{}   %
%\author[ ]{姜~骏\thanks{jiangjun@bcc.ac.cn}}   %
%\affil[ ]{北京市计算中心}    %
%\author[a]{Author A}									     %
%\author[a]{Author B}									     %
%\author[a]{Author C \thanks{Corresponding author: email@mail.com}}			     %
%%\author[a]{Author/通讯作者 C \thanks{Corresponding author: cores-email@mail.com}}     	     %
%\author[b]{Author D}									     %
%\author[b]{Author/作者 D}								     %
%\author[b]{Author E}									     %
%\affil[a]{Department of Computer Science, \LaTeX\ University}				     %
%\affil[b]{Department of Mechanical Engineering, \LaTeX\ University}			     %
%\affil[b]{作者单位-2}			    						     %
											     %
%%% 使用 \thanks 定义通讯作者								     %
											     %
%\renewcommand*{\Authfont}{\small\rm} % 修改作者的字体与大小				     %
%\renewcommand*{\Affilfont}{\small\it} % 修改机构名称的字体与大小			     %
%\renewcommand\Authands{ and } % 去掉 and 前的逗号					     %
%\renewcommand\Authands{ , } % 将 and 换成逗号					     %
%\date{} % 去掉日期									     %
%\date{2020-12-30}									     %

%%%%%%%%%%%%%%%%%%%%%%%%%%%%%%%%%%%%%%%%%%  不使用 authblk 包制作标题  %%%%%%%%%%%%%%%%%%%%%%%%%%%%%%%%%%%%%%%%%%%%%%
%-------------------------------The Title of The Report-----------------------------------------%
%\title{报告标题:~}   %
%-----------------------------------------------------------------------------

%----------------------The Authors and the address of The Paper--------------------------------%
%\author{
%\small
%Author1, Author2, Author3\footnote{Communication author's E-mail} \\    %Authors' Names	       %
%\small
%(The Address,City Post code)						%Address	       %
%}
%\affil[$\dagger$]{清华大学~材料加工研究所~A213}
%\affil{清华大学~材料加工研究所~A213}
%\date{}					%if necessary					       %
%----------------------------------------------------------------------------------------------%
%%%%%%%%%%%%%%%%%%%%%%%%%%%%%%%%%%%%%%%%%%%%%%%%%%%%%%%%%%%%%%%%%%%%%%%%%%%%%%%%%%%%%%%%%%%%%%%%%%
%%%%%%%%%%%%%%%%%%%%%%%%%%%%%%%%%%%%%%%%%%%%%%%%%%%%%%%%%%%%%%%%%%%%%%%%%%%%%%%%%%%%%%%%%%%%%%%%%%%%%%%%%%%%%%%%%%%%%
各位领导、各位专家,大家下午好。我是来自计算中心的姜骏,很有幸能有这样一个机会向大家汇报一下我近年来的工作。
\begin{itemize}
	\item 第2页\\
这是我的评审申请口头报告的内容,分为四个部分:~个人基本情况,研究的主要内容,科研业绩以及其他的一些内容。
\item 第3页\\
这是我的基本情况:~理学博士毕业,2016年到计算中心云平台事业部工作,2019年起评的副研究员。现在申报~材料科学领域-材料科学基础学科方向-成果转化与技术转移类型的研究员。
\item 第4页\\
我从参加工作以来,长期从事材料微观尺度模拟方向的研究工作,主要是材料计算方法与软件的开发,以高通量并发式自动流程为主;以及材料基本结构与性质模拟,这方面更多的是金属和半导体材料的微观结构与材料性质的数据的积累。

这些工作主要依托我承担和参与的3项国家级科研任务完成的,其中结题的有一项。\\
这三项任务分别是
新材料重大专项中的``人工智能支持的跨尺度材料分子动力学软件的力学性质功能模块开发'',它主要包括材料计算软件开发和材料力学性质模拟两部分内容;
国家自然基金项目中的``低维材料的激子性质第一原理计算软件的功能模块开发'',它也主要包括计算软件开发、微观结构计算和材料物性模拟;
已经结题的重点研发计划项目中的``高通量并发式材料计算自动流程软件'',它主要涉及材料计算软件的软件开发。

不难看出,我的研究内容是紧扣这几个方面展开的。
\item 第5页\\
以这些研究任务为基础,我们的工作也取得了一些成果,这些成果紧密贴和AI、特别是材料研究涉及的AI for Science方向迫切需要解决的计算自动化,也就是数据生产和数据积累的问题,而微观材料数据更是材料学研究最基础的数据底座。

所以我们在为科研团队提供技术服务的时候,特别注意到对材料结构和物性数据的积累,形成我们的自己的材料数据库。
\item 第6页\\
	在我们的研究成果还包括发明专利和软著,专利是以确定材料晶体结构和电子结构为特色;软著是开发材料计算自动流程并结合材料数据管理特色取得的。
\item 第7页\\
	这些年,在应用上述成果提供技术服务过程中,我也发表了一些论文,这些论文中前3篇也紧密围绕材料高效率和高通量计算,后3篇则主要是关于材料结构与性质模拟与分析,与之间前的研究任务密切关联。
\item 第8页\\
	此外还出版了一本专著,该书的第二部分,即计算理论和第四部分,即实践应用部分,是我执笔撰写的。这本书现在在各电商网站上已经是紧缺的溢价书,京东上卖到1300多元。
\item 第9页\\
	我单位作为国有企业,我们的研究成果(材料计算平台和结果分析)主要的转化和技术服务就是为各研究机构提供技术服务,在申评阶段,技术服务合同金额累计730万元;特别值得一提的是,由我完成的``金属与半导体材料微观尺度晶体和物理性质数据集''获得数据知识产权登记,并入选北京市数据知识产权登记十大典型案例。
\item 第10页\\
	我在计算中心组建了本单位计算材料团队,确立的团队目标是以微观尺度材料计算应用为牵引,以计算材料数据库建设为重点,为微观尺度材料计算的软件和计算服务。团队现有博士2人,硕士2人,出站博士后1人。\\
	2023年9月起,担任北京城市学院信息学院校外导师\\
	今年``数据要素x''大赛北京分赛``科技创新''赛道评审专家\\
	最近,我成为首届AI+新材料大会 组委会 专家组成员
\item 第11页\\
	以上是我近年来的工作业绩,作为研究员申报的基本信息,请各位专家评估。
\end{itemize}

