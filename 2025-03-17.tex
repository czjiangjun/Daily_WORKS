%---------------------- TEMPLATE FOR REPORT ------------------------------------------------------------------------------------------------------%

%\thispagestyle{fancy}   % 插入页眉页脚                                        %
%%%%%%%%%%%%%%%%%%%%%%%%%%%%% 用 authblk 包 支持作者和E-mail %%%%%%%%%%%%%%%%%%%%%%%%%%%%%%%%%
%\title{More than one Author with different Affiliations}				     %
%\title{\rm{VASP}的电荷密度存储文件\rm{CHGCAR}}
%\title{面向高温合金材料设计的计算模拟软件中的几个主要问题}
\title{计算材料团队近年来工作情况概述}
\author[ ]{}   %
%\author[ ]{姜~骏\thanks{jiangjun@bcc.ac.cn}}   %
%\affil[ ]{北京市计算中心}    %
%\author[a]{Author A}									     %
%\author[a]{Author B}									     %
%\author[a]{Author C \thanks{Corresponding author: email@mail.com}}			     %
%%\author[a]{Author/通讯作者 C \thanks{Corresponding author: cores-email@mail.com}}     	     %
%\author[b]{Author D}									     %
%\author[b]{Author/作者 D}								     %
%\author[b]{Author E}									     %
%\affil[a]{Department of Computer Science, \LaTeX\ University}				     %
%\affil[b]{Department of Mechanical Engineering, \LaTeX\ University}			     %
%\affil[b]{作者单位-2}			    						     %
											     %
%%% 使用 \thanks 定义通讯作者								     %
											     %
\renewcommand*{\Authfont}{\small\rm} % 修改作者的字体与大小				     %
\renewcommand*{\Affilfont}{\small\it} % 修改机构名称的字体与大小			     %
\renewcommand\Authands{ and } % 去掉 and 前的逗号					     %
\renewcommand\Authands{ , } % 将 and 换成逗号					     %
\date{} % 去掉日期									     %
%\date{2020-12-30}									     %

%%%%%%%%%%%%%%%%%%%%%%%%%%%%%%%%%%%%%%%%%%  不使用 authblk 包制作标题  %%%%%%%%%%%%%%%%%%%%%%%%%%%%%%%%%%%%%%%%%%%%%%
%-------------------------------The Title of The Report-----------------------------------------%
%\title{报告标题:~}   %
%-----------------------------------------------------------------------------

%----------------------The Authors and the address of The Paper--------------------------------%
%\author{
%\small
%Author1, Author2, Author3\footnote{Communication author's E-mail} \\    %Authors' Names	       %
%\small
%(The Address,City Post code)						%Address	       %
%}
%\affil[$\dagger$]{清华大学~材料加工研究所~A213}
%\affil{清华大学~材料加工研究所~A213}
%\date{}					%if necessary					       %
%----------------------------------------------------------------------------------------------%
%%%%%%%%%%%%%%%%%%%%%%%%%%%%%%%%%%%%%%%%%%%%%%%%%%%%%%%%%%%%%%%%%%%%%%%%%%%%%%%%%%%%%%%%%%%%%%%%%%%%%%%%%%%%%%%%%%%%%
\maketitle

%\thispagestyle{fancy}   % 首页插入页眉页脚 
北京市计算中心有限公司的计算材料团队主要的研究任务包括:~应用第一原理和分子动力学方法及软件模拟和优化材料的功能、性质,构建材料结构-性质的计算数据库以及从事材料计算模拟相关的软件开发。团队当前共有博士四人,硕士一人(一人为在读博士)。近年来,团队通过承担国家重点研发计划项目、重大专项和国家自然科学基金面上项目等纵向研发课题,围绕材料学计算和数据方向的以下方面开展:
\begin{itemize}
%	\item %``高通量并发式材料计算算法和软件''(编号:~\textrm{2017YFB0701500})和``产学研用协同的高通量材料计算融合服务平台''(编号:~\textrm{2018YFB0704300})
%	\item 北科院青年骨干计划项目``基于甲烷催化燃烧机理的材料计算自动流程设计''(编号:~\textrm{YC201820})
	\item 反应堆材料的抗辐照行为与微观机理研究
	\item 材料微观晶体和性质计算数据库建设
	\item 碳催化-反应机理及催化材料与反应知识图谱建设
\end{itemize}

团队通过完成国家重点专项(均已结题)``高通量并发式材料计算算法和软件''(项目编号:~\textrm{2017YFB0701500})和``产学研用协同的高通量材料计算融合服务平台''(项目编号:~\textrm{2018YFB0704300}),开发了针对于高温合金电子结构计算的高通量并发式计算的自动流程软件,并与上海交通大学团队协同提升材料电子计算单线程与整体性能的算法优化、调度算法;独立完成晶体对称性分析工具模块,支持晶体空间群分析和电子能带路径的标准化自动生成。共发表论文3篇,获得软件著作权1项,专利1项。

目前,团队在研的纵向课题包括国家自然基金面上项目``低维材料等离和激子极化激元的第一性原理研究''(项目编号:~\textrm{12474217})和重大专项``基于人工智能技术的高性能多尺度分子动力学模拟平台''(项目编号:~\textrm{2024ZD0606900}),主要围绕金属慢化反应堆的主要材料锆化氢的动力学行为展开研究。氢化锆因具有较高的热稳定性、较高的氢密度、低的中子捕获截面以及良好的导热性,成为核反应中常用的固体中子慢化材料。已在重量轻、体积小的反应堆及液态金属冷却的热中子反应堆中付诸应用。但是将氢化锆用于反应堆,必须解决一个重要的问题是工作温度对氢化锆影响问题。作为中子慢化剂,氢化锆的工作温度在$\mathrm{650\sim750}^{\circ}\mathrm{C}$之间,在该温度范围内,原子比大于1.8的氢化锆的氢分解压远远高于一个大气压,即氢化锆中的氢将很容易析出。氢的析出一方面增大包壳中的气相压力,另一方面则使氢化锆中原子比减小(氢含量降低),会使其中子慢化能力减弱,最终失去中子慢化的能力。因此,阻止或减缓氢的析出,是能否将氢化锆应用于小型反应堆而必须解决的一个关键问题。%国内外对这方面的研究资料表明,为了阻止或减少氢的析出,在材料表面建立防氢渗透层。
研究氢化锆中氢原子的扩散和辐照动力学行为对发展空间技术、开展深空探测具有重要意义。为了提升作为慢化剂部件的氢化锆的质量金和可靠性,团队联合北京科技大学、中国有色金属研究院,针对氢化锆固态辐照模型开展第一性原理和分子动力学模拟:~通过累积体系基态能量和原子受力的第一原理计算数据,应用数据挖掘技术,构建氢化锆的高精度机器学习势函数,在此基础上应用\textrm{GPU}加速的分子动力学模拟,研究中子辐照下氢化锆辐照缺陷和元素偏析的形成与演化,以及氢析出的动力学机制。建设氢化锆计算-实验-生产-应用的数据库,开展数据驱动的氢化锆成分和工艺的优化设计。

团队在完成支持合金、多相催化的异构化材料计算自动流程开发过程中,对现有国际常用材料计算流程与数据库及其代码实现作了系统剖析,掌握了包括\textrm{Material Projects}、\textrm{ASE}在内的多种计算软件自动流程与数据库实现方案。基于自动流程,搭建了稀土和半导体材料微观晶体和物理性质数据库。该数据库涵盖稀土金属硼化物、过渡金属氧化物、镍基单晶高温合金模型的晶体结构和电子结构(能带、态密度)和磁学、光学性质等,总数据量约为1000余条,有效可读数据1G。数据以``北京-稀土与碱土金属化合物、半导体材料电子结构数据''形式在北京国际大数据交易有限公司完成了数据资产登记,并以``金属与半导体微观尺度晶体和物理性质数据集''形式完成了数据知识产权登记。氢化锆材料计算模拟过程中,也将形成相当可观的数据规模,这些数据也将以反应堆材料结构与性质数据形式保存到数据库中。

随着人工智能技术影响的不断扩大,知识图谱作为整合垂直领域行业知识的重要方式,实现了相关概念描述的实体-属性的有效关联。以化学-化工领域为例,通过组织、表示和存储化学科学、化工和领域知识的特定类型的概念,可以构建并整合适应化学-化工垂直领域相关的实体(如化合物、元素、分子、反应等)、关系(如化学反应、化学结构性质等)以及属性(如物理性质、化学性质、命名等)的多维知识网络。在人工智能通用模型的基础上,此类具体的知识网络无疑将有助于模型对垂直领域知识的更好地理解和分析,并基于自然语言处理和信息检索技术,实现智能体对垂直领域知识的准确检索和表达,用户也可以通过问答或关键词来获取相关的知识和信息。团队联合中科合成油技术股份有限公司,针对煤化工研究的专业知识,搭建了一个基于化工知识的知识图谱系统,通过集成实体识别、实体查询、关系查询以及化工知识概览等功能,为用户提供便捷、高效的煤化工相关化工知识获取途径。借助图谱技术,将有助于煤化工研究中涉及的专业知识和信息的可视化展示,并形成知识点之间的有效关联结构,促进煤化工研究中知识点的规范化表示与整合。

%\section{院青年骨干项目}
%开发适合催化材料微观机理模拟的\textrm{DFT-MD}自动流程软件,研究动力学约化耦合算法、高通量基元反应\textrm{Kohn-Sham}方程筛选与调度平衡算法。以$\mathrm{CH}_4$催化燃烧反应研究为牵引,面向微观尺度下异相界面催化燃烧反应动力学机理模拟计算的高效率实现。通过基元反应活化能确定影响进程的决速步,筛选对反应机理干扰的大量自由基反应;优化决速步反应得到的势能面(\textrm{DFT-MD}耦合的关键)提升分子反应动力学计算的迭代稳定性为共性特征,形成含有多决速步或非关联第一原理计算流程作业框架,结合具体的燃烧反应动力学求解流程实例化,得到适用于催化反应动力学研究的流程版本,经优化的并发\textrm{DFT}求解流程可将电子步计算自动流程的计算效率提高20\%左右。研制适应多决速步化学反应的复杂反应动力学机理研究的自动流程软件,并在商业计算机和国产高性能超级计算机上成功测试运行,成为可跨平台典型催化燃烧反应动力学模拟的自动流程软件。完成该项目,本单位发表论文1篇,获得软件著作权1项。

此外,自2022年秋起,团队与北京航空航天大学、北京低碳清洁能源研究院中科院化学所的相关课题组展开合作,基于第一原理和分子动力学的理论和方法,研究低维材料等离和激子极化激元、$\mathrm{CO}_2$催化还原的催化机理(特别是针对金属氧化物$\mathrm{CeO2}_2$担载金属$\mathrm{Ni}$的界面,对$\mathrm{CO}_2$经$\cdot\mathrm{COOH}$还原为$\mathrm{CO}$,研究可能的催化动力学过程;$\mathrm{CO}$在金属$\mathrm{Ni}$表面与$\mathrm{H}$作用,形成$\mathrm{CH}_4$的可能动力学机理)以及有机手性分子的堆叠动力学过程涉及的反应机制,相关研究工作正陆续开展。%团队的计算表明,$\mathrm{CO}_2$在过渡金属氧化物表面的吸附和缺陷作用是$\mathrm{CO}_2\rightarrow\mathrm{CO}$的重要催化过程,而过渡金属$\mathrm{Ni}$的主要作用是分解还原剂$\mathrm{H}_2$,为还原反应提供足够的活化$\mathrm{H}$原子。因此较好地说明了金属氧化物担载金属催化剂的耦合作用。相关研究结果正在整理成研究论文。

完成上述工作的同时,团队成员合作完成专著《计算材料科学理论与实践》的编写,并于2021年由人民邮电出版社出版。
