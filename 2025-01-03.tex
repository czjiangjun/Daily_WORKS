%---------------------- TEMPLATE FOR REPORT ------------------------------------------------------------------------------------------------------%

%\thispagestyle{fancy}   % 插入页眉页脚                                        %
%%%%%%%%%%%%%%%%%%%%%%%%%%%%% 用 authblk 包 支持作者和E-mail %%%%%%%%%%%%%%%%%%%%%%%%%%%%%%%%%
%\title{More than one Author with different Affiliations}				     %
%\title{\rm{VASP}的电荷密度存储文件\rm{CHGCAR}}
%\title{面向高温合金材料设计的计算模拟软件中的几个主要问题}
\title{化学-化工知识图谱的建设与应用}
\author[ ]{北京北科融智云计算科技有限公司}%{北京市计算中心有限公司}   %
%\author[ ]{姜~骏\thanks{jiangjun@bcc.ac.cn}}   %
%\affil[ ]{北京市计算中心}    %
%\author[a]{Author A}									     %
%\author[a]{Author B}									     %
%\author[a]{Author C \thanks{Corresponding author: email@mail.com}}			     %
%%\author[a]{Author/通讯作者 C \thanks{Corresponding author: cores-email@mail.com}}     	     %
%\author[b]{Author D}									     %
%\author[b]{Author/作者 D}								     %
%\author[b]{Author E}									     %
%\affil[a]{Department of Computer Science, \LaTeX\ University}				     %
%\affil[b]{Department of Mechanical Engineering, \LaTeX\ University}			     %
%\affil[b]{作者单位-2}			    						     %
											     %
%%% 使用 \thanks 定义通讯作者								     %
											     %
\renewcommand*{\Authfont}{\small\rm} % 修改作者的字体与大小				     %
\renewcommand*{\Affilfont}{\small\it} % 修改机构名称的字体与大小			     %
\renewcommand\Authands{ and } % 去掉 and 前的逗号					     %
\renewcommand\Authands{ , } % 将 and 换成逗号					     %
\date{} % 去掉日期									     %
%\date{2020-12-30}									     %

%%%%%%%%%%%%%%%%%%%%%%%%%%%%%%%%%%%%%%%%%%  不使用 authblk 包制作标题  %%%%%%%%%%%%%%%%%%%%%%%%%%%%%%%%%%%%%%%%%%%%%%
%-------------------------------The Title of The Report-----------------------------------------%
%\title{报告标题:~}   %
%-----------------------------------------------------------------------------

%----------------------The Authors and the address of The Paper--------------------------------%
%\author{
%\small
%Author1, Author2, Author3\footnote{Communication author's E-mail} \\    %Authors' Names	       %
%\small
%(The Address,City Post code)						%Address	       %
%}
%\affil[$\dagger$]{清华大学~材料加工研究所~A213}
%\affil{清华大学~材料加工研究所~A213}
%\date{}					%if necessary					       %
%----------------------------------------------------------------------------------------------%
%%%%%%%%%%%%%%%%%%%%%%%%%%%%%%%%%%%%%%%%%%%%%%%%%%%%%%%%%%%%%%%%%%%%%%%%%%%%%%%%%%%%%%%%%%%%%%%%%%%%%%%%%%%%%%%%%%%%%

% 目录
\newpage
\pagestyle{empty}    % 清空页码                                      %
\renewcommand{\contentsname}{\centering 目录} % 自定义目录标题并居中
\tableofcontents
\newpage
\pagestyle{fancy}   % 插入页眉页脚 
\setcounter{page}{1}
\maketitle
%\thispagestyle{fancy}   % 首页插入页眉页脚 
\section{引言}
知识图谱\textrm{(Knowledge~Graph)}是一种用于组织、表示和存储知识的图形化数据结构形式,目的是仿照人类对于知识的认知、理解方式,将实体\textrm{(Entities)}、关系\textrm{(Relationship)}和属性\textrm{(Attributes)}以图形的形式呈现出来,使计算机能够更好地认知、理解和推理知识。早在1960年代,就曾出现过以模拟仿真实人类思维-决策流程的专家系统,称为知识工程\textrm{(Knowledge Engineering, KE)}\upcite{Knowledge_Engineering}。此后不久涌现了多种商务的专家与知识管理系统,一般统称为知识库系统\textrm{(knowledge-based system)}。在知识库系统基础上,通过用户人机问答式互动,可以构建结构化的机器可读与生成式的文本知识,而运维知识库系统的,通常是行业和领域专家(如图\ref{Fig:Knowledge-based_system}所示)。
\begin{figure}[h!]
\centering
\includegraphics[height=1.85in,width=5.85in,viewport=0 0 1500 475,clip]{Figures/Knowledge-based_system.png}
\caption{\small\textrm{The minimal components of a knowledge-based system. cite from~\cite{ACR56-128_2023}}}%(与文献\cite{EPJB33-47_2003}图1对比)
\label{Fig:Knowledge-based_system}
\end{figure}

%\subsection{知识的逻辑-推理}
人类基于逻辑对知识的推理形式为演绎\textrm{(deductive)}、归纳\textrm{(inductive)}和溯因\textrm{(abductive)}。在人工智能系统中,溯因已经是广为接受但也是最优挑战的推理形式。从人类认知的角度看,逻辑推理减少的脑力思维称为启发。但是启发也模糊了“规则”和“类似”。 虽然归纳和演绎方法论上的差别,由于其推理对已有前提深度依赖,所以即使前提数据不完备(比如存在个别例外),认知的结果可能仍然是一致的。但是在溯因推理中,这种情形会需要更高层次的抽象,这对于人类专家也是非常大的挑战。

进入21世纪以来,得益于计算能力的大幅度提升和语义网\textrm{(Semantic Web)}\upcite{SA284-34_2001}技术与开源软件的发展,知识图谱作为继知识工程后的计算机辅助学习和智能数据支撑工具得到了长足发展\upcite{SA141-112948_2020,IEEETNNLS33-494_2022},特别是在%以下是一些关于知识图谱在近年发展趋势:
自然语言处理\textrm{(Natural Language Processing,~NLP)}、%中语言处理任务的改进与问答系统的文档与摘要生成,搜索引擎优化中的知识卡片及相关结果的丰富: 知识图谱被广泛用于改进自然语言处理任务,如问答系统和文档摘要生成。此外,搜索引擎如Google使用知识图谱来提供更丰富的搜索结果,例如知识卡片和相关问题。 
医疗和生命科学中的辅助精准诊疗与决策、%: 知识图谱在医疗诊断、药物研发和疾病管理方面的应用迅速增长。它们用于。 
智能城市和物联网\textrm{(Internet of Things,~IoT)}、%: 知识图谱在智能城市项目中发挥关键作用,帮助城市管理者优化资源分配、交通管理和环境监测。此外,它们在IoT中用于设备和传感器之间的数据整合和智能控制。 
金融和风险管理中的复杂金融关系与风险识别、个性化投资建议等领域,都起到了重要的业务提升和精准服务的作用。%此外在: 在金融领域,知识图谱用于分析复杂的金融关系、识别风险和欺诈行为,以及提供个性化的投资建议。它们有助于机构更好地理解市场动态和客户行为。 
%智能助手和虚拟助手: 知识图谱被广泛用于智能助手和虚拟助手,如Siri、Alexa和Google助手。这些助手使用知识图谱来回答用户的问题、提供建议和执行任务。 
%教育和培训: 在教育领域,知识图谱被用于个性化学习路径的设计,帮助学生更有效地学习和掌握知识。 
%企业知识管理: 企业越来越多地使用知识图谱来管理内部知识资源,以促进知识共享、问题解决和决策支持。 
%可持续发展和环境保护、: 在可持续发展领域,知识图谱有助于整合环境数据、能源消耗和碳排放等信息,以支持可持续发展决策。 
%卫生和流行病学、: 知识图谱在流行病学研究中有应用潜力,可以帮助疾病控制机构追踪疾病传播、预测爆发并采取预防措施。 
%多模态知识图谱等不同领域,: 近年来,研究者开始探索将文本数据、图像、声音和其他多媒体数据整合到知识图谱中,以更全面地表示世界的知识。
知识图谱是基于语义网技术发展起来的,网格化结构的整合能力强,可实现异质数据的相互链接,确保了知识图谱可以被软件自动接受。知识图谱对人类专家系统的决策仿真,除了传统的问答式输出外,在软件支持下,拥有了学习和推理能力,也具备初级的创造知识的能力。因此非常适合跨专业领域、跨空间的应用场景。%在化学和化工研究领域,具有专业深度知识背景的图谱还处于起步建设阶段。%数据当前的科学数据来源,中文文献检索数据主要源于知网\textrm{(CNKI)},英文文献则主要来源\textrm{(Web of science,~WOS)}核心库 
近年来,机器学习广泛应用于化学知识的提取,特别是对于有足够的关系清晰的数据,效果非常显著。原则上,机器学习基于离散数据回归来处理数据,并不需要理解数据间的行为和关联。如果数据关联是基于统计的相关性,机器学习也可以视为归纳推理。另一方面,知识图谱是基于知识和领域专家经验构建的,当数据稀少时,就可以应用图谱算法而不必清洗或处理数据。\upcite{JACS144-11713_2022}

%煤化工技术是指以煤为原料生产各种能源或化工产品的工艺过程,一般包括煤炭转化和后续加工
机器仿真接受和感知知识的实现,是通过对数据赋予一定的意义,并寻找数据间的关系来完成的。传统的关系型文本数据间是通过表格直接相互关联的,一旦知识结构发生变化,这种表达形式成为极大的制约。知识图谱以图形结构的方式,通常使用节点(实体)和连线/边(关系)的形式来表示知识。这种结构使得知识点之间的关联关系更加清晰和可视化。当有新的数据和关系加入知识图谱时,不会影响原有的知识结构。它主要的关键词为:
\begin{itemize}
	\item 实体\textrm{(Entities)}:~知识图谱中的实体是指具体的事物、概念、人物、地点等,每个实体都有一个唯一的标识符。例如,在一个化学知识图谱中,分子、合成体、密度等都可以是实体。
	\item 关系\textrm{(Relationships)}:~实体之间的关系表示不同实体之间的连接或互动。这些关系可以是有向的或无向的,用于描述实体之间的各种联系,如"具有"、"属于"、"值为"等。
	\item 属性\textrm{(Attributes)}:~实体可以有一些描述性的属性,这些属性是与实体相关的额外信息。例如,一个分子实体可以有属性包括分子量、化合价、密度等。
\end{itemize}

知识图谱是一种强大的工具,用于组织、存储和分析知识,它在推动人工智能、自然语言处理和数据科学领域的发展中发挥着重要作用。知识图谱的应用领域在不断扩展,对于解决复杂的问题和推动科学研究具有巨大潜力。其主要的特点是:
\begin{itemize}
	\item 语义丰富:~知识图谱不仅仅是简单的数据存储结构,它还具有语义信息,可以帮助计算机更好地理解知识。例如,通过关系的定义,计算机可以知道"父亲"关系是指一个人与另一个人之间的亲属关系。
	\item 查询和推理:~知识图谱支持复杂的查询和推理操作,允许用户通过查询知识图谱来获取有关实体和关系的信息,并进行逻辑推理以回答复杂的问题。
	\item 应用领域丰富:~知识图谱在多个领域有广泛的应用,包括自然语言处理、搜索引擎优化、智能助手、数据挖掘、推荐系统、医疗诊断、金融风险分析等。
	\item 标准化:~知识图谱的创建和管理通常遵循特定的标准和规范,以确保数据的一致性和互操作性。例如,资源描述框架\textrm{(Resource Description Framework, RDF)}和编程语言\textrm{WOL~(Web Ontology Language)},是常用于表示知识图谱的标准。
	\item 可持续更新:~知识图谱需要不断更新和维护,以反映知识的最新发展和变化。需要自动化的数据抓取和人工编辑的结合。
\end{itemize}

随着人工智能\textrm{(Artificial Intelligence AI)}、数据科学和信息管理技术的不断进步和应用领域的扩展,可以预见知识图谱也有望发挥更大的作用。%未来化学和化工知识图谱的发展有望进一步推动科学研究、新材料开发和绿色化学等领域的创新。

知识图谱仿真人类认知解决和认知问题的三个阶段如图\ref{Fig:Three-main-stages-in-KE-pro-dev}所示:~标准化\textrm{(specification)}-概念化\textrm{(conceptualization)}-应用\textrm{(implementation)}
\begin{figure}[h!]
\centering
\includegraphics[height=1.85in,width=5.85in,viewport=0 0 1500 430,clip]{Figures/Three-main-stages-in-KE-project-development.png}
\caption{\small\textrm{The three main stages in Knowledge Graph project development. cite from~\cite{ACR56-128_2023}}}%(与文献\cite{EPJB33-47_2003}图1对比)
\label{Fig:Three-main-stages-in-KE-pro-dev}
\end{figure}
在标准化阶段,领域专家根据认知反馈,确定其理解的知识点和如何认知知识点的规范过程,支持团队定义一套投喂知识图谱认知所需的完整问答/判断。这两项操作将有效聚焦知识图谱的认知对象,为概念化提供坚实的基础。概念映射将是知识的半正式表示,并对关联的实体有基本的确定,领域专家可以根据概念映射定义或设计适合问答/判断的算法。

在应用阶段,概念映射的实体实现本体化\textrm{(ontologized)},专家在本体化形式基础上梳理信息并实例化,由此完成\textrm{ABox}。然后,软件在设计的算法基础上程式化地传递知识并作出推断。整个系统经过验证后运行,遍历三阶段反复迭代也并不罕见,可以得到更好的输出结果\upcite{Handbook-Ontology}。

\section{化学-化工知识图谱}
在化学-化工领域,可以通过组织、表示和存储化学科学、化学工程和领域知识的特定类型的知识图谱。构建化学-化工知识图谱的目标是捕捉和整合与化学科学、化学工程和领域相关的实体(如化合物、元素、分子、反应等)、关系(如化学反应、化学结构、性质等)以及属性(如物理性质、化学性质、命名等),以便计算机可以更好地理解和分析化学与应用化学的信息。以下是有关化学-化工知识图谱的一些关键方面: 
\begin{itemize}
	\item 实体:~化学科学的实体包括元素、化合物、分子、离子、反应、化学方程式、化学领域的科学家和研究机构等;化学工程的实体包括化工工艺、化工设备、化学反应、化工原料、化工产品、化工企业、化工工程师等。每个实体都具有唯一的标识符。
	\item 关系:~关系用于描述不同实体之间的连接或互动,例如,化合物之间的化学反应、元素之间的关系、实验数据与化学实体之间的关联等;化工工艺中的流程步骤、设备之间的连接、原料与产品之间的转化等。
	\item 属性:~化学知识图谱中的实体可以具有各种属性,包括物理性质(如密度、熔点、沸点)、化学性质(如酸碱性、溶解性)、结构信息(如分子式、分子结构)、命名规范、工艺参数(如产量、效率)等。
	\item 图形结构:化学知识图谱通常以图形结构的形式表示,其中节点表示化学-化工实体,边表示实体之间的关系。这种结构有助于可视化和查询化学科学与化学工程的信息。
	\item 领域知识的表示:化学知识图谱可以捕捉领域专家的知识,包括领域特定的术语、概念、分类体系、工艺流程、设备设计规范、化工安全标准和规则等。
	\item 查询和推理:化学知识图谱支持各种查询操作,使用户能够检索与特定化学实体或关系相关的信息。还可以进行推理操作,以推断新的化学关系或性质。
	\item 应用领域:化学知识图谱在药物研发、材料科学、环境科学、化学教育、化学信息检索、化学工程、工业生产、工艺优化、化工安全、环境保护等领域有广泛的应用。它有助于提高科学研究与工业生产效率、减少环境风险,并促进化学科学研究者与化工工程师的决策制定。
	\item 数据来源:构建化学知识图谱需要整合来自各种数据源的信息,包括文献、化学软件、工艺设计文档、计算与实验数据、安全手册、化工数据库等等。
	\item 标准和本体:化学-化工知识图谱的创建和管理通常需要依靠化学-化工领域的标准和本体,以确保数据的一致性和互操作性。
\end{itemize} 
总的来说,化学知识图谱有助于推动化学领域的研究和应用,使科学家和工程师能够更好地利用化学知识来解决问题、设计新材料和药物,以及改进和加速化学与化工工程的创新和发展,提高生产效率,并确保工业过程的可持续性和安全性。这种图谱的发展有助于研究人员、工程师和决策者更好地利用和分享化工知识,也将加速化学领域的创新和发展。

\section{国外的化学-化工知识图谱建设现状}
语义网概念出现不久,化学信息学研究者就考虑如何助力化学家\upcite{JCIM46-939_2006,Nature451-648_2008}。\textrm{M.~Kraft}等\upcite{TRSA368-3633_2010}从更广泛的层面考虑化学语义实例的应用,在该文中,作者通过讨论两个专门主题(化工复合物和燃烧的环境影响)的语义实例,试图建立分子尺度化学对宏观现象中复杂性、环境和健康的影响。他们构建了基本的化学-化工图谱\textrm{J-Park Simulator~(JPS)},\textrm{JPS}包含了除化工过程对产品和环境的影响之外的很多内容,如物流、基础设施和能耗与废物处理等\upcite{EP75-1536_2015,AE175-305_2016,CCE118-49_2018,CCE131-106586_2019,CCE130-106577_2019}。通过创建此数字化工具,产生了一个更宏大的规划,即\textrm{The World Avatar~(TWA)}项目\upcite{DCE1-e6_2020,DCE2-e10_2021}。\textrm{TWA}项目的层次规划如图\ref{Fig:Three_Layers-of-TWA}所示。
\begin{figure}[h!]
\centering
\includegraphics[height=3.55in,width=5.85in,viewport=0 0 1380 800,clip]{Figures/Three_Layers-of-TWA-digital-twin-real_world.png}
\caption{\small\textrm{Three layer of TWA (\url{www.theworldavatar.com}) digital twin of the real world. cite from~\cite{ACR56-128_2023}}}%(与文献\cite{EPJB33-47_2003}图1对比)
\label{Fig:Three_Layers-of-TWA}
\end{figure}
\textrm{TWA}可以看成基于语义网技术的通用数字孪生\textrm{(Digital twins)}技术\footnote{数字孪生系统可以根据人员、设备或系统的基础,在信息化平台上创造一个数字版的“克隆体”,“克隆体”可以模拟实际设备或系统的发展走向。数字孪生的本质是信息建模,旨在为现实世界中的实体对象在数字虚拟世界中构建完全一致的数字模型,但数字孪生涉及的信息建模已不再是基于传统的底层信息传输格式的建模。}知识图谱映射到整个现实世界,\textrm{TWA}的数据满足\textrm{FAIR}原理\footnote{\textrm{FAIR}原理表示数据\textrm{Findable}、\textrm{Accessible}、\textrm{Interoperable}、\textrm{Reusable}\upcite{SD3-160018_2016}。}。

当前很多高科技公司,包括\textrm{Google}、\textrm{IBM}、\textrm{Microsoft}和\textrm{eBay}等都在应用企业级的知识图谱\upcite{Queue17-48_2019}。制药公司\textrm{AstraZeneca}是新药制备领域应用知识图谱的领先者\upcite{NC13-1667_2022}。

%\section{知识图谱建设}
知识图谱具有知识构建的能力,借助机器学习\textrm{(Machine Learning,~ML)},可以在现有知识基础上产生(如外推)新的知识。必须解决化学-化工知识的格式化表示\textrm{(Formal Representation)}、基本逻辑推理和化学知识的产生等相关问题。
\subsection{化学知识的格式化表示}
知识图谱的知识点表示的逻辑结构\textrm{(schema)},将术语(或实体)相关的整体表示成\textrm{TBox}\footnote{引入\textit{Box}的概念后,实体依然被视作点,而关系则被视作$n$维空间的中的边,点和边组成的区域就被称为\textit{Box}。}\textrm{(terminological box)},这种实体-关系的描述习惯上称为\textrm{Ontology}\footnote{\textrm{Ontology}直译为本体论,这里是指使用逻辑形式化的方法规定概念与关系,也就是“主\textrm{(subject)}-谓\textrm{predicate}-宾\textrm{(object)}”的谓词逻辑,使人或机器可以用统一的、准确的推理方法处理数据。}由\textrm{Ontology}出发表示的实体-实体关系为\textrm{ABox~(Assertion component)}。图\ref{Fig:Mapping-relationship-molecule-synthon}给出分子与合成体的图谱表示。 
\begin{figure}[h!]
\centering
\includegraphics[height=4.90in,width=5.85in,viewport=0 0 950 790,clip]{Figures/Mapping-the-relationship-between-molecule-and-synthon.png}
\caption{\small\textrm{Mapping the relationship molecule (chemical) and synthon (abstract) concepts and illustrating them with instrances. cite from~\cite{ACR56-128_2023}}}%(与文献\cite{EPJB33-47_2003}图1对比)
\label{Fig:Mapping-relationship-molecule-synthon}
\end{figure}

\subsection{作为知识生态组成的化学知识图谱}
文献\cite{ACR56-128_2023}详细讨论了作为\textrm{TWA}知识生态\textrm{(knowledge ecosystem)}构成的中化学知识图谱及相关软件。
%\subsection{化学物种}
在\textrm{TWA}中,化学物种及其性质由\textrm{ontology~OntoSpecies}表示,如图\ref{Fig:OntoSpecies-to-segments-TWA}所示。\textrm{OntoSpecies}中的化学物种主要纪录分子式、电荷、分子量和自旋多重度。不同同位素、电荷和自旋态表示的化学物种不相同。因为\textrm{IRIs~(Internationalized Resource Identifiers)}包含\textrm{UUIDs~(universally unique identifiers)},因此可通过对化学物种赋不同的\textrm{IRIs},\textrm{OntoSpecies}就可以数字表示同位素,以区分实验的、氧化-还原与电化学驱动过程和光化学下的物种。对于反应物模拟,\textrm{OntoSpecies}纪录了特定反应条件下的反应标准生成焓\textrm{(standrd enthalpy)}。\upcite{CCE137-106813_2020}
\begin{figure}[h!]
\centering
\includegraphics[height=3.40in,width=5.85in,viewport=0 0 1170 700,clip]{Figures/Connection-of-OntoSpecies-to-segments-of-KG.png}
\caption{\small\textrm{Connection of OntoSpecies to other segments of TWA KG. cite from~\cite{ACR56-128_2023}}}%(与文献\cite{EPJB33-47_2003}图1对比)
\label{Fig:OntoSpecies-to-segments-TWA}
\end{figure}

用\textrm{IRIs}标签化学物种是机器可操作的,但不方便化学家识别,因此\textrm{TWA}中的化学物种需要进一步添加通用化学信息学标签\upcite{COC26-33_2019},如\textrm{\textrm{InChI}}、\textrm{InChIKey}和\textrm{CAS}注册码、\textrm{PubChemCID}和\textrm{SMILES}等,如图\ref{Fig:Key-OntoSpecies-and-external-concepts}所示。
\begin{figure}[h!]
\centering
\includegraphics[height=3.20in,width=4.35in,viewport=0 0 990 750,clip]{Figures/Key_OntoSpecies-and-external_concepts.png}
\caption{\small\textrm{Key OntoSpecies (black) and external (blue) concepts, along with a number of properties (green) used to describe chemical species in TWA KG. cite from~\cite{ACR56-128_2023}}}%(与文献\cite{EPJB33-47_2003}图1对比)
\label{Fig:Key-OntoSpecies-and-external-concepts}
\end{figure}

\subsection{反应复杂性研究}
很多化学反应和自组装过程都是由亚稳的简单分子前驱体(化学物种)引发的,对化学反应过程的模拟和反应机理的理解需要考虑热力学和动力学因素。为通过语义学研究化学反应的物种,\textrm{M.~Kraft}开发了\textrm{OntoKin}和\textrm{OntoCompChem}\upcite{JCIM59-3154_2019},如图\ref{Fig:Automated-linking-between-OntoSpecies-Kin-CompChem}所示。\textrm{OntoKin}是用于计算机辅助设计的标准命名下表示反应机理的算法的\textrm{ontology},化学反应过程中,反应机理由成比例的化学物种表示。在\textrm{OntoKin}中,化学反应由反应物和产物描述,反应物和产物由\textrm{OntoSpecoes~IRIs}的热力学和输运模型标签。在\textrm{OntoKin}中还会标记反应发生的形式(如气相、表面等)。每个化学反应的速度由\textrm{Arrhenius}模型表示,可通过调节温度、压力来控制气相反应动力学(即计算反应速率)。描述一个化学反应可能涉及很多反应步骤,所以\textrm{OntoKin}结合\textrm{OntoSpecies}就可以提供数据和模型,并可与文献中报导的动力学、热力学或输运模型对比\upcite{ACSO5-18342_2020}。所以这样的知识图谱框架可用于评估专家经验判断的可靠程度。\textrm{OntoCompChem}目前主要关注分子,图谱用密度泛函理论\textrm{(DFT)}的输入输出表示\upcite{JCIM59-3154_2019}。\textrm{OntoCompChem}是基于由\textrm{CompChem}规定的\textrm{CML~(Chemical Markup Language)}\upcite{JCI4-15_2012}的语义概念发展起来的。\textrm{OntoCompCHem}中对计算的描述包括:
\begin{itemize}
	\item 计算对象(单点计算,几何结构优化和频率计算)
	\item 使用的计算软件(如\textrm{Gaussian~16})
	\item 计算中采用的方法,包括泛函和基组(如\textrm{B3LYP,6-31G(d)})
	\item 电荷与自旋的极化
\end{itemize}
图谱中也纪录了前线轨道\textrm{frontier orbitals}和收敛的自洽迭代能量。对几何结构优化,纪录最终优化的几何结构;而频率计算,纪录的是零点能的校正和几何结构对应的完整的振动频率。

利用组织衔接软件\footnote{组织衔接软件是能够感知环境、进行决策和执行动作的智能处理软件,也称为\textrm{Agent}。该软件的设计和训练,需要结合机器学习和人工智能技术,如强化学习、深度学习等。},可以将化合物、化学反应和\textrm{DFT}衔接在一起\upcite{CCE137-106813_2020}。因为\textrm{OntoKin}中可能涉及到成千上万的化学物种和化学反应,所以衔接软件\textrm{Linking}是非常必要的\upcite{ACSO5-18342_2020}。组织衔接软件工作方式类似于人类代理:~能接收输入数据(如传感器信息、文本、图像等),通过分析和处理数据,理解环境和任务要求,并做出相应的决策和行动。通过与环境的交互和反馈,组织衔接软件可以逐步改进性能和表现,实现好的任务执行能力。组织衔接软件的核心功能是感知、推理和决策:
\begin{itemize}
	\item 感知:~通过传感器等方式获取环境信息的能力,例如通过摄像头获取图像或通过麦克风获取声音
	\item 推理:~基于获取的信息进行逻辑推理和分析的能力,以了解环境和任务需求
	\item 决策:~根据推理结果做出相应的决策,并执行相应的动作 
\end{itemize}
图\ref{Fig:Automated-linking-between-OntoSpecies-Kin-CompChem}给出火箭推进的氢燃料燃烧反应涉及的10种化学物种和40个基元反应,\textrm{Linking}软件可以追索每个反应,有一个反应是\ch{H2O2}~+\ch{*OH}~$\rightleftharpoons$~\ch{H2O}~+\ch{*OOH}。
\begin{figure}[h!]
\centering
\includegraphics[height=3.40in,width=4.05in,viewport=0 0 1010 750,clip]{Figures/Automated-linking-between-OntoSepcies-Kin-CompChem.png}
\caption{\small\textrm{Automated linking between OntoSpecies, OntoKin and OntoCompChem. cite from~\cite{ACR56-128_2023}}}%(与文献\cite{EPJB33-47_2003}图1对比)
\label{Fig:Automated-linking-between-OntoSpecies-Kin-CompChem}
\end{figure}
\subsection{合理的自组装材料的自动化设计}
金属有机多面体\textrm{(Metal-organic polyhedra, MOPs)}是由有机的和金属的化学单元\textrm{(Chemical Building Units, CBUs)}组装,并重构成规则的多面体\upcite{CR120-8987_2020}。金属有机多面体和其它的笼状结构过去主要由领域行业专家设计的,为了设计新的金属有机多面体,专家必须考虑化学的和空间组成的影响,但小朋友搭玩具模型时并不具备多面体的几何知识背景,可见直觉想象力是参与推理的。如图\ref{Fig:OntoMOPs-MOPs}所示,在\textrm{OntoMOPs}中,将组装模型\textrm{(Assembly Models, AMs)}的概念和通用组装单元\textrm{(Generic Building Units, GBUs)}引入到\textrm{MOPs}的合理设计。而\textrm{MOPs}的化学单元也用\textrm{OntoSpecies}标注,并在图谱中标注该\textrm{MOPs}最初的出处文献。而\textrm{MOP}发现软件的算法思路是设置规定\textrm{CBUs}的允许组合,又不能有太大的应力。
\begin{figure}[h!]
\centering
\includegraphics[height=2.70in,width=5.05in,viewport=0 0 1330 700,clip]{Figures/Key_concepts-in-OntoMOPs-and-designed-MOPs.png}
\caption{\small\textrm{Key concepts in OntoMOPs (left) and examples of newly rationally designed MOPs (right). cite from~\cite{ACR56-128_2023}}}%(与文献\cite{EPJB33-47_2003}图1对比)
\label{Fig:OntoMOPs-MOPs}
\end{figure}
\subsection{界面友好的\rm{TWA}知识图谱}
问答式知识图谱用的查询语言(如\textrm{SPARQL})并需要了解知识图谱中如何组织知识,因此对于缺少知识图谱基本了解的用户,缺乏使用知识图谱的动力,所以界面友好的知识图谱的需求非常迫切。在\textrm{TWA}项目中,化学-化工类的知识图谱中,问答模块\textrm{Marie}就是允许用户用自然语言提问,在后台问题描述的场景会转换成机器可读的问答形式\upcite{DCE3-100032_2022}。为了实现这些功能,\textrm{Marie}使用自然语言处理和网络软件来确定主题、问题类型和用户提问的实体。一旦问题被理清,软件会将相关信息提交给查询软件,查询软件会将信息转递到\textrm{SPARQL}软件,\textrm{SPARQL}软件查询知识图谱并向用户返回信息。图\ref{Fig:TWA-KG-Marie}给出典型的用户提问\textrm{Marie}要求展示芳香\ch{C-H}化合物模型。
\begin{figure}[h!]
\centering
\includegraphics[height=3.70in,width=3.35in,viewport=0 0 750 790,clip]{Figures/TWA-KG-Marie.png}
\caption{\small\textrm{Marie's back-end operations involved in querying information that is in TWA KG and one that it is generated through agent operation. Printed results of queries are adapted with permission from (i) ref \cite{JCIM61-3868_2021}. cite from ~\cite{ACR56-128_2023}}}%(与文献\cite{EPJB33-47_2003}图1对比)
\label{Fig:TWA-KG-Marie}
\end{figure}
为了节约知识存储的成本,很多知识可通过推演或计算得到。所以如果\textrm{Marie}在知识图谱中没有找到答案,它会采取另外的技术路线:~通过激活发现软件,该软件会寻找适合问题的软件,合适的软件再向知识图谱查询并计算相关结果。一个典型的例子是用户提问查询\ch{CO2}的热容时,软件\textrm{Thermo}会在诸如\textrm{OntoCompChem}中计算\ch{CO2}的热容。
\subsection{小结}
知识图谱结合处理软件可以实现复杂决策并通过计算、新的属性和自主的实验产生新的知识\upcite{JACSAu2-292_2022}。因此可以预见,在不远的将来,通过软件和知识图谱的复杂的组合会发现和创造新的分子或材料。因此如果化学知识形成的生态系统,将会通过化学知识空间的探索,有效地发现更多的新知识。

\include*{01_Latex_Template/Content-4-Bibliography}             %% 普通论文:~参考文献 	      %
\section{系统架构与关键技术}
\subsection{整体架构} 
本项目的系统架构%采用业界成熟通行的实现方案和技术体系作为本次架构设计的基础。
在设计中以信息安全和行业信息化标准为开展工作的基础保障,充分采用\textrm{SOA(Service-Oriented Architecture)}的设计思想,将系统进行纵向切分包括数据层、基础支撑层、服务支撑层、应用层等几个层面。在切分的过程中对具体的应用系统与具体的数据存储通过中间的业务服务体系进行解耦,从而使得系统中的业务应用与数据保持相对独立,减少应用系统各功能模块间的依赖关系,通过定义良好的访问接口与通讯协议形成松散耦合型系统,在保证系统间信息交换的同时,还能保持各系统之间的相对独立的运行。图\ref{KG_Chem-Frame}为化工知识图谱系统项目整体架构图。
\begin{figure}[h!]
%\vspace*{-0.05in}
\centering
\includegraphics[height=3.08in,width=6.00in,viewport=0 0 245 113,clip]{Figures/KG_Chem-Frame.png}
\caption{项目总体技术框架.}%(与文献\cite{EPJB33-47_2003}图1对比)
\label{KG_Chem-Frame}
\end{figure} 

本项目建设是以满足化工知识知识图谱与知识库建设的实际需要为目标,充分利用现有数据和监测成果,依托现有计算机网络环境及资源环境,遵循统一的技术架构。
\begin{itemize}
	\item 数据层\\ 
数据层作为应用系统的底层,经过总集标段数据中台处理后形成统一标准、统一格式的数据及业务系统自有数据为支撑层提供算据支撑。数据中台提供的数据包括基础数据、业务数据、实时监测数据、实体数据和文件数据。
\item 支撑层\\ 
支撑层作为系统整体服务能力核心,包括已建设的公用组件支撑、大数据支撑服务和知识引擎。其中知识处理模块针对原始知识库进行数据清洗、知识表示与建模、知识抽取、知识融合、知识加工、知识存储等处理,最终形成基础知识库;为应用和外部应用提供智能支撑。
\item 接口层\\ 
接口层封装集成支撑层技术服务逻辑,形成统一标准的知识接口和数据接口向上为系统应用提供服务。
\item 应用层\\ 
应用层主要以微服务的方式为用户提供各种应用服务,包括实体识别、实体查询、关系查询、化工知识概览等服务。外部应用的服务主要包括化工知识图谱展现等业务系统,需要通过数据中台调用知识库的知识接口服务。
\item 展示层\\
展示层为用户提供友好的界面展示效果和交互式操作。针对不同终端展示界面进行设计和适配,确保不同界面下展示信息的全面和美观。 
\end{itemize}

%\subsection{数据采集} 
%在项目建设过程中,首先需要广泛的数据采集工作,包括收集和整理化学化工领域的文献、数据集、专利信息等。因此必须建立规范的数据采集流程,确保数据的准确性和完整性。同时,%我们对业务进行了建模,
%明确%了
%化学化工领域的概念和关系,为后续的知识图谱构建奠定了基础。

%\subsection{知识库建设和本体构建}
%基于采集到的数据和业务建模的结果,我们开始着手构建化学化工知识库和本体。我们采用了先进的本体构建技术,将领域知识转化为计算机可理解的形式。我们创建了本体的概念、属性和关系,并进行了本体的验证和优化。同时,我们对知识库进行了组织和分类,使得用户可以方便地浏览和检索相关信息。

\subsection{语义认知算法和分析挖掘算法} 
为了提升知识图谱的智能化和实用性,引入了语义认知算法和多种分析挖掘算法。语义认知算法能够理解用户的查询意图,提供更准确和个性化的搜索结果。语义分析的基础是清华大学自然语言处理与社会人文计算实验室研制推出的一套中文词法分析工具\textrm{THULAC~(THU Lexical Analyzer for Chinese)},它具有中文分词和词性标注功能,并利用大规模人工分词和词性标注中文语料库训练而成,模型性能强大,处理速度快。

基于\textrm{THULAC}可以将输入的中文文本按照词语的边界进行切分,把连续的汉字序列分割成一个个独立的词语。这是中文自然语言处理中的基础且关键的步骤,为后续的文本分析、信息检索等任务提供了基础。在完成分词的基础上,为每个词语标注其词性,如名词、动词、形容词、副词等。通过词性标注结果,有助于理解文本的语法结构和语义信息。此外用户可以设置词与词性间的分隔符、使用过滤器去除一些没有意义的词语等。这些功能增加了工具的适用性和灵活性,能够满足不同用户在不同场景下的需求。用户也可以根据自己的特定需求添加自定义词典,这对于一些专业领域或具有特定术语的文本处理非常有用。用户词典中的词会被打上特定的标签,方便在后续的分析中进行识别和处理。

利用集成的目前世界上规模最大的人工分词和词性标注中文语料库(约含~5800~万字)训练而成,在标准数据集\textrm{Chinese Treebank~(CTB5)}上分词的\textrm{F1}值可达\textrm{97.3\%},词性标注的\textrm{F1}值可达到~\textrm{92.9\%},与该数据集上最好方法效果相当。同时进行分词和词性标注速度为\textrm{300KB/s},每秒可处理约15万字;只进行分词速度可达到\textrm{1.3MB/s},能够满足大规模文本处理的需求。

目前该算法对新词的识别能力有限,还缺乏动态学习。语言是不断发展变化的,会不断涌现出各种新词、热词以及特定领域的专业术语等。基于\textrm{THULAC}已有的训练数据进行学习和分析,对于训练数据中未曾出现过的新词,识别能力相对较弱。比如一些网络流行语、新兴的科技或商业术语等,可能无法准确地将其识别为独立的词语进行分词,从而影响到后续的词性标注和文本分析的准确性。对特定领域文本的适应性有待提高,这是通用模型的局限性造成的,因为基础的\textrm{THULAC}采用的是通用的词法分析模型,虽然在通用文本上能够取得较好的效果,但对于一些特定领域的专业文本,如医学、法律、金融等,由于这些领域的文本具有较强的专业性和独特的语言表达方式,可能无法准确理解其中的专业术语和复杂句式,导致分词和词性标注的准确性下降。

%缺乏领域针对性训练:与一些专门针对特定领域进行优化训练的词法分析工具相比,THULAC 在处理特定领域文本时缺乏针对性的训练和优化,无法充分满足这些领域的专业需求。
此外,在处理复杂句式和歧义问题上仍有不足,复杂句式处理能力有限,对于一些结构复杂的长句子,尤其是包含嵌套结构、并列结构、省略结构等复杂句式的文本,在分词和词性标注时可能会出现错误或不准确的情况。例如一些带有多个修饰成分的长句,可能无法准确判断各个词语之间的修饰关系和语义关系,从而影响到分词和词性标注的结果。歧义消解不够完善,中文语言中存在大量的歧义现象,如一词多义、句子结构歧义等。虽然\textrm{THULAC}已经在一定程度上通过上下文信息来消解一些歧义,但对于一些较为复杂的歧义问题,仍然存在无法准确消解的情况。例如``乒乓球拍卖完了''这句话,存在``乒乓球/拍卖/完了''和``乒乓球拍/卖/完了''等多种分词方式,当前算法可能无法准确判断其正确的分词方式。

%4. 可定制性和灵活性相对较弱:

%参数设置有限:THULAC 虽然提供了一些参数供用户进行设置,如是否进行词性标注、是否进行简繁转换、是否过滤冗余词汇等,但总体来说参数设置的选项相对较少,无法满足用户对于不同文本处理需求的精细化调整。
%
%与其他工具的集成度不高:在实际的自然语言处理项目中,往往需要将词法分析工具与其他的自然语言处理工具或系统进行集成。然而,THULAC 与其他工具的集成度相对不高,在与其他工具进行协同工作时可能需要进行较多的额外开发和调试工作,增加了项目的开发成本和难度。
%
%5. 缺乏可视化界面和交互功能:
%
%不便于非专业用户使用:对于一些不具备编程基础和自然语言处理专业知识的非专业用户来说,THULAC 的命令行操作方式和编程接口可能过于复杂和难以理解,缺乏直观的可视化界面和交互功能,使得这些用户在使用 THULAC 时存在较大的困难,无法方便地进行文本分析和处理。
%
%不利于结果的直观展示:在一些需要对文本分析结果进行直观展示和可视化呈现的场景下,THULAC 无法直接提供相应的可视化功能,需要用户自行将分析结果进行整理和转换后再使用其他的可视化工具进行展示,增加了用户的操作步骤和工作量。
%
知识图谱构建中的关键步骤包括实体识别、实体关系抽取、事件抽取等,其中讨论最多的就是如何实现实体识别和实体关系抽取,分析挖掘算法则可以从大量的数据中发现隐藏的关联和规律,为化学化工行业的研究和决策提供有价值的信息。这里对实体关系分类采取的是\textrm{fastText}是快速文本分类算法,与基于神经网络的分类算法相比,\textrm{fastText}有两大优点:(1)\textrm{fastText}在保持高精度的情况下加快了训练速度和测试速度;(2)\textrm{fastText}不需要预训练好的词向量,\textrm{fastText}会自己训练词向量。
%    3、fastText两个重要的优化:Hierarchical Softmax、N-gram 
\textrm{fastText}结合了自然语言处理和机器学习中最成功的理念。包括使用词袋以及\textrm{n-gram}袋表征语句,还有使用子字\textrm{(subword)}信息,并通过隐藏表征在类别间共享信息。另外采用\textrm{softmax}层级(利用了类别不均衡分布的优势)来加速。

文本分类后,再应用\textit{k}-最近邻(\textit{k}-\textrm{nearest neighbors, kNN})算法来定义页面的相似度。\textrm{kNN}算法利用数据点空间距离的类似性,不再训练,因此对于处理快速任务特别具有吸引力。简言之,如果$d$-维空间有训练集数据$\{\mathbf{x}^{(\mathrm{i})}\}$,\textrm{kNN}计算未知数据点与这些数据点之间的空间距离
\begin{displaymath}
	d(\mathbf{x},\mathbf{x}^{(\mathrm{i})})=\|\mathbf{x}-\mathbf{x}^{(\mathrm{i})}\|_p
\end{displaymath}
这里的$p$是维度参数。一旦得到$\mathbf{x}$到空间各点的距离,$\mathbf{x}$点归入与其有最近邻$k$值最大的类中,如果没有最大类,则随机归入最近邻点的最常使用的标注类中。显然,对连续的标签值求平均,就是基于\textrm{kNN}的回归。类似地$k$值的选取对于分类很敏感,不同的$k$值很可能得到完全不同的数据分类。\textrm{kNN}算法无需表示成向量,比较相似度即可:~\textrm{k}值通过网格搜索得到。

两个页面的相似度$\mathrm{sim(p1,p2)}$的定义:
\begin{itemize}
	\item 计算\textrm{title}之间的词向量的余弦相似度(利用\textrm{fasttext}计算的词向量能够避免out of vocabulary);
	\item 计算两组\textrm{openType}之间的词向量的余弦相似度的平均值;
	\item 计算具有相同的\textrm{baseInfoKey}的\textrm{IDF}值之和(因为‘中文名’这种属性贡献应该比较小);
	\item 统计具有相同\textrm{baseInfoKey}下\textrm{baseInfoValue}相同的个数;
	\item 预测一个页面时,由于\textrm{kNN}要将该页面和训练集中所有页面进行比较,因此每次预测的复杂度是$O(n)$,$n$为训练集规模。在这个过程中,可以统计各个分相似度的\textrm{IDF}值、均值、方差、标准差,然后对4个相似度进行标准化:
		\begin{displaymath}
			(x-\bar{x})/\mathrm{D}(x)
		\end{displaymath}
		这里$\bar{x}$是均值,$\mathrm{D}(x)$是方差。
	\item 上述四个部分的相似度的加权和为最终的两个页面的相似度,权值由向量\textrm{weight}控制,通过10折叠交叉验证$+$网格搜索得到
\end{itemize}

\section{化学化工知识图谱项目建设方案}
\subsection{整体概况}
\subsection{系统简介} 
化学化工知识图谱项目的目标是通过制定化学化工数据采集方法,采用``本体-要素-概念''三位一体的技术实现业务建模,支持智慧语义认知算法和多种分析挖掘算法,从而完成化学化工知识库的建设,构建化学化工行业知识图谱。我们的目标是建立一个全面、准确、可靠的知识图谱,为化学化工行业的研究、应用和决策提供有力支持。
\subsection{建设目标} 
本系统旨在构建一个全面的化工知识图谱系统,通过集成实体识别、实体查询、关系查询以及化工知识概览等功能,为用户提供便捷、高效的化工知识获取途径。
\begin{enumerate}
	\item 实体识别开发实体搜索引擎,能够准确识别实体信息,这一过程不仅迅速,而且全面,能够覆盖数据集中的各个角落,确保用户获取到最全面、最准确的信息。
	\item 实体查询建立化工实体数据库,包含各类化工实体的详细信息,开发实体查询接口,允许用户根据实体名称或属性进行查询,获取相关实体的详细信息。
	\item 关系查询构建化工实体关系网络,明确各类实体之间的关联和相互作用,开发关系查询功能,允许用户根据实体名称或属性查询与之相关的其他实体及其关系,提供可视化关系展示功能,帮助用户更直观地理解实体之间的关系。
	\item 化工知识概览,整合化工领域的基础知识、专业术语、行业标准等,构建化工知识库开发知识概览功能,以树形结构或分类列表的形式展示化工知识,方便用户快速了解领域全。
\end{enumerate}

\subsection{建设要求} 
建设化工知识图谱系统相关的知识库,涉及到多个方面的知识和技术。下面是一些建设内容和要求规范,以便实现实体识别、实体查询、关系查询、化工知识概览等功能。 

1.~{知识库建设} 
  收集和整理相关领域的知识:建立一个包含化工基础数据、化工关系数据、化工类型及相关概念等领域的知识库。

维护知识的准确性和及时性:及时更新知识库中的信息,确保系统所使用的知识与最新的科学研究和实践相符合。

组织知识的结构:使用合适的分类和标签系统,将知识库中的信息组织成易于搜索和浏览的形式。

持续更新和维护:定期更新和维护知识库,跟踪新的研究成果、政策法规的变化,并及时更新知识图谱中的内容。

测试和优化:进行系统测试和评估,验证知识库的准确性、完整性和性能。根据用户反馈和需求,对系统进行优化和改进,提高用户体验和系统效能。

2.~{实体识别功能} 
基于自然语言处理和信息检索技术,实现实体搜索功能,用户可以通过输入实体信息或关键词来获取相关的知识和信息。

支持关键词匹配和语义理解:系统能够理解用户输入的查询意图,并根据输入的关键词或问题匹配合适的知识库中的内容。

提供相关度排序和过滤功能:将搜索结果按照相关度排序,并提供过滤选项,使用户能够快速找到所需的信息。 

3.~{实体查询功能} 
用户只需简单输入实体信息或关键词,即可轻松获取详尽的实体信息。这一功能不仅支持对特定实体的深度检索,还能智能地搜索出与该实体相关联的其他实体,以及它们之间错综复杂的关系网络。

用户可以根据自己的需求,灵活设定查询条件,如实体名称、属性特征或特定关系类型等,系统随即展开全面而精确的搜索。通过这一功能,用户可以迅速发现某一实体在数据集中所处的位置,以及它与周围实体的连接点和相互作用方式。

此外,系统还会以直观易懂的方式呈现这些关系,如关系图谱,帮助用户更加清晰地理解实体之间的内在联系。

\subsection{关系查询功能} 
关系查询中,我们添加两个实体信息,选择对应关系。可查询出所有相关实体信息,不仅极大地扩展了关系查询的广度和深度,更使得我们能够深入挖掘出那些隐藏在复杂关系网络中的、令人意想不到的隐含关系。

通过这一功能,用户可以轻松地在庞大的数据集中,找到两个看似毫无关联的实体之间,通过一系列中间节点所构成的最短路径。这些路径可能揭示了实体之间未曾被注意到的联系,为数据分析提供了全新的视角和洞见。

\subsection{化工知识概览功能} 
知识概览部分,我们为用户提供了一个直观且易于导航的界面,能够清晰地列出某一特定分类下的所有词条列表。这些词条不仅涵盖了该分类下的核心概念、关键术语和重要信息,还通过精心设计的组织方式,帮助用户快速构建对该分类领域的全面认知。

当用户点击某一词条时,我们将采用创新的树形结构展示方法,动态地呈现与该词条相关的概念体系。这一树形结构以词条为核心节点,通过分支和子节点的方式,层层递进地展示出与核心词条相关联的其他概念、属性和关系。用户可以通过浏览这一结构,深入了解词条之间的层级关系、逻辑联系和相互作用,从而更加系统地掌握该分类领域的知识体系

每个相关实体都会以超链接的形式呈现。用户点击,即可跳转到该实体的详细词条页面。

以上是化工知识图谱系统相关知识库建设思路和要求规范的一些方面。建设过程中,需要综合考虑实际需求、技术可行性和用户体验,确保系统能够准确、高效地满足用户的需求。

\section{系统功能介绍} 
本系统是一个专注于化工领域的化工知识图谱系统,旨在为用户提供高效、全面的化工知识查询,系统集成了多项先进功能,包括实体识别、实体查询、关系查询以及化工知识概览。

\subsection{实体识别} 
收集和整理相关领域的知识:建立一个包含化工知识专家、化工名词(有机化工、无机化工)、实体查询、关系查询等领域的知识库。

\begin{figure}[h!]
	\vskip -10pt
\centering
\includegraphics[height=2.40in,width=4.85in,viewport=0 0 145 80,clip]{Figures/KG_Chem-entity_search.png}
\caption{\small\textrm{实体文本检索示例}}%(与文献\cite{EPJB33-47_2003}图1对比)
\label{Fig:KG-Chem_entity_search}
\end{figure}
实体识别不仅需要高准确性,还要考虑到速度和可扩展性,特别是在处理大规模数据集时。因此,选择合适的实体识别技术和优化算法是至关重要的。图\ref{Fig:KG-Chem_entity_search}示意了实体文本检索示例:~输入搜索文本,并点击提交查询按钮,系统便会立即启动高效精准的搜索引擎,为用户筛选出与搜索文本高度相关的实体信息。这一过程不仅迅速,而且全面,能够覆盖数据集中的各个角落,确保用户获取到最全面、最准确的信息。

图\ref{Fig:KG-Chem_entity}给出实体文本识别的示意,搜索结果将以直观、易读的方式展现给用户,每个相关实体都会以超链接的形式呈现。用户点击,即可跳转到该实体的详细词条页面。在词条页面上,用户可以深入了解实体的详细信息,包括其定义、属性、特征、背景等,以及其他关联的实体。

\begin{figure}[h!]
\centering
\includegraphics[height=2.40in,width=4.85in,viewport=0 0 145 80,clip]{Figures/KG_Chem-entity.png}
\caption{\small\textrm{实体文本识别示例}}%(与文献\cite{EPJB33-47_2003}图1对比)
\label{Fig:KG-Chem_entity}
\end{figure}
实体识别不仅需要高准确性,还要考虑到速度和可扩展性,特别是在处理大规模数据集时。因此,选择合适的实体识别技术和优化算法是至关重要的。
\subsection{实体查询} 
高级关系查询,用户只需简单输入查询条件,即可轻松获取详尽的实体信息。这一功能不仅支持对特定实体的深度检索,还能智能地搜索出与该实体相关联的其他实体,以及它们之间错综复杂的关系网络。

用户可以根据自己的需求,灵活设定查询条件,如实体名称、属性特征或特定关系类型等,系统随即展开全面而精确的搜索。通过这一功能,用户可以迅速发现某一实体在数据集中所处的位置,以及它与周围实体的连接点和相互作用方式。

\begin{figure}[h!]
\centering
\includegraphics[height=2.40in,width=4.85in,viewport=0 0 145 80,clip]{Figures/KG_Chem-entity-check.png}
\caption{\small\textrm{实体查询示例}}%(与文献\cite{EPJB33-47_2003}图1对比)
\label{Fig:KG-Chem_entity_check}
\end{figure}
图\ref{Fig:KG-Chem_entity_check}给出实体查询的示例。实体识别不仅需要高准确性,还要考虑到速度和可扩展性,特别是在处理大规模数据集时。因此,选择合适的实体识别技术和优化算法是至关重要的。
此外,系统还会以直观易懂的方式呈现这些关系,如关系图谱,帮助用户更加清晰地理解实体之间的内在联系。

\subsection{关系查询} 
关系查询中,我们添加两个实体信息,选择对应关系。可查询出所有相关实体信息,不仅极大地扩展了关系查询的广度和深度,更使得我们能够深入挖掘出那些隐藏在复杂关系网络中的、令人意想不到的隐含关系。图\ref{Fig:KG-Chem_entity_relation}给出实体关系查询的示例。

\begin{figure}[h!]
\centering
\includegraphics[height=2.40in,width=4.85in,viewport=0 0 145 80,clip]{Figures/KG_Chem-entity-relation.png}
\caption{\small\textrm{实体关系查询示例}}%(与文献\cite{EPJB33-47_2003}图1对比)
\label{Fig:KG-Chem_entity_relation}
\end{figure}
实体识别不仅需要高准确性,还要考虑到速度和可扩展性,特别是在处理大规模数据集时。因此,选择合适的实体识别技术和优化算法是至关重要的。
通过这一功能,用户可以轻松地在庞大的数据集中,找到两个看似毫无关联的实体之间,通过一系列中间节点所构成的最短路径。这些路径可能揭示了实体之间未曾被注意到的联系,为数据分析提供了全新的视角和洞见。

\subsection{化工知识概览} 
知识概览部分,我们为用户提供了一个直观且易于导航的界面,能够清晰地列出某一特定分类下的所有词条列表。这些词条不仅涵盖了该分类下的核心概念、关键术语和重要信息,还通过精心设计的组织方式,帮助用户快速构建对该分类领域的全面认知。图\ref{Fig:KG-Chem_entity_overview}给出当前的化学-化工知识的概览的示例。

当用户点击某一词条时,我们将采用创新的树形结构展示方法,动态地呈现与该词条相关的概念体系。这一树形结构以词条为核心节点,通过分支和子节点的方式,层层递进地展示出与核心词条相关联的其他概念、属性和关系。用户可以通过浏览这一结构,深入了解词条之间的层级关系、逻辑联系和相互作用,从而更加系统地掌握该分类领域的知识体系。

\begin{figure}[h!]
\centering
\includegraphics[height=2.40in,width=4.85in,viewport=0 0 145 80,clip]{Figures/KG_Chem-entity-overview.png}
\caption{\small\textrm{化学-化工知识概览示例}}%(与文献\cite{EPJB33-47_2003}图1对比)
\label{Fig:KG-Chem_entity_overview}
\end{figure}
实体识别不仅需要高准确性,还要考虑到速度和可扩展性,特别是在处理大规模数据集时,图\ref{Fig:KG-Chem_entity_overview-hyperlink}给出每个实体存在超链接形式的示例。因此,选择合适的实体识别技术和优化算法是至关重要的。

\begin{figure}[h!]
\centering
\includegraphics[height=2.40in,width=4.85in,viewport=0 0 145 85,clip]{Figures/KG_Chem-entity-overview-hyperlink.png}
\caption{\small\textrm{每个相关实体都可以超链接的形式呈现,用户点击,即可跳转到该实体的详细词条页面}}%(与文献\cite{EPJB33-47_2003}图1对比)
\label{Fig:KG-Chem_entity_overview-hyperlink}
\end{figure}

\subsection{智能搜索} 
基于自然语言处理和信息检索技术,实现智能搜索功能,用户可以通过输入问题或关键词来获取相关的知识和信息。

支持关键词匹配和语义理解:系统能够理解用户输入的查询意图,并根据输入的关键词或问题匹配合适的知识库中的内容。图\ref{Fig:KG-Chem_smart_search}给出知识图谱的智能搜索示例。
\begin{figure}[h!]
\centering
\includegraphics[height=2.40in,width=4.85in,viewport=0 0 145 85,clip]{Figures/KG_Chem-smart_search.png}
\caption{\small\textrm{知识图谱智能搜索示例}}%(与文献\cite{EPJB33-47_2003}图1对比)
\label{Fig:KG-Chem_smart_search}
\end{figure}
实体识别不仅需要高准确性,还要考虑到速度和可扩展性,特别是在处理大规模数据集时。因此,选择合适的实体识别技术和优化算法是至关重要的。

提供相关度排序和过滤功能:将搜索结果按照相关度排序,并提供过滤选项,使用户能够快速找到所需的信息。

\subsection{图谱可视化} 
使用图谱技术将知识库中的信息进行可视化展示,形成知识之间的关联和结构。

图\ref{Fig:KG-Chem_visualization}展示实体关系和属性:通过图谱可视化,展示化学化工行业中的各种实体(如化工专家、有机化工、无机化工等)之间的关系和属性。
\begin{figure}[h!]
\centering
\includegraphics[height=3.40in,width=4.85in,viewport=0 0 145 110,clip]{Figures/KG_Chem-visualization.png}
\caption{\small\textrm{知识图谱可视化示例}}%(与文献\cite{EPJB33-47_2003}图1对比)
\label{Fig:KG-Chem_visualization}
\end{figure}
实体识别不仅需要高准确性,还要考虑到速度和可扩展性,特别是在处理大规模数据集时。因此,选择合适的实体识别技术和优化算法是至关重要的。

支持交互和导航:用户可以通过交互方式探索图谱,浏览相关实体和关系,深入了解相关知识。

\section{数据及业务流程说明}
\subsection{业务数据说明} 
化工知识图谱系统旨在通过集成实体识别、实体查询、关系查询以及化工知识概览等功能,构建一个全面、高效、易用的化工知识管理与应用平台。本系统以化工领域的专业知识为核心,通过整合各类相关数据资源,为用户提供便捷的化工知识获取与应用服务。图\ref{Fig:KG_Chem-Tech_Frame}给出知识图谱的业务流程的示意图%以下为本系统的业务数据说明。
\begin{figure}[h!]
\centering
\includegraphics[height=4.00in,width=5.85in,viewport=0 0 240 150,clip]{Figures/KG_Chem-Tech_Frame.png}
\caption{\small\textrm{知识图谱的业务流程的框架示意图}}%(与文献\cite{EPJB33-47_2003}图1对比)
\label{Fig:KG_Chem-Tech_Frame}
\end{figure}
实体识别不仅需要高准确性,还要考虑到速度和可扩展性,特别是在处理大规模数据集时。因此,选择合适的实体识别技术和优化算法是至关重要的。

\subsubsection{基础数据} 
基础数据是化工知识图谱系统的核心构成,它涵盖了化学元素周期表、化学分子式、化学反应类型等关键化学知识,这些数据来源于化工知识图谱系统基础信息数据库,确保了其准确性和权威性。在系统中,基础数据为实体识别、关系查询等功能的实现提供了坚实的基础支撑,是用户深入理解和应用化工知识的重要基石。

\subsubsection{化工实体数据} 
化工实体数据涵盖了化工品、化工类型、实体类型、相关概念等化工领域的核心实体数据,这些数据来源于系统内置的实体识别功能对图片、文档的自动提取,以及用户的手动输入或上传,为实体查询、关系查询及化工知识概览等功能的实现提供了丰富且准确的实体基础。

\subsubsection{关系数据} 
关系数据是化工知识图谱中连接各个实体的桥梁,它详细记录了化学反应的原料与产物关系、化学品的合成路径等实体间的关联与相互作用。这些数据既源自系统内部通过关系查询功能对知识库的智能提取。它们共同构成了化工知识概览与关系查询功能不可或缺的关系基础,为用户深入探索和理解化工领域的复杂关系提供了强有力的支持。

\subsubsection{知识库数据} 
除了直接对接总集标段各种不同类型的知识数据源,知识库平台预留人工录入的方式,支持专家等用户,可按照知识录入的标准规范,手动录入相关行业专家经验等知识信息。

将各类原始知识数据抽样抽析加工,形成调度规则、业务规则、专家经验、历史档案等数据库,以图的方式进行组织,构建全景全域图谱与个体画像。

值得注意的是,知识数据需要结合业务场景,按照业务规则、使用频率、模型要求进行实时与不定时更新,更好支撑业务。

\subsection{数据采集和业务建模} 
在项目建设过程当中,我们进行了广泛的数据采集工作,包括收集和整理化学化工领域的文献、数据集、专利信息等。我们建立了一个规范的数据采集流程,确保数据的准确性和完整性。同时,我们对业务进行了建模,明确了化学化工领域的概念和关系,为后续的知识图谱构建奠定了基础。

\subsubsection{知识库建设和本体构建} 
基于采集到的数据和业务建模的结果,我们开始着手构建化学化工知识库和本体。我们采用了先进的本体构建技术,将领域知识转化为计算机可理解的形式。我们创建了本体的概念、属性和关系,并进行了本体的验证和优化。同时,我们对知识库进行了组织和分类,使得用户可以方便地浏览和检索相关信息。图\ref{Fig:KG-Chem_Data-flow}给出的是知识本体构建的示意。

\begin{figure}[h!]
\centering
\includegraphics[height=2.00in,width=5.85in,viewport=0 0 145 45,clip]{Figures/KG_Chem-Data-flow.png}
\caption{\small\textrm{知识本体的构建示意图}}%(与文献\cite{EPJB33-47_2003}图1对比)
\label{Fig:KG-Chem_Data-flow}
\end{figure}
%实体识别不仅需要高准确性,还要考虑到速度和可扩展性,特别是在处理大规模数据集时。因此,选择合适的实体识别技术和优化算法是至关重要的。
%2.4智慧语义认知算法和分析挖掘算法
%
%为了提升知识图谱的智能化和实用性,我们引入了智慧语义认知算法和多种分析挖掘算法。智慧语义认知算法能够理解用户的查询意图,提供更准确和个性化的搜索结果。而分析挖掘算法则可以从大量的数据中发现隐藏的关联和规律,为化学化工行业的研究和决策提供有价值的信息。
\subsection{知识处理流程说明} 
通过数据资源统筹规划与汇集,建立数据模型,推动数据治理入库、数据开发及数据运维,实现化工数据与基础数据多次元的融合,并以数据服务的方式提供各类分析数据与成果,为化工知识图谱功能提供服务的支撑。

\subsubsection{数据处理与整合} 
数据清洗与标准化,采用先进的数据清洗算法,对来自不同渠道的数据进行去重、修正和格式统一,确保数据的准确性和一致性。

数据融合与整合,运用数据整合技术,将不同来源、不同格式的数据进行融合,形成统一的数据视图,为系统提供全面的数据支持。

\subsubsection{知识图谱构建} 
实体识别与抽取,利用自然语言处理\textrm{(NLP)}和机器学习技术,从图片、文档等中自动识别和抽取化工实体,如化工品、化工类型等。

关系抽取与建模,通过关系抽取算法,从文本中自动挖掘实体间的关联与相互作用,如化工反应的原料与产物关系,构建丰富的关系网络。

知识表示与存储,采用图数据库等先进技术,对化工知识进行高效的表示和存储,支持快速的查询和推理。

\subsubsection{多源数据汇集与治理} 
基于已有的业务,本着服务调用,资源共享的原则,减少数据的冗余存储,通过定义标准的汇集策略,对标准业务数据,非标准矢量、文本、图片等非格式化数据进行汇集,满足化工知识图谱业务的需要。

可通过ETL工具,各类接口及标准协议,在不同的层级进行抽取、转换、编码、关联与标签等操作,实现多源数据的对接与交换,构建标准元数据及主题库,能在平台上展示与应用。数据采集入库的流程依托数据汇集功能实现,数据汇集功能采用基于工作流的汇聚引擎,按照``定义数据源、定义采集方式配置存储位置、配置调度策略、执行采集任务''的流程进行数据采集。数据汇集功能能够接入多种类型的数据源;可以自定义数据采集策略,配置任务调度规则、执行周期;采集过程可以与质量控制进行融合,自动化汇聚数据的过程中执行质量检查。

\subsection{知识图谱分析服务说明} 
知识图谱分析服务以知识图谱库为基础,支持知识图谱计算功能服务,为知识图谱应用提供数据和功能支撑。本项目知识图谱分析服务建设内容包括但不限于知识图谱引擎建设、知识图谱结构设计及数据融合处理入库、图谱分析服务功能建设。

\subsubsection{定义与分类} 
知识图谱是一种通过图形结构表达知识的方法,它通过节点(实体)和边(关系)来表示和存储现实世界中的各种对象及其相互联系。这些实体和关系构成了一个复杂的网络,使得知识的存储不再是孤立的,而是相互关联和支持的。

知识图谱根据其内容和应用领域可以分为多种类型。例如,通用知识图谱旨在覆盖广泛的领域知识,如Google的Knowledge Graph;而领域知识图谱则专注于特定领域,如医疗、金融等。此外,根据构建方法的不同,知识图谱还可以分为基于规则的、基于统计的和混合型知识图谱。

\subsubsection{核心组成} 
知识图谱的核心组成元素包括实体、关系和属性。实体是知识图谱中的基本单位,代表现实世界中的对象,如人、地点、组织等。关系则描述了实体之间的各种联系,例如“属于”、“位于”等。属性是对实体的具体描述,如年龄、位置等。这些元素共同构成了知识图谱的骨架,使得知识的组织和检索变得更加高效和精确。

\subsubsection{数据源选择} 
知识图谱构建的首要步骤是确定和获取数据源。数据源的选择直接影响知识图谱的质量和应用范围。通常,数据源可以分为两大类:公开数据集和私有数据。公开数据集,如Wikipedia、Freebase、DBpedia等,提供了丰富的通用知识,适用于构建通用知识图谱。而私有数据,如企业内部数据库、专业期刊等,则更适用于构建特定领域的知识图谱。

选择数据源时,应考虑数据的可靠性、相关性、完整性和更新频率。可靠性保证了数据的准确性,相关性和完整性直接影响知识图谱的应用价值,而更新频率则关系到知识图谱的时效性。在实践中,通常需要结合多个数据源,以获取更全面和深入的知识覆盖。

获取数据后,下一步是数据清洗。这一过程涉及从原始数据中移除错误、重复或不完整的信息。数据清洗的方法包括去噪声、数据规范化、缺失值处理等。去噪声是移除数据集中的错误和无关数据,例如,去除格式错误的记录或非相关领域的信息。数据规范化涉及将数据转换为一致的格式,如统一日期格式、货币单位等。对于缺失值,可以采用插值、预测或删除不完整记录的方法处理。

数据清洗不仅提高了数据的质量,还能增强后续处理的效率和准确性。因此,这一步骤在知识图谱构建中至关重要。

实体识别是指从文本中识别出知识图谱中的实体,这是构建知识图谱的核心步骤之一。实体识别通常依赖于自然语言处理\textrm{(NLP)}技术,特别是命名实体识别\textrm{(NER)}。NER技术能够从非结构化的文本中识别出具有特定意义的片段,如人名、地名、机构名等。

实体识别的方法多种多样,包括基于规则的方法、统计模型以及近年来兴起的基于深度学习的方法。基于规则的方法依赖于预定义的规则来识别实体,适用于结构化程度较高的领域。统计模型,如隐马尔可夫模型\textrm{(HMM)}、条件随机场\textrm{(CRF)}等,通过学习样本数据中的统计特征来识别实体。而基于深度学习的方法,如使用长短时记忆网络\textrm{(LSTM)}或\textrm{BERT}等预训练模型,能够更有效地处理语言的复杂性和多样性,提高识别的准确率和鲁棒性。

\begin{figure}[h!]
\centering
\includegraphics[height=2.90in,width=5.85in,viewport=0 0 145 62,clip]{Figures/KG_Chem-Data.png}
\caption{\small\textrm{知识图谱中知识的表示关系示意图}}%(与文献\cite{EPJB33-47_2003}图1对比)
\label{Fig:KG-Chem_Data}
\end{figure}
实体识别不仅需要高准确性,还要考虑到速度和可扩展性,特别是在处理大规模数据集时。因此,选择合适的实体识别技术和优化算法是至关重要的。图\ref{Fig:KG-Chem_Data}给出的是知识图谱中知识的表示关系的示意。

\subsection{项目配套支撑条件说明} 
数据资源:本项目涵盖了大量的化学化工领域的数据资源,包括文献、数据集、专利信息等。我们通过与合作伙伴、研究机构和数据库提供商合作,获得了丰富的数据资源,并进行了规范的整理和处理。

技术平台:为了支持知识图谱的构建和运行,我们建立了适用的技术平台。该平台包括硬件设备、软件工具和数据库系统等,能够满足项目对计算能力、存储能力和数据处理能力的要求。

专家团队:我们组建了一支专业的团队,包括化学化工领域的专家、数据科学家和工程师等。他们具有丰富的专业知识和实践经验,能够有效地指导和支持项目的进行。

合作伙伴:我们与相关的合作伙伴建立了紧密的合作关系。这些合作伙伴包括研究机构、高校和企业等,他们提供了专业知识、数据资源和技术支持,共同推动项目的进展。

数据采集方法和标准:为了确保数据的准确性和一致性,我们制定了数据采集方法和标准。这些方法和标准涵盖了数据来源、采集流程、数据验证和清洗等方面,确保采集到的数据符合项目的需求和质量要求。

法律和隐私保护:在数据采集和使用过程中,我们严格遵守相关的法律法规和隐私政策。我们采取措施保护数据的安全性和隐私性,确保项目在合规的框架下进行。
