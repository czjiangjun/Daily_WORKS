\textrm{VASP}中保存体系电荷密度的文件为\textrm{CHGCAR},该文件中存储的内容包括
\begin{itemize}
	\item 计算体系重复单元(晶胞)的晶格矢量
	\item 计算体系所含原子的分数坐标
	\item 计算体系的平面波贡献的电荷密度与晶体体积乘积$\rho(\vec r)\cdot V_{\mathrm{cell}}$在倒空间网格中的分布
	\item 计算体系中每个原子在缀加原子球内的电荷密度$\rho(\vec r)$径向分布
\end{itemize}

不难看出,要分析构成晶体的各组分与其原子态的电荷密度变化(电荷密度差)定义为:

\begin{displaymath}
	\delta\rho = \rho_{\mathrm{sc}} - \rho_{\mathrm{atom}}
\end{displaymath}

上式中$\rho_{\mathrm{sc}}$为自洽得到总的电荷密度

需要指出的是,在计算电荷密度变化时,仍用原来自洽计算时的四个输入文件,但\textrm{INCAR}中需要将有关参数设置为{\textbf{ICHARG=12}},{\textbf{NSW=1}}和{\textbf{NELM=0}},其他设置不变,以便得到原子的电荷密度($\rho_{\mathrm{atom}})
