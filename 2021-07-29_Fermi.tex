\documentclass{article}      % Specifies the document class

%%%%%%%%%%%%%%%%% CJK 中文版面控制  %%%%%%%%%%%%%%%%%%%%%%%%%%%%%%
%\usepackage{CJK} % CTEX-CJK 中文支持                            %
\usepackage{xeCJK} % seperate the english and chinese		 %
%\usepackage{CJKutf8} % Texlive 中文支持                         %
\usepackage{CJKnumb} %中文序号                                   %
\usepackage{indentfirst} % 中文段落首行缩进                      %
%\setlength\parindent{22pt}       % 段落起始缩进量               %
\renewcommand{\baselinestretch}{1.2} % 中文行间距调整            %
\setlength{\textwidth}{16cm}                                     %
\setlength{\textheight}{24cm}                                    %
\setlength{\topmargin}{-1cm}                                     %
\setlength{\oddsidemargin}{0.1cm}                                %
\setlength{\evensidemargin}{\oddsidemargin}                      %
%%%%%%%%%%%%%%%%%%%%%%%%%%%%%%%%%%%%%%%%%%%%%%%%%%%%%%%%%%%%%%%%%%

\usepackage{latexsym}
\usepackage{amsmath,amsthm,amsfonts,amssymb,bm}          %数学公式
\usepackage{mathrsfs}                                    %英文花体
\usepackage{xcolor}                                        %使用默认允许使用颜色
%\usepackage{hyperref} 
\usepackage{graphicx}
\usepackage{subfigure}           %图片跨页
\usepackage{animate}		 %插入动画
\usepackage{caption}
\captionsetup{font=footnotesize}

%\usepackage[version=3]{mhchem}		%化学公式
\usepackage{chemfig}		%化学公式
\usepackage{chemformula}	%化学分子式

\usepackage{fontspec} % use to set font
\setCJKmainfont{SimSun}
\XeTeXlinebreaklocale "zh"  % Auto linebreak for chinese
\XeTeXlinebreakskip = 0pt plus 1pt % Auto linebreak for chinese

\usepackage{longtable}                                   %使用长表格
\usepackage{multirow}
\usepackage{makecell}		%允许单元格内换行

\usepackage{arydshln}
\newcommand{\adots}{\mathinner{\mkern2mu%
\raisebox{0.1em}{.}\mkern2mu\raisebox{0.4em}{.}%
\mkern2mu\raisebox{0.7em}{.}\mkern1mu}}
%%%%%%%%%%%%%%%%%%%%%%%%%  参考文献引用 %%%%%%%%%%%%%%%%%%%%%%%%%%%
%%尽量使用 BibTeX(含有超链接,数据库的条目URL即可)                %
%%%%%%%%%%%%%%%%%%%%%%%%%%%%%%%%%%%%%%%%%%%%%%%%%%%%%%%%%%%%%%%%%%%

\usepackage[numbers,sort&compress]{natbib} %紧密排列             %
\usepackage[sectionbib]{chapterbib}        %每章节单独参考文献   %
\usepackage{hypernat}                                                                         %
\usepackage[bookmarksopen=true,pdfstartview=FitH,CJKbookmarks]{hyperref}              %
\hypersetup{bookmarksnumbered,colorlinks,linkcolor=green,citecolor=blue,urlcolor=red}         %
%参考文献含有超链接引用时需要下列宏包,注意与natbib有冲突        %
%\usepackage[dvipdfm]{hyperref}                                  %
%\usepackage{hypernat}                                           %
\newcommand{\upcite}[1]{\hspace{0ex}\textsuperscript{\cite{#1}}} %

%%%%%%%%%%%%%%%%%%%%%%%%%%%%%%%%%%%%%%%%%%%%%%%%%%%%%%%%%%%%%%%%%%%%%%%%%%%%%%%%%%%%%%%%%%%%%%%
%\AtBeginDvi{\special{pdf:tounicode GBK-EUC-UCS2}} %CTEX用dvipdfmx的话,用该命令可以解决      %
%						   %pdf书签的中文乱码问题		      %
%%%%%%%%%%%%%%%%%%%%%%%%%%%%%%%%%%%%%%%%%%%%%%%%%%%%%%%%%%%%%%%%%%%%%%%%%%%%%%%%%%%%%%%%%%%%%%%

%%%%%%%%%%%%%%%%%%%%%  % EPS 图片支持  %%%%%%%%%%%%%%%%%%%%%%%%%%%
\usepackage{graphicx}                                            %
%%%%%%%%%%%%%%%%%%%%%%%%%%%%%%%%%%%%%%%%%%%%%%%%%%%%%%%%%%%%%%%%%%


\begin{document}
%\CJKindent     %在CJK环境中,中文段落起始缩进2个中文字符
%\indent
\graphicspath{{/home/jun_jiang/Documents/Latex_art_beamer/Presentation_Beamer/Figures/}}
%\graphicspath{{/media/Windows_7/Documents/Presentation_Beamer/Figures/}}
%
\renewcommand{\abstractname}{\small{\CJKfamily{hei} 摘\quad 要}} %\CJKfamily{hei} 设置中文字体,字号用\big \small来设
\renewcommand{\refname}{\centering\CJKfamily{hei} 参考文献}
%\renewcommand{\figurename}{\CJKfamily{hei} 图.}
\renewcommand{\figurename}{{\bf Fig}.}
%\renewcommand{\tablename}{\CJKfamily{hei} 表.}
\renewcommand{\tablename}{{\bf Tab}.}

%将图表的Caption写成 图(表) Num. 格式
\makeatletter
\long\def\@makecaption#1#2{%
  \vskip\abovecaptionskip
  \sbox\@tempboxa{#1. #2}%
  \ifdim \wd\@tempboxa >\hsize
    #1. #2\par
  \else
    \global \@minipagefalse
    \hb@xt@\hsize{\hfil\box\@tempboxa\hfil}%
  \fi
  \vskip\belowcaptionskip}
\makeatother

\newcommand{\keywords}[1]{{\hspace{0\ccwd}\small{\CJKfamily{hei} 关键词:}{\hspace{2ex}{#1}}\bigskip}}

%%%%%%%%%%%%%%%%%%中文字体设置%%%%%%%%%%%%%%%%%%%%%%%%%%%
%默认字体 defalut fonts \TeX 是一种排版工具 \\		%
%{\bfseries 粗体 bold \TeX 是一种排版工具} \\		%
%{\CJKfamily{song}宋体 songti \TeX 是一种排版工具} \\	%
%{\CJKfamily{hei} 黑体 heiti \TeX 是一种排版工具} \\	%
%{\CJKfamily{kai} 楷书 kaishu \TeX 是一种排版工具} \\	%
%{\CJKfamily{fs} 仿宋 fangsong \TeX 是一种排版工具} \\	%
%%%%%%%%%%%%%%%%%%%%%%%%%%%%%%%%%%%%%%%%%%%%%%%%%%%%%%%%%

%\addcontentsline{toc}{section}{Bibliography}

%-------------------------------The Title of The Paper-----------------------------------------%
\title{\rm{VASP~}软件的\rm{Fermi}能计算问题的讨论}
%----------------------------------------------------------------------------------------------%

%----------------------The Authors and the address of The Paper--------------------------------%
\author{
\small
%Author1, Author2, Author3\footnote{Communication author's E-mail} \\    %Authors' Names	       %
\small
%(The Address,City Post code)						%Address	       %
}
\date{}					%if necessary					       %
%----------------------------------------------------------------------------------------------%
\maketitle

%-------------------------------------------------------------------------------The Abstract and the keywords of The Paper----------------------------------------------------------------------------%
%\begin{abstract}
%The content of the abstract
%\end{abstract}

%\keywords {Keyword1; Keyword2; Keyword3}
%-----------------------------------------------------------------------------------------------------------------------------------------------------------------------------------------------------%

%----------------------------------------------------------------------------------------The Body Of The Paper----------------------------------------------------------------------------------------%
%Introduction
\section{周期体系计算中的能量零点的移动与\textrm{Fermi}能}
\subsection{问题的提出}
在电子结构计算中,对分子、原子等有限尺度体系,习惯上将能量零点取在无穷远,即无穷远处的静止电子的能量为零。这样选择的能量参考点,束缚态的电子能量都是负值,并且基态最高占据态的电子能级与第一电离能的负值对应。但是对于理想的周期体系来说,“无穷远”因为引入周期性而消失,所以必须另外选择能量零点。\upcite{JPC-SSP12-4409_1979,XIE-LU}

\subsection{晶体总能量计算与能量零点选择}
一般地,晶体中的基态总能量$E_T$可以表示成晶格中的电子能量$E_{e-e}$与离子实排斥能$E_{N-N}$之和:~
	\begin{equation}
		E_T=E_{e-e}+E_{N-N}=T[\rho]+E_{ext}+E_{\mathrm{Coul}}+E_{\mathrm{XC}}+E_{N-N}
		\label{eq:Crystal_ENE_R}
	\end{equation}
根据密度泛函理论(\textrm{Density-Functional Theory, DFT})和\textrm{Kohn-Sham}方程\upcite{PRB136-864_1964,PRA140-1133_1965},电子本征态方程为:~
\begin{equation}
	\bigg[\dfrac12\nabla^2+V_{ext}(\vec r)+V_{\mathrm{Coul}}(\vec r)+V_{\mathrm{XC}}[\rho(\vec r)]\bigg]|\psi_i(\vec r)\rangle=\varepsilon_i|\psi_i(\vec r)\rangle
	\label{eq:DFT}
\end{equation}
动能泛函用单电子能量表示为
\begin{equation}
	T[{\rho}]=\sum_in_i\langle\psi_i|\varepsilon_i-V_{\mathrm{KS}}|\psi_i\rangle
	\label{eq:DFT_Kin}
\end{equation}
$n_i$是$\psi_i$上的电子占据数,$\varepsilon_i$是其能量本征值,因此总能量的泛函表示为:
\begin{equation}
	E_T=\sum_in_i\varepsilon_i-\dfrac12\int\int\mathrm{d}\vec r\mathrm{d}\vec r\dfrac{\rho(\vec r)\rho(\vec r^{\prime})}{|\vec r-\vec r^{\prime}|}+\int\mathrm{d}\vec r\rho(\vec r)[\epsilon_{\mathrm{XC}}(\vec r)-V_{\mathrm{XC}}(\vec r)]+E_{N-N}
	\label{eq:DFT_ENE_R}
	\end{equation}

对于周期体系来说,因为电子的能量本征态是与动量空间($\vec K$空间)相关联,即\textrm{Kohn-Sham}方程表示为:
\begin{equation}
	\bigg[\dfrac12\vec k^2+V_{ext}(\vec k)+V_{\mathrm{Coul}}(\vec k)+V_{\mathrm{XC}}[\rho(\vec k)]\bigg]|\psi_i^{\vec k}(\vec r)\rangle=\varepsilon_i^{\vec k}|\psi_i^{\vec k}(\vec r)\rangle
	\label{eq:DFT-k}
\end{equation}
显然,总能量在动量空间中计算更方便:~
\begin{equation}
	E_T=\sum_{i,\vec k}n_i\varepsilon_i^{\vec k}-\dfrac{\Omega}2\sum_{\vec k}\rho^{\ast}(\vec k)V_{\mathrm{Coul}}(\vec k)+\Omega\sum_{\vec k}\rho^{\ast}(\vec k)[\epsilon_{\mathrm{XC}}(\vec k)-V_{\mathrm{XC}}(\vec k)]+E_{N-N}
	\label{eq:DFT_ENE_G}
\end{equation}
其中$V_{\mathrm{Coul}}(\vec k)$、$\epsilon_{\mathrm{XC}}(\vec k)$与$\rho^{\ast}(\vec k)$分别是\textrm{Coulomb}相互作用、单个电子的交换-相关能、交换-相关势和电子密度的\textrm{Fourier}分量。

实际计算中需要作一些数学处理:~
\begin{itemize}
	\item 交换-相关势和交换-相关能的计算一般先在实空间计算$\epsilon_{\mathrm{XC}}(\vec r)$和$V_{\mathrm{XC}}(\vec r)$后,再通过\textrm{Fourier~}变换到动量空间,得到$\epsilon_{\mathrm{XC}}(\vec k)$和$V_{\mathrm{XC}}(\vec k)$。
	\item 由\textrm{Poisson}方程
\begin{equation}
	\nabla^2V_{\mathrm{Coul}}(\vec r)=-4\pi\rho(\vec r)
	\label{eq:Poisson}
\end{equation}
的\textrm{Fourier}展开有
\begin{equation}
	V_{\mathrm{Coul}}(\vec k)=\dfrac{4\pi\rho^{\ast}(\vec k)}{|\vec k|^2}
	\label{eq:FFT_Poisson}
\end{equation}
显然$V_{\mathrm{Coul}}(\vec k=0)$是发散的;
	\item 考虑离子间\textrm{Coulomb}相互作用能之和
	\begin{equation}
		E_{N-N}=\dfrac12\sum_{\vec R,s}\sideset{}{^{\prime}}\sum_{\vec R^{\prime},\vec s^{\prime}}\dfrac{Z_sZ_{s^{\prime}}}{|\vec R+\vec r_s-\vec R^{\prime}-\vec r_s^{\prime}|}
		\label{eq:Ion_Coulomb_ENE}
	\end{equation}
这里$Z_s$是离子实的电荷数,$\vec R$表示晶格点的位矢,$\vec r_s$代表元胞内原子的相对位矢。因为$E_{N-N}$求和包含无穷多项,是发散的;
	\item 用于求解能量本征态的式\eqref{eq:DFT-k}中$V_{ext}$的\textrm{Fourier}分量在$\vec k=0$处也是发散的。
\end{itemize}
因此总能量泛函中,$E_{N-N}$、$V_{\mathrm{Coul}}(\vec k=0)$和$V_{ext}(\vec k=0)$这三项单独都是发散的,但因为整个体系出于电中性,所以这些发散项相互抵消,应是一个常数。

因此实际的总能计算中,首先在求解\textrm{Kohn-Sham}方程时,先将$V_{\mathrm{Coul}}(\vec k=0)$和$V_{ext}(\vec k=0)$同时置为零,这相当于势能作一平移,或者说重新定义势能零点。由此得到的总能泛函为:~
\begin{equation}
	E_T=\sum_{i,\textcolor{red}{\vec k\neq0}}n_i\varepsilon_i^{\vec k}-\dfrac{\Omega}2\sum_{\textcolor{red}{\vec k\neq 0}}\rho^{\ast}(\vec k)V_{\mathrm{Coul}}(\vec k)+\Omega\sum_{\vec k}\rho^{\ast}(\vec k)[\epsilon_{\mathrm{XC}}(\vec k)-V_{\mathrm{XC}}(\vec k)]+E_{N-N}
	\label{eq:DFT_ENE_G-2}
\end{equation}
最后在总能量计算中,考虑补偿势能零点的这一平移。

\subsection{发散项的处理}
根据上面的讨论,总能量中发散项之和为:~
	\begin{equation}
		\begin{aligned}
			\lim_{\vec k\rightarrow0}\Omega&\bigg[\dfrac12V_{\mathrm{Coul}}(\vec k)+\sum_sv_{ext}^s(\vec k)\bigg]\rho^{\ast}(\vec k)+\dfrac12\sum_{\vec R,s}\sideset{}{^{\prime}}\sum_{\vec R^{\prime},\vec s^{\prime}}\dfrac{Z_sZ_{s^{\prime}}}{|\vec R+\vec r_s-\vec R^{\prime}-\vec r_s^{\prime}|}\\
			=&\sum_s\alpha_s\sum_sZ_s+E_{\mathrm{Ewald}}
		\end{aligned}
		\label{eq:diver-term}
	\end{equation}
	
$V_{ext}$在不存在其他外场时,一般只考虑离子-电子的\textrm{Coulomb}相互作用,
	\begin{equation}
		\begin{aligned}
			V_{ext}(\vec r)&=\sum_{\vec R,s}\dfrac{-Z_s}{|\vec r-\vec R-\vec r_s|}\\
			&\equiv\sum_{\vec R,s}v_{ext}^s(\vec r-\vec R-\vec r_s)
		\end{aligned}
		\label{eq:Ion-ele_Coulomb}
	\end{equation}

对于形如$Z_s/r$的外场,其\textrm{Fourier}分量在$\vec k=0$附近展开
	\begin{equation}
		v_{ext}^s(\vec k)=-\dfrac{4\pi Z_s}{\Omega|\vec k|^2}+\alpha_s+O(\vec k)
		\label{eq:V_ext}
	\end{equation}
展开$\rho^{\ast}(\vec k)$,有
	\begin{equation}
		\lim_{\vec k\rightarrow 0}\rho^{\ast}(\vec k)=\dfrac{\sum_sZ_s}{\Omega}+\beta|\vec k|^2+O(\vec k)
		\label{eq:rho_ext}
	\end{equation}
去掉高次项,有
\begin{equation}
	\begin{aligned}
		\lim_{\vec k\rightarrow 0}&\bigg[\boxed{\textcolor{blue}{\dfrac{\Omega}2\dfrac{4\pi[\rho^{\ast}(\vec k)]^2}{|\vec k|^2}}}+\boxed{\Omega}\bigg(\boxed{\textcolor{blue}{-\dfrac{4\pi\sum_sZ_s}{\Omega|\vec k|^2}}}+\sum_s\alpha_s\bigg)\boxed{\rho^{\ast}(\vec k)}+\boxed{\textcolor{red}{\dfrac12\dfrac{4\pi(\sum_sZ_s)^2}{\Omega|\vec k|^2}}}\bigg]\\
		&+\boxed{\dfrac12\sum_{\vec R,s}\sideset{}{^{\prime}}\sum_{\vec R^{\prime},\vec s^{\prime}}\dfrac{Z_sZ_{s^{\prime}}}{|\vec R+\vec r_s-\vec R^{\prime}-\vec r_{s^{\prime}}|}-\lim_{\vec k\rightarrow0}\textcolor{red}{\dfrac12\dfrac{4\pi(\sum_sZ_s)^2}{\Omega|\vec k|^2}}}\\
		=&\sum_s\alpha_s\sum_sZ_s+\textcolor{magenta}{E_{\mathrm{Ewald}}}
	\end{aligned}
	\label{eq:V_ext_exp2}
\end{equation}
其中离子间排斥势采用\textrm{Ewald~}方法得到\upcite{Born-Huang,R.Martin}:~对于形如点电荷形式的静电势$\dfrac{e^2}r$,可引入\textrm{Gauss~}误差函数\upcite{Grosso-Parravicini}
\begin{equation}
	\begin{aligned}
		&\mathrm{erf}(x)=\dfrac2{\sqrt{\pi}}\int_0^{x}\mathrm{e}^{-t^2}\mathrm{d}t\\
		&\mathrm{erfc}(x)=\dfrac2{\sqrt{\pi}}\int_x^{\infty}\mathrm{e}^{-t^2}\mathrm{d}t\\
		\mbox{满足}\quad&\mathrm{erf}(x)+\mathrm{erfc}(x)=1
	\end{aligned}
	\label{eq:err_fun}
\end{equation}
得到恒等式(见图\ref{Error_Function}):
\begin{equation}
	\dfrac{e^2}r\equiv\dfrac{e^2}r\mathrm{erf}(\sqrt{\eta}r)+\dfrac{e^2}r\mathrm{erfc}(\sqrt{\eta}r)
	\label{eq:err_fun_comp}
\end{equation}
\begin{figure}[h!]
\centering
\vspace*{-0.10in}
\includegraphics[height=2.55in,width=5.8in,viewport=0 0 1100 455,clip]{Ewald_method.png}
\caption{\small \textrm{Decomposition of the potential $-e^2/r$ (singular at the origin and of long-range nature) into a contribution $-(e^2/r)\mathrm{erf}(\sqrt{\eta}r)$(regular at the origin of long-range) and a contribution $-(e^2/r)\mathrm{erfc}(\sqrt{\eta}r)$ (singular at the origin and of short-range nature). Here $\sqrt{\eta}=1 (\mathrm{Bohr radius unit})$ is chosen.}\upcite{Grosso-Parravicini}}%(与文献\cite{EPJB33-47_2003}图1对比)
\label{Error_Function}
\end{figure}

	\begin{equation}
		\begin{aligned}
			E_{\textrm{Ewald}}=&\dfrac12\sum_{\vec R,s}\sideset{}{^{\prime}}\sum_{\vec R^{\prime},\vec s^{\prime}}\dfrac{Z_sZ_{s^{\prime}}}{|\vec R+\vec r_s-\vec R^{\prime}-\vec r_{s^{\prime}}|}-\lim_{\vec k\rightarrow0}\dfrac12\times\dfrac{4\pi(\sum_sZ_s)^2}{\Omega|\vec k|^2}\\
			=&\dfrac12\sum_{\vec R,s}\sideset{}{^{\prime}}\sum_{\vec R^{\prime},\vec s^{\prime}}\dfrac{Z_sZ_{s^{\prime}}}{|\vec R+\vec r_s-\vec R^{\prime}-\vec r_{s^{\prime}}|}-\dfrac1{2\Omega}\sum_{s,s^{\prime}}\int\mathrm{d}\vec r\dfrac{Z_sZ_{s^{\prime}}}r\\
			=&\sum_{s,s^{\prime}}Z_sZ_{s^{\prime}}\bigg\{\dfrac{2\pi}{\Omega}\sum_{\vec k\neq 0}\cos[\vec k\cdot(\vec r_s-\vec r_{s^{\prime}})]\dfrac{\mathrm{e}^{-|\vec k|^2/4\eta}}{|\vec k|^2}\\
			&-\dfrac{\pi}{2\eta\Omega}+\dfrac14\sum_{\vec R}\dfrac{\mathrm{erf}(\sqrt{\eta}x)}x\bigg|_{\vec R+\vec r_s-\vec r_s^{\prime}\neq0}-\sqrt{\dfrac{\eta}{\pi}}\delta_{s,s^{\prime}}\bigg\}
		\end{aligned}
		\label{eq:Ewald_ENE}
	\end{equation}
	$\mathrm{erf}(x)$是误差函数,$\sqrt{\eta}$原则上是任意参数。$\alpha_s$由$v_{ext}^s(\vec r)$确定:~
	\begin{equation}
		\alpha_s=\lim_{\vec k\rightarrow0}\bigg[v_{ext}^s(\vec k)+\dfrac{4\pi Z_s}{\Omega|\vec k|^2}\bigg]=\dfrac1{\Omega}\int\mathrm{d}\vec r\bigg[v_{ext}^s(\vec r)+\dfrac{Z_s}r\bigg]
		\label{eq:alpha_s}
	\end{equation}
由此得到的总能量表达式是:
\begin{equation}
	\begin{aligned}
		E_T=&\sum_i\varepsilon_i-\dfrac{\Omega}2\sum_{\vec k\neq0}\rho^{\ast}(\vec k)V_{\mathrm{Coul}}(\vec k)\\
		&+\Omega\sum_{\vec k}\rho^{\ast}(\vec k)[\epsilon_{\mathrm{XC}}(\vec k)-V_{\mathrm{XC}}(\vec k)]\\
		&+\sum_s\alpha_s\sum_sZ_s+E_{\mathrm{Ewald}}
	\end{aligned}
	\label{eq:TOT_ENE_Finial}
\end{equation}

\textrm{VASP~}软件的总能量计算即遵照式\eqref{eq:TOT_ENE_Finial}计算的。图\ref{TOTEN_VASP}给出就是\textrm{VASP~}总能计算的输出形式:~
\begin{figure}[h!]
\centering
\vspace*{-0.12in}
\includegraphics[height=3.85in,width=4.2in,viewport=0 0 600 495,clip]{VASP_Total_ENE.png}
\caption{\small \textrm{The Total-E calculated by VASP.}}%(与文献\cite{EPJB33-47_2003}图1对比)
\label{TOTEN_VASP}
\end{figure}

%根据\textrm{Ewald}的势能计算方法,$\boxed{\dfrac12\dfrac{4\pi(\sum_sZ_s)^2}{\Omega|\vec k|^2}}$表示的电子势能在$\vec k=0$处的贡献,可分为
%	\begin{itemize}
%		\item \textcolor{purple}{对应于实空间电子势能的长程可收敛部分:~}式\eqref{eq:Ewald_ENE}中第三项
%			\begin{displaymath}
%				-(\sum\limits_{s,s^{\prime}}Z_sZ_{s^{\prime}})\dfrac14\sum_{\vec R}\dfrac{\mathrm{erf}(\sqrt{\eta}x)}x\bigg|_{\vec R+\vec r_s-\vec r_s^{\prime}\neq0}
%			\end{displaymath}
%		\item \textcolor{purple}{对应于实空间电子势能的短程发散部分:~}式\eqref{eq:Ewald_ENE}中第二项
%			\begin{displaymath}
%				(\sum_{s,s^{\prime}}Z_sZ_{s^{\prime}})\dfrac{\pi}{2\eta\Omega}
%			\end{displaymath}
%		\textcolor{red}{注意:~实际计算中,因为误差函数的参数$\sqrt{\eta}$不为零,因此该发散部分表示为一个大数而不是$\infty$}。
%	\end{itemize}
%	类似地,不难看出,式\eqref{eq:Ewald_ENE}中\textcolor{blue}{第一项}和\textcolor{magenta}{第四项}分别对应离子-电子的\textrm{Coulomb}相互作用
%	\begin{displaymath}
%		\boxed{\dfrac12\sum_{\vec R,s}\sideset{}{^{\prime}}\sum_{\vec R^{\prime},\vec s^{\prime}}\dfrac{Z_sZ_{s^{\prime}}}{|\vec R+\vec r_s-\vec R^{\prime}-\vec r_{s^{\prime}}|}}
%	\end{displaymath}
%	的\textcolor{blue}{长程收敛}和\textcolor{magenta}{短程发散}部分。
%
%以\textrm{FCC-Al}为例,采用\textrm{VASP~}计算得到有关数值如下:~
%\begin{displaymath}
%	\begin{aligned}
%	&\mathrm{E-fermi}:~&7.4406\\
%	&\sum\alpha_iZ_i:~&-0.1949\\
%	&\mathrm{Ewald-Energy}:~&-72.4621\\
%	&\mathrm{XC(G=0)}:~&-10.00040 \\
%	\end{aligned}
%\end{displaymath}
%将\textrm{VASP~}计算中的\textrm{Ewald-Energy}按式\eqref{eq:Ewald_ENE}分解,各部分对应的数值为:~
%\begin{displaymath}
%	\begin{aligned}
%	&\mathrm{Part-1}:~&1.320907 \\
%	&\mathrm{Part-2}:~&-50.176618 \\ 
%	&\mathrm{Part-3}:~&1.481905 \\
%	&\mathrm{Part-4}:~&-25.088309 \\
%	\end{aligned}
%\end{displaymath}
%\textcolor{red}{不难看出,这里第二项和第四项分别代表两部分势能在$\vec k=0$的发散项贡献,因此数值的绝对值比另外两项大得多。}

\section{能量零点移动对能量本征态的影响}
根据上述讨论,因为能量零点的平移,周期体系计算的能量本征值$\varepsilon_i$的数值一般不绝对为负值。习惯上在能带和态密度表示时,常常将\textrm{Fermi~}能设置成零。

参照总能计算中能量零点移动的讨论,可以计算\textrm{Kohn-Sham~}方程中势能零点引起的能量本征值的移动,%根据检索\textrm{VASP~}的代码发现:~\textcolor{red}{\textrm{VASP~}程序在构造\textrm{Fock~}矩阵的时候,已经包括了$V_{\mathrm{XC}}(\vec k=0)$的贡献~},即\textcolor{purple}{$\mathrm{XC(G=0)}$}对应的数值(见图\ref{TOTEN_VASP});~
$V_{\mathrm{Coul}}(\vec k=0)$和$V_{ext}[\rho(\vec k=0)]$。
即
\begin{equation}
	\lim_{\vec k\rightarrow0}\bigg[V_{\mathrm{Coul}}(\vec k)+\sum_sv_{ext}^s(\vec k)\bigg]
	\label{eq:part_diver-term}
\end{equation}
式\eqref{eq:part_diver-term}\textbf{势的移动}并不简单对应式\eqref{eq:diver-term}中\textbf{能量移动}的发散项求和。

在$\vec k=0$附近,将式\eqref{eq:V_ext}和\eqref{eq:rho_ext}代入式\eqref{eq:part_diver-term},去掉高次项,可有
\begin{equation}
	\begin{aligned}
		&\lim_{\vec k\rightarrow 0}\bigg[\dfrac{4\pi\rho^{\ast}(\vec k)}{|\vec k|^2}+\bigg(-\dfrac{4\pi\sum_sZ_s}{\Omega|\vec k|^2}+\sum_s\alpha_s\bigg)\bigg]\\
		=&\lim_{\vec k\rightarrow 0}\bigg[\boxed{\dfrac{4\pi}{\Omega}\dfrac{\sum_sZ_s}{|\vec k|^2}}+4\pi\beta\boxed{-\dfrac{4\pi\sum_sZs}{\Omega|\vec k^2|}}+\sum_s\alpha_s\bigg]\\
		=&\sum_s\alpha_s+4\pi\beta
	\end{aligned}
	\label{eq:V_shift-term}
\end{equation}
式\eqref{eq:V_shift-term}对应\textrm{VASP~}软件中给出的\textcolor{purple}{$\mathrm{alpha+bet}$}的数值,见图\ref{TOTEN_VASP}(在\textrm{VASP~}中,\textcolor{purple}{bet}项对应式\eqref{eq:V_shift-term}的$4\pi\beta$)。

上述推导也验证了赝势理论的基本思想:~对于电中性的周期体系,在倒空间中,电子\textrm{Coulomb~}势与原子核的吸引势相互抵消后,净的作用可近似为高阶奇点和一个平缓的势函数。\textrm{VASP~}软件中,$\alpha_s$和$\beta$的数值计算方式如下:~
\begin{itemize}
	\item \textcolor{blue}{$\alpha_s$取原子赝势在径向的第一个点(即离$\vec k=0$最近)的数值}
	\item \textcolor{blue}{$\beta$由各原子赝电荷密度前5个点(即离$\vec k=0$足够近)的数值两阶差分后求和得到}
\end{itemize}

传统的求解\textrm{Kohn-Sham~}方程计算能量本征值时,将势函数中包括核-电子吸引和电子间排斥排斥势的全部$\vec k=0$部分的贡献扣除。如果考虑补偿函数式\eqref{eq:V_shift-term}的贡献,则利用了上述两项的奇点能量部分抵消的特性,保留了势能零点在无穷远时的部分特征(\textcolor{red}{注意:~采用该能量补偿方案,高阶奇点仍然存在!})。

综上所述,在\textrm{VASP~}中,如果考虑周期体系的势能零点移动,则\textrm{Fermi~}的数值可取为\textbf{两项之和}:~
\begin{displaymath}
	\mathrm{E_{fermi}}=\textcolor{blue}{\mathrm{E-fermi}}+\textcolor{purple}{\mathrm{alpha+bet}}
\end{displaymath}
这就是排除高阶奇点后的\textrm{Fermi~}能,与传统的分子、原子中能量计算结果相近。
%\subsection{一点讨论}
%我们对\textrm{FCC-Al}的计算表明,
%\begin{displaymath}
%	\begin{aligned}
%	&\mathrm{E-fermi}:~&7.4406\\
%	&\sum\alpha_iZ_i:~&-0.1949\\
%	&\mathrm{Ewald-Energy}:~&-72.4621\\
%	&\mathrm{XC(G=0)}:~&-10.00040 \\
%	&\mathrm{alpha+bet}:~-&14.2459\\
%	\end{aligned}
%\end{displaymath}
%因此考虑势能零点移动修正的\textrm{Fermi}能应为(\textcolor{blue}{包含\textbf{OUTCAR}中$\mathrm{alpha+bet}$项}):
%\begin{displaymath}
%	7.4406-14.2459=-6.80\;\mathrm{eV}
%\end{displaymath}
%$\ast$注:上述计算验证~
%\begin{itemize}
%	\item 在\textrm{VASP~}计算中,$\mathrm{alpha+bet\mbox{项}}<0$恒成立
%	\item \textrm{VASP~}中$|\mathrm{E-Fermi}|<|\mathrm{alpha+bet}|$成立
%\end{itemize}

\section{\rm{VASP}计算结果对上述推导的检验}
考虑周期体系,如果原子间距离足够远,则周期体系计算结果应该逼近原子分子体系的计算结果,基于该思想,可以用\textrm{VASP}软件对假想的\textrm{Si}和\textrm{Fe}孤立原子(分别在足够大的晶胞中),然后逐渐减少原子间距离,考察\textrm{OUTCAR}文件中\textcolor{purple}{\textrm{alpha+bet}}的变化,结果列于图\ref{Fig:VASP_alpha+bet}。
\begin{figure}[h!]
\centering
\hspace*{-0.7in}
\vspace*{-0.2in}
\includegraphics[width=1.2\textwidth, viewport=0 0 1520 780, clip]{VASP_Fermi_alpha-bet.pdf}
\caption{\small Compare the \textcolor{purple}{\textrm{alpha+bet}} with the distance of atom.}%(与文献\cite{EPJB33-47_2003}图1对比)
\label{Fig:VASP_alpha+bet}
\end{figure}

不难看出,随着原子间距离的逐步增大,\textcolor{purple}{\textrm{alpha+bet}}的数值逐渐趋向于零,相应的\textrm{Fermi}数值逐渐逼近原子能级(数据未在此列出)。一般地认为原子间距离超过10~\AA ,可以原子间没有相互作用,考虑本次极端模型中原子采用\textrm{FCC}密堆积,原子间距离为10~\AA~时对应的晶胞参数为15.874~\AA ,当晶胞参数大于15.874~\AA ,\textcolor{purple}{\textrm{alpha+bet}}的数值将趋向于0,这一结论与图\ref{Fig:VASP_alpha+bet}中曲线趋向零的起始位置吻合得很好。因此,趋于极端情况的假想模型计算结果也验证了\textcolor{purple}{\textrm{alpha+bet}}表示的就是排除高阶奇点后的能量本征值修正项,与前一节讨论的结论一致。
%{
%\frametitle{发展统一理论框架下的材料计算程序}
%\begin{itemize}
%	\item
%\end{itemize}
%}

%-------------------The Figure Of The Paper------------------
%\begin{figure}[h!]
%\centering
%\includegraphics[height=3.35in,width=2.85in,viewport=0 0 400 475,clip]{PbTe_Band_SO.eps}
%\hspace{0.5in}
%\includegraphics[height=3.35in,width=2.85in,viewport=0 0 400 475,clip]{EuTe_Band_SO.eps}
%\caption{\small Band Structure of PbTe (a) and EuTe (b).}%(与文献\cite{EPJB33-47_2003}图1对比)
%\label{Pb:EuTe-Band_struct}
%\end{figure}

%-------------------The Equation Of The Paper-----------------
%\begin{equation}
%\varepsilon_1(\omega)=1+\frac2{\pi}\mathscr P\int_0^{+\infty}\frac{\omega'\varepsilon_2(\omega')}{\omega'^2-\omega^2}d\omega'
%\label{eq:magno-1}
%\end{equation}

%\begin{equation} 
%\begin{split}
%\varepsilon_2(\omega)&=\frac{e^2}{2\pi m^2\omega^2}\sum_{c,v}\int_{BZ}d{\vec k}\left|\vec e\cdot\vec M_{cv}(\vec k)\right|^2\delta [E_{cv}(\vec k)-\hbar\omega] \\
% &= \frac{e^2}{2\pi m^2\omega^2}\sum_{c,v}\int_{E_{cv}(\vec k=\hbar\omega)}\left|\vec e\cdot\vec M_{cv}(\vec k)\right|^2\dfrac{dS}{\nabla_{\vec k}E_{cv}(\vec k)}
% \end{split}
%\label{eq:magno-2}
%\end{equation}

%-------------------The Table Of The Paper----------------------
%\begin{table}[!h]
%\tabcolsep 0pt \vspace*{-12pt}
%%\caption{The representative $\vec k$ points contributing to $\sigmA_2^{xy}$ of interband transition in EuTe around 2.5 eV.}
%\label{Table-EuTe_Sigma}
%\begin{minipage}{\textwidth}
%%\begin{center}
%\centering
%\def\temptablewidth{0.84\textwidth}
%\rule{\temptablewidth}{1pt}
%\begin{tabular*} {\temptablewidth}{|@{\extracolsep{\fill}}c|@{\extracolsep{\fill}}c|@{\extracolsep{\fill}}l|}

%-------------------------------------------------------------------------------------------------------------------------
%&Peak (eV)  & {$\vec k$}-point            &Band{$_v$} to Band{$_c$}  &Transition Orbital
%Components\footnote{波函数主要成分后的括号中,$5s$、$5p$和$5p$、$4f$、$5d$分别指碲和铕的原子轨道。} &Gap (eV)   \\ \hline
%-------------------------------------------------------------------------------------------------------------------------
%&2.35       &(0,0,0)         &33$\rightarrow$34    &$4f$(31.58)$5p$(38.69)$\rightarrow$$5p$      &2.142   \\% \cline{3-7}
%&       &(0,0,0)         &33$\rightarrow$34    &$4f$(31.58)$5p$(38.69)$\rightarrow$$5p$      &2.142   \\% \cline{3-7}
%-------------------------------------------------------------------------------------------------------------------------
%\end{tabular*}
%\rule{\temptablewidth}{1pt}
%\end{minipage}{\textwidth}
%\end{table}

%-------------------The Long Table Of The Paper--------------------
%\begin{small}
%%\begin{minipage}{\textwidth}
%%\begin{longtable}[l]{|c|c|cc|c|c|} %[c]指定长表格对齐方式
%\begin{longtable}[c]{|c|c|p{1.9cm}p{4.6cm}|c|c|}
%\caption{Assignment for the peaks of EuB$_6$}
%\label{tab:EuB6-1}\\ %\\长表格的caption中换行不可少
%\hline
%%
%--------------------------------------------------------------------------------------------------------------------------------
%\multicolumn{2}{|c|}{\bfseries$\sigmA_1(\omega)$谱峰}&\multicolumn{4}{c|}{\bfseries部分重要能带间电子跃迁\footnotemark}\\ \hline
%\endfirsthead
%--------------------------------------------------------------------------------------------------------------------------------
%%
%\multicolumn{6}{r}{\it 续表}\\
%\hline
%--------------------------------------------------------------------------------------------------------------------------------
%标记 &峰位(eV) &\multicolumn{2}{c|}{有关电子跃迁} &gap(eV)  &\multicolumn{1}{c|}{经验指认} \\ \hline
%\endhead
%--------------------------------------------------------------------------------------------------------------------------------
%%
%\multicolumn{6}{r}{\it 续下页}\\
%\endfoot
%\hline
%--------------------------------------------------------------------------------------------------------------------------------
%%
%%\hlinewd{0.5$p$t}
%\endlastfoot
%--------------------------------------------------------------------------------------------------------------------------------
%%
%% Stuff from here to \endlastfoot goes at bottom of last page.
%%
%--------------------------------------------------------------------------------------------------------------------------------
%标记 &峰位(eV)\footnotetext{见正文说明。} &\multicolumn{2}{c|}{有关电子跃迁\footnotemark} &gap(eV) &\multicolumn{1}{c|}{经验指认\upcite{PRB46-12196_1992}}\\ \hline
%--------------------------------------------------------------------------------------------------------------------------------
%
%     &0.07 &\multicolumn{2}{c|}{电子群体激发$\uparrow$} &--- &电子群\\ \cline{2-5}
%\raisebox{2.3ex}[0pt]{$\omega_f$} &0.1 &\multicolumn{2}{c|}{电子群体激发$\downarrow$} &--- &体激发\\ \hline
%--------------------------------------------------------------------------------------------------------------------------------
%
%     &1.50 &\raisebox{-2ex}[0pt][0pt]{20-22(0,1,4)} &2$p$(10.4)4$f$(74.9)$\rightarrow$ &\raisebox{-2ex}[0pt][0pt]{1.47} &\\%\cline{3-5}
%     &1.50$^\ast$ & &2$p$(17.5)5$d_{\mathrm E}$(14.0)$\uparrow$ & &4$f$$\rightarrow$5$d_{\mathrm E}$\\ \cline{3-5}
%     \raisebox{2.3ex}[0pt][0pt]{$a$} &(1.0$^\dagger$) &\raisebox{-2ex}[0pt][0pt]{20-22(1,2,6)} &\raisebox{-2ex}[0pt][0pt]{4$f$(89.9)$\rightarrow$2$p$(18.7)5$d_{\mathrm E}$(13.9)$\uparrow$}\footnotetext{波函数主要成分后的括号中,2$s$、2$p$和5$p$、4$f$、5$d$、6$s$分别指硼和铕的原子轨道;5$d_{\mathrm E}$、5$d_{\mathrm T}$分别指铕的(5$d_{z^2}$,5$d_{x^2-y^2}$和5$d_{xy}$,5$d_{xz}$,5$d_{yz}$)轨道,5$d_{\mathrm{ET}}$(或5$d_{\mathrm{TE}}$)则指5个5$d$轨道成分都有,成分大的用脚标的第一个字母标示;2$ps$(或2$sp$)表示同时含有硼2$s$、2$p$轨道成分,成分大的用第一个字母标示。$\uparrow$和$\downarrow$分别标示$\alpha$和$\beta$自旋电子跃迁。} &\raisebox{-2ex}[0pt][0pt]{1.56} &激子跃迁。 \\%\cline{3-5}
%     &(1.3$^\dagger$) & & & &\\ \hline
%--------------------------------------------------------------------------------------------------------------------------------

%     & &\raisebox{-2ex}[0pt][0pt]{19-22(0,0,1)} &2$p$(37.6)5$d_{\mathrm T}$(4.5)4$f$(6.7)$\rightarrow$ & & \\\nopagebreak %\cline{3-5}
%     & & &2$p$(24.2)5$d_{\mathrm E}$(10.8)4$f$(5.1)$\uparrow$ &\raisebox{2ex}[0pt][0pt]{2.78} &a、b、c峰可能 \\ \cline{3-5}
%     & &\raisebox{-2ex}[0pt][0pt]{20-29(0,1,1)} &2$p$(35.7)5$d_{\mathrm T}$(4.8)4$f$(10.0)$\rightarrow$ & &包含有复杂的\\ \nopagebreak%\cline{3-5}
%     &2.90 & &2$p$(23.2)5$d_{\mathrm E}$(13.2)4$f$(3.8)$\uparrow$ &\raisebox{2ex}[0pt][0pt]{2.92} &强激子峰。$^{\ast\ast}$\\ \cline{3-5}
%$b$  &2.90$^\ast$ &\raisebox{-2ex}[0pt][0pt]{19-22(0,1,1)} &2$p$(33.9)4$f$(15.5)$\rightarrow$ & &B2$s$-2$p$的价带 \\ \nopagebreak%\cline{3-5}
%     &3.0 & &2$p$(23.2)5$d_{\mathrm E}$(13.2)4$f$(4.8)$\uparrow$ &\raisebox{2ex}[0pt][0pt]{2.94} &顶$\rightarrow$B2$s$-2$p$导\\ \cline{3-5}
%     & &12-15(0,1,2) &2$p$(39.3)$\rightarrow$2$p$(25.2)5$d_{\mathrm E}$(8.6)$\downarrow$ &2.83 &带底跃迁。\\ \cline{3-5}
%     & &14-15(1,1,1) &2$p$(42.5)$\rightarrow$2$p$(29.1)5$d_{\mathrm E}$(7.0)$\downarrow$ &2.96 & \\\cline{3-5}
%     & &13-15(0,1,1) &2$p$(40.4)$\rightarrow$2$p$(28.9)5$d_{\mathrm E}$(6.6)$\downarrow$ &2.98 & \\ \hline
%--------------------------------------------------------------------------------------------------------------------------------
%%\hline
%%\hlinewd{0.5$p$t}
%\end{longtable}
%%\end{minipage}{\textwidth}
%%\setlength{\unitlength}{1cm}
%%\begin{picture}(0.5,2.0)
%%  \put(-0.02,1.93){$^{1)}$}
%%  \put(-0.02,1.43){$^{2)}$}
%%\put(0.25,1.0){\parbox[h]{14.2cm}{\small{\\}}
%%\put(-0.25,2.3){\line(1,0){15}}
%%\end{picture}
%\end{small}

%-----------------------------------------------------------------------------------------------------------------------------------------------------------------------------------------------------%

\newpage
%--------------------------------------------------------------------------The Biblography of The Paper-----------------------------------------------------------------%
%\newpage																				%
%-----------------------------------------------------------------------------------------------------------------------------------------------------------------------%
%\begin{thebibliography}{99}																		%
%%\bibitem{PRL58-65_1987}H.Feil, C. Haas, {\it Phys. Rev. Lett.} {\bf 58}, 65 (1987).											%
%\end{thebibliography}																			%
%-----------------------------------------------------------------------------------------------------------------------------------------------------------------------%
%																					%
\phantomsection\addcontentsline{toc}{section}{Bibliography}	 %直接调用\addcontentsline命令可能导致超链指向不准确,一般需要在之前调用一次\phantomsection命令加以修正	%
\bibliography{/home/jun_jiang/Documents/Latex_art_beamer/ref/Myref}																			%
\bibliographystyle{/home/jun_jiang/Documents/Latex_art_beamer/ref/mybib}																		%
%  \nocite{*}																				%
%-----------------------------------------------------------------------------------------------------------------------------------------------------------------------%

\clearpage     
%\end{CJK} 前加上\clearpage是CJK的要求
\end{document}
