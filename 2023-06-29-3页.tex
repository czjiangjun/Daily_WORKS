%---------------------- TEMPLATE FOR REPORT ------------------------------------------------------------------------------------------------------%

%\thispagestyle{fancy}   % 插入页眉页脚                                        %
%%%%%%%%%%%%%%%%%%%%%%%%%%%%% 用 authblk 包 支持作者和E-mail %%%%%%%%%%%%%%%%%%%%%%%%%%%%%%%%%
%\title{More than one Author with different Affiliations}				     %
%\title{\rm{VASP}的电荷密度存储文件\rm{CHGCAR}}
\title{面向高温合金材料设计的计算模拟软件中的几个主要问题}
%\title{面向高温合金材料模拟的计算方法发展和创新研究}
\author[ ]{姜~骏}   %
%\author[ ]{姜~骏\thanks{jiangjun@bcc.ac.cn}}   %
%\affil[ ]{北京市计算中心}    %
%\author[a]{Author A}									     %
%\author[a]{Author B}									     %
%\author[a]{Author C \thanks{Corresponding author: email@mail.com}}			     %
%%\author[a]{Author/通讯作者 C \thanks{Corresponding author: cores-email@mail.com}}     	     %
%\author[b]{Author D}									     %
%\author[b]{Author/作者 D}								     %
%\author[b]{Author E}									     %
%\affil[a]{Department of Computer Science, \LaTeX\ University}				     %
%\affil[b]{Department of Mechanical Engineering, \LaTeX\ University}			     %
%\affil[b]{作者单位-2}			    						     %
											     %
%%% 使用 \thanks 定义通讯作者								     %
											     %
\renewcommand*{\Authfont}{\small\rm} % 修改作者的字体与大小				     %
\renewcommand*{\Affilfont}{\small\it} % 修改机构名称的字体与大小			     %
\renewcommand\Authands{ and } % 去掉 and 前的逗号					     %
\renewcommand\Authands{ , } % 将 and 换成逗号					     %
\date{} % 去掉日期									     %
%\date{2020-12-30}									     %

%%%%%%%%%%%%%%%%%%%%%%%%%%%%%%%%%%%%%%%%%%  不使用 authblk 包制作标题  %%%%%%%%%%%%%%%%%%%%%%%%%%%%%%%%%%%%%%%%%%%%%%
%-------------------------------The Title of The Report-----------------------------------------%
%\title{报告标题:~}   %
%-----------------------------------------------------------------------------

%----------------------The Authors and the address of The Paper--------------------------------%
%\author{
%\small
%Author1, Author2, Author3\footnote{Communication author's E-mail} \\    %Authors' Names	       %
%\small
%(The Address,City Post code)						%Address	       %
%}
%\affil[$\dagger$]{清华大学~材料加工研究所~A213}
%\affil{清华大学~材料加工研究所~A213}
%\date{}					%if necessary					       %
%----------------------------------------------------------------------------------------------%
%%%%%%%%%%%%%%%%%%%%%%%%%%%%%%%%%%%%%%%%%%%%%%%%%%%%%%%%%%%%%%%%%%%%%%%%%%%%%%%%%%%%%%%%%%%%%%%%%%%%%%%%%%%%%%%%%%%%%
\maketitle
%\thispagestyle{fancy}   % 首页插入页眉页脚 
\section{引言}
\ch{Ni}-基单晶高温合金是航空发动机中核心部件的主要材料,其成元素与结构非常复杂。尽管近年来通用计算机的运算能力有了很大提升,但是传统的材料计算方式,仍难以模拟此类合金材料的力学和热力学行为。必须针对复杂合金的特点,创新和发展材料模拟与计算的模式:~从计算物理的角度出发,除了对单晶高温合金进行合理的建模外,针对单晶高温合金的计算规模,应考虑发展适合复杂合金模型的结构对称性分析方法;~根据主要计算工具\textrm{VASP}软件在原子、电子层次计算的特点,应在提升软件计算效率方面实现创新。此外,鉴于合金材料中有大量含有~\textit{d}-电子的过渡金属,其中重元素的旋-轨耦合效应也将难以忽略,有必要将旋-轨耦合效应对体系性能的影响纳入计算。 在此基础上,才能通过材料基本属性的计算,揭示其高温条件下物理机理的深刻内涵。

\section{对称性与能带路径的标准化}
对称性伴随的是系统在变换过程中物理量的守恒,利用对称性,能在有限条件下获得更多的物理信息。对于高温合金材料而言,因为组成元素及其组分比较复杂,为合理反映微量元素的影响,设计的合金模型必须足够大。相应的计算规模也将大大超过普通材料模拟,这是高温合金模拟的主要困难之一。如何利用体系的晶体对称性,有效降低计算规模、提升模拟计算的能力,进而建立微观空间对称划分与宏观材料性质的关系,变得非常重要。

在之前的工作中,通过对\textrm{VASP}软件的代码剖析,我们注意到\textrm{VASP}的对称性分析功能只有点群模块,缺少晶体空间群模块,我们从晶体学理论出发,按照点群-空间群生成对应法则,编写了全部空间群生成代码,完善了\textrm{VASP}软件的晶体原子和空间对称性分析功能。空间群分析功能的引入,除了有助于高温合金材料模型的对称性判断,还可利用空间群的划分,明确多组元复杂模型微观空间结构与宏观力学特性的关系,简化高温合金材料的力学性质模拟。

周期体系的电子结构都可用能带表示,能带显示了体系中电子能量沿高对称方向的色散关系。绘制能带图时所需的高对称性$\vec k$~点方向,主要取决研究者的经验和习惯,缺乏统一的规则,所以有很大的随意性。%\textrm{Setyawan}等\upcite{CMS49-299_2010}以\textrm{BCC}结构的\ch{GeF4}的能带为例,指出
单晶高温合金材料模型往往具有高度近似的电子结构,确定合理的电子结构表示路径,规范统一的能带路径,不仅方便模型的电子结构数据对比,也将方便数据挖掘技术在能带领域的应用。我们开发了嵌入式功能模块,实现了能带表示的$\vec k$~点路径标准化,并将模块集成到\textrm{VASP}软件中。对软件对称性分析模块功能的发展,提升现有软件的计算能力,也为数据挖掘技术应用于高温合金材料的电子结构解析提供了必要的基础。

\section{\rm{VASP}软件的并行与计算性能的提升}
当前高通量材料计算软件主要面向简单材料,对高温合金材料的软件模拟很难提供有效支持,因此必须在现有软件基础上发展和创新计算模式,才能完成对高温合金材料的高效模拟。\textrm{VASP}作为材料模拟软件的代表,具有良好的材料物性计算模拟能力,有着出众的并行效率。通过分析\textrm{VASP}软件的并行与计算特色,可有针对性地考虑适应高温合金的计算模拟性能提升的策略。
%这里主要围绕核心计算软件\textrm{VASP}的并行能力和计算特点,从自动计算流程对核心计算软件的调度能力、自动流程软件与系统作业管理系统的交互能力几个方向发展相关方法,从整体上提高高通量材料计算自动流程处理复杂模型的尺度与规模。

\subsection{\rm{VASP}的并行规模和\rm{FFT}}
%核心计算软件是材料计算最重要的部分,所有的材料模拟和物性计算都要通过核心计算软件完成,
传统材料模拟主要依赖操作系统支持的\textrm{MPI}并行接口,完成大规模的自洽迭代和矩阵对角化,但这一模式的并行扩展性只能到数核-十数核,而且由于软件的\textrm{MPI}并行性能差异极大,扩展性也较差(一般不超过32核)。当组织多核完成复杂体系模拟时,常常造成计算资源的浪费。\textrm{VASP}的并行和扩展性都有突出的表现,一般的通用计算机上,\textrm{VASP}在64-128核能保持良好的的并行线性度。源于\textrm{VASP}在\textrm{MPI}并行的基础上,引入轮询调度算法\textrm{(Round-Robin Scheduling)},实现计算任务在计算资源(节点或核)分配上的负载均衡。材料模拟计算的自洽迭代求解偏微分方程\textrm{(Partial Differential Equations,~PDE)}是资源消耗的主要部分,\textrm{VASP}在\textrm{PDE}方程自洽迭代求解中应用优化算法(包括共轭梯度法、\textrm{RMM-DIIS}方法等)通过约束对角化矩阵的规模,限制节点间通信。保证了\textrm{MPI}并行的效率和扩展性。\textrm{VASP}的并行上限可以突破256核,但在500核数量级上,并行效率下降比较明显。如果对并行系统与\textrm{VASP}结合作深度改造(如国家超算天津中心的方案),\textrm{VASP}的并行扩展可以到$10^4$核级别,这一改造需要对底层代码和计算框架作较大规模改动,一般科研工作者不具备这样的软件改写能力。

\textrm{VASP}的自洽迭代中,涉及电荷密度、势以及波函数计算过程存在大量的快速\textrm{Fourier}变换(\textrm{Fast Fourier Tranform},这些\textrm{FFT}需要大量的内存开销,对并行效率影响较大,如果能从算法层面降低\textrm{FFT}的存储和通信,有助于进一步提升\textrm{VASP}的并行,提高软件处理合金模型的能力。这方面有待进一步深入研究。
%由数据结构出发,构建支撑数据到计算资源的均衡负载的软件框架,将有望系统地提升材料计算核心软件的并行能力和扩展性,从而提升软件处理更大规模计算模型的能力。针对\ch{Ni}-基单晶高温合金材料的模拟,考虑到模型的复杂度(过渡金属原子多、原子成分变化大),软件框架的开发将重点解决软件的并行扩展性问题,通过建立以密度矩阵理论为基础的线性标度算法及软件框架,原理性改变哈密顿矩阵结构,优化作业并行效率。

\subsection{\rm{VASP}的计算特色和计算能力提升策略}
\textrm{VASP}软件是目前最流行的材料微观模拟软件之一,有着优异的并行能力。其计算特色主要包括:

(1)~物理思想层面,\textrm{VASP}采用的投影子缀加波\textrm{(PAW)}方法很好地平衡了计算的精度和效率,既保留了平面波-赝势方法的高效率,又包含了\textrm{LAPW}-全势方法的高精度计算思想,特别是针对含有大量\textit{d}-电子的高温合金体系,\textrm{PAW}从物理原理上保证了计算结果的可靠与高效。

(2)~\textrm{VASP}的离子-电子自洽迭代采用的是从头算分子动力学\textrm{AIMD}方法,通过在动力学系统中引入经典力学的绝热能量,实现电子与离子运动既可在同一动力学框架内处理,并在时间尺度上将两者分离。电子弛豫过程与分子动力学可以用类似迭代方式处理,不但大大简约了程序的复杂度,而且使得分子动力学与密度泛函理论紧密结合,无需诉诸跨尺度的势函数(力场)模拟。

(3)~算法层面上,\textrm{VASP}引入了大量的优化算法(如共轭梯度算法、\textrm{Davison}算法和\textrm{RMM-DIIS}算法等),有效地保证了矩阵对角化和自洽迭代的快速收敛及稳定性问题。高温合金材料的设计更多的关注体系的热力学与结构性质,\textrm{VASP}的多层次优化算法支持的迭代收敛的特色,方便了高温合金材料的能量学数据的获得,为热力学与力学参数的计算提供了保障。

面向\ch{Ni}-基单晶高温合金这样的复杂体系模拟,必须从计算方法层面发展和提升软件对单晶高温合金的计算效率。除去软件已有的\textrm{AIMD}自洽迭代模式和\textrm{VASP}现有的共轭梯度、\textrm{RMM-DIIS}等经典的优化算法,有必要引入机器学习算法(如神经网络)在内的多种优化算法,实现复杂模型的自洽迭代快速稳定收敛。

\section{提高计算流程的并发与均衡、计算资源的管理和调度能力}
当前的高通量材料计算自动流程软件大部分应用\textrm{Python}语言开发,有较好的灵活性,参照\textrm{VASP}软件的并行扩展度提高策略,通过重新设计计算流程,引入均衡负载算法,将可独立并发的计算任务根据计算资源均衡分配,提升计算流程的水平扩展\textrm{(Scale Out)}能力。

高通量材料计算流程软件\textrm{Mater Projects}的计算流程组织、管理和参数传递,都基于数据库实现,大大方便了复杂计算流程中的子任务的有序组装、分配。\ch{Ni}-基单晶高温合金材料的模拟流程流程,已远超现有自动流程软件的支持范围,复杂体系的材料模拟生成海量的初始结构,结构优化子过程的并发将有助于提升计算效率。利用数据库技术,针对单晶高温合金材料的计算模拟过程,开发新型自动流程软件,将每个子过程与传递参数都分解为数据库元素,组织并优化成并发度高的计算流程(如图\ref{CH4_comp_BCC}所示),将会提升计算流程对计算资源的利用率,防止计算流程设计不合理引起的资源浪费,达到计算资源的合理、有效地调度和分配。 
\begin{figure}[h!]
\centering
\vskip -2pt
\includegraphics[height=2.25in]{CH4_complex_machine.png}
\caption{面向复杂体系材料模拟顺序流程的子过程并发化示意图}%
\label{CH4_comp_BCC}
\end{figure}
%\subsection{改进计算流程对计算资源的管理和调度能力}

\section{\it{d}-电子与高温合金材料}
高温合金材料中,除了\ch{Al}元素外,含有大量的过渡金属元素(含有未成对\textit{d}-电子),\textit{d}-电子的轨道形态丰富,根据电荷密度分布分析也不难看出,金属原子间的局域电荷密度的组分有明显的\textit{d}-电子的贡献,表明过渡金属间\textit{d}-电子有很明确的成键趋势,\textit{d}电子参与成键的化学局域性好,有利于合金元素的稳定;~此外,过渡金属原子的\textit{d}-电子分布在\ch{Al}原子的周围,将有助于\ch{Al}与周围原子的稳定,因为\ch{Al}的\textit{p}-电子易于离域,一旦有过渡金属元素的\textit{d}-电子组分,形变能力和成键能力会明显提升,这将大大提升\ch{Al}原子的稳定性。考虑到不同元素的\textit{d}-电子的含量差别很大,因此要通过定量分析高温合金中的\textit{d}-电子的具体功能和作用,为后续高温合金材料的性能提升,必须对体系进行高精度的第一原理计算。需要指出的是,由于\textit{d}-电子本身比较局域,电子相关(\textrm{electron correlation})效应显著,现有的\textrm{DFT}理论框架无法很精确地考虑类似高温合金这种复杂体系的\textrm{d}-电子相关效应。对高温合金的\textit{d}-电子效应的贡献,可能需要通过\textrm{DMFT}或\textrm{Green~function}等方法获得比较精确的电子相关估算。

\section{旋-轨耦合与高温合金}
功能材料中,当含有过渡金属为是重元素时,其价电子自旋与轨道角动量会发生显著的相互作用,对体系的物理性质有较显著的影响,这称为旋-轨耦合\textrm{(SOC)}效应(以价层及附近的\textit{p}、\textit{d}电子较为显著)。旋-轨耦合本质上是当原子核很重,电子在原子核附近运动速度接近光速时,相对论效应不能忽略引起的。对于轻元素,旋-轨耦合效应大约为$10^{-3}\sim10^{-2}\mathrm{eV}$,一般可以忽略。随着核电荷数的增加,旋-轨耦合的贡献将明显增加。高温合金材料中的过渡金属合金组分多且复杂,旋-轨耦合效应在$10^{-2}\sim10^{-1}\mathrm{eV}$范围内,在考虑电子结构有关的性质时,一般不应忽略。对于非功能材料,体系的热力学性质、力学性质主要取决于体系的原子-原子、分子-分子间相互作用,其大小范围在$1\sim10^3\mathrm{eV}$不等,比旋-轨耦合作用强得多,因此无须考虑旋-轨耦合效应的贡献。只有当旋-轨耦合效应大小与原子/分子间相互作用相当时,才需要考虑。只是精确估算旋-轨耦合效应时,计算成本将大大增加。所以对于高温合金材料,一个可行的方案是选择若干典型的模型,用标量相对论方法完成旋-轨耦合计算,估计体系中旋-轨耦合贡献的大小;将得到的数值与体系的原子/分子间相互作用势比较,如果两者数据接近,则旋-轨耦合效应不能忽略,否则在有关力学、热力学参数计算时,可暂时不予考虑。

\section{小结}
材料模拟与计算在第四代\ch{Ni}-基单晶高温合金的研发过程中,将起到非常重要的探索和牵引作用,如何有效地管理、分配和利用计算资源,是应用高通量自动流程方式计算需要重点考虑的问题,现有的材料计算流程软件无法支持具有如此复杂度材料的高效模拟过程。通过剖析核心计算软件\textrm{VASP}的代码、发展相关功能,我们对\textrm{VASP}软件有了比较深入的掌握,我们讨论了\textrm{VASP}的对称性功能模块、并行性能和计算特点,从高温合金材料含有过渡金属元素的\textit{d}-电子和旋-轨耦合效应对计算结果可能的影响,提出了针对性的解决方案,重点围绕单晶高温合金材料计算模拟的复杂度,结合现有软件特点和计算方法的发展方向,提出主要面向单晶高温合金体系模拟核心计算和流程软件发展的技术路线。
