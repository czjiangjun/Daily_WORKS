%---------------------- TEMPLATE FOR REPORT ------------------------------------------------------------------------------------------------------%

%\thispagestyle{fancy}   % 插入页眉页脚                                        %
%%%%%%%%%%%%%%%%%%%%%%%%%%%%% 用 authblk 包 支持作者和E-mail %%%%%%%%%%%%%%%%%%%%%%%%%%%%%%%%%
%\title{More than one Author with different Affiliations}				     %
%\title{\rm{VASP}的电荷密度存储文件\rm{CHGCAR}}
%\title{面向高温合金材料设计的计算模拟软件中的几个主要问题}
\title{计算材料团队近年来工作情况概述}
\author[ ]{}   %
%\author[ ]{姜~骏\thanks{jiangjun@bcc.ac.cn}}   %
%\affil[ ]{北京市计算中心}    %
%\author[a]{Author A}									     %
%\author[a]{Author B}									     %
%\author[a]{Author C \thanks{Corresponding author: email@mail.com}}			     %
%%\author[a]{Author/通讯作者 C \thanks{Corresponding author: cores-email@mail.com}}     	     %
%\author[b]{Author D}									     %
%\author[b]{Author/作者 D}								     %
%\author[b]{Author E}									     %
%\affil[a]{Department of Computer Science, \LaTeX\ University}				     %
%\affil[b]{Department of Mechanical Engineering, \LaTeX\ University}			     %
%\affil[b]{作者单位-2}			    						     %
											     %
%%% 使用 \thanks 定义通讯作者								     %
											     %
\renewcommand*{\Authfont}{\small\rm} % 修改作者的字体与大小				     %
\renewcommand*{\Affilfont}{\small\it} % 修改机构名称的字体与大小			     %
\renewcommand\Authands{ and } % 去掉 and 前的逗号					     %
\renewcommand\Authands{ , } % 将 and 换成逗号					     %
\date{} % 去掉日期									     %
%\date{2020-12-30}									     %

%%%%%%%%%%%%%%%%%%%%%%%%%%%%%%%%%%%%%%%%%%  不使用 authblk 包制作标题  %%%%%%%%%%%%%%%%%%%%%%%%%%%%%%%%%%%%%%%%%%%%%%
%-------------------------------The Title of The Report-----------------------------------------%
%\title{报告标题:~}   %
%-----------------------------------------------------------------------------

%----------------------The Authors and the address of The Paper--------------------------------%
%\author{
%\small
%Author1, Author2, Author3\footnote{Communication author's E-mail} \\    %Authors' Names	       %
%\small
%(The Address,City Post code)						%Address	       %
%}
%\affil[$\dagger$]{清华大学~材料加工研究所~A213}
%\affil{清华大学~材料加工研究所~A213}
%\date{}					%if necessary					       %
%----------------------------------------------------------------------------------------------%
%%%%%%%%%%%%%%%%%%%%%%%%%%%%%%%%%%%%%%%%%%%%%%%%%%%%%%%%%%%%%%%%%%%%%%%%%%%%%%%%%%%%%%%%%%%%%%%%%%%%%%%%%%%%%%%%%%%%%
\maketitle
%\thispagestyle{fancy}   % 首页插入页眉页脚 
北京市计算中心有限公司的计算材料团队主要工作以第一原理和分子动力学材料物性模拟研究、基础计算方法和稀土和半导体材料微观晶体和物理性质料数据库建设为主。团队当前共有博士四人,硕士二人(一人为在读博士)。团队近年的工作主要包括:
%\begin{itemize}
%	\item 国家重点研发计划项目``高通量并发式材料计算算法和软件''(编号:~\textrm{2017YFB0701500})和``产学研用协同的高通量材料计算融合服务平台''(编号:~\textrm{2018YFB0704300})
%	\item 北科院青年骨干计划项目``基于甲烷催化燃烧机理的材料计算自动流程设计''(编号:~\textrm{YC201820})
%	\item 
%	\item 稀土和半导体材料微观晶体和物理性质数据库建设
%	\item 二氧化碳-甲烷催化还原机理研究
%\end{itemize}

\section{异构计算材料软件:~面向多尺度计算与催化机理}
国家重点专项``高通量并发式材料计算算法和软件''主要基于材料基因组基本理念及其核心科学内涵,以发展高通量多尺度并发式集成计算算法和相应的软件系统为基础,研发国家重大需求的资源化高温合金:~多尺度科学以表述物质跨空间或时间尺度的参数传递或耦合现象为特点,化学多元复杂结构合金的物性局域性或广延性规律内禀于其多尺度关联或跨越之中。本团队在参与该专项时,主要承担的任务是协同清华大学、中科院数学所发展用于高温合金的高通量并发式计算算法及基础软件自动流程,负责高精度设计、软件及对称性分析及高通量并发式计算中$\geqslant5000$中的部分作业建模及效率分析;~与上海交通大学协同提升单线程与整体性能方面的算法优化、调度算法等方面的工作;~注释\textrm{VASP}软件中微动弹性能带\textrm{(NEB)}部分的代码。完成该项目,本单位发表论文3篇,获得软件著作权1项,专利1项。

%\section{院青年骨干项目}
院青年骨干项目``基于甲烷催化燃烧机理的材料计算自动流程设计''主要针对$\mathrm{CH}_4$燃烧机理研究:~天然气的主要成分$\mathrm{CH}_4$催化燃烧过程复杂性的制约,催化剂与$\mathrm{CH}_4$相互作用的微观机理一直并不清楚,开发适合天然气在$377\sim877^{\circ}\mathrm{C}$范围燃烧的催化剂,多年来一直是研究的重点和难点。%在没有催化剂的条件下,甲烷在空气中直接燃烧,温度高达1600$^{\circ}\mathrm{C}$左右,并生成氮氧化物($\mathrm{NO}_x$)等污染物质。使用催化剂,不仅可以降低$\mathrm{CH}_4$ 的起燃温度和燃烧峰值温度,减少污染物生成;并且燃烧利用率可以达到99.9\%,接近完全氧化,基本不会形成\textrm{CO}和碳氢化合物,因此$\mathrm{CH}_4$ 催化燃烧可以达到近零污染排放。
开发适合催化材料微观机理模拟的\textrm{DFT-MD}自动流程软件,研究动力学约化耦合算法、高通量基元反应\textrm{Kohn-Sham}方程筛选与调度平衡算法。以$\mathrm{CH}_4$催化燃烧反应研究为牵引,面向微观尺度下异相界面催化燃烧反应动力学机理模拟计算的高效率实现。通过基元反应活化能确定影响进程的决速步,筛选对反应机理干扰的大量自由基反应;优化决速步反应得到的势能面(\textrm{DFT-MD}耦合的关键)提升分子反应动力学计算的迭代稳定性为共性特征,形成含有多决速步或非关联第一原理计算流程作业框架,结合具体的燃烧反应动力学求解流程实例化,得到适用于催化反应动力学研究的流程版本,经优化的并发\textrm{DFT}求解流程可将电子步计算自动流程的计算效率提高20\%左右。研制适应多决速步化学反应的复杂反应动力学机理研究的自动流程软件,并在商业计算机和国产高性能超级计算机上成功测试运行,成为可跨平台典型催化燃烧反应动力学模拟的自动流程软件。完成该项目,本单位发表论文1篇,获得软件著作权1项。

\section{并行计算架构与资源}
团队长期从事云计算、大数据等领域的系统研发工作。%、\textrm{Apache~Spark}是一个用于大规模数据处理的开源分布式计算引擎,旨在通过内存计算提高大数据处理的效率。随着数据量和计算复杂度的增加,\textrm{Spark}的性能优化变得至关重要。
\textrm{Spark}作为一种通用数据处理框架,默认设置在各种场景下可能并不总是最优的,因此在处理特定的工作负载时,进行优化可以大幅提高系统的性能。团队对执行计划优化\textrm{(Execution Plan Optimization)}、数据分区与调度优化\textrm{(Data Partitioning and Task Scheduling Optimization)}以及内存管理与缓存优化\textrm{(Memory Management and Caching Optimization)}等方向对\textrm{Spark}应用以及\textrm{Spark}系统进行优化等方向进行研究,并在工程实践上获得了良好效果。此外,团队还参与北科智算平台的设计和维护,该平台是基于四个``材料基因工程''的成果凝练而成,构建了一套高效、安全的材料研发生态系统。团队还负责科学计算\textrm{(GPUMD,VASP,Lammps,CP2K等)}、工程仿真、图形图像等高性能计算领域的行业软件的技术咨询、安装、编译、维护管理工作。%提供技术咨询或其他技术支持,调研平台用户的使用需求。处理平台用户在使用中出现的问题。%协助同事实现网页的基本框架和交互功能,优化网页内容。
%围绕基于\textrm{FP-LAPW}高精度第一原理计算软件\textrm{WIEN2k}的总体重构:\\
%\textrm{WIEN2k}代码陈旧、数据结构零散、数据读写效率低下,大大制约了该软件在实际应用中的影响力,但\textrm{FP-LAPW}方法本身具有极高的精度,而且不依赖原子赝势。本人对\textrm{WIEN2k}代码进行全面梳理,提出针对\textrm{WIEN2k}代码重构和效率提升有总体的方案,并就核心代码(自洽迭代部分)加速提出具体的实现路线。

%围绕基于\textrm{PAW}方法的原子数据集(赝势)的构造:\\
%\textrm{VASP}的原子数据集集文件\textrm{POTCAR}性能优异,但仅有原子信息文件,而无生成方案;开源软件\textrm{QE}、\textrm{PWSCF}、\textrm{ABINIT}等的原子数据集生成方案不及\textrm{VASP}。本人通过对相关原子数据集和\textrm{PAW}方法的深入研究,探索了支持\textrm{VASP}原子数据集生成的可行方案

%此间完成专著一部(与人合著)。
\section{稀土和半导体材料微观晶体和物理性质数据库建设}
通过国家重点专项和院骨干项目的实施,团队完成支持合金与多相催化的异构化材料计算自动流程的开发,对现有国际常用材料计算流程与数据库及其代码实现作了系统剖析,掌握了包括\textrm{Material Projects}、\textrm{ASE}在内的多种计算软件自动流程与数据库实现方案,以该自动流程为基础,搭建了稀土和半导体材料微观晶体和物理性质数据库。该数据库涵盖稀土金属硼化物、过渡金属氧化物、镍基单晶高温合金模型的晶体结构和电子结构(能带、态密度)和磁学、光学性质等,总数据量约为1000余条,有效可读数据1G。数据以``北京-稀土与碱土金属化合物、半导体材料电子结构数据''形式在北京国际大数据交易有限公司完成了数据资产登记,并以``金属与半导体微观尺度晶体和物理性质数据集''形式完成了数据知识产权登记。

\section{二氧化碳-甲烷催化还原机理研究}
自2022年秋,在完成$\mathrm{CH}_4$催化燃烧机理的基础上,与北京科技大学大学、北京航空航天大学和北京低碳清洁能源研究院相关课题组合作,基于第一原理和分子动力学的理论和方法,研究$\mathrm{CO}_2$催化还原的催化机理,特别是针对金属氧化物$\mathrm{CeO2}_2$担载金属$\mathrm{Ni}$的界面,对$\mathrm{CO}_2$经$\cdot\mathrm{COOH}$还原为$\mathrm{CO}$,研究可能的催化动力学过程;$\mathrm{CO}$在金属$\mathrm{Ni}$表面与$\mathrm{H}$作用,形成$\mathrm{CH}_4$的可能动力学机理。团队的计算表明,$\mathrm{CO}_2$在过渡金属氧化物表面的吸附和缺陷作用是$\mathrm{CO}_2\rightarrow\mathrm{CO}$的重要催化过程,而过渡金属$\mathrm{Ni}$的主要作用是分解还原剂$\mathrm{H}_2$,为还原反应提供足够的活化$\mathrm{H}$原子。因此较好地说明了金属氧化物担载金属催化剂的耦合作用。相关研究结果正在整理成研究论文。

完成上述工作的同时,团队成员完成专著《计算材料科学理论与实践》的编写,并于2021年由人民邮电出版社出版。
