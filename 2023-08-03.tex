%---------------------- TEMPLATE FOR REPORT ------------------------------------------------------------------------------------------------------%

%\thispagestyle{fancy}   % 插入页眉页脚                                        %
%%%%%%%%%%%%%%%%%%%%%%%%%%%%% 用 authblk 包 支持作者和E-mail %%%%%%%%%%%%%%%%%%%%%%%%%%%%%%%%%
%\title{More than one Author with different Affiliations}				     %
%\title{\rm{VASP}的电荷密度存储文件\rm{CHGCAR}}
\title{高温合金材料计算模拟中的几个问题}
%\title{面向高温合金材料模拟的计算方法发展和创新研究}
\author[ ]{}   %
%\author[ ]{姜~骏\thanks{jiangjun@bcc.ac.cn}}   %
%\affil[ ]{北京市计算中心}    %
%\author[a]{Author A}									     %
%\author[a]{Author B}									     %
%\author[a]{Author C \thanks{Corresponding author: email@mail.com}}			     %
%%\author[a]{Author/通讯作者 C \thanks{Corresponding author: cores-email@mail.com}}     	     %
%\author[b]{Author D}									     %
%\author[b]{Author/作者 D}								     %
%\author[b]{Author E}									     %
%\affil[a]{Department of Computer Science, \LaTeX\ University}				     %
%\affil[b]{Department of Mechanical Engineering, \LaTeX\ University}			     %
%\affil[b]{作者单位-2}			    						     %
											     %
%%% 使用 \thanks 定义通讯作者								     %
											     %
\renewcommand*{\Authfont}{\small\rm} % 修改作者的字体与大小				     %
\renewcommand*{\Affilfont}{\small\it} % 修改机构名称的字体与大小			     %
\renewcommand\Authands{ and } % 去掉 and 前的逗号					     %
\renewcommand\Authands{ , } % 将 and 换成逗号					     %
\date{} % 去掉日期									     %
%\date{2020-12-30}									     %

%%%%%%%%%%%%%%%%%%%%%%%%%%%%%%%%%%%%%%%%%%  不使用 authblk 包制作标题  %%%%%%%%%%%%%%%%%%%%%%%%%%%%%%%%%%%%%%%%%%%%%%
%-------------------------------The Title of The Report-----------------------------------------%
%\title{报告标题:~}   %
%-----------------------------------------------------------------------------

%----------------------The Authors and the address of The Paper--------------------------------%
%\author{
%\small
%Author1, Author2, Author3\footnote{Communication author's E-mail} \\    %Authors' Names	       %
%\small
%(The Address,City Post code)						%Address	       %
%}
%\affil[$\dagger$]{清华大学~材料加工研究所~A213}
%\affil{清华大学~材料加工研究所~A213}
%\date{}					%if necessary					       %
%----------------------------------------------------------------------------------------------%
%%%%%%%%%%%%%%%%%%%%%%%%%%%%%%%%%%%%%%%%%%%%%%%%%%%%%%%%%%%%%%%%%%%%%%%%%%%%%%%%%%%%%%%%%%%%%%%%%%%%%%%%%%%%%%%%%%%%%
\maketitle
%\thispagestyle{fancy}   % 首页插入页眉页脚 
\section{引言}
镍基单晶高温合金是航空发动机核心部件的主要材料,组分多元且结构复杂,通常的材料计算软件很难模拟合此类金材料的力学和热力学行为。改进和提升软件对复杂合金的模拟能力,发展适合高温合金材料设计的多学科融合研究方式已成为高温合金材料研究的重要内容。
%以及针对主要计算工具维也纳原子模拟软件包~(\textrm{VASP})~在原子、电子层次计算的特点,在加速计算方面做必要的发展。此外,鉴于合金材料中有大量含有~\textit{d}-电子的金属元素,由于\textit{d}-电子的对称性复杂,大量的\textit{d}电子使得能带结构的复杂度显著增加;~在高精度计算中,合金中重元素的旋-轨耦合效应也将难以忽略,有必要从计算方法层次考虑计算这部分贡献对体系性能的影响。 在此基础上,我们才有望完整考虑单晶高温合金材料的电子能带结构及其与晶格的相互作用,确定材料的基本特性,进而研究材料在高温条件下产生的物理效应,揭示其物理机理的深刻内涵。

\section{模拟软件的计算优势与改进}
高温合金材料的设计重点围绕体系的热力学与结构性质,维也纳原子模拟软件\textrm{(VASP)}是此类材料性质计算最优异的微观模拟软件之一,为材料热力学与力学参数的计算提供了保障。
\subsection{维也纳原子模拟软件的计算特色}
维也纳原子模拟软件的计算性能突出表现在以下几方面:

(1) 软件采用的投影子缀加波\textrm{(PAW)}方法,较好地平衡了计算精度和效率。

(2) 软件通过使用大量的优化算法,(如共轭梯度算法、\textrm{Davison}算法和\textrm{RMM-DIIS}算法等),有效地保证了矩阵对角化和自洽迭代的快速收敛及稳定性问题。

(3) 软件的从头算分子动力学计算,将电子与离子运动在同一动力学框架内处理,并在时间尺度上将两者分离。不但大大简约了程序的复杂度,而且使得分子动力学与密度泛函理论紧密结合,无需诉诸跨尺度的势函数(力场)模拟。

%对于简单材料模拟,维也纳原子模拟软件的计算能力足以支持,但是对单晶高温合金这样的复杂体系模拟,还需要从计算方法层面予以发展,提升软件对单晶高温合金的计算效率。单晶高温合金模拟的高精度计算主要是电子结构和从头计算分子动力学的自洽迭代,除了其已支持的共轭梯度等经典的优化算法,有必要引入包括随机梯度下降优化算法和机器学习算法(如神经网络)在内的多种优化算法,实现复杂模型的自洽迭代快速稳定收敛。
%由数据结构出发,构建支撑数据到计算资源的均衡负载的软件框架,将有望系统地提升材料计算核心软件的并行能力和扩展性,从而提升软件处理更大规模计算模型的能力。针对\ch{Ni}-基单晶高温合金材料的模拟,考虑到模型的复杂度(过渡金属原子多、原子成分变化大),软件框架的开发将重点解决软件的并行扩展性问题,通过建立以密度矩阵理论为基础的线性标度算法及软件框架,原理性改变哈密顿矩阵结构,优化作业并行效率n。

\subsection{利用对称性提升软件计算速度}
对称性伴随系统变换过程中物理量的守恒,利用对称性,能在有限条件下获得更多物理信息。根据晶体学理论,材料模型中平移对称性与点群对称性相互约束,可将周期体系划分为32个点群和230个空间群。对于高温合金材料而言,因为组成元素及其组分比较复杂,要合理反映微量元素的影响,必须将合金模型设计得足够大。因此合金材料的计算规模远远超过普通材料的模拟,这是高温合金模拟的主要困难之一。如何利用晶体对称性,有效地降低计算规模、提升模拟计算的能力,建立微观对称划分与宏观材料性质的关系,是研究中的重要问题。

在此前工作中,我们通过剖析维也纳原子模拟软件的核心代码,发现软件的对称性分析功能只包含点群模块,缺少晶体空间群部分,我们从晶体学理论出发,按照点群-空间群生成对应法则,编写了全部空间群生成代码,完善了软件的晶体原子和空间对称性分析功能。空间群功能的引入,将有助于确定多组元复杂模型微观空间结构与宏观力学特性的关系,简化高温合金材料的力学性质模拟。

\subsection{\rm{VASP}的计算能力提升策略}
面向镍-基单晶高温合金这样的复杂体系模拟,必须从计算方法层面发展和提升软件对单晶高温合金的计算效率。除去软件已有的第一原理分子动力学自洽迭代模式和现有的共轭梯度、\textrm{RMM-DIIS}等经典的优化算法,有必要引入机器学习算法(如神经网络)在内的多种优化算法,实现复杂模型的自洽迭代快速稳定收敛。

维也纳原子模拟软件在并行扩展性方面有优异的表现,在上百核的通用计算机上能保持良好的的并行线性度。该软件通过轮询调度算法\textrm{(Round-Robin Scheduling)},实现计算资源分配的均衡。现有软件版本的并行上限约为256核,在超过500核的环境下,并行效率下降比较明显。通过对软件的运行监测我们发现,计算过程中使用了大量快速傅立叶变换(\textrm{Fast Fourier Transform}),内存开销很大,如能从算法层面降低傅立叶变换的存储和通信,必将有助于提升软件的并行效能,提高软件处理合金模型的能力。

\section{\textit{d}-电子和旋-轨耦合}
高温合金材料中除了铝元素外,大部分是重过渡元素。高温合金的电荷密度分布分析表明,金属原子间有明显的\textit{d}-电子的贡献。合金中的不同过渡元素原子的\textit{d}-电子含量不同,因此必须对体系进行高精度的第一原理计算,解析\textit{d}-电子成键的组分贡献,为高温合金材料的性能提升提供支持。此外,过渡金属的价电子自旋与轨道角动量的相互作用,对体系的物理性质有较显著的影响。高温合金材料中的旋-轨耦合效应在$10^{-2}\sim10^{-1}\mathrm{eV}$范围内,在电子结构计算中,一般不应忽略,这也将使得高温合金材料的性质模拟变得复杂。

\section{小结}
材料模拟与计算在第四代单晶高温合金的研发过程中,将起到重要的探索和牵引作用,如何有效地管理、分配和利用计算资源,是应用高通量自动流程方式计算需要重点考虑的问题。通过对核心计算软件的代码剖析、发展相关功能,我们对维也纳原子模拟软件有了较深入的掌握,重点围绕单晶高温合金材料计算模拟的复杂度,结合现有软件特点和计算方法的发展方向,提出主要面向单晶高温合金体系模拟计算物理和计算方法相融合发展的技术路线。
