%---------------------- TEMPLATE FOR REPORT ------------------------------------------------------------------------------------------------------%

%\thispagestyle{fancy}   % 插入页眉页脚                                        %
%%%%%%%%%%%%%%%%%%%%%%%%%%%%% 用 authblk 包 支持作者和E-mail %%%%%%%%%%%%%%%%%%%%%%%%%%%%%%%%%
%\title{More than one Author with different Affiliations}				     %
%\title{\rm{VASP}的电荷密度存储文件\rm{CHGCAR}}
%\title{面向高温合金材料设计的计算模拟软件中的几个主要问题}
\title{氢化锆中氢原子扩散动力学研究}
\author[ ]{北京市计算中心有限公司}   %{北京北科融智云计算科技有限公司}%
%\author[ ]{姜~骏\thanks{jiangjun@bcc.ac.cn}}   %
%\affil[ ]{北京市计算中心}    %
%\author[a]{Author A}									     %
%\author[a]{Author B}									     %
%\author[a]{Author C \thanks{Corresponding author: email@mail.com}}			     %
%%\author[a]{Author/通讯作者 C \thanks{Corresponding author: cores-email@mail.com}}     	     %
%\author[b]{Author D}									     %
%\author[b]{Author/作者 D}								     %
%\author[b]{Author E}									     %
%\affil[a]{Department of Computer Science, \LaTeX\ University}				     %
%\affil[b]{Department of Mechanical Engineering, \LaTeX\ University}			     %
%\affil[b]{作者单位-2}			    						     %
											     %
%%% 使用 \thanks 定义通讯作者								     %
											     %
\renewcommand*{\Authfont}{\small\rm} % 修改作者的字体与大小				     %
\renewcommand*{\Affilfont}{\small\it} % 修改机构名称的字体与大小			     %
\renewcommand\Authands{ and } % 去掉 and 前的逗号					     %
\renewcommand\Authands{ , } % 将 and 换成逗号					     %
\date{} % 去掉日期									     %
%\date{2020-12-30}									     %

%%%%%%%%%%%%%%%%%%%%%%%%%%%%%%%%%%%%%%%%%%  不使用 authblk 包制作标题  %%%%%%%%%%%%%%%%%%%%%%%%%%%%%%%%%%%%%%%%%%%%%%
%-------------------------------The Title of The Report-----------------------------------------%
%\title{报告标题:~}   %
%-----------------------------------------------------------------------------

%----------------------The Authors and the address of The Paper--------------------------------%
%\author{
%\small
%Author1, Author2, Author3\footnote{Communication author's E-mail} \\    %Authors' Names	       %
%\small
%(The Address,City Post code)						%Address	       %
%}
%\affil[$\dagger$]{清华大学~材料加工研究所~A213}
%\affil{清华大学~材料加工研究所~A213}
%\date{}					%if necessary					       %
%----------------------------------------------------------------------------------------------%
%%%%%%%%%%%%%%%%%%%%%%%%%%%%%%%%%%%%%%%%%%%%%%%%%%%%%%%%%%%%%%%%%%%%%%%%%%%%%%%%%%%%%%%%%%%%%%%%%%%%%%%%%%%%%%%%%%%%%
\maketitle
%\thispagestyle{fancy}   % 首页插入页眉页脚 
在核反应堆中,中子慢化过程至关重要,对核反应效率和核材料的利用效率都有着深远的影响。热中子反应堆中,裂变主要由低能量的热中子\textrm{(<0.1~eV,2200~m/s)}引发。但是核裂变产生的中子初始能量较高,平均能量约为\textrm{2~MeV},这样高速的快中子要有效地参与链式裂变反应,就必须减速成热中子,这是慢化剂的核心任务。如果没有慢化剂的作用,快中子就很容易泄漏出堆芯,无法维持稳定的链式反应,核反应堆也难以高效运行。例如,在常见的压水堆和沸水堆中,轻水作为慢化剂,通过与中子的多次弹性散射,不断降低中子的能量,使其达到热中子能区,从而大大提高裂变反应的几率。

长期以来,锆合金因为具有较低的热中子吸收截面、优良的耐腐蚀性能和优异的高温力学性能,被广泛用作压水堆核燃料包壳管和压力管。金属锆对氢具有极强的亲和力,但室温下氢在锆中的固溶度却非常低\textrm{($<1\mathrm{ppm}$)},导致锆合金在生产、加工及长期服役过程中不可避免地吸收环境中的氢,并析出脆性的氢化物。大量氢化物的析出,将使得锆合金包壳管韧脆转变温度大幅升高,形成多种多样的局部脆性,从而对核反应堆的安全运行造成威胁。所以氢化锆主要作为锆合金包壳材料中氢化物的形核、析出、长大和致脆机理的研究对象,长期存在于核反应堆的安全运行相关的讨论中。

近年来,关于氢化锆作为中子慢化材料越来越受到重视,关于氢化锆的研究方向也得到了拓展。氢化锆作为反应堆慢化剂具有独特的优势:~氢化锆中的氢原子含量丰富。氢的质量数与中子相近,根据中子散射理论,当中子与氢核发生弹性散射时,能量损失效率高,使得氢化锆能够在相对较小的体积内,有效地将快中子慢化为热中子。%其次,氢化锆的负温度系数特性。当反应堆温度升高时,氢化锆的慢化性能会有所减弱,导致中子能谱变硬,参与裂变的中子数量减少,进而抑制核反应的强度,使反应堆功率下降。这种负反馈机制就像一个自动的 “调节器”,能够在反应堆温度异常升高时,自发地对反应进行调节,增强了反应堆的固有安全性,降低了因温度失控而引发事故的风险。
此外,氢化锆具备出色的高温稳定性,能够在较高温度$600\sim650^{\circ}\mathrm{C}$下稳定工作,且无需像液态慢化剂那样依赖高压容器来维持状态,这使得采用氢化锆作为慢化剂的反应堆在设计上更加灵活,体积和质量得以显著减小,尤其适用于一些对空间和重量有严格限制的应用场景,如空间核反应堆、移动型核反应堆等。

%随着吸氢量的增加,锆合金中会先后形成$\zeta$氢化物\textrm{(\ch{Zr2H},\textrm{HCP}-密排六方结构)}、$\gamma$氢化物\textrm{(\ch{ZrH},\textrm{FCT}-面心四方结构)}、$\delta$氢化物\textrm{(\ch{ZrH}$_{1.66}$,\textrm{FCC}-面心立方结构)}和$\epsilon$氢化物\textrm{(\ch{ZrH2},\textrm{FCT}-面心四方结构)}。锆合金中常见的氢化物为$\gamma$氢化物和$\delta$氢化物,前者是亚稳态氢化物,后者为比较稳定的氢化物。在没有外加应力的情况下,氢化物通常以锆基体的基面为惯习面,沿基面上的$<\mathrm{a}>$方向生长。在外加拉应力情况下,氢化锆会发生再取向,其惯习面会随着拉应力的增加逐步从基面转向\textrm{\{10-1$i$\} ($i$=1-7)}锥面,直至最后以柱面\{10-10\}为惯习面。再取向过程中,氢化物的生长方向始终沿着$<\!\mathrm{a}\!>$方向。在核燃料包壳管中,初始的氢化物都沿管子的周向分布,这与挤压管的初始织构密切相关,即锆包壳管具有基面沿管子周向分布的特征。在核反应堆服役过程中,核燃料在中子辐照下发生体积膨胀,使得包壳管被撑大,此时包壳管沿周向受到一定的拉应力,这种应力称之为环向应力\textrm{(hoop stress)}。在环向应力的作用下,当核反应堆冷却时,锆合金包壳管中的氢化物析出就会发生再取向,新的惯习面会沿着锥面或柱面。当环向拉应力超过\textrm{100~MPa}时,氢化物再取向主要会以柱面为惯习面。此时,若对比氢化物的初始分布和再取向分布,可以发现氢化物相当于转动了$90^{\circ}$,形成了大量的径向氢化物,沿锆合金包壳管厚度方向分布。这种再取向的氢化物会更容易造成锆合金包壳的失效。

%1.2 氢化锆慢化机理的特殊性
关于锆合金吸氢形成氢化物的研究表明,氢化锆具有丰富的固态晶体结构,会先后形成$\zeta$氢化物\textrm{(\ch{Zr2H},\textrm{HCP}-密排六方结构)}、$\gamma$氢化物\textrm{(\ch{ZrH},\textrm{FCT}-面心四方结构)}、$\delta$氢化物\textrm{(\ch{ZrH}$_{1.66}$,\textrm{FCC}-面心立方结构)}和$\epsilon$氢化物\textrm{(\ch{ZrH2},\textrm{FCT}-面心四方结构)}。锆合金中常见的氢化物为$\gamma$氢化物和$\delta$氢化物,前者是亚稳态氢化物,后者为比较稳定的氢化物。%这决定了其慢化机理的特殊性。
氢化锆晶体中,氢原子处于特定的晶格位置,与锆原子形成稳定的化学键,形成稳定的束缚态。

当中子与氢化锆晶体相互作用时,%会发生一种特殊的量子化能量交换过程,即中子声子散射。声子是晶体中原子集体振动的能量量子,当中子与氢化锆晶体碰撞时,
有可能激发晶体的振动态,产生或吸收声子%,中子可通过这种碰撞得到或失去声子,
从而引起能量转移,%并且这种能量交换是量子化的,中子可获得或失去 0.137eV 整数倍的能量。
当入射中子能量较高,%高于 0.137eV 时,每次散射都有较大概率使中子损失 0.137eV 的能量,
激发高频的声学模,能够迅速将中子慢化,此时氢化锆是很好的慢化剂;而当%能量低于 0.137eV 时,
中子能量较低时,只能通过效率较低的激发声学模失去能量,慢化效果相对减弱。
%温度变化对氢化锆慢化性能的影响机制较为复杂,在学术界存在正 / 负温度系数的争议。一般观点认为,当氢化锆温度升高后,处于激发态的晶体数目增多,热中子从氢化锆晶体的激发态获得量子化能量的几率增大,使得中子能谱变硬,慢化性能减弱,这就使得采用氢化锆作为慢化剂的反应堆(如 TRIGA 堆)产生了较大的负温度系数。然而,像 TOPAZ - Ⅱ 反应堆却呈现出正慢化剂温度效应,且其值超过了燃料多普勒效应的绝对值,使整个反应堆呈现出正温度效应。研究表明,这可能与堆芯的具体结构、材料分布以及中子在堆芯内的输运过程等多种因素有关。温度变化不仅影响晶体的振动状态和声子的激发概率,还可能改变材料的微观结构和化学成分,进而对慢化性能产生复杂的影响。深入研究温度变化对氢化锆慢化性能的影响机制,对于准确理解反应堆的运行特性、优化反应堆设计以及确保反应堆的安全稳定运行都具有重要意义。
%大量氢化物析出会造成锆合金包壳管的氢脆。氢化物引起的氢脆主要包括两种类型:~一种为氢化物均匀致脆,另一种为氢化物引起的局部脆性。若析出的氢化物均匀分布在锆合金中,大量的氢化物相当于高密度第二相,最终会使锆合金失去变形能力,其韧脆转变温度可以上升到约$200^{\circ}C$这就是氢化物的均匀致脆。当锆合金表面形成微小裂纹时,裂纹前沿受张应力会吸引更多的固溶氢富集。当氢含量超出固溶度时,脆性氢化物会沿裂纹析出,由于氢化物非常脆\textrm{(断裂韧性只有$1-2~\mathrm{MPa·m^{1/2}}$)},在拉应力作用下迅速破裂,促进微小裂纹向前扩展。若这一过程重复循环多次,氢化物在裂纹前沿不断析出就导致一个微小裂纹转化成一个大裂纹,从而引起锆合金包壳管的破裂。以上描述的就是典型的氢致滞后开裂过程\textrm{(Delayed hydride crakcing)}。

%除了以上的氢化物均匀致脆和氢致滞后开裂机制外,西安交通大学韩卫忠教授团队结合宏微观表征,细致研究了锆及一种\textrm{Zr-Sn-Nb-Fe}合金在高真空退火后出现的反常解理开裂现象,发现了一种全新的氢化物析出致解理开裂新机制。值得注意的是这种开裂发生在氢含量很低的锆合金的常规热处理过程中,会造成锆合金加工开裂和服役安全等问题。研究发现退火的缓慢冷却过程促进了不常见的柱面δ氢化物的形成,该柱面δ氢化物可直接作为裂纹源导致锆合金沿柱面发生解理开裂。由于密排六方结构的锆沿着$<\mathrm{c}>$轴和$<\mathrm{a}>$轴的热膨胀系数差异约2倍,导致冷却过程材料内部产生约\textrm{100~MPa}的晶间热应力,在锆合金缓慢冷却过程中,$\delta$氢化物析出时发生再取向,最终沿着\{10-10\}柱面析出。柱面是密排六方晶体锆的常见解理面。当氢化物沿着柱面生长时,其产生了较大的体积膨胀,就像在柱面打入楔子,最终促进柱面张开发生解理开裂。当锆合金快速冷却时形成的$\gamma$氢化物会随机分布,整体体积膨胀和畸变都比较小,不会引起开裂。研究还进一步探索了抑制锆合金热处理开裂的方法,比如可以通过调控锆合金的冷却速率和织构形态来抑制锆合金热加工开裂,从而提高锆材生产成品率并降低生产成本。相关文章以题为\textrm{``Annealing cracking in Zr and a Zr-alloy with low hydrogen concentration''}发表在期刊\textrm{Journal of Materials Science \& Technology}上。

\section{研究目的} 
氢化锆在其工作温度范围内$(400\sim700^{\circ}\mathrm{C})$,反应平衡向氢析出的方向移动,造成氢损失,从而降低氢化锆的中子慢化效率。通过分子动力学方法研究(1)氢化锆固体中氢原子在高温密闭真空环境下的扩散行为,(2)在氢化锆表面形成阻氢氧化锆膜层后,氢原子在体相氢化锆-氧化锆膜层界面的扩散和氧化锆膜层的反应动力学行为,(3)模拟氢化锆晶格的声子振动频率。%构建氢扩散系数低、致密性好的氧化膜层,是解决氢化锆高温失氢问题的有效途径。
在此基础上,分析氢扩散引起氢气、水等小分子的形成和扩散机制,探索氢化锆材料在高温密闭真空环境中失氢的反应动力学机理,为优化氢化锆中子慢化材料的稳定性给出理论阐述。

\section{研究内容与方法}
%\textcolor{red}{\hl{2025-03-04}增补:}~\\
%与传统的包壳材料不同,氢化锆(\ch{ZrH}$_{1.85-1.88}$)是作为慢化中子、提高反应堆中\ce{^{235}U}的利用率而存在于反应堆中。固体在高温(如$T=600^{\circ}\mathrm{C}$)密闭环境(真实体系为\ch{CO2}和\ch{He}气氛)中,如果体系温度均匀,则认为氢原子在体系中均匀分布,氢原子在材料中扩散,到氢化锆表面形成\ch{H2},如果存在温度梯度,则氢原子将向低温区域聚集,并可能产生氢气(\ch{H2})
\subsection{氢化锆材料的原子间相互作用}
精确的原子间相互作用函数(力场)是应用分子动力学方法研究微观尺度下原子扩散和反应动力学的重要基础。反应动力学计算涉及化学键的形成和重组,伴随电子的得失与转移,而传统的分子动力学力场一般无法给出电子层次的精确描述,基于\textrm{DFT}计算的机器学习势大幅度提高了分子动力学力场的描述精度,有望用于反应动力学模拟。对于氢化锆体系(包括氢化锆本体和氧化锆涂层表面)中的氢扩散,在体相氢化锆需要考虑锆-锆相互作用、锆-氢相互作用;~在氧化锆涂层表面需要考虑锆-锆相互作用、锆-氧相互作用;~特别地,在体相和涂层表面的界面,还需要考虑锆-锆、锆-氢、锆-氧的复杂相互作用。在此基础上,考虑氢在体相中的扩散,伴随空位形成、氢填隙等缺陷结构的影响和温度的影响,因此基于\textrm{DFT}的分子动力学势函数构建主要包括:
\begin{itemize}
	\item 势函数的分类
\begin{itemize}
	\item \ch{Zr-Zr}的势函数:~工况温度下锆原子之间的相互作用,是氢原子扩散的环境影响,也是描述锆化氢体相和氧化锆涂层表面相互作用的重要函数
	\item \ch{Zr-H}的势函数:~工况温度下氢原子与锆原子的相互作用,是描述氢化锆中扩散问题最重要的势函数之一
	\item \ch{Zr-O}的势函数:~工况温度下氧化锆涂层表面致密性的主要函数
	\item \ch{H-O}的势函数:~氢原子扩散进入表面涂层的相互作用函数,也是实验中检出水分子的重要指标
	\item \ch{H-H}的势函数:~氢原子在体相和表面扩散并形成气体逸出的主要描述函数,也是实验中检出氢气分子的重要指标
	\item \ch{O-O}的势函数:~氧原子在体相扩散和形成水分子、氧气分子的主要指标
\end{itemize}
	\item 建模(如表面模型、真空层等):~包括纯锆、氢化锆、氧化锆等\\
各类模型初始结构的来源
		\begin{itemize}
			\item \url{https://oqmd.org/materials/entry/10100}
			\item \url{https://next-gen.materialsproject.org/materials?formula=Y}
			\item \url{http://crystalium.materialsvirtuallab.org/}
		\end{itemize}
	\item \textrm{NEP}势函数构造\\
		每一类势函数的训练与构造,基于基态结构,至少采用150个百原子($10^2$量级)的模型,通过\textrm{DFT}计算获得基态能量,总的原子模型约$10^3$量级,每次并发模型计算任务为5-20,平均每个模型计算时间为1天,以当前的计算资源估计,预计至少两月完成相关模型的计算任务。考虑温度的影响,将训练获得的势函数,预测$300\sim1200^{\circ}\mathrm{C}$范围内的原子行为,并用\textrm{DFT}计算结果检验和标定,由此得到各类不同环境的\textrm{NEP}势函数
%本研究的主要难点是获得相关原子的势函数,作为分子动力学模拟的基础,包括
\end{itemize}
\subsection{氢化锆材料中的原子扩散}
研究氢化锆中的原子扩散,重点考虑氢原子的扩散行为,特别是扩散行为受温度变化的影响
\begin{enumerate}
	\item 氢化锆(\ch{ZrH}$_{1.85-1.88}$)中氢原子由体相向外部空间中扩散的物理过程、扩散量及相应的扩散系数:\\
		通过氢化锆中氢原子经基本氢原子位置、填隙(四面体配位和八面体配位)空位、缺陷空位向体系表面的扩散行为描述,考虑\textrm{NVT}系综,得到氢原子在氢化锆体相的扩散系数;考虑氢原子结合形成氢气后,氢气分子在氢化锆中的动力学行为,计算氢分子在氢化锆体相的扩散系数。
	\item 氢化锆(\ch{ZrH}$_{1.85-1.88}$)固体表面存在\ch{ZrO2}涂层的条件下,氢原子脱离氢化锆向外部空间中扩散的物理过程、扩散量及相应的扩散系数:\\
		通过氢原子穿越氢化锆-氧化锆界面、进入氧化锆涂层表面的动力学过程,以及在氧化锆涂层内与氧原子的扩散行为;考虑氢原子与氧原子的结合,在高温下形成水分子后的扩散,并计算扩散系数
	\item 给定密闭空间体积,氢原子扩散后,整个扩散过程达到平衡时氢气的浓度\\
		通过模拟氢化锆表面氢气吸附并形成氢原子并进入氢化锆体相,确定高温下,氢化锆与表面氢气表面平衡时的氢气分压估计,得到平衡太氢气的浓度
	\item 氢化锆(\ch{ZrH}$_x$)固体中,不同浓度比例的氢(如$x=1.70$、$x=1.80$等)所对应的起始扩散温度:\\
		根据\textrm{NVT}系综,结合氢气分压,确定不同氢比例下,氢分子扩散的起始温度,并用实验结果检验
	\item 上述参数与表面\ch{ZrO2}涂层的厚度的关系\\
		改变模型中表面涂层的厚度(\ch{ZrO2}层数),研究表面图层\ch{ZrO2}的影响
\end{enumerate}

\subsection{氢化锆-氧化锆中可能存在的化学反应}
%\textcolor{red}{\hl{2025-03-04}}增补:~\hl{x-y-z}
工程和实验上,研究氢原子和氢气在氢化锆体相、氢化锆-氧化锆界面的扩散,每隔一定的时间,测定密闭气相的组分分压、分离气体测定色谱,确定其中\ch{H2}的含量。根据实现中在位分离的气体推测,可能的化学反应主要包括
{\centering
\ce{ZrH\textit{x} = Zr + $\frac{x}2$H2 ^}\\
\ce{H2 (\textit{g}) + CO2 (\textit{g}) = CO (\textit{g}) + H2O} \\
\ce{Zr + 2 CO2 (\textit{g}) = ZrO + 2 CO (\textit{g})} \\
\ce{Zr + 2 CO2 (\textit{g}) = ZrC + ZrO2}\\
}
应用理论模拟方法确定发生的化学反应,主要从反应热力学和动力学角度考虑:
\begin{enumerate}
	\item 化学热力学:\\
		主要借助\textrm{DFT}计算,确定在工作温度范围内$(400\sim700^{\circ}\mathrm{C})$各反应的反应自由能,初步确定可自发进行的化学反应。注意在上述反应中,考虑了固体中存在碳,使得反应更为复杂
	\item 化学动力学:\\
		根据热力学条件确定可能的动力学反应后,选用高精度的动力学力场,模拟若干原子(一般不超过10个原子,包括氢化锆、氢气、氧化锆、二氧化碳等)彼此接近和远离的可能组合,获得上述反应可能的动力学过程,确定各反应物-产物形成过程中形成的中间态结构,确定反应中的活化能和决速步,并通过第一原理分子动力学的\textrm{CI-NEB}方法检验反应动力学的可能性。
\end{enumerate}
化学反应涉及化学键的形成与断裂,存在电子转移过程,特别是有关过渡态的模拟与推测,对计算精度的要求高,因此对原子间势函数和\textrm{DFT}能量计算的要求都比较高,即使应用机器学习势函数,整体的计算量仍将会相当可观。

\subsection{氢化锆的晶格振动声子}
应用机器学习势,可以比较精确地得到工作温度下,氢化锆的声子振动模式。因为氢原子在氢化锆体相扩散,将引起声子振动模式的变化,声子模式随温度和原子扩散的变化,可以确定工况温度下最优氢含量的范围。前述计算已经确定,当阻氢涂层氧化锆的存在,氢的扩散将会受到影响。而优化氢含量可以作为理论估算的阻氢表面涂层厚度的依据。将由声子振动优化得到的涂层厚度与前述计算得到的涂层厚度对比,检验两类计算的合理性。

