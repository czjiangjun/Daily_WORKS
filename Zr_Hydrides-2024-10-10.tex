%---------------------- TEMPLATE FOR REPORT ------------------------------------------------------------------------------------------------------%

%\thispagestyle{fancy}   % 插入页眉页脚                                        %
%%%%%%%%%%%%%%%%%%%%%%%%%%%%% 用 authblk 包 支持作者和E-mail %%%%%%%%%%%%%%%%%%%%%%%%%%%%%%%%%
%\title{More than one Author with different Affiliations}				     %
%\title{\rm{VASP}的电荷密度存储文件\rm{CHGCAR}}
%\title{面向高温合金材料设计的计算模拟软件中的几个主要问题}
\title{氢化锆中氢原子扩散过程研究与包壳材料的辐照动力学研究}
\author[ ]{北京市计算中心有限公司}   %{北京北科融智云计算科技有限公司}%
%\author[ ]{姜~骏\thanks{jiangjun@bcc.ac.cn}}   %
%\affil[ ]{北京市计算中心}    %
%\author[a]{Author A}									     %
%\author[a]{Author B}									     %
%\author[a]{Author C \thanks{Corresponding author: email@mail.com}}			     %
%%\author[a]{Author/通讯作者 C \thanks{Corresponding author: cores-email@mail.com}}     	     %
%\author[b]{Author D}									     %
%\author[b]{Author/作者 D}								     %
%\author[b]{Author E}									     %
%\affil[a]{Department of Computer Science, \LaTeX\ University}				     %
%\affil[b]{Department of Mechanical Engineering, \LaTeX\ University}			     %
%\affil[b]{作者单位-2}			    						     %
											     %
%%% 使用 \thanks 定义通讯作者								     %
											     %
\renewcommand*{\Authfont}{\small\rm} % 修改作者的字体与大小				     %
\renewcommand*{\Affilfont}{\small\it} % 修改机构名称的字体与大小			     %
\renewcommand\Authands{ and } % 去掉 and 前的逗号					     %
\renewcommand\Authands{ , } % 将 and 换成逗号					     %
\date{} % 去掉日期									     %
%\date{2020-12-30}									     %

%%%%%%%%%%%%%%%%%%%%%%%%%%%%%%%%%%%%%%%%%%  不使用 authblk 包制作标题  %%%%%%%%%%%%%%%%%%%%%%%%%%%%%%%%%%%%%%%%%%%%%%
%-------------------------------The Title of The Report-----------------------------------------%
%\title{报告标题:~}   %
%-----------------------------------------------------------------------------

%----------------------The Authors and the address of The Paper--------------------------------%
%\author{
%\small
%Author1, Author2, Author3\footnote{Communication author's E-mail} \\    %Authors' Names	       %
%\small
%(The Address,City Post code)						%Address	       %
%}
%\affil[$\dagger$]{清华大学~材料加工研究所~A213}
%\affil{清华大学~材料加工研究所~A213}
%\date{}					%if necessary					       %
%----------------------------------------------------------------------------------------------%
%%%%%%%%%%%%%%%%%%%%%%%%%%%%%%%%%%%%%%%%%%%%%%%%%%%%%%%%%%%%%%%%%%%%%%%%%%%%%%%%%%%%%%%%%%%%%%%%%%%%%%%%%%%%%%%%%%%%%
\maketitle
%\thispagestyle{fancy}   % 首页插入页眉页脚 
在核反应堆中,中子慢化过程至关重要,对核反应效率和核材料的利用效率都有着深远的影响。热中子反应堆中,裂变主要由低能量的热中子\textrm{(<0.1~eV,2200~m/s)}引发。但是核裂变产生的中子初始能量较高,平均能量约为\textrm{2~MeV},这样高速的快中子要有效地参与链式裂变反应,就必须减速成热中子,这便是慢化剂的核心任务。如果没有慢化剂的作用,快中子就很容易泄漏出堆芯,无法维持稳定的链式反应,核反应堆也难以高效运行。例如,在常见的压水堆和沸水堆中,轻水作为慢化剂,通过与中子的多次弹性散射,不断降低中子的能量,使其达到热中子能区,从而大大提高裂变反应的几率。

长期以来,锆合金因为具有较低的热中子吸收截面、优良的耐腐蚀性能和优异的高温力学性能,被广泛用作压水堆核燃料包壳管和压力管。金属锆对氢具有极强的亲和力,但室温下氢在锆中的固溶度却非常低\textrm{($<1\mathrm{ppm}$)},导致锆合金在生产、加工及长期服役过程中不可避免地吸收环境中的氢,并析出脆性的氢化物。大量氢化物的析出,将使得锆合金包壳管韧脆转变温度大幅升高,形成多种多样的局部脆性,从而对核反应堆的安全运行造成威胁。所以氢化锆主要作为锆合金包壳材料中氢化物的形核、析出、长大和致脆机理的研究对象,长期存在于核反应堆的安全运行相关的讨论中。

近年来,关于氢化锆作为中子慢化材料越来越受到重视,关于氢化锆的研究方向也得到了拓展。氢化锆作为反应堆慢化剂具有独特的优势:~氢化锆中的氢原子含量丰富。氢的质量数与中子相近,根据中子散射理论,当中子与氢核发生弹性散射时,能量损失效率高,使得氢化锆能够在相对较小的体积内,有效地将快中子慢化为热中子。%其次,氢化锆的负温度系数特性。当反应堆温度升高时,氢化锆的慢化性能会有所减弱,导致中子能谱变硬,参与裂变的中子数量减少,进而抑制核反应的强度,使反应堆功率下降。这种负反馈机制就像一个自动的 “调节器”,能够在反应堆温度异常升高时,自发地对反应进行调节,增强了反应堆的固有安全性,降低了因温度失控而引发事故的风险。
此外,氢化锆具备出色的高温稳定性,能够在较高温度$600\sim650^{\circ}\mathrm{C}$下稳定工作,且无需像液态慢化剂那样依赖高压容器来维持状态,这使得采用氢化锆作为慢化剂的反应堆在设计上更加灵活,体积和质量得以显著减小,尤其适用于一些对空间和重量有严格限制的应用场景,如空间核反应堆、移动型核反应堆等。

%随着吸氢量的增加,锆合金中会先后形成$\zeta$氢化物\textrm{(\ch{Zr2H},\textrm{HCP}-密排六方结构)}、$\gamma$氢化物\textrm{(\ch{ZrH},\textrm{FCT}-面心四方结构)}、$\delta$氢化物\textrm{(\ch{ZrH}$_{1.66}$,\textrm{FCC}-面心立方结构)}和$\epsilon$氢化物\textrm{(\ch{ZrH2},\textrm{FCT}-面心四方结构)}。锆合金中常见的氢化物为$\gamma$氢化物和$\delta$氢化物,前者是亚稳态氢化物,后者为比较稳定的氢化物。在没有外加应力的情况下,氢化物通常以锆基体的基面为惯习面,沿基面上的$<\mathrm{a}>$方向生长。在外加拉应力情况下,氢化锆会发生再取向,其惯习面会随着拉应力的增加逐步从基面转向\textrm{\{10-1$i$\} ($i$=1-7)}锥面,直至最后以柱面\{10-10\}为惯习面。再取向过程中,氢化物的生长方向始终沿着$<\!\mathrm{a}\!>$方向。在核燃料包壳管中,初始的氢化物都沿管子的周向分布,这与挤压管的初始织构密切相关,即锆包壳管具有基面沿管子周向分布的特征。在核反应堆服役过程中,核燃料在中子辐照下发生体积膨胀,使得包壳管被撑大,此时包壳管沿周向受到一定的拉应力,这种应力称之为环向应力\textrm{(hoop stress)}。在环向应力的作用下,当核反应堆冷却时,锆合金包壳管中的氢化物析出就会发生再取向,新的惯习面会沿着锥面或柱面。当环向拉应力超过\textrm{100~MPa}时,氢化物再取向主要会以柱面为惯习面。此时,若对比氢化物的初始分布和再取向分布,可以发现氢化物相当于转动了$90^{\circ}$,形成了大量的径向氢化物,沿锆合金包壳管厚度方向分布。这种再取向的氢化物会更容易造成锆合金包壳的失效。

%1.2 氢化锆慢化机理的特殊性
关于锆合金吸氢形成氢化物的研究表明,氢化锆具有丰富的固态晶体结构,会先后形成$\zeta$氢化物\textrm{(\ch{Zr2H},\textrm{HCP}-密排六方结构)}、$\gamma$氢化物\textrm{(\ch{ZrH},\textrm{FCT}-面心四方结构)}、$\delta$氢化物\textrm{(\ch{ZrH}$_{1.66}$,\textrm{FCC}-面心立方结构)}和$\epsilon$氢化物\textrm{(\ch{ZrH2},\textrm{FCT}-面心四方结构)}。锆合金中常见的氢化物为$\gamma$氢化物和$\delta$氢化物,前者是亚稳态氢化物,后者为比较稳定的氢化物。%这决定了其慢化机理的特殊性。
氢化锆晶体中,氢原子处于特定的晶格位置,与锆原子形成稳定的化学键,形成稳定的束缚态。

当中子与氢化锆晶体相互作用时,%会发生一种特殊的量子化能量交换过程,即中子声子散射。声子是晶体中原子集体振动的能量量子,当中子与氢化锆晶体碰撞时,
有可能激发晶体的振动态,产生或吸收声子%,中子可通过这种碰撞得到或失去声子,
从而引起能量转移,%并且这种能量交换是量子化的,中子可获得或失去 0.137eV 整数倍的能量。
当入射中子能量较高,%高于 0.137eV 时,每次散射都有较大概率使中子损失 0.137eV 的能量,
激发高频的声学模,能够迅速将中子慢化,此时氢化锆是很好的慢化剂;而当%能量低于 0.137eV 时,
中子中子能量较低时,只能通过效率较低的激发声学模失去能量,慢化效果相对减弱。
%温度变化对氢化锆慢化性能的影响机制较为复杂,在学术界存在正 / 负温度系数的争议。一般观点认为,当氢化锆温度升高后,处于激发态的晶体数目增多,热中子从氢化锆晶体的激发态获得量子化能量的几率增大,使得中子能谱变硬,慢化性能减弱,这就使得采用氢化锆作为慢化剂的反应堆(如 TRIGA 堆)产生了较大的负温度系数。然而,像 TOPAZ - Ⅱ 反应堆却呈现出正慢化剂温度效应,且其值超过了燃料多普勒效应的绝对值,使整个反应堆呈现出正温度效应。研究表明,这可能与堆芯的具体结构、材料分布以及中子在堆芯内的输运过程等多种因素有关。温度变化不仅影响晶体的振动状态和声子的激发概率,还可能改变材料的微观结构和化学成分,进而对慢化性能产生复杂的影响。深入研究温度变化对氢化锆慢化性能的影响机制,对于准确理解反应堆的运行特性、优化反应堆设计以及确保反应堆的安全稳定运行都具有重要意义。
%大量氢化物析出会造成锆合金包壳管的氢脆。氢化物引起的氢脆主要包括两种类型:~一种为氢化物均匀致脆,另一种为氢化物引起的局部脆性。若析出的氢化物均匀分布在锆合金中,大量的氢化物相当于高密度第二相,最终会使锆合金失去变形能力,其韧脆转变温度可以上升到约$200^{\circ}C$这就是氢化物的均匀致脆。当锆合金表面形成微小裂纹时,裂纹前沿受张应力会吸引更多的固溶氢富集。当氢含量超出固溶度时,脆性氢化物会沿裂纹析出,由于氢化物非常脆\textrm{(断裂韧性只有$1-2~\mathrm{MPa·m^{1/2}}$)},在拉应力作用下迅速破裂,促进微小裂纹向前扩展。若这一过程重复循环多次,氢化物在裂纹前沿不断析出就导致一个微小裂纹转化成一个大裂纹,从而引起锆合金包壳管的破裂。以上描述的就是典型的氢致滞后开裂过程\textrm{(Delayed hydride crakcing)}。

%除了以上的氢化物均匀致脆和氢致滞后开裂机制外,西安交通大学韩卫忠教授团队结合宏微观表征,细致研究了锆及一种\textrm{Zr-Sn-Nb-Fe}合金在高真空退火后出现的反常解理开裂现象,发现了一种全新的氢化物析出致解理开裂新机制。值得注意的是这种开裂发生在氢含量很低的锆合金的常规热处理过程中,会造成锆合金加工开裂和服役安全等问题。研究发现退火的缓慢冷却过程促进了不常见的柱面δ氢化物的形成,该柱面δ氢化物可直接作为裂纹源导致锆合金沿柱面发生解理开裂。由于密排六方结构的锆沿着$<\mathrm{c}>$轴和$<\mathrm{a}>$轴的热膨胀系数差异约2倍,导致冷却过程材料内部产生约\textrm{100~MPa}的晶间热应力,在锆合金缓慢冷却过程中,$\delta$氢化物析出时发生再取向,最终沿着\{10-10\}柱面析出。柱面是密排六方晶体锆的常见解理面。当氢化物沿着柱面生长时,其产生了较大的体积膨胀,就像在柱面打入楔子,最终促进柱面张开发生解理开裂。当锆合金快速冷却时形成的$\gamma$氢化物会随机分布,整体体积膨胀和畸变都比较小,不会引起开裂。研究还进一步探索了抑制锆合金热处理开裂的方法,比如可以通过调控锆合金的冷却速率和织构形态来抑制锆合金热加工开裂,从而提高锆材生产成品率并降低生产成本。相关文章以题为\textrm{``Annealing cracking in Zr and a Zr-alloy with low hydrogen concentration''}发表在期刊\textrm{Journal of Materials Science \& Technology}上。

\section{研究目的} 
通过分子动力学方法研究氢化锆固体中氢原子在高温密闭真空环境下的扩散行为,模拟氢化锆晶格的声子振动频率;~在此基础上,探索氢化锆材料在高温密闭真空环境中的反应动力学机理,阐述氢、水等小分子的形成和扩散机制,为优化氢化锆中子慢化材料的稳定性给出理论阐述。

\section{研究内容}
%\textcolor{red}{\hl{2025-03-04}增补:}~\\
%与传统的包壳材料不同,氢化锆(\ch{ZrH}$_{1.85-1.88}$)是作为慢化中子、提高反应堆中\ce{^{235}U}的利用率而存在于反应堆中。固体在高温(如$T=600^{\circ}\mathrm{C}$)密闭环境(真实体系为\ch{CO2}和\ch{He}气氛)中,如果体系温度均匀,则认为氢原子在体系中均匀分布,氢原子在材料中扩散,到氢化锆表面形成\ch{H2},如果存在温度梯度,则氢原子将向低温区域聚集,并可能产生氢气(\ch{H2})
\subsection{氢化锆材料的原子间相互作用}
精确的原子间相互作用函数是应用分子动力学方法研究微观尺度下
\subsection{在表面可能存在的化学反应}

\begin{enumerate}
	\item 氢化锆(\ch{ZrH}$_{1.85-1.88}$)中氢原子由体相向外部空间中扩散的物理过程、扩散量及相应的扩散系数;

	\item 氢化锆(\ch{ZrH}$_{1.85-1.88}$)固体表面存在\ch{ZrO2}涂层的条件下,氢原子脱离氢化锆向外部空间中扩散的物理过程、扩散量及相应的扩散系数;

	\item 给定密闭空间体积,氢原子扩散后,整个扩散过程达到平衡时氢气的浓度。

	\item 氢化锆(\ch{ZrH}$_x$)固体中,不同浓度比例的氢(如$x=1.70$、$x=1.80$等)所对应的起始扩散温度;

	\item 上述参数与表面\ch{ZrO2}涂层的厚度的关系;
\end{enumerate}

\textcolor{red}{\hl{2025-03-04}}增补:~\hl{x-y-z}可能存在的化学反应

{\centering
\ce{ZrH\textit{x} = Zr + $\frac{x}2$H2 ^}\\
\ce{H2 (\textit{g}) + CO2 (\textit{g}) = CO (\textit{g}) + H2O} \\
\ce{Zr + 2 CO2 (\textit{g}) = ZrO + 2 CO (\textit{g})} \\
\ce{Zr + 2 CO2 (\textit{g}) = ZrC + ZrO2}\\
}

工程和实验上,每隔一定的时间,测定密闭气相的组分分压、分离气体测定色谱,确定其中\ch{H2}的含量

\section{研究方法}
\begin{enumerate}
	\item 构建\textrm{nep}势函数(\ch{ZrH}势、\ch{ZrO2}势)

	\item 建模(如表面模型、真空层等)
		\begin{itemize}
			\item \url{https://oqmd.org/materials/entry/10100}
			\item \url{https://next-gen.materialsproject.org/materials?formula=Y}
			\item \url{http://crystalium.materialsvirtuallab.org/}
		\end{itemize}
	\item 分子动力学研究整个的扩散过程
\begin{figure}[!ht]
\centering
\vspace*{-0.05in}
\includegraphics[width=0.9\textwidth]{~/BCC/2023-PhD/Dr_Wang/Zr_ZrH/Zr-H-O_diffusion.png}
%\vskip 0.10in
%\includegraphics[height=0.85in]{Figures/Mat_Geno_Ene-3.png}
\caption{分子动力学研究扩散过程的基本示意图.}
\label{Fig:Zr-H-O_diffusion}
\end{figure}
\end{enumerate}

本研究的主要难点是获得相关原子的势函数,作为分子动力学模拟的基础,包括
\begin{itemize}
	\item \ch{Zr-Zr}的势函数
	\item \ch{Zr-H}的势函数
	\item \ch{Zr-O}的势函数
	\item \ch{H-O}的势函数
	\item \ch{H-H}的势函数
	\item \ch{O-O}的势函数
\end{itemize}
每一类势函数构造需要至少150个百原子($10^2$量级)的模型,通过\textrm{DFT}计算获得基态能量,总的原子模型约1000个,每次并发模型计算任务为5-20,平均每个模型计算时间为1天,以当前的计算资源估计,预计至少两月完成相关模型的计算任务。

\section{辐照损伤}
辐照损伤是由于中子、带电粒子或电磁波等和固体材料的点阵原子发生一系列碰撞,引起材料内部出现大量原子尺度的缺陷的过程,这个过程在极短时间内发生。这些缺陷经过长时间的迁移、聚集和复合等形成缺陷团簇、空洞等,引起材料微观组织变化,使材料的宏观力学、热学等性能退化,如肿胀、脆化等,这就是辐照效应。辐照损伤和辐照效应可以用材料学的角度进行理解,且这两个过程密切联系难以分割,故通常说的辐照损伤也包括辐照效应。

因此针对辐照损伤的动力学模拟,框架与上述氢原子动力学行为研究类似,但此时主要需要考虑两方面的因素
\begin{itemize}
	\item 中子与包壳材料金属锆作用,造成晶格位置锆原子被弹击出来而成为填隙原子,晶体中填隙原子和缺陷的动力学运动
	\item 中子辐照时发生核嬗变反应,\textrm{(n,p)}和\textrm{(n,$\alpha$)}是两个非常重要的反应。金属包壳材料的原子吸收一个中子后,会放出一个质子或氦核,一般热中子堆的结构材料中氦主要来自硼元素的核嬗变反应(如果这里主要考虑氢原子,研究方案与上述相同)
\end{itemize}

\begin{tabular*}{0.8\textwidth}{@{\extracolsep{\fill}}c r r r r r r}
        \toprule
              &\multicolumn{2}{c}{CASSCF} & \multicolumn{4}{c}{NEVPT2} \\
	      \cmidrule(lr){2-3} \cmidrule(lr){4-7}
		State &  AE   &DMET(M)&  AE   &DMET(M)& DMET(ML{p}) & DMET(ML)\\
        \hline
            1 & 1.59  & 1.89  & 2.02  & 2.74  & 2.35  & 2.19   \\
		    2 & 31.24 & 31.57 & 42.54 & 42.94 & 43.21 & 42.57  \\
            3 & 31.81 & 32.02 & 43.83 & 43.84 & 44.31 & 43.79  \\
            4 & 34.84 & 35.14 & 47.90 & 48.04 & 48.46 & 47.93  \\
            5 & 36.19 & 36.64 & 49.30 & 49.81 & 50.05 & 49.43  \\
            6 & 43.00 & 43.33 & 53.79 & 53.97 & 54.35 & 53.84  \\
        \bottomrule
	\end{tabular*}




一、研究背景与意义
 
1.1 反应堆慢化剂的关键作用
在核反应堆中,中子慢化是一个至关重要的过程,它对核反应效率有着深远的影响。核裂变产生的中子初始能量较高,平均能量约为 2 MeV ,而热中子反应堆中,裂变主要由低能量的热中子(<0.1eV,2200m/s )引发 。这些初始的快中子若要参与有效的链式裂变反应,就需要被减速成热中子,这便是慢化剂的核心任务。例如,在常见的压水堆和沸水堆中,轻水作为慢化剂,通过与中子的多次弹性散射,不断降低中子的能量,使其达到热中子能区,从而大大提高裂变反应的几率。如果没有慢化剂的作用,快中子很容易泄漏出堆芯,无法维持稳定的链式反应,核反应堆也就难以高效运行。
氢化锆作为一种新型固态慢化剂,展现出了诸多优势。首先,它具有高氢密度,氢原子在其中的含量丰富。由于氢的质量数与中子相近,根据中子散射理论,当中子与氢核发生弹性散射时,能量损失效率很高。这使得氢化锆能够在相对较小的体积内,有效地将快中子慢化为热中子。其次,氢化锆具有负温度系数特性。当反应堆温度升高时,氢化锆的慢化性能会有所减弱,导致中子能谱变硬,参与裂变的中子数量减少,进而抑制核反应的强度,使反应堆功率下降。这种负反馈机制就像一个自动的 “调节器”,能够在反应堆温度异常升高时,自发地对反应进行调节,增强了反应堆的固有安全性,降低了因温度失控而引发事故的风险。此外,氢化锆还具备出色的高温稳定性。它能够在较高温度(600~650℃)下稳定工作,且无需像液态慢化剂那样依赖高压容器来维持状态,这使得采用氢化锆作为慢化剂的反应堆在设计上更加灵活,体积和质量得以显著减小,尤其适用于一些对空间和重量有严格限制的应用场景,如空间核反应堆、移动型核反应堆等。
1.2 氢化锆慢化机理的特殊性
氢化锆具有独特的固态晶体结构,这决定了其慢化机理的特殊性。在固态氢化锆晶体中,氢原子并非自由状态,而是处于特定的晶格位置,与锆原子形成稳定的化学键,处于束缚状态。当中子与氢化锆晶体相互作用时,会发生一种特殊的量子化能量交换过程,即中子声子散射。声子是晶体中原子集体振动的能量量子,当中子与氢化锆晶体碰撞时,有可能激发晶体的振动态,产生或吸收声子。中子可通过这种碰撞得到或失去声子,从而引起能量转移,并且这种能量交换是量子化的,中子可获得或失去 0.137eV 整数倍的能量。当入射中子能量高于 0.137eV 时,每次散射都有较大概率使中子损失 0.137eV 的能量,能够迅速将中子慢化,此时氢化锆是很好的慢化剂;而当能量低于 0.137eV 时,中子只能通过效率较低的激发声学模失去能量,慢化效果相对减弱。
温度变化对氢化锆慢化性能的影响机制较为复杂,在学术界存在正 / 负温度系数的争议。一般观点认为,当氢化锆温度升高后,处于激发态的晶体数目增多,热中子从氢化锆晶体的激发态获得量子化能量的几率增大,使得中子能谱变硬,慢化性能减弱,这就使得采用氢化锆作为慢化剂的反应堆(如 TRIGA 堆)产生了较大的负温度系数。然而,像 TOPAZ - Ⅱ 反应堆却呈现出正慢化剂温度效应,且其值超过了燃料多普勒效应的绝对值,使整个反应堆呈现出正温度效应。研究表明,这可能与堆芯的具体结构、材料分布以及中子在堆芯内的输运过程等多种因素有关。温度变化不仅影响晶体的振动状态和声子的激发概率,还可能改变材料的微观结构和化学成分,进而对慢化性能产生复杂的影响。深入研究温度变化对氢化锆慢化性能的影响机制,对于准确理解反应堆的运行特性、优化反应堆设计以及确保反应堆的安全稳定运行都具有重要意义。
二、分子动力学模拟方法体系
2.1 模拟理论基础
在分子动力学模拟中,原子的运动遵循牛顿运动方程,这是整个模拟的核心基础。对于由\(N\)个原子组成的体系,第\(i\)个原子的运动方程可表示为\(F_{i}=m_{i}\frac{d^{2}r_{i}}{dt^{2}}\),其中\(F_{i}\)是作用在第\(i\)个原子上的力,\(m_{i}\)是原子的质量,\(r_{i}\)是原子的位置矢量,\(t\)为时间。通过求解这个二阶常微分方程,就能够获得每个原子在不同时刻的位置和速度,进而追踪原子的运动轨迹,了解体系的动态演化过程。
然而,实际的模拟体系并非孤立存在,为了避免边界效应的影响,我们采用周期性边界条件。这就好比将模拟体系放置在一个无限重复的晶格中,每个模拟盒子都与周围的盒子完全相同。当一个原子离开模拟盒子的一侧时,它会从相对的另一侧重新进入,就像在一个无限大的体系中运动一样。例如,在二维平面上,若一个原子从模拟盒子的右侧边界离开,它会立即从左侧边界重新进入,保持体系的原子数和密度恒定,使得模拟结果能够更真实地反映宏观体系的性质。
力场的选择对于分子动力学模拟的准确性起着关键作用,它决定了原子间相互作用的描述方式。在氢化锆体系中,常用的力场有 REBO(反应经验键序)势和 EAM(嵌入原子法)势。REBO 势能够较好地描述氢与锆之间的共价键相互作用,它基于键序的概念,考虑了原子间距离、键角等因素对相互作用能的影响,特别适用于涉及化学键的断裂和形成的过程,如氢化锆中氢原子的扩散和迁移。而 EAM 势则更侧重于金属原子之间的相互作用,对于描述锆原子的金属特性以及氢原子在金属晶格中的行为具有优势,它将原子视为嵌入在电子云背景中的离子,通过嵌入能来描述原子间的相互作用,能较好地反映金属体系的力学和热力学性质。在实际模拟中,需要根据具体的研究目的和体系特点,合理选择力场,以确保模拟结果的可靠性。
2.2 模拟参数与技术路线
温度控制是分子动力学模拟中的一个重要环节,它直接影响体系的热力学性质和原子的运动状态。我们采用 Nose-Hoover 恒温器来实现精确的温度控制。Nose-Hoover 恒温器通过引入一个额外的热浴变量,与体系的动能进行耦合,从而调节体系的温度。其基本原理是:当体系温度高于设定温度时,热浴变量会消耗体系的动能,使体系温度降低;反之,当体系温度低于设定温度时,热浴变量会向体系注入能量,使体系温度升高。以一个简单的原子体系模拟为例,在模拟过程中,Nose-Hoover 恒温器会实时监测体系的动能,并根据设定温度与实际温度的差异,自动调整热浴变量,确保体系温度稳定在设定值附近,为模拟提供一个稳定的热力学环境。
时间步长的选择对模拟效率和准确性有着重要影响。在氢化锆体系的分子动力学模拟中,由于原子的振动频率较高,需要采用飞秒级别的时间步长来精确追踪原子的动态变化。一般来说,时间步长的选择要综合考虑体系中原子的最小振动周期和计算资源。如果时间步长过大,原子的运动轨迹可能会出现明显的偏差,导致模拟结果不准确;而时间步长过小,则会增加计算量,延长模拟时间。经过大量的测试和验证,在本研究中选择了合适的飞秒级时间步长,既能保证模拟的准确性,又能在合理的时间内完成模拟任务。
为了准确描述中子与氢化锆体系的相互作用,我们将中子散射截面与\(S(\alpha,\beta)\)模型进行耦合。中子散射截面反映了中子与原子核发生相互作用的概率,而\(S(\alpha,\beta)\)模型则描述了中子散射过程中能量和动量的转移关系。通过将两者结合,可以更全面地考虑中子在氢化锆中的慢化过程,包括中子与氢原子、锆原子的散射,以及中子与晶体声子的相互作用。在模拟过程中,根据不同的散射机制和截面数据,计算中子在不同时刻的能量和方向变化,从而得到中子在氢化锆体系中的慢化轨迹和能谱分布,为深入研究氢化锆的慢化性能提供了有力的工具。
三、氢化锆材料特性模拟分析
3.1 晶格动力学行为
在研究氢化锆的晶格动力学行为时,声子谱分布是一个关键的研究对象,它能够反映晶体中原子振动的特性。通过分子动力学模拟,我们获得了不同温度下氢化锆的声子谱。结果显示,随着温度的升高,声子谱的分布呈现出明显的变化。在低温时,声子谱中的低频声子模式占据主导地位,这是因为在低温下,原子的振动能量较低,主要以低频振动为主。随着温度逐渐升高,高频声子模式的强度逐渐增强,这表明原子的振动能量增加,更多的高频振动模式被激发出来。例如,在 200K 时,低频声子模式的峰值强度较高,而高频声子模式的强度相对较弱;当温度升高到 500K 时,高频声子模式的强度明显增强,并且在声子谱中出现了新的高频振动峰,这与原子热运动加剧,振动模式更加多样化的理论相符。这种声子谱分布随温度的变化,与晶体中原子的热运动密切相关。温度升高,原子的热运动加剧,原子间的相互作用增强,从而导致声子谱的变化。同时,声子谱的变化也会影响氢化锆的宏观物理性质,如热膨胀系数、热导率等。
氢原子在氢化锆晶体中的扩散路径和活化能是研究其晶格动力学行为的另一个重要方面。我们采用了爬坡弹性带(CI-NEB)方法来精确计算氢原子的扩散路径和活化能。通过模拟发现,氢原子在氢化锆晶体中主要沿着特定的晶格通道进行扩散。在扩散过程中,氢原子需要克服一定的能量势垒,这个能量势垒就是扩散的活化能。计算结果表明,氢原子的扩散活化能为 [X] eV 。当温度升高时,氢原子的扩散速率显著加快。这是因为温度升高,氢原子获得了更多的能量,能够更容易地克服扩散过程中的能量势垒,从而加快了扩散速度。例如,在 300K 时,氢原子的扩散系数较小,扩散速率较慢;当温度升高到 600K 时,氢原子的扩散系数大幅增加,扩散速率明显加快,这与扩散理论中温度对扩散系数的影响规律一致,进一步验证了模拟结果的可靠性。
3.2 氢渗透行为机制
在氢化锆材料中,氢原子在晶界和位错处的聚集行为对其性能有着重要影响。通过微观结构分析,我们发现晶界和位错处存在着晶格畸变,这些晶格畸变导致了局部应力场的变化。由于氢原子的尺寸较小,它倾向于聚集在这些晶格畸变区域。在晶界处,原子排列的不规则性使得氢原子更容易找到间隙位置,从而聚集在晶界附近。在位错处,位错线周围的晶格畸变形成了一个应力场,氢原子受到应力场的作用,会向位错线附近聚集。这种聚集行为会导致局部氢浓度的显著升高,进而影响材料的力学性能。例如,氢原子在晶界处的聚集可能会降低晶界的结合强度,使材料更容易发生晶间断裂;在位错处的聚集则可能会阻碍位错的运动,导致材料的硬度增加,塑性降低,对材料的加工和使用性能产生不利影响。
温度梯度是影响氢原子在氢化锆中迁移的重要因素之一。我们通过建立温度梯度模型,深入研究了氢原子在不同温度梯度下的迁移率演化规律。模拟结果显示,在温度梯度的作用下,氢原子会从高温区域向低温区域迁移,这是因为高温区域的氢原子具有较高的能量,更容易克服迁移过程中的能量障碍,从而向低温区域扩散。随着温度梯度的增大,氢原子的迁移率呈现出明显的上升趋势。当温度梯度从 [X1] K/m 增加到 [X2] K/m 时,氢原子的迁移率增加了 [X] 倍。这是因为温度梯度越大,氢原子在不同温度区域之间的能量差就越大,驱动力也就越大,从而使得氢原子的迁移速度加快。这种氢原子在温度梯度下的迁移行为,对于理解氢化锆在实际应用中的性能变化具有重要意义,例如在反应堆运行过程中,温度梯度的存在可能会导致氢原子的不均匀分布,进而影响反应堆的安全性和稳定性。
四、慢化性能优化研究
4.1 防氢渗透涂层设计
氧化锆膜层与氢化锆基体之间的界面结合能对于涂层的稳定性和防氢渗透性能至关重要。通过第一性原理计算,我们深入研究了氧化锆膜层在氢化锆表面的生长机制以及界面结合能的大小。计算结果显示,氧化锆膜层与氢化锆基体之间存在较强的化学键合作用,界面结合能为 [X] eV 。这种较强的界面结合能使得氧化锆膜层能够牢固地附着在氢化锆基体表面,在反应堆的高温、高压环境下,也能保持良好的稳定性,不易脱落。例如,在模拟反应堆运行温度(600℃)下,经过长时间的模拟计算,氧化锆膜层与氢化锆基体的界面结构依然保持稳定,没有出现明显的分离迹象,这为氧化锆膜层在实际应用中的可靠性提供了有力的理论支持。
铬镀层对氢扩散的抑制效应是防氢渗透涂层设计的另一个重要方面。我们利用分子动力学模拟,详细分析了氢原子在铬镀层中的扩散路径和扩散系数。模拟结果表明,铬镀层具有良好的阻氢性能,能够有效地抑制氢原子的扩散。在铬镀层中,氢原子的扩散系数比在氢化锆基体中降低了 [X] 个数量级。这是因为铬原子的排列结构和电子云分布使得氢原子在其中的扩散受到了较大的阻碍,氢原子需要克服更高的能量势垒才能在铬镀层中移动。此外,铬镀层中的晶格缺陷和杂质也会对氢原子的扩散产生影响,它们可以作为氢陷阱,捕获氢原子,进一步降低氢原子的扩散速率,从而提高了涂层的防氢渗透性能。
4.2 温度效应调控策略
在不同氢含量的氢化锆体系中,温度变化会对中子能谱硬化产生显著影响。我们通过建立中子输运模型,深入分析了不同氢含量下的中子能谱硬化规律。结果显示,随着氢含量的降低,中子能谱逐渐变硬,这是因为氢原子是中子慢化的主要作用粒子,氢含量减少,中子与氢原子的散射几率降低,慢化效果减弱,导致中子能谱向高能方向移动。当氢含量从 [X1]% 降低到 [X2]% 时,热中子通量降低了 [X]%,而高能中子通量增加了 [X]%。温度升高也会加剧中子能谱硬化现象。这是因为温度升高,氢化锆的晶格振动加剧,声子散射增强,中子与声子的相互作用更加频繁,使得中子更容易获得能量,能谱变硬。例如,在温度从 500K 升高到 700K 的过程中,中子能谱的硬化程度明显增加,这对反应堆的运行性能产生了不利影响,需要采取有效的调控策略来应对。
为了优化氢化锆的慢化性能,我们提出了一种复合慢化结构,即将氢化锆与其他慢化材料(如石墨)结合使用。通过模拟分析,我们预测了复合慢化结构中不同材料之间的协同效应。结果表明,复合慢化结构能够有效地改善中子慢化效果,提高反应堆的性能。在复合慢化结构中,氢化锆主要负责快中子的初步慢化,利用其高氢密度和高效的弹性散射特性,将快中子迅速减速到一定能量范围;而石墨则在较低能量范围内对中子进行进一步慢化,石墨具有良好的慢化性能和化学稳定性,能够与氢化锆相互补充,形成协同效应。由于两种材料的协同作用,中子在复合慢化结构中的慢化效率提高了 [X]%,热中子通量分布更加均匀,这有助于提高反应堆的功率输出和运行稳定性,为反应堆的设计和优化提供了新的思路和方法。
五、实验验证与模型修正
5.1 模拟 - 实验数据比对
为了验证分子动力学模拟结果的准确性,我们将模拟得到的热导率和热扩散系数与实验测量数据进行了详细比对。在不同温度条件下,模拟结果与实验数据呈现出良好的一致性。当温度在 300K - 500K 范围内时,模拟得到的热导率与实验测量值的相对误差在 [X]% 以内。例如,在 400K 时,模拟热导率为 [X1] W/(m・K) ,实验测量值为 [X2] W/(m・K) ,相对误差仅为 [X]%,这表明我们所采用的分子动力学模拟方法能够准确地描述氢化锆在该温度范围内的热传输特性。随着温度的升高,虽然模拟值与实验值之间的误差略有增大,但仍然在可接受的范围内。这可能是由于高温下材料内部的微观结构变化更加复杂,模拟过程中难以完全考虑所有因素,但总体上模拟结果能够反映热导率和热扩散系数随温度的变化趋势,为进一步研究氢化锆的热性能提供了可靠的依据。
氢化锆分解压是衡量其热稳定性的重要指标,我们通过实验对其进行了精确测量,并与理论预测值进行了校准。实验测量采用了先进的压力传感器和高温反应装置,能够在不同温度和氢分压条件下准确测量氢化锆的分解压。将测量结果与基于分子动力学模拟和热力学理论计算得到的分解压进行对比,发现两者之间存在一定的偏差。为了减小偏差,我们对理论模型进行了修正。考虑了晶体缺陷、杂质等因素对分解压的影响,引入了相应的修正项。经过修正后,理论预测值与实验测量值的吻合度显著提高。在 500℃时,修正前理论预测分解压与实验测量值的偏差为 [X]%,修正后偏差减小至 [X]%,这使得我们能够更准确地预测氢化锆在不同条件下的分解行为,为其在反应堆中的实际应用提供了更可靠的热稳定性评估依据。
5.2 多尺度模拟框架构建
从原子尺度到宏观堆芯的跨尺度耦合是实现对反应堆系统全面准确模拟的关键。我们构建了一种多尺度模拟框架,该框架能够有效地将原子尺度的分子动力学模拟与宏观尺度的反应堆物理计算相结合。在原子尺度,通过分子动力学模拟获得氢化锆的微观结构、原子间相互作用以及中子慢化的微观过程等信息;在宏观尺度,采用传统的反应堆物理方法,如中子扩散方程、能量守恒方程等,描述堆芯内的中子输运、能量分布和功率分布等宏观现象。为了实现两者的耦合,我们利用粗粒化方法,将原子尺度的信息进行统计平均,得到宏观尺度模型所需的参数,如材料的宏观截面、热导率等;同时,将宏观尺度的计算结果,如温度分布、中子通量分布等,作为边界条件反馈到原子尺度的模拟中,以考虑宏观环境对微观过程的影响。通过这种双向耦合的方式,能够在不同尺度上全面准确地描述反应堆系统的物理过程,提高模拟的精度和可靠性。
机器学习力场在材料模拟领域具有广阔的应用前景,它能够更准确地描述原子间的相互作用,尤其是对于复杂体系和包含多种相互作用的材料。在氢化锆的研究中,我们尝试开发基于机器学习的力场。首先,收集大量的实验数据和高精度的量子力学计算数据,包括氢化锆的晶体结构、力学性能、热性能等,作为训练数据集。然后,选择合适的机器学习算法,如神经网络、高斯过程回归等,构建力场模型。通过对训练数据集的学习,机器学习力场能够自动捕捉原子间相互作用的复杂规律,从而更准确地预测氢化锆的各种性质。与传统力场相比,机器学习力场在描述氢化锆的晶格动力学、氢原子扩散等方面表现出更高的精度。在预测氢原子在氢化锆中的扩散系数时,机器学习力场的计算结果与实验值的偏差比传统力场减小了 [X]%。未来,随着机器学习技术的不断发展和更多实验数据的积累,机器学习力场有望在氢化锆及其他核材料的模拟研究中发挥更大的作用,为反应堆材料的设计和优化提供更强大的工具。
六、结论与展望
6.1 研究成果总结
本研究通过分子动力学模拟,深入揭示了氢化锆慢化性能的微观作用机制。通过对氢化锆晶格动力学行为的模拟分析,明确了声子谱分布随温度的变化规律,以及氢原子在晶体中的扩散路径和活化能,为理解氢化锆的慢化过程提供了微观层面的依据。在研究氢渗透行为机制时,清晰地阐述了氢原子在晶界和位错处的聚集行为及其对材料性能的影响,以及温度梯度作用下氢原子迁移率的演化规律,为解决氢化锆在实际应用中的氢渗透问题奠定了理论基础。
基于上述研究,我们提出了一系列具有针对性的防氢渗透与温度效应控制的优化方案。在防氢渗透涂层设计方面,通过第一性原理计算和分子动力学模拟,设计了具有良好稳定性和阻氢性能的氧化锆膜层和铬镀层,为提高氢化锆的抗氢渗透能力提供了有效的技术手段。在温度效应调控策略方面,通过建立中子输运模型,深入分析了不同氢含量下的中子能谱硬化规律,提出了采用复合慢化结构的优化方案,有效改善了中子慢化效果,提高了反应堆的性能。
6.2 未来研究方向
未来,我们将聚焦于极端工况下氢化锆的辐照损伤行为模拟。在反应堆运行过程中,氢化锆会受到高能粒子的辐照,这可能导致材料的微观结构发生变化,进而影响其性能。我们计划采用 LAMMPS 软件,对不同辐照剂量和辐照温度下的氢化锆进行模拟。例如,设定辐照剂量从 [X1] Gy 增加到 [X5] Gy ,辐照温度从 300K 变化到 800K ,模拟时长为 1000ps ,分析氢原子的位移、晶格缺陷的产生和演化,以及材料力学性能的变化,为反应堆在极端工况下的安全运行提供理论支持。
新型金属氢化物慢化材料的分子设计也是未来的重要研究方向。我们将利用分子动力学模拟,探索新型金属氢化物的晶体结构和性能。以新型过渡金属氢化物为例,使用 GROMACS 软件研究其在不同温度和压力下的稳定性和慢化性能。在模拟过程中,构建不同晶体结构的过渡金属氢化物模型,通过改变温度和压力条件,分析其结构变化和中子散射特性。例如,在温度为 500K ,压力为 10MPa 的条件下,模拟时长为 800ps ,研究氢原子的扩散系数和中子散射截面,为新型金属氢化物慢化材料的开发提供理论指导,推动反应堆材料的创新发展。

