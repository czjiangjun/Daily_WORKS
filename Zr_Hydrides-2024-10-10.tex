%---------------------- TEMPLATE FOR REPORT ------------------------------------------------------------------------------------------------------%

%\thispagestyle{fancy}   % 插入页眉页脚                                        %
%%%%%%%%%%%%%%%%%%%%%%%%%%%%% 用 authblk 包 支持作者和E-mail %%%%%%%%%%%%%%%%%%%%%%%%%%%%%%%%%
%\title{More than one Author with different Affiliations}				     %
%\title{\rm{VASP}的电荷密度存储文件\rm{CHGCAR}}
%\title{面向高温合金材料设计的计算模拟软件中的几个主要问题}
\title{氢化锆中氢原子扩散过程研究与包壳材料的辐照动力学研究}
\author[ ]{}   %
%\author[ ]{姜~骏\thanks{jiangjun@bcc.ac.cn}}   %
%\affil[ ]{北京市计算中心}    %
%\author[a]{Author A}									     %
%\author[a]{Author B}									     %
%\author[a]{Author C \thanks{Corresponding author: email@mail.com}}			     %
%%\author[a]{Author/通讯作者 C \thanks{Corresponding author: cores-email@mail.com}}     	     %
%\author[b]{Author D}									     %
%\author[b]{Author/作者 D}								     %
%\author[b]{Author E}									     %
%\affil[a]{Department of Computer Science, \LaTeX\ University}				     %
%\affil[b]{Department of Mechanical Engineering, \LaTeX\ University}			     %
%\affil[b]{作者单位-2}			    						     %
											     %
%%% 使用 \thanks 定义通讯作者								     %
											     %
\renewcommand*{\Authfont}{\small\rm} % 修改作者的字体与大小				     %
\renewcommand*{\Affilfont}{\small\it} % 修改机构名称的字体与大小			     %
\renewcommand\Authands{ and } % 去掉 and 前的逗号					     %
\renewcommand\Authands{ , } % 将 and 换成逗号					     %
\date{} % 去掉日期									     %
%\date{2020-12-30}									     %

%%%%%%%%%%%%%%%%%%%%%%%%%%%%%%%%%%%%%%%%%%  不使用 authblk 包制作标题  %%%%%%%%%%%%%%%%%%%%%%%%%%%%%%%%%%%%%%%%%%%%%%
%-------------------------------The Title of The Report-----------------------------------------%
%\title{报告标题:~}   %
%-----------------------------------------------------------------------------

%----------------------The Authors and the address of The Paper--------------------------------%
%\author{
%\small
%Author1, Author2, Author3\footnote{Communication author's E-mail} \\    %Authors' Names	       %
%\small
%(The Address,City Post code)						%Address	       %
%}
%\affil[$\dagger$]{清华大学~材料加工研究所~A213}
%\affil{清华大学~材料加工研究所~A213}
%\date{}					%if necessary					       %
%----------------------------------------------------------------------------------------------%
%%%%%%%%%%%%%%%%%%%%%%%%%%%%%%%%%%%%%%%%%%%%%%%%%%%%%%%%%%%%%%%%%%%%%%%%%%%%%%%%%%%%%%%%%%%%%%%%%%%%%%%%%%%%%%%%%%%%%
\maketitle
%\thispagestyle{fancy}   % 首页插入页眉页脚 
锆合金具有较低的热中子吸收截面、优良的耐腐蚀性能和优异的高温力学性能,被广泛用作压水堆核燃料包壳管和压力管。锆合金对氢具有极强的亲和力,但室温下氢在锆中的固溶度却非常低\textrm{($<1\mathrm{ppm}$)},导致其在生产加工及长期服役过程中不可避免地吸收环境中的氢,并大量析出脆性氢化物。大量氢化物的析出导致锆合金包壳管韧脆转变温度大幅升高,形成多种多样的局部脆性,对核反应堆的安全运行造成威胁。为了确保核反应堆的安全运行并进一步提高核燃料的燃耗,工业界和学术界对锆包壳中氢化物的形核、析出、长大和致脆机理开展了长期研究。

随着吸氢量的增加,锆合金中会先后形成$\zeta$氢化物\textrm{(\ch{Zr2H},HCP-密排六方结构)}、$\gamma$氢化物\textrm{(\ch{ZrH},FCT-面心四方结构)}、$\delta$氢化物\textrm{(\ch{ZrH}$_{1.66}$,FCC-面心立方结构)}和$\epsilon$氢化物\textrm{(\ch{ZrH2},FCT-面心四方结构)}。锆合金中常见的氢化物为$\gamma$氢化物和$\delta$氢化物,前者是亚稳态氢化物,后者为比较稳定的氢化物。在没有外加应力的情况下,氢化物通常以锆基体的基面为惯习面,沿基面上的$<\mathrm{a}>$方向生长。在外加拉应力情况下,氢化锆会发生再取向,其惯习面会随着拉应力的增加逐步从基面转向\textrm{\{10-1$i$\} ($i$=1-7)}锥面,直至最后以柱面\{10-10\}为惯习面。再取向过程中,氢化物的生长方向始终沿着$<\mathrm{a}>$方向。在核燃料包壳管中,初始的氢化物都沿管子的周向分布,这与挤压管的初始织构密切相关,即锆包壳管具有基面沿管子周向分布的特征。在核反应堆服役过程中,核燃料在中子辐照下发生体积膨胀,使得包壳管被撑大,此时包壳管沿周向受到一定的拉应力,这种应力称之为环向应力\textrm{(hoop stress)}。在环向应力的作用下,当核反应堆冷却时,锆合金包壳管中的氢化物析出就会发生再取向,新的惯习面会沿着锥面或柱面。当环向拉应力超过\textrm{100~MPa}时,氢化物再取向主要会以柱面为惯习面。此时,若对比氢化物的初始分布和再取向分布,可以发现氢化物相当于转动了$90^{\circ}$,形成了大量的径向氢化物,沿锆合金包壳管厚度方向分布。这种再取向的氢化物会更容易造成锆合金包壳的失效。


大量氢化物析出会造成锆合金包壳管的氢脆。氢化物引起的氢脆主要包括两种类型:一种为氢化物均匀致脆,另一种为氢化物引起的局部脆性。若析出的氢化物均匀分布在锆合金中,大量的氢化物相当于高密度第二相,最终会使锆合金失去变形能力,其韧脆转变温度可以上升到约$200^{\circ}C$这就是氢化物的均匀致脆。当锆合金表面形成微小裂纹时,裂纹前沿受张应力会吸引更多的固溶氢富集。当氢含量超出固溶度时,脆性氢化物会沿裂纹析出,由于氢化物非常脆\textrm{(断裂韧性只有$1-2~\mathrm{MPa·m^{1/2}}$)},在拉应力作用下迅速破裂,促进微小裂纹向前扩展。若这一过程重复循环多次,氢化物在裂纹前沿不断析出就导致一个微小裂纹转化成一个大裂纹,从而引起锆合金包壳管的破裂。以上描述的就是典型的氢致滞后开裂过程\textrm{(Delayed hydride crakcing)}。

除了以上的氢化物均匀致脆和氢致滞后开裂机制外,西安交通大学韩卫忠教授团队结合宏微观表征,细致研究了锆及一种\textrm{Zr-Sn-Nb-Fe}合金在高真空退火后出现的反常解理开裂现象,发现了一种全新的氢化物析出致解理开裂新机制。值得注意的是这种开裂发生在氢含量很低的锆合金的常规热处理过程中,会造成锆合金加工开裂和服役安全等问题。研究发现退火的缓慢冷却过程促进了不常见的柱面δ氢化物的形成,该柱面δ氢化物可直接作为裂纹源导致锆合金沿柱面发生解理开裂。由于密排六方结构的锆沿着$<\mathrm{c}>$轴和$<\mathrm{a}>$轴的热膨胀系数差异约2倍,导致冷却过程材料内部产生约\textrm{100~MPa}的晶间热应力,在锆合金缓慢冷却过程中,$\delta$氢化物析出时发生再取向,最终沿着\{10-10\}柱面析出。柱面是密排六方晶体锆的常见解理面。当氢化物沿着柱面生长时,其产生了较大的体积膨胀,就像在柱面打入楔子,最终促进柱面张开发生解理开裂。当锆合金快速冷却时形成的$\gamma$氢化物会随机分布,整体体积膨胀和畸变都比较小,不会引起开裂。研究还进一步探索了抑制锆合金热处理开裂的方法,比如可以通过调控锆合金的冷却速率和织构形态来抑制锆合金热加工开裂,从而提高锆材生产成品率并降低生产成本。相关文章以题为\textrm{``Annealing cracking in Zr and a Zr-alloy with low hydrogen concentration''}发表在期刊\textrm{Journal of Materials Science \& Technology}上。

\section{研究目的} 
通过理论计算研究氢化锆固体中氢原子在高温密闭真空环境的扩散行为,模拟实验结果;~在此基础上,探索锆包壳材料接受辐照后损伤的微观动力学机理

\section{研究内容}
\textcolor{red}{\hl{2025-03-04}增补:}~\\
与传统的包壳材料不同,氢化锆(\ch{ZrH}$_{1.85-1.88}$)是作为慢化中子、提高反应堆中\ce{^{235}U}的利用率而存在于反应堆中。固体在高温(如$T=600^{\circ}\mathrm{C}$)密闭环境(真实体系为\ch{CO2}和\ch{He}气氛)中,如果体系温度均匀,则认为氢原子在体系中均匀分布,氢原子在材料中扩散,到氢化锆表面形成\ch{H2},如果存在温度梯度,则氢原子将向低温区域聚集,并可能产生氢气(\ch{H2})

在表面可能存在的化学反应

\begin{enumerate}
	\item 氢化锆(\ch{ZrH}$_{1.85-1.88}$)中氢原子由体相向外部空间中扩散的物理过程、扩散量及相应的扩散系数;

	\item 氢化锆(\ch{ZrH}$_{1.85-1.88}$)固体表面存在\ch{ZrO2}涂层的条件下,氢原子脱离氢化锆向外部空间中扩散的物理过程、扩散量及相应的扩散系数;

	\item 给定密闭空间体积,氢原子扩散后,整个扩散过程达到平衡时氢气的浓度。

	\item 氢化锆(\ch{ZrH}$_x$)固体中,不同浓度比例的氢(如$x=1.70$、$x=1.80$等)所对应的起始扩散温度;

	\item 上述参数与表面\ch{ZrO2}涂层的厚度的关系;
\end{enumerate}

\textcolor{red}{\hl{2025-03-04}}增补:~\hl{x-y-z}可能存在的化学反应

{\centering
\ce{ZrH\textit{x} = Zr + $\frac{x}2$H2 ^}\\
\ce{H2 (\textit{g}) + CO2 (\textit{g}) = CO (\textit{g}) + H2O} \\
\ce{Zr + 2 CO2 (\textit{g}) = ZrO + 2 CO (\textit{g})} \\
\ce{Zr + 2 CO2 (\textit{g}) = ZrC + ZrO2}\\
}

工程和实验上,每隔一定的时间,测定密闭气相的组分分压、分离气体测定色谱,确定其中\ch{H2}的含量

\section{研究方法}
\begin{enumerate}
	\item 构建\textrm{nep}势函数(\ch{ZrH}势、\ch{ZrO2}势)

	\item 建模(如表面模型、真空层等)
		\begin{itemize}
			\item \url{https://oqmd.org/materials/entry/10100}
			\item \url{https://next-gen.materialsproject.org/materials?formula=Y}
			\item \url{http://crystalium.materialsvirtuallab.org/}
		\end{itemize}
	\item 分子动力学研究整个的扩散过程
\begin{figure}[!ht]
\centering
\vspace*{-0.05in}
\includegraphics[width=0.9\textwidth]{~/BCC/2023-PhD/Dr_Wang/Zr_ZrH/Zr-H-O_diffusion.png}
%\vskip 0.10in
%\includegraphics[height=0.85in]{Figures/Mat_Geno_Ene-3.png}
\caption{分子动力学研究扩散过程的基本示意图.}
\label{Fig:Zr-H-O_diffusion}
\end{figure}
\end{enumerate}

本研究的主要难点是获得相关原子的势函数,作为分子动力学模拟的基础,包括
\begin{itemize}
	\item \ch{Zr-Zr}的势函数
	\item \ch{Zr-H}的势函数
	\item \ch{Zr-O}的势函数
	\item \ch{H-O}的势函数
	\item \ch{H-H}的势函数
	\item \ch{O-O}的势函数
\end{itemize}
每一类势函数构造需要至少150个百原子($10^2$量级)的模型,通过\textrm{DFT}计算获得基态能量,总的原子模型约1000个,每次并发模型计算任务为5-20,平均每个模型计算时间为1天,以当前的计算资源估计,预计至少两月完成相关模型的计算任务。

\section{辐照损伤}
辐照损伤是由于中子、带电粒子或电磁波等和固体材料的点阵原子发生一系列碰撞,引起材料内部出现大量原子尺度的缺陷的过程,这个过程在极短时间内发生。这些缺陷经过长时间的迁移、聚集和复合等形成缺陷团簇、空洞等,引起材料微观组织变化,使材料的宏观力学、热学等性能退化,如肿胀、脆化等,这就是辐照效应。辐照损伤和辐照效应可以用材料学的角度进行理解,且这两个过程密切联系难以分割,故通常说的辐照损伤也包括辐照效应。

因此针对辐照损伤的动力学模拟,框架与上述氢原子动力学行为研究类似,但此时主要需要考虑两方面的因素
\begin{itemize}
	\item 中子与包壳材料金属锆作用,造成晶格位置锆原子被弹击出来而成为填隙原子,晶体中填隙原子和缺陷的动力学运动
	\item 中子辐照时发生核嬗变反应,\textrm{(n,p)}和\textrm{(n,$\alpha$)}是两个非常重要的反应。金属包壳材料的原子吸收一个中子后,会放出一个质子或氦核,一般热中子堆的结构材料中氦主要来自硼元素的核嬗变反应(如果这里主要考虑氢原子,研究方案与上述相同)
\end{itemize}

\begin{tabular*}{0.8\textwidth}{@{\extracolsep{\fill}}c r r r r r r}
        \toprule
              &\multicolumn{2}{c}{CASSCF} & \multicolumn{4}{c}{NEVPT2} \\
	      \cmidrule(lr){2-3} \cmidrule(lr){4-7}
		State &  AE   &DMET(M)&  AE   &DMET(M)& DMET(ML{p}) & DMET(ML)\\
        \hline
            1 & 1.59  & 1.89  & 2.02  & 2.74  & 2.35  & 2.19   \\
		    2 & 31.24 & 31.57 & 42.54 & 42.94 & 43.21 & 42.57  \\
            3 & 31.81 & 32.02 & 43.83 & 43.84 & 44.31 & 43.79  \\
            4 & 34.84 & 35.14 & 47.90 & 48.04 & 48.46 & 47.93  \\
            5 & 36.19 & 36.64 & 49.30 & 49.81 & 50.05 & 49.43  \\
            6 & 43.00 & 43.33 & 53.79 & 53.97 & 54.35 & 53.84  \\
        \bottomrule
	\end{tabular*}
