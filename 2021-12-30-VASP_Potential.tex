%%%%%%%%%%%%%%%%%%%%%%%%%%%%% 用 authblk 包 支持作者和E-mail %%%%%%%%%%%%%%%%%%%%%%%%%%%%%%%%%
%\title{More than one Author with different Affiliations}				     %
\title{\hei{\textbf{VASP}的赝势生成}}
%\author[a]{Author/作者 A}   %
%\author[a]{Author B}									     %
%\author[a]{Author/通讯作者 C \thanks{Corresponding author: cores-email@mail.com}}     %
%\author[b]{Author/作者 D}									     %
%\author[b]{Author E}									     %
%\affil[a]{Department of Computer Science, \LaTeX\ University}				     %
%\affil[a]{作者单位-1 \authorcr 地址}    %\authorcr表示换行
%\affil[b]{Department of Mechanical Engineering, \LaTeX\ University}			     %
%\affil[b]{作者单位-2}			     %
\author[]{姜骏\thanks{E-mal:~czjiangjun@gmail.com}}   %
\affil[]{北京市计算中心}    %
											     %
%%% 使用 \thanks 定义通讯作者								     %
%%\affil命令后的{}中的内容,如果觉得需要换行的话,换行命令是\authorcr(不是\\)。
%%Email中可以吧相同邮箱的人@前面的内容写在一个{}里,用逗号隔开。注意{和}前面要加\。例如:
%%\affil[*]{单位1, \authorcr Email: \{zuozhe1, zuozhe2\}@yahoo.com, zuozhe3@sina.com}
											     %
\renewcommand*{\Authfont}{\small\rm} % 修改作者的字体与大小				     %
\renewcommand*{\Affilfont}{\small\it} % 修改机构名称的字体与大小			     %
\renewcommand\Authands{ and } % 去掉 and 前的逗号					     %
\renewcommand\Authands{ , } % 将 and 换成逗号					     %
\date{} % 去掉日期									     %
%\date{2020-12-30}									     %
%%%%%%%%%%%%%%%%%%%%%%%%%%%%%%%%%%%%%%%%%%%%%%%%%%%%%%%%%%%%%%%%%%%%%%%%%%%%%%%%%%%%%%%%%%%%%%
%-------------------------------The Title of The Paper-----------------------------------------%
%\title{标题}
%----------------------------------------------------------------------------------------------%

%----------------------The Authors and the address of The Paper--------------------------------%
%\author{
%作者:
%\small
%Author1, Author2, Author3\footnote{Communication author's E-mail} \\    %Authors' Names	       %
%\small
%(The Address,City Post code)						%Address	       %
%}
%\affil[$\dagger$]{清华大学~材料加工研究所~A213}
%\affil{清华大学~材料加工研究所~A213}
%\date{}					%if necessary					       %
%----------------------------------------------------------------------------------------------%
%%%%%%%%%%%%%%%%%%%%%%%%%%%%%%%%%%%%%%%%%%%%%%%%%%%%%%%%%%%%%%%%%%%%%%%%%%%%%%%%%%%%%%%%%%%%%%%%%%%%%%%%%%%%%%%%%%%%%
\maketitle
%\thispagestyle{fancy}   % 首页插入页眉页脚 

%-------------------------------------------------------------------------------The Abstract and the keywords of The Paper----------------------------------------------------------------------------%
%\begin{abstract}
%The content of the abstract
%\end{abstract}

%\keywords{Keyword1; Keyword2; Keyword3}

%-------------------------------------------------------------------------------The Content of The Paper----------------------------------------------------------------------------------------------%
%\tableofcontents %% 制作目录(目录是根据标题自动生成的)
%-----------------------------------------------------------------------------------------------------------------------------------------------------------------------------------------------------%

%\newpage	        % 每个新的/newpage 即可有新的\thispagestyle 引领      %
%\thispagestyle{fancy}   % 插入页眉页脚                                        %
%----------------------------------------------------------------------------------------The Body Of The Paper----------------------------------------------------------------------------------------%
%Introduction
\section{参数关系}
\begin{itemize}
	\item \textrm{AUTOA}:~ \textrm{AUTOA}=0.529177249
	\item \textrm{RYTOEV}:~ \textrm{RYTOEV} = 13.605826
	\item \textrm{FELECT}: $V(r)\cdot r$(对应单位)
		\begin{displaymath}
			2\times\textrm{AUTOA}\times\textrm{RYTOEV}
		\end{displaymath}
		表示$2e^2$的单位换算关系
\end{itemize}
%\section{Introduction}
\section{实空间电荷密度的计算表达式}
原子价电子波函数积分
\begin{equation}
	\begin{aligned}
		\int\psi_{lm}(\vec r)|\psi_{l^{\prime}m^{\prime}}(\vec r)\mathrm{d}\vec r=&\underbrace{\int\phi_l(r)\phi_{l^{\prime}}(r)\mathrm{d}r}&\underbrace{\int Y_{lm}(\theta,\varphi)Y_{l^{\prime}m^{\prime}}(\theta,\varphi)\mathrm{d}\Omega} \\%=4\pi\int\rho_v\mathrm{d}r
		&\mbox{径向部分积分} &\mbox{角度部分积分}
	\end{aligned}
	\label{eq:wave-integral}
\end{equation}
这里$\int\mathrm{d}\Omega$表示角度部分的积分微元(单位立体角)的积分$\int\sin\theta\mathrm{d}\theta\mathrm{d}\varphi$
%原子价电子径向波函数$\phi_i(r)$与价电子密度:
%\begin{equation}
%	\int|\phi_i(r)|^2\mathrm{d}r =4\pi\int\rho_v\mathrm{d}r
%	\label{eq:wave-charge}
%\end{equation}
对于角度部分,当前的被积函数$Y_{lm}(\theta,\varphi)Y_{l^{\prime}m^{\prime}}(\theta,\varphi)$不包含总角动量信息,称为\textcolor{magenta}{非耦合表象},可记作$|l,m;l^{\prime},m^{\prime}\rangle$。如果考虑角动量耦合产生总的角动量的贡献,称为\textcolor{magenta}{耦合表象},记作$|l,l^{\prime};L,M\rangle$。\footnote{两种表象都包含四个量子数,这里不能确定六个量子数。但因为有$L_z=l_z+l_z^{\prime}$,因此$M=m+m^{\prime}$在两种表象下都成立}

\subsection{\rm{CG}系数}
耦合表象$|l,l^{\prime};L,M\rangle$可以用非耦合表象$|l,m;l^{\prime},m^{\prime}\rangle$的线性组合表示
\begin{equation}
	|l,l^{\prime};L,M\rangle=\sum_{mm^{\prime}}C_{lm,l^{\prime}m^{\prime}}^{LM}|l,m;l^{\prime},m^{\prime}\rangle \qquad(m+m^{\prime}=M)
	\label{eq:couple_vs_uncouple}
\end{equation}
这里系数$C_{lm,l^{\prime}m^{\prime}}^{LM}$称为\textrm{Clebsch-Gordan~(CG)}系数。

\textrm{CG}系数与\textrm{Wignerr~3-j}符号有密切关系
\begin{equation}
	\begin{pmatrix}l &l^{\prime} &L\\m &m^{\prime} &M\end{pmatrix}\equiv\dfrac{(-1)^{l-l^{\prime}-M}}{2L+1}\langle l,m;l^{\prime},m^{\prime}|L,-M\rangle
	\label{eq:CG-vs-Wigner_3-j}
\end{equation}

\subsection{\rm{CG}系数与球谐函数}
\textrm{CG}系数常用于计算球谐函数乘积的积分
\begin{equation}
	\begin{aligned}
		&\int Y_{lm}^{\ast}(\theta,\varphi)Y_{l^{\prime}m^{\prime}}(\theta,\varphi)^{\ast}Y_{LM}(\theta,\varphi)\mathrm{d}\Omega\\
		=&(-1)^M\sqrt{\dfrac{(2l+1)(2l^{\prime}+1)}{4\pi(2L+1)}}\begin{bmatrix}l &l^{\prime} &L\\0 &0 &0\end{bmatrix}\begin{bmatrix}l &l^{\prime} &L\\m &m^{\prime} &-M\end{bmatrix}\\
			=&\sqrt{\dfrac{(2l+1)(2l^{\prime}+1)(2L+1)}{4\pi}}\langle l0l^{\prime}0|L0\rangle\langle lml^{\prime}m^{\prime}|LM\rangle\\
			=&\sqrt{\dfrac{(2l+1)(2l^{\prime}+1)(2L+1)}{4\pi}}\begin{pmatrix}l &l^{\prime} &L\\ 0 &0 &0\end{pmatrix}\begin{pmatrix}l &l^{\prime} &L\\m &m^{\prime} &M\end{pmatrix}
	\end{aligned}
	\label{CG-spherical_harmonics}
\end{equation}
因此典型的三个球谐函数乘积的积分可以表示为两个\textrm{CG}系数或\textrm{Wigner~3-j}符号相乘。

由式\eqref{CG-spherical_harmonics}和球谐函数的正交归一性
\begin{equation}
	\int Y_{l^{\prime}m^{\prime}}^{\ast}(\theta,\varphi)Y_{lm}(\theta,\varphi)\mathrm{d}\Omega=\delta_{ll^{\prime}}\delta_{mm^{\prime}}
	\label{SH-orthonormality}
\end{equation}
可得出结论:~两个球谐函数的乘积用另一个球谐函数展开时,展开系数可用\textrm{CG}系数表示
\begin{equation}
	Y_{lm}(\theta,\varphi)Y_{l^{\prime}m^{\prime}}(\theta,\varphi)=\sum_{LM}\sqrt{\dfrac{(2l+1)(2l^{\prime}+1)}{4\pi(2L+1)}}\langle l0l^{\prime}0|L0\rangle\langle lml^{\prime}m^{\prime}|LM\rangle Y_{LM}(\theta,\varphi)
	\label{eq:CG-spherical_harmonics_expand}
\end{equation}

\subsection{电荷密度的计算}
应用\textrm{CG}系数,可以将价电子密度的\textcolor{red}{分波表示}
\begin{equation}
	\begin{aligned}
		\rho^{LM}(r)=\sum_{ll^{\prime}}&\underbrace{\sum_{mm^{\prime}}n_{lm,l^{\prime}m^{\prime}}C_{lm,l^{\prime}m^{\prime}}^{LM}}\phi_l(r)\phi_{l^{\prime}}(r)\\
		&\mbox{变量\textrm{RHOLM}的计算}
	\end{aligned}
	\label{eq:rho_LM}
\end{equation}
这里$\rho^{LM}(r)$对应变量\textrm{RHO(:,:,1)},用$C_{lm,l^{\prime}m^{\prime}}^{LM}$表示\textrm{CG}系数。

体系的密度$\rho$
\begin{equation}
	\rho(r)=\sum_{LM}\rho^{LM}(r)
	\label{eq:density}
\end{equation}

\section{实空间的势函数计算}
\subsection{价电子的势函数}
精确冻芯条件下价电子势\textrm{POTAE}遵照表达式
\begin{equation}
	\mathrm{POTAE} = V_{\mathrm{H}}[n_v]+V_{\mathrm{XC}}[n_v+n_c]
	\label{eq:POTAE}
\end{equation}
计算。

精确冻芯的原子芯势\textrm{POTAEC}遵照表达式
\begin{equation}
	\mathrm{POTAEC} = V_{\mathrm{H}}[n_c]-\dfrac{Z~e}r
	\label{eq:POTAEC}
\end{equation}

为构造局域有效势$V_{\mathrm{eff}}(r)$,文献建议,\textcolor{red}{由冻芯价电子势\textrm{POTAE}“截断”(取$L=0$)获得}

理论上,有效势\textcolor{red}{局域赝势}$V_{\mathrm{eff}}(r)$的计算表达式为
\begin{equation}
	\begin{aligned}
		V_{\mathrm{eff}}(r) =& A\dfrac{\sin(q_{loc}r)}r\\
		\Rightarrow & V_{\mathrm{H}}[n_v+n_{Zc}]+V_{\mathrm{XC}}[n_v+n_c]
	\end{aligned}
	\label{eq:POT_EFF}
\end{equation}

构造赝势时$\tilde{V}_{\mathrm{loc}}(r)$,衔接半径$r_{\mathrm{vloc}}$与赝化波函数的截断半径$r_c$不同($r_{\mathrm{vloc}}$比$r_cs$小)
赝波函数引起的屏蔽势\textrm{POTPS}严格遵照表达式
\begin{equation}
	\mathrm{POTPS}:~\tilde{V}_{\mathrm{loc}}(r) = V_{\mathrm{H}}[\tilde{n}_v+\hat{n}_v]+V_{\mathrm{XC}}[\tilde{n}_v+\hat{n}_v+\tilde{n}_c]
	\label{eq:POTPS}
\end{equation}
计算。

\textcolor{blue}{去屏蔽的}\textcolor{blue}{可移植}\textcolor{red}{局域赝势}\textrm{POTPSC}的构造方案

\subsubsection{可移植局域赝势的构造(1)}
有限赝势去屏蔽,即可获得可移植局域赝势(离子势)
\begin{equation}
	\begin{aligned}
		\mathrm{POTPSC}:~\tilde{V}^{\mathrm{ion}}_{\mathrm{loc}}(r) =& V_{\mathrm{eff}}(r)-V_{\mathrm{loc}}\\
		=&V_{\mathrm{eff}}(r)-(V_{\mathrm{H}}[\tilde{n}_v+\hat{n}_v]+V_{\mathrm{XC}}[\tilde{n}_v+\hat{n}_v+\tilde{n}_c])
	\end{aligned}
	\label{eq:POTPSC-unsreen}
\end{equation}
优点:~无需引入额外参数

\subsubsection{可移植局域赝势的构造(2)}
去屏蔽的可移植离子赝势,可以考虑由冻芯原子芯势\textrm{POTAEC}的“截断”(取$L=0$)获得
\begin{equation}
	\begin{aligned}
		\mathrm{POTPSC}:~\tilde{V}^{\mathrm{ion}}_{\mathrm{loc}}(r) =& A\dfrac{\sin(q_{\mathrm{ion}}r)}r\\
		\Rightarrow & V_{\mathrm{H}}[n_{Zc}]=V_{\mathrm{H}}[n_c]-\dfrac{Z~e}r
	\end{aligned}
	\label{eq:POTPSC-sine}
\end{equation}

构造离子赝势时$V_{\mathrm{ion}}(r)$,需指定衔接半径$r_{\mathrm{vion}}$与有效局域赝势的衔接半径比较接近\textcolor{blue}{($r_{\mathrm{vion}}$可以比$r_{\mathrm{vloc}}$略大)}


\section{倒空间的势函数计算}
\subsection{径向函数的\rm{Fourier}变换}
用$\sin$函数表示静电相互作用势的\textrm{Fourier}变换:
\begin{displaymath}
	\begin{aligned}
		V(q)=&e^2\int\dfrac{\mathrm{d}r^3}r\mathrm{e}^{\mathrm{i}\vec q\dot\vec r}\\
		=&2\pi e^2\int_0^{\infty}r\mathrm{d}r\int_{-1}^{+1}\mathrm{d}(\cos\theta)\mathrm{e}^{\mathrm{i}qr\cos\theta}\\
		=&\dfrac{2\pi~e^2}{\mathrm{i}q}\int_0^{\infty}\mathrm{d}r(\mathrm{e}^{\mathrm{i}qr}-\mathrm{e}^{-\mathrm{i}qr})\\
		=&\dfrac{4\pi~e^2}{q}\int_0^{\infty}\mathrm{d}r\sin{qr}\\
		=&\dfrac{4\pi~e^2}{q^2}\bigg(1-\lim\limits_{r\rightarrow\infty}\cos(qr)\bigg)
	\end{aligned}
\end{displaymath}
该积分在$r\rightarrow\infty$时出现振荡,假设振荡可以忽略,则有
\begin{equation}
	V(q)=\dfrac{4\pi~e^2}{q^2}
	\label{eq:V-reciprocal}
\end{equation}
因此点电荷在倒空间产生的电势表示为
\begin{displaymath}
		V(q)=\dfrac{4\pi~Z~e^2}{q^2}
\end{displaymath}
电荷密度$n(q)$在倒空间产生的电势表示为
\begin{displaymath}
		V(q)=\dfrac{4\pi~n(q)~e^2}{q^2}
\end{displaymath}

\subsection{实空间变量$\tilde V_{\mathrm{loc}}^{\mathrm{ion}}(r)$与倒变量变量$V_{loc}^{ion}(G)$的对应}
在冻芯近似下,实空间局域离子赝势可以分解为三部分贡献
\begin{equation}
	\begin{aligned}
		\tilde{V}^{\mathrm{ion}}_{\mathrm{loc}}(r)=&\underbrace{\boxed{Z_{\mathrm{core}}e/r-\tilde{V}_{\mathrm{H}}[n_c]}}+\textcolor{red}{\dfrac{Z_{\mathrm{val}}^{\ast}~e}{r}}\\
		&\mbox{冻芯原子的电荷抵消}
	\end{aligned}
	\label{eq:POTAEC-real-devided}
\end{equation}
其中$Z_{\mathrm{val}^{\ast}}$表示被屏蔽的核电荷(价)部分(实空间):
\begin{displaymath}
	Z_{\mathrm{val}}^{\ast}=Z_{\mathrm{val}}\times\mathrm{ERFC}(r/\mathrm{\AA})
\end{displaymath}

根据代码\textrm{setlocalpp.F}中,子程序\textrm{POTTORHO}的描述,冻芯原子的电荷抵消部分(实空间常数势)与倒空间常数势的\textrm{Fourier}变换关系
利用变换关系
\begin{displaymath}
	\delta~q=2\pi\int_0^{\infty}\sin(qr)\mathrm{d}r
\end{displaymath}
实空间和倒空间的\textrm{Fourier}变换的关系为
\begin{equation}
	\begin{aligned}
		F(q)\cdot q=4\pi\int\sin(qr)f(r)r\mathrm{d}r\\
		G(r)\cdot r=\dfrac1{2\pi^2}\int\sin(qr)g(q)q\mathrm{d}q
	\end{aligned}
	\label{eq:Fourier-Transform}
\end{equation}
因此倒空间局域赝势的表示分为两部分
\begin{equation}
	\mathrm{PSP}:~\tilde{V}^{\mathrm{ion}}_{\mathrm{loc}}(q)=\boxed{\mbox{冻芯原子的电荷抵消:~倒空间表示}}+\textcolor{red}{\dfrac{4\pi~Z_{\mathrm{val}}^{\ast}~e^2}{q^2}}
	\label{eq:POTAEC-reciprocal-devided}
\end{equation}
其中$Z_{\mathrm{val}^{\ast}}$表示被屏蔽的核电荷(价)部分(倒空间):
\begin{displaymath}
	Z_{\mathrm{val}}^{\ast}=Z_{\mathrm{val}}\times\bigg[1-\mathrm{e}^{-\dfrac{}4\big(\dfrac{q}{\mathrm{\AA}^{-1}}\big)^2}\bigg]
\end{displaymath}

子程序{\textrm{FOURPOT\_TO\_Q\_CHECK}}完成局域离子赝势$\tilde{V}^{\mathrm{ion}}_{\mathrm{loc}}(r)$向$\tilde{V}^{\mathrm{ion}}_{\mathrm{loc}}(q)$的变换

实际实施\textrm{FFT}变换的部分是(\textcolor{magenta}{这是赝势的核心本质!!})
\begin{displaymath}
	\boxed{\mbox{冻芯原子电荷抵消:~正空间表示}}\stackrel{\mathrm{FFT}}{\Longrightarrow}\boxed{\mbox{冻芯原子电荷抵消:~倒空间表示}}
\end{displaymath}
注意:~\textrm{VASP}程序中对$\tilde{V}^{\mathrm{ion}}_{\mathrm{loc}}(r)$\textcolor{red}{取负值},$\boxed{\mbox{冻芯原子电荷抵消:~正空间表示}}$的表达式的正确书写

\section{倒空间的赝电荷表示}
\subsection{赝芯电荷的倒空间表示}
\begin{displaymath}
	\mathrm{PSPCOR}:~\tilde{n}_c(q) \stackrel{\mathrm{FFT}}{\Longleftarrow}\dfrac{\boxed{\mathrm{RHOPS}}}{r^2}\times\dfrac1{\mathrm{SCALE}}
\end{displaymath}
上式中\textrm{SCALE}表示$2\sqrt{\pi}$是因为\textrm{VASP}实现中对$l=0$的值表示时
\begin{displaymath}
	\mathrm{RHOPS}:~\tilde{n}_c(r) = 2\sqrt{\pi}\times\tilde{n}(r)\cdot r^2
\end{displaymath}
\subsection{赝价电荷的倒空间表示}
与芯电荷类似,赝价电子密度的倒空间表示
\begin{displaymath}
	\mathrm{PSPRHO}:~\tilde{n}_v(q) \stackrel{\mathrm{FFT}}{\Longleftarrow}\boxed{\mathrm{RHOPS00}}\times\dfrac1{\mathrm{SCALE}}
\end{displaymath}
上式中\textrm{SCALE}表示$2\sqrt{\pi}$是因为\textrm{VASP}实现中对$l=0$的值表示时
\begin{displaymath}
	\mathrm{RHOPS00}:~\tilde{n}_v(r) = 2\sqrt{\pi}\times\tilde{n}(r)
\end{displaymath}

\subsection{\textcolor{magenta}{由赝芯电荷表示可移植局域赝势}}
考虑可移植局域赝势的倒空间表示的分解(\textcolor{blue}{电荷在倒空间无法直接作芯-价电子分割})
\begin{itemize}
	\item 冻芯原子的电荷抵消:~倒空间表示$V(q)=0$
	\item 被屏蔽的核电荷(价)部分
\begin{equation}
	\begin{aligned}
		\tilde{V}_{\mathrm{eff}}^{\mathrm{ion}(1)}(q) =& A\dfrac{\sin(qr_{loc})}r\\
		\Rightarrow & \dfrac{4\pi~Z_{\mathrm{val}}^{\ast}~e^2}{q^2} = \dfrac{4\pi~Z_{\mathrm{val}}~e^2}{q^2}\times\bigg[1-\mathrm{e}^{-\dfrac{}4\big(\dfrac{q}{\mathrm{\AA}^{-1}}\big)^2}\bigg] 
	\end{aligned}
	\label{eq:POT_G_core-EFF}
\end{equation}
构造赝势时$\tilde{V}_{\mathrm{eff}}^{\mathrm{ion}}(q)$,需要指定衔接半径$q_{\mathrm{ion}}$
	\item 芯电荷在倒空间的贡献
		\begin{equation}
			\tilde{V}_{\mathrm{eff}}^{\mathrm{ion}(2)}(q) = -\dfrac{4\pi\tilde{n}_c(q)e^2}{q^2}\times\bigg[1-\mathrm{e}^{-\dfrac{}4\big(\dfrac{q}{\mathrm{\AA}^{-1}}\big)^2}\bigg] 
			\label{eq:POT_G_core}
		\end{equation}
\end{itemize}
在倒空间,可直接表示局域可移植赝势
\begin{displaymath}
	\tilde{V}_{\mathrm{eff}}^{\mathrm{ion}}(q) = \tilde{V}_{\mathrm{eff}}^{\mathrm{ion}(1)}(q) - \dfrac{4\pi\tilde{n}_c(q)e^2}{q^2}\times\bigg[1-\mathrm{e}^{-\dfrac{}4\big(\dfrac{q}{\mathrm{\AA}^{-1}}\big)^2}\bigg] 
\end{displaymath}

\section{投影函数的构造}

\section{离子修正项$D_{ij}^{\mathrm{ion}}$的计算}
\begin{equation}
	\begin{aligned}
		\mathrm{D}_{ij}^{\mathrm{ion}}=&\langle\phi_i|\mathbf{T}+V_{\mathrm{eff}}|\phi_j\rangle - \langle\tilde{\phi}_i|\mathbf{T}+\tilde{V}_{\mathrm{eff}}|\tilde{\phi}_j\rangle\\
		-&\int\hat{Q}_{ij}^{00}(r)\tilde{V}_{\mathrm{eff}}(r)\mathrm{d}r
	\end{aligned}
	\label{eq:DION}
\end{equation}
这里的各项可分解为
\begin{itemize}
	\item 算符$\mathbf{T}$的定义为
\begin{equation}
	\mathbf{T}|\psi_i\rangle=\bigg[-\nabla^2-\dfrac{l(l+2)}2\bigg]|\psi_i\rangle
	\label{eq:Kinetic}
\end{equation}
\item 贡献$\mathrm{D}_{ij}$的定义为
	\begin{equation}
		\mathrm{D}_{ij} = \langle\phi_i|\mathbf{T}+V_{\mathrm{eff}}|\phi_j\rangle - \langle\tilde{\phi}_i|\mathbf{T}+\tilde{V}_{\mathrm{eff}}|\tilde{\phi}_j\rangle
		\label{eq:DIJ}
	\end{equation}
\item \textrm{PAW}电量差
	\begin{equation}
		\hat{Q}_{\mathrm{PAW}}=\int(\phi_i\phi_j-\tilde{\phi}_i\tilde{\phi}_j)r^Lr^2\mathrm{d}r
		\label{eq:Q-PAW}
	\end{equation}
\item 去屏蔽项贡献
	\begin{equation}
		-\int\hat{Q}^{00}_{ij}(r)\tilde{V}_{\mathrm{eff}}(r)\mathrm{d}r
		\label{eq:unsreened}
	\end{equation}
\end{itemize}

%\section{正文章节}
%参考文献的引用方式1\upcite{QCQC_2014}
%-------------------The Figure Of The Paper------------------
%\begin{figure}[h!]
%\centering
%\includegraphics[height=3.35in,width=2.85in,viewport=0 0 400 475,clip]{PbTe_Band_SO.eps}
%\hspace{0.5in}
%\includegraphics[height=3.35in,width=2.85in,viewport=0 0 400 475,clip]{EuTe_Band_SO.eps}
%\caption{\small Band Structure of PbTe (a) and EuTe (b).}%(与文献\cite{EPJB33-47_2003}图1对比)
%\label{Pb:EuTe-Band_struct}
%\end{figure}

%-------------------The Equation Of The Paper-----------------
%\begin{equation}
%\varepsilon_1(\omega)=1+\frac2{\pi}\mathscr P\int_0^{+\infty}\frac{\omega'\varepsilon_2(\omega')}{\omega'^2-\omega^2}d\omega'
%\label{eq:magno-1}
%\end{equation}



%\section{正文章节}
%参考文献的引用方式1\upcite{QCQC_2014}
%-------------------The Figure Of The Paper------------------
%\begin{figure}[h!]
%\centering
%\includegraphics[height=3.35in,width=2.85in,viewport=0 0 400 475,clip]{PbTe_Band_SO.eps}
%\hspace{0.5in}
%\includegraphics[height=3.35in,width=2.85in,viewport=0 0 400 475,clip]{EuTe_Band_SO.eps}
%\caption{\small Band Structure of PbTe (a) and EuTe (b).}%(与文献\cite{EPJB33-47_2003}图1对比)
%\label{Pb:EuTe-Band_struct}
%\end{figure}

%-------------------The Equation Of The Paper-----------------
%\begin{equation}
%\varepsilon_1(\omega)=1+\frac2{\pi}\mathscr P\int_0^{+\infty}\frac{\omega'\varepsilon_2(\omega')}{\omega'^2-\omega^2}d\omega'
%\label{eq:magno-1}
%\end{equation}

%\begin{equation} 
%\begin{split}
%\varepsilon_2(\omega)&=\frac{e^2}{2\pi m^2\omega^2}\sum_{c,v}\int_{BZ}d{\vec k}\left|\vec e\cdot\vec M_{cv}(\vec k)\right|^2\delta [E_{cv}(\vec k)-\hbar\omega] \\
% &= \frac{e^2}{2\pi m^2\omega^2}\sum_{c,v}\int_{E_{cv}(\vec k=\hbar\omega)}\left|\vec e\cdot\vec M_{cv}(\vec k)\right|^2\dfrac{dS}{\nabla_{\vec k}E_{cv}(\vec k)}
% \end{split}
%\label{eq:magno-2}
%\end{equation}

%-------------------The Table Of The Paper----------------------
%\begin{table}[!h]
%\tabcolsep 0pt \vspace*{-12pt}
%%\caption{The representative $\vec k$ points contributing to $\sigma_2^{xy}$ of interband transition in EuTe around 2.5 eV.}
%\label{Table-EuTe_Sigma}
%\begin{minipage}{\textwidth}
%%\begin{center}
%\centering
%\def\temptablewidth{0.84\textwidth}
%\rule{\temptablewidth}{1pt}
%\begin{tabular*} {\temptablewidth}{|@{\extracolsep{\fill}}c|@{\extracolsep{\fill}}c|@{\extracolsep{\fill}}l|}

%-------------------------------------------------------------------------------------------------------------------------
%&Peak (eV)  & {$\vec k$}-point            &Band{$_v$} to Band{$_c$}  &Transition Orbital
%Components\footnote{波函数主要成分后的括号中,$5s$、$5p$和$5p$、$4f$、$5d$分别指碲和铕的原子轨道。} &Gap (eV)   \\ \hline
%-------------------------------------------------------------------------------------------------------------------------
%&2.35       &(0,0,0)         &33$\rightarrow$34    &$4f$(31.58)$5p$(38.69)$\rightarrow$$5p$      &2.142   \\% \cline{3-7}
%&       &(0,0,0)         &33$\rightarrow$34    &$4f$(31.58)$5p$(38.69)$\rightarrow$$5p$      &2.142   \\% \cline{3-7}
%-------------------------------------------------------------------------------------------------------------------------
%\end{tabular*}
%\rule{\temptablewidth}{1pt}
%\end{minipage}{\textwidth}
%\end{table}

%-------------------The Long Table Of The Paper--------------------
%\begin{small}
%%\begin{minipage}{\textwidth}
%%\begin{longtable}[l]{|c|c|cc|c|c|} %[c]指定长表格对齐方式
%\begin{longtable}[c]{|c|c|p{1.9cm}p{4.6cm}|c|c|}
%\caption{Assignment for the peaks of EuB$_6$}
%\label{tab:EuB6-1}\\ %\\长表格的caption中换行不可少
%\hline
%%
%--------------------------------------------------------------------------------------------------------------------------------
%\multicolumn{2}{|c|}{\bfseries$\sigma_1(\omega)$谱峰}&\multicolumn{4}{c|}{\bfseries部分重要能带间电子跃迁\footnotemark}\\ \hline
%\endfirsthead
%--------------------------------------------------------------------------------------------------------------------------------
%%
%\multicolumn{6}{r}{\it 续表}\\
%\hline
%--------------------------------------------------------------------------------------------------------------------------------
%标记 &峰位(eV) &\multicolumn{2}{c|}{有关电子跃迁} &gap(eV)  &\multicolumn{1}{c|}{经验指认} \\ \hline
%\endhead
%--------------------------------------------------------------------------------------------------------------------------------
%%
%\multicolumn{6}{r}{\it 续下页}\\
%\endfoot
%\hline
%--------------------------------------------------------------------------------------------------------------------------------
%%
%%\hlinewd{0.5$p$t}
%\endlastfoot
%--------------------------------------------------------------------------------------------------------------------------------
%%
%% Stuff from here to \endlastfoot goes at bottom of last page.
%%
%--------------------------------------------------------------------------------------------------------------------------------
%标记 &峰位(eV)\footnotetext{见正文说明。} &\multicolumn{2}{c|}{有关电子跃迁\footnotemark} &gap(eV) &\multicolumn{1}{c|}{经验指认\upcite{PRB46-12196_1992}}\\ \hline
%--------------------------------------------------------------------------------------------------------------------------------
%
%     &0.07 &\multicolumn{2}{c|}{电子群体激发$\uparrow$} &--- &电子群\\ \cline{2-5}
%\raisebox{2.3ex}[0pt]{$\omega_f$} &0.1 &\multicolumn{2}{c|}{电子群体激发$\downarrow$} &--- &体激发\\ \hline
%--------------------------------------------------------------------------------------------------------------------------------
%
%     &1.50 &\raisebox{-2ex}[0pt][0pt]{20-22(0,1,4)} &2$p$(10.4)4$f$(74.9)$\rightarrow$ &\raisebox{-2ex}[0pt][0pt]{1.47} &\\%\cline{3-5}
%     &1.50$^\ast$ & &2$p$(17.5)5$d_{\mathrm E}$(14.0)$\uparrow$ & &4$f$$\rightarrow$5$d_{\mathrm E}$\\ \cline{3-5}
%     \raisebox{2.3ex}[0pt][0pt]{$a$} &(1.0$^\dagger$) &\raisebox{-2ex}[0pt][0pt]{20-22(1,2,6)} &\raisebox{-2ex}[0pt][0pt]{4$f$(89.9)$\rightarrow$2$p$(18.7)5$d_{\mathrm E}$(13.9)$\uparrow$}\footnotetext{波函数主要成分后的括号中,2$s$、2$p$和5$p$、4$f$、5$d$、6$s$分别指硼和铕的原子轨道;5$d_{\mathrm E}$、5$d_{\mathrm T}$分别指铕的(5$d_{z^2}$,5$d_{x^2-y^2}$和5$d_{xy}$,5$d_{xz}$,5$d_{yz}$)轨道,5$d_{\mathrm{ET}}$(或5$d_{\mathrm{TE}}$)则指5个5$d$轨道成分都有,成分大的用脚标的第一个字母标示;2$ps$(或2$sp$)表示同时含有硼2$s$、2$p$轨道成分,成分大的用第一个字母标示。$\uparrow$和$\downarrow$分别标示$\alpha$和$\beta$自旋电子跃迁。} &\raisebox{-2ex}[0pt][0pt]{1.56} &激子跃迁。 \\%\cline{3-5}
%     &(1.3$^\dagger$) & & & &\\ \hline
%--------------------------------------------------------------------------------------------------------------------------------

%     & &\raisebox{-2ex}[0pt][0pt]{19-22(0,0,1)} &2$p$(37.6)5$d_{\mathrm T}$(4.5)4$f$(6.7)$\rightarrow$ & & \\\nopagebreak %\cline{3-5}
%     & & &2$p$(24.2)5$d_{\mathrm E}$(10.8)4$f$(5.1)$\uparrow$ &\raisebox{2ex}[0pt][0pt]{2.78} &a、b、c峰可能 \\ \cline{3-5}
%     & &\raisebox{-2ex}[0pt][0pt]{20-29(0,1,1)} &2$p$(35.7)5$d_{\mathrm T}$(4.8)4$f$(10.0)$\rightarrow$ & &包含有复杂的\\ \nopagebreak%\cline{3-5}
%     &2.90 & &2$p$(23.2)5$d_{\mathrm E}$(13.2)4$f$(3.8)$\uparrow$ &\raisebox{2ex}[0pt][0pt]{2.92} &强激子峰。$^{\ast\ast}$\\ \cline{3-5}
%$b$  &2.90$^\ast$ &\raisebox{-2ex}[0pt][0pt]{19-22(0,1,1)} &2$p$(33.9)4$f$(15.5)$\rightarrow$ & &B2$s$-2$p$的价带 \\ \nopagebreak%\cline{3-5}
%     &3.0 & &2$p$(23.2)5$d_{\mathrm E}$(13.2)4$f$(4.8)$\uparrow$ &\raisebox{2ex}[0pt][0pt]{2.94} &顶$\rightarrow$B2$s$-2$p$导\\ \cline{3-5}
%     & &12-15(0,1,2) &2$p$(39.3)$\rightarrow$2$p$(25.2)5$d_{\mathrm E}$(8.6)$\downarrow$ &2.83 &带底跃迁。\\ \cline{3-5}
%     & &14-15(1,1,1) &2$p$(42.5)$\rightarrow$2$p$(29.1)5$d_{\mathrm E}$(7.0)$\downarrow$ &2.96 & \\\cline{3-5}
%     & &13-15(0,1,1) &2$p$(40.4)$\rightarrow$2$p$(28.9)5$d_{\mathrm E}$(6.6)$\downarrow$ &2.98 & \\ \hline
%--------------------------------------------------------------------------------------------------------------------------------
%%\hline
%%\hlinewd{0.5$p$t}
%\end{longtable}
%%\end{minipage}{\textwidth}
%%\setlength{\unitlength}{1cm}
%%\begin{picture}(0.5,2.0)
%%  \put(-0.02,1.93){$^{1)}$}
%%  \put(-0.02,1.43){$^{2)}$}
%%\put(0.25,1.0){\parbox[h]{14.2cm}{\small{\\}}
%%\put(-0.25,2.3){\line(1,0){15}}
%%\end{picture}
%\end{small}

%------------------------------------直-接-插-入-文-件--------------------------------------------------------------------------------------
%\textcolor{red}{\textbf{直接插入文件}}:\verbatiminput{/home/jun_jiang/Documents/Latex_art_beamer/Daily_WORKS/Report-2020_model.tex} %为保险:~选用文件名绝对路径
%\textcolor{red}{\textbf{备忘录}}:\verbatiminput{/home/jun_jiang/Documents/备忘录.txt}
%---------------------------------------------------------------------------------------------------------------------------------------------%

%-------------------------------------------------------------------------Thanks------------------------------------------------------------------------------------------------
%\newpage %%
%\newpage %%
%\thispagestyle{fancy}   % 首页插入页眉页脚 
\section{致谢}
致谢内容
%-----------------------------------------------------------------------------------------------------------------------------------------------------------------------%

