%%%%%%%%%%%%%%%%%%%%%%%%%%%%% 用 authblk 包 支持作者和E-mail %%%%%%%%%%%%%%%%%%%%%%%%%%%%%%%%%
%\title{More than one Author with different Affiliations}				     %
\title{\hei{\textbf{VASP}的赝势生成}}
%\author[a]{Author/作者 A}   %
%\author[a]{Author B}									     %
%\author[a]{Author/通讯作者 C \thanks{Corresponding author: cores-email@mail.com}}     %
%\author[b]{Author/作者 D}									     %
%\author[b]{Author E}									     %
%\affil[a]{Department of Computer Science, \LaTeX\ University}				     %
%\affil[a]{作者单位-1 \authorcr 地址}    %\authorcr表示换行
%\affil[b]{Department of Mechanical Engineering, \LaTeX\ University}			     %
%\affil[b]{作者单位-2}			     %
\author[]{姜骏\thanks{E-mal:~czjiangjun@gmail.com}}   %
%\affil[]{天宁问道斋 \authorcr 北京市西城区}    %\authorcr表示换行
											     %
%%% 使用 \thanks 定义通讯作者								     %
%%\affil命令后的{}中的内容,如果觉得需要换行的话,换行命令是\authorcr(不是\\)。
%%Email中可以吧相同邮箱的人@前面的内容写在一个{}里,用逗号隔开。注意{和}前面要加\。例如:
%%\affil[*]{单位1, \authorcr Email: \{zuozhe1, zuozhe2\}@yahoo.com, zuozhe3@sina.com}
											     %
\renewcommand*{\Authfont}{\small\rm} % 修改作者的字体与大小				     %
\renewcommand*{\Affilfont}{\small\it} % 修改机构名称的字体与大小			     %
\renewcommand\Authands{ and } % 去掉 and 前的逗号					     %
\renewcommand\Authands{ , } % 将 and 换成逗号					     %
\date{} % 去掉日期									     %
%\date{2020-12-30}									     %
%%%%%%%%%%%%%%%%%%%%%%%%%%%%%%%%%%%%%%%%%%%%%%%%%%%%%%%%%%%%%%%%%%%%%%%%%%%%%%%%%%%%%%%%%%%%%%

%----------------------The Authors and the address of The Paper--------------------------------%
%\author{
%作者:
%\small
%Author1, Author2, Author3\footnote{Communication author's E-mail} \\    %Authors' Names	       %
%\small
%(The Address,City Post code)						%Address	       %
%}
%\affil[$\dagger$]{清华大学~材料加工研究所~A213}
%\affil{清华大学~材料加工研究所~A213}
%\date{}					%if necessary					       %
%----------------------------------------------------------------------------------------------%
%%%%%%%%%%%%%%%%%%%%%%%%%%%%%%%%%%%%%%%%%%%%%%%%%%%%%%%%%%%%%%%%%%%%%%%%%%%%%%%%%%%%%%%%%%%%%%%%%%%%%%%%%%%%%%%%%%%%%
\maketitle
%\thispagestyle{fancy}   % 首页插入页眉页脚 

%-------------------------------------------------------------------------------The Abstract and the keywords of The Paper----------------------------------------------------------------------------%
%\begin{abstract}
%The content of the abstract
%\end{abstract}

%\keywords{Keyword1; Keyword2; Keyword3}

%-------------------------------------------------------------------------------The Content of The Paper----------------------------------------------------------------------------------------------%
%\tableofcontents %% 制作目录(目录是根据标题自动生成的)
%-----------------------------------------------------------------------------------------------------------------------------------------------------------------------------------------------------%

%\newpage	        % 每个新的/newpage 即可有新的\thispagestyle 引领      %
%\thispagestyle{fancy}   % 插入页眉页脚                                        %
%----------------------------------------------------------------------------------------The Body Of The Paper----------------------------------------------------------------------------------------%
%Introduction
\section{参数关系}
\begin{itemize}
	\item \textrm{AUTOA}:~ \textrm{AUTOA}=0.529177249
	\item \textrm{RYTOEV}:~ \textrm{RYTOEV} = 13.605826
	\item \textrm{FELECT}: $V(r)\cdot r$(对应单位)
		\begin{displaymath}
			2\times\textrm{AUTOA}\times\textrm{RYTOEV}
		\end{displaymath}
		表示$2e^2$的单位换算关系
\end{itemize}
%\section{Introduction}
\section{变量表达式}
原子价电子波函数积分
\begin{equation}
	\begin{aligned}
		\int\psi_{lm}(\vec r)|\psi_{l^{\prime}m^{\prime}}(\vec r)\mathrm{d}\vec r=&\underbrace{\int\phi_l(r)\phi_{l^{\prime}}(r)\mathrm{d}r}&\underbrace{\int Y_{lm}(\theta,\varphi)Y_{l^{\prime}m^{\prime}}(\theta,\varphi)\mathrm{d}\Omega} \\%=4\pi\int\rho_v\mathrm{d}r
		&\mbox{径向部分积分} &\mbox{角度部分积分}
	\end{aligned}
	\label{eq:wave-integral}
\end{equation}
这里$\int\mathrm{d}\Omega$表示角度部分的积分微元(单位立体角)的积分$\int\sin\theta\mathrm{d}\theta\mathrm{d}\varphi$
%原子价电子径向波函数$\phi_i(r)$与价电子密度:
%\begin{equation}
%	\int|\phi_i(r)|^2\mathrm{d}r =4\pi\int\rho_v\mathrm{d}r
%	\label{eq:wave-charge}
%\end{equation}
对于角度部分,当前的被积函数$Y_{lm}(\theta,\varphi)Y_{l^{\prime}m^{\prime}}(\theta,\varphi)$不包含总角动量信息,称为\textcolor{magenta}{非耦合表象},可记作$|l,m;l^{\prime},m^{\prime}\rangle$。如果考虑角动量耦合产生总的角动量的贡献,称为\textcolor{magenta}{耦合表象},记作$|l,l^{\prime};L,M\rangle$。\footnote{两种表象都包含四个量子数,这里不能确定六个量子数。但因为有$L_z=l_z+l_z^{\prime}$,因此$M=m+m^{\prime}$在两种表象下都成立}

\subsection{\rm{CG}系数}
耦合表象$|l,l^{\prime};L,M\rangle$可以用非耦合表象$|l,m;l^{\prime},m^{\prime}\rangle$的线性组合表示
\begin{equation}
	|l,l^{\prime};L,M\rangle=\sum_{mm^{\prime}}C_{lm,l^{\prime}m^{\prime}}^{LM}|l,m;l^{\prime},m^{\prime}\rangle \qquad(m+m^{\prime}=M)
	\label{eq:couple_vs_uncouple}
\end{equation}
这里系数$C_{lm,l^{\prime}m^{\prime}}^{LM}$称为\textrm{Clebsch-Gordan~(CG)}系数。

\textrm{CG}系数与\textrm{Wignerr~3-j}符号有密切关系
\begin{equation}
	\begin{pmatrix}l &l^{\prime} &L\\m &m^{\prime} &M\end{pmatrix}\equiv\dfrac{(-1)^{l-l^{\prime}-M}}{2L+1}\langle l,m;l^{\prime},m^{\prime}|L,-M\rangle
	\label{eq:CG-vs-Wigner_3-j}
\end{equation}

\subsection{\rm{CG}系数与球谐函数}
\textrm{CG}系数常用于计算球谐函数乘积的积分
\begin{equation}
	\begin{aligned}
		&\int Y_{lm}^{\ast}(\theta,\varphi)Y_{l^{\prime}m^{\prime}}(\theta,\varphi)^{\ast}Y_{LM}(\theta,\varphi)\mathrm{d}\Omega\\
		=&(-1)^M\sqrt{\dfrac{(2l+1)(2l^{\prime}+1)}{4\pi(2L+1)}}\begin{bmatrix}l &l^{\prime} &L\\0 &0 &0\end{bmatrix}\begin{bmatrix}l &l^{\prime} &L\\m &m^{\prime} &-M\end{bmatrix}\\
			=&\sqrt{\dfrac{(2l+1)(2l^{\prime}+1)(2L+1)}{4\pi}}\langle l0l^{\prime}0|L0\rangle\langle lml^{\prime}m^{\prime}|LM\rangle\\
			=&\sqrt{\dfrac{(2l+1)(2l^{\prime}+1)(2L+1)}{4\pi}}\begin{pmatrix}l &l^{\prime} &L\\ 0 &0 &0\end{pmatrix}\begin{pmatrix}l &l^{\prime} &L\\m &m^{\prime} &M\end{pmatrix}
	\end{aligned}
	\label{CG-spherical_harmonics}
\end{equation}
因此典型的三个球谐函数乘积的积分可以表示为两个\textrm{CG}系数或\textrm{Wigner~3-j}符号相乘。

由式\eqref{CG-spherical_harmonics}和球谐函数的正交归一性
\begin{equation}
	\int Y_{l^{\prime}m^{\prime}}^{\ast}(\theta,\varphi)Y_{lm}(\theta,\varphi)\mathrm{d}\Omega=\delta_{ll^{\prime}}\delta_{mm^{\prime}}
	\label{SH-orthonormality}
\end{equation}
可得出结论:~两个球谐函数的乘积用另一个球谐函数展开时,展开系数可用\textrm{CG}系数表示
\begin{equation}
	Y_{lm}(\theta,\varphi)Y_{l^{\prime}m^{\prime}}(\theta,\varphi)=\sum_{LM}\sqrt{\dfrac{(2l+1)(2l^{\prime}+1)}{4\pi(2L+1)}}\langle l0l^{\prime}0|L0\rangle\langle lml^{\prime}m^{\prime}|LM\rangle Y_{LM}(\theta,\varphi)
	\label{eq:CG-spherical_harmonics_expand}
\end{equation}

\subsection{电荷密度的计算}
应用\textrm{CG}系数,可以将价电子密度的\textcolor{red}{分波表示}
\begin{equation}
	\begin{aligned}
		\rho^{LM}(r)=\sum_{ll^{\prime}}&\underbrace{\sum_{mm^{\prime}}n_{lm,l^{\prime}m^{\prime}}C_{lm,l^{\prime}m^{\prime}}^{LM}}\phi_l(r)\phi_{l^{\prime}}(r)\\
		&\mbox{变量\textrm{RHOLM}的计算}
	\end{aligned}
	\label{eq:rho_LM}
\end{equation}
这里$\rho^{LM}(r)$对应变量\textrm{RHO(:,:,1)},用$C_{lm,l^{\prime}m^{\prime}}^{LM}$表示\textrm{CG}系数。

体系的密度$\rho$
\begin{equation}
	\rho(r)=\sum_{LM}\rho^{LM}(r)
	\label{eq:density}
\end{equation}

变量\textrm{POTAE}遵照表达式
\begin{equation}
	\mathrm{POTAE} = V_{\mathrm{H}}[n_v]+V_{\mathrm{XC}}[n_v+n_c]
	\label{eq:POTAE}
\end{equation}
计算。

为构造局域有效势,文献建议,\textcolor{red}{由全电子势\textrm{POTAE}“截断”(取$L=0$)获得}

理论上,有效势$V_{\mathrm{eff}}(r)$的计算表达式为
\begin{equation}
	\begin{aligned}
		V_{\mathrm{eff}}(r) =& A\dfrac{\sin(q_{loc}r)}r\\
		\Rightarrow & V_{\mathrm{H}}[n_v+n_{Zc}]+V_{\mathrm{XC}}[\rho_v+\rho_c]
	\end{aligned}
	\label{eq:POT_EFF}
\end{equation}

变量\textrm{POTPS}严格遵照表达式
\begin{equation}
	\mathrm{POTPS} = V_{\mathrm{H}}[\tilde{n}_v+\hat{n}_v]+V_{\mathrm{XC}}[\tilde{n}_v+\hat{n}_v+\tilde{n}_c]
	\label{eq:POTPS}
\end{equation}
计算。

变量\textrm{POTPSC},即去屏蔽的可移植赝势
\begin{equation}
	\begin{aligned}
		\mathrm{POTPSC} =& V_{\mathrm{eff}}(r)-\mathrm{POTPS}\\
		=&V_{\mathrm{eff}}(r)-(V_{\mathrm{H}}[\tilde{n}_v+\hat{n}_v]+V_{\mathrm{XC}}[\tilde{n}_v+\hat{n}_v+\tilde{n}_c])
	\end{aligned}
	\label{eq:POTPSC}
\end{equation}

\subsection{实空间变量$\tilde V_{\mathrm{eff}}^{\mathrm{ion}}(r)$与倒变量变量\textrm{PSP}}
在冻芯近似下,离子球形势表示为
\begin{equation}
	\mathrm{POTAEC} = V_{\mathrm{H}}[n_{Zc}](r)= V_{\mathrm{H}}[n_c](r)+V_{\mathrm{H}}[Z](r) 
	\label{eq:POTAEC}
\end{equation}
根据代码\textrm{setlocalpp.F}中,子程序\textrm{POTTORHO}的描述,
%\begin{equation}
%		\mathrm{POTPSC} = V_{\mathrm{H}}[\tilde{n}_c](r)+V_{\mathrm{H}}[Z](r)
%	\label{eq:POTPSC}
%\end{equation}
代码中$\tilde V_{\mathrm{eff}}^{\mathrm{ion}}(r)$选用\textrm{POTAEC}的0阶球函数的表示
\begin{equation}
	\begin{aligned}
		\tilde V_{\mathrm{eff}}^{\mathrm{ion}}(r)=&A\dfrac{\sin(q_{loc}r)}r\\
		\Rightarrow & V_{\mathrm{H}}[n_v+n_{Zc}]+V_{\mathrm{XC}}[\rho_v+\rho_c]
	\end{aligned}
	\label{eq:POT_ION}
\end{equation}
定义核电荷中价电荷部分贡献势的误差函数
\begin{equation}
	\Delta V_{\mathrm{H}}[\tilde{Z}_v](r)=\dfrac{Z_v\mathrm{ERF}(r/\alpha)}r\times2e^2
	\label{eq:Delta_Z_valence}
\end{equation}
由此确定,\textrm{FFT}变换中与实空间对应部分的表达式是
\begin{displaymath}
	\mathrm{POTPSC} \textcolor{red}{-} \Delta V_{\mathrm{H}}[Z_v](r) = V_{\mathrm{H}}[\tilde{n}_c](r)+V_{\mathrm{H}}[Z](r) \textcolor{red}{-} \Delta V_{\mathrm{H}}[Z_v](r)
\end{displaymath}

在倒空间,\textrm{FFT}变换部分表示为
\begin{displaymath}
	\mathrm{PSP} \textcolor{blue}{+} V_{\mathrm{H}}[Z_v](\vec{q}) \textcolor{blue}{-} \Delta V_{\mathrm{H}}^{\prime}[\tilde{Z}_v](\vec q)
\end{displaymath}
其中
\begin{displaymath}
	\begin{aligned}
		V_{\mathrm{H}}[Z_v](\vec q)=&\dfrac{4\pi Z_v}{\vec q^2}\times2e^2\\
		\Delta V_{\mathrm{H}}^{\prime}[\tilde{Z}_v](\vec q)=&\dfrac{4\pi Z_v}{\vec q^2}\mathrm{e}^{-(\vec q/2\alpha)^2}\times2e^2
	\end{aligned}
\end{displaymath}

\textbf{由FFT变换对应于$V_{\mathrm{H}}[\tilde{n}_c]+V_{\mathrm{H}}[\tilde{Z}]$}(\textcolor{red}{这是赝势的核心本质!!}),可推定\textrm{PSP}相当于倒空间表示的
\begin{equation}
	\begin{aligned}
		\mathrm{PSP}&=V_{\mathrm{H}}[\tilde{n}_c](\vec q)+V_{\mathrm{H}}[Z](q)-V_{\mathrm{H}}[Z_v](q)-\Delta V_{\mathrm{H}}[Z_v](q)+\Delta V_{\mathrm{H}}^{\prime}[Z_v](q)\\
		&=V_{\mathrm{H}}[\tilde{n}_c](\vec q)+V_{\mathrm{H}}[Z_c](\vec q)-\Delta V_{\mathrm{H}}[Z_v](q)+\Delta V_{\mathrm{H}}^{\prime}[Z_v](q)
	\end{aligned}
	\label{eq:PSP}
\end{equation}
根据\textrm{Ewald}求和规则,
\begin{displaymath}
	\Delta V_{\mathrm{H}}[Z_v](q)-\Delta V_{\mathrm{H}}^{\prime}[Z_v](q)
\end{displaymath}
抵消。\\
因此\textrm{PSP}对应的势是
\begin{displaymath}
	\mathrm{PSP} = V_{\mathrm{H}}[\tilde{n}_c](\vec q)+V_{\mathrm{H}}[Z_c](\vec q)
\end{displaymath}
这是真正可以平移的赝势。
%\section{正文章节}
%参考文献的引用方式1\upcite{QCQC_2014}
%-------------------The Figure Of The Paper------------------
%\begin{figure}[h!]
%\centering
%\includegraphics[height=3.35in,width=2.85in,viewport=0 0 400 475,clip]{PbTe_Band_SO.eps}
%\hspace{0.5in}
%\includegraphics[height=3.35in,width=2.85in,viewport=0 0 400 475,clip]{EuTe_Band_SO.eps}
%\caption{\small Band Structure of PbTe (a) and EuTe (b).}%(与文献\cite{EPJB33-47_2003}图1对比)
%\label{Pb:EuTe-Band_struct}
%\end{figure}

%-------------------The Equation Of The Paper-----------------
%\begin{equation}
%\varepsilon_1(\omega)=1+\frac2{\pi}\mathscr P\int_0^{+\infty}\frac{\omega'\varepsilon_2(\omega')}{\omega'^2-\omega^2}d\omega'
%\label{eq:magno-1}
%\end{equation}



%\section{正文章节}
%参考文献的引用方式1\upcite{QCQC_2014}
%-------------------The Figure Of The Paper------------------
%\begin{figure}[h!]
%\centering
%\includegraphics[height=3.35in,width=2.85in,viewport=0 0 400 475,clip]{PbTe_Band_SO.eps}
%\hspace{0.5in}
%\includegraphics[height=3.35in,width=2.85in,viewport=0 0 400 475,clip]{EuTe_Band_SO.eps}
%\caption{\small Band Structure of PbTe (a) and EuTe (b).}%(与文献\cite{EPJB33-47_2003}图1对比)
%\label{Pb:EuTe-Band_struct}
%\end{figure}

%-------------------The Equation Of The Paper-----------------
%\begin{equation}
%\varepsilon_1(\omega)=1+\frac2{\pi}\mathscr P\int_0^{+\infty}\frac{\omega'\varepsilon_2(\omega')}{\omega'^2-\omega^2}d\omega'
%\label{eq:magno-1}
%\end{equation}

%\begin{equation} 
%\begin{split}
%\varepsilon_2(\omega)&=\frac{e^2}{2\pi m^2\omega^2}\sum_{c,v}\int_{BZ}d{\vec k}\left|\vec e\cdot\vec M_{cv}(\vec k)\right|^2\delta [E_{cv}(\vec k)-\hbar\omega] \\
% &= \frac{e^2}{2\pi m^2\omega^2}\sum_{c,v}\int_{E_{cv}(\vec k=\hbar\omega)}\left|\vec e\cdot\vec M_{cv}(\vec k)\right|^2\dfrac{dS}{\nabla_{\vec k}E_{cv}(\vec k)}
% \end{split}
%\label{eq:magno-2}
%\end{equation}

%-------------------The Table Of The Paper----------------------
%\begin{table}[!h]
%\tabcolsep 0pt \vspace*{-12pt}
%%\caption{The representative $\vec k$ points contributing to $\sigma_2^{xy}$ of interband transition in EuTe around 2.5 eV.}
%\label{Table-EuTe_Sigma}
%\begin{minipage}{\textwidth}
%%\begin{center}
%\centering
%\def\temptablewidth{0.84\textwidth}
%\rule{\temptablewidth}{1pt}
%\begin{tabular*} {\temptablewidth}{|@{\extracolsep{\fill}}c|@{\extracolsep{\fill}}c|@{\extracolsep{\fill}}l|}

%-------------------------------------------------------------------------------------------------------------------------
%&Peak (eV)  & {$\vec k$}-point            &Band{$_v$} to Band{$_c$}  &Transition Orbital
%Components\footnote{波函数主要成分后的括号中,$5s$、$5p$和$5p$、$4f$、$5d$分别指碲和铕的原子轨道。} &Gap (eV)   \\ \hline
%-------------------------------------------------------------------------------------------------------------------------
%&2.35       &(0,0,0)         &33$\rightarrow$34    &$4f$(31.58)$5p$(38.69)$\rightarrow$$5p$      &2.142   \\% \cline{3-7}
%&       &(0,0,0)         &33$\rightarrow$34    &$4f$(31.58)$5p$(38.69)$\rightarrow$$5p$      &2.142   \\% \cline{3-7}
%-------------------------------------------------------------------------------------------------------------------------
%\end{tabular*}
%\rule{\temptablewidth}{1pt}
%\end{minipage}{\textwidth}
%\end{table}

%-------------------The Long Table Of The Paper--------------------
%\begin{small}
%%\begin{minipage}{\textwidth}
%%\begin{longtable}[l]{|c|c|cc|c|c|} %[c]指定长表格对齐方式
%\begin{longtable}[c]{|c|c|p{1.9cm}p{4.6cm}|c|c|}
%\caption{Assignment for the peaks of EuB$_6$}
%\label{tab:EuB6-1}\\ %\\长表格的caption中换行不可少
%\hline
%%
%--------------------------------------------------------------------------------------------------------------------------------
%\multicolumn{2}{|c|}{\bfseries$\sigma_1(\omega)$谱峰}&\multicolumn{4}{c|}{\bfseries部分重要能带间电子跃迁\footnotemark}\\ \hline
%\endfirsthead
%--------------------------------------------------------------------------------------------------------------------------------
%%
%\multicolumn{6}{r}{\it 续表}\\
%\hline
%--------------------------------------------------------------------------------------------------------------------------------
%标记 &峰位(eV) &\multicolumn{2}{c|}{有关电子跃迁} &gap(eV)  &\multicolumn{1}{c|}{经验指认} \\ \hline
%\endhead
%--------------------------------------------------------------------------------------------------------------------------------
%%
%\multicolumn{6}{r}{\it 续下页}\\
%\endfoot
%\hline
%--------------------------------------------------------------------------------------------------------------------------------
%%
%%\hlinewd{0.5$p$t}
%\endlastfoot
%--------------------------------------------------------------------------------------------------------------------------------
%%
%% Stuff from here to \endlastfoot goes at bottom of last page.
%%
%--------------------------------------------------------------------------------------------------------------------------------
%标记 &峰位(eV)\footnotetext{见正文说明。} &\multicolumn{2}{c|}{有关电子跃迁\footnotemark} &gap(eV) &\multicolumn{1}{c|}{经验指认\upcite{PRB46-12196_1992}}\\ \hline
%--------------------------------------------------------------------------------------------------------------------------------
%
%     &0.07 &\multicolumn{2}{c|}{电子群体激发$\uparrow$} &--- &电子群\\ \cline{2-5}
%\raisebox{2.3ex}[0pt]{$\omega_f$} &0.1 &\multicolumn{2}{c|}{电子群体激发$\downarrow$} &--- &体激发\\ \hline
%--------------------------------------------------------------------------------------------------------------------------------
%
%     &1.50 &\raisebox{-2ex}[0pt][0pt]{20-22(0,1,4)} &2$p$(10.4)4$f$(74.9)$\rightarrow$ &\raisebox{-2ex}[0pt][0pt]{1.47} &\\%\cline{3-5}
%     &1.50$^\ast$ & &2$p$(17.5)5$d_{\mathrm E}$(14.0)$\uparrow$ & &4$f$$\rightarrow$5$d_{\mathrm E}$\\ \cline{3-5}
%     \raisebox{2.3ex}[0pt][0pt]{$a$} &(1.0$^\dagger$) &\raisebox{-2ex}[0pt][0pt]{20-22(1,2,6)} &\raisebox{-2ex}[0pt][0pt]{4$f$(89.9)$\rightarrow$2$p$(18.7)5$d_{\mathrm E}$(13.9)$\uparrow$}\footnotetext{波函数主要成分后的括号中,2$s$、2$p$和5$p$、4$f$、5$d$、6$s$分别指硼和铕的原子轨道;5$d_{\mathrm E}$、5$d_{\mathrm T}$分别指铕的(5$d_{z^2}$,5$d_{x^2-y^2}$和5$d_{xy}$,5$d_{xz}$,5$d_{yz}$)轨道,5$d_{\mathrm{ET}}$(或5$d_{\mathrm{TE}}$)则指5个5$d$轨道成分都有,成分大的用脚标的第一个字母标示;2$ps$(或2$sp$)表示同时含有硼2$s$、2$p$轨道成分,成分大的用第一个字母标示。$\uparrow$和$\downarrow$分别标示$\alpha$和$\beta$自旋电子跃迁。} &\raisebox{-2ex}[0pt][0pt]{1.56} &激子跃迁。 \\%\cline{3-5}
%     &(1.3$^\dagger$) & & & &\\ \hline
%--------------------------------------------------------------------------------------------------------------------------------

%     & &\raisebox{-2ex}[0pt][0pt]{19-22(0,0,1)} &2$p$(37.6)5$d_{\mathrm T}$(4.5)4$f$(6.7)$\rightarrow$ & & \\\nopagebreak %\cline{3-5}
%     & & &2$p$(24.2)5$d_{\mathrm E}$(10.8)4$f$(5.1)$\uparrow$ &\raisebox{2ex}[0pt][0pt]{2.78} &a、b、c峰可能 \\ \cline{3-5}
%     & &\raisebox{-2ex}[0pt][0pt]{20-29(0,1,1)} &2$p$(35.7)5$d_{\mathrm T}$(4.8)4$f$(10.0)$\rightarrow$ & &包含有复杂的\\ \nopagebreak%\cline{3-5}
%     &2.90 & &2$p$(23.2)5$d_{\mathrm E}$(13.2)4$f$(3.8)$\uparrow$ &\raisebox{2ex}[0pt][0pt]{2.92} &强激子峰。$^{\ast\ast}$\\ \cline{3-5}
%$b$  &2.90$^\ast$ &\raisebox{-2ex}[0pt][0pt]{19-22(0,1,1)} &2$p$(33.9)4$f$(15.5)$\rightarrow$ & &B2$s$-2$p$的价带 \\ \nopagebreak%\cline{3-5}
%     &3.0 & &2$p$(23.2)5$d_{\mathrm E}$(13.2)4$f$(4.8)$\uparrow$ &\raisebox{2ex}[0pt][0pt]{2.94} &顶$\rightarrow$B2$s$-2$p$导\\ \cline{3-5}
%     & &12-15(0,1,2) &2$p$(39.3)$\rightarrow$2$p$(25.2)5$d_{\mathrm E}$(8.6)$\downarrow$ &2.83 &带底跃迁。\\ \cline{3-5}
%     & &14-15(1,1,1) &2$p$(42.5)$\rightarrow$2$p$(29.1)5$d_{\mathrm E}$(7.0)$\downarrow$ &2.96 & \\\cline{3-5}
%     & &13-15(0,1,1) &2$p$(40.4)$\rightarrow$2$p$(28.9)5$d_{\mathrm E}$(6.6)$\downarrow$ &2.98 & \\ \hline
%--------------------------------------------------------------------------------------------------------------------------------
%%\hline
%%\hlinewd{0.5$p$t}
%\end{longtable}
%%\end{minipage}{\textwidth}
%%\setlength{\unitlength}{1cm}
%%\begin{picture}(0.5,2.0)
%%  \put(-0.02,1.93){$^{1)}$}
%%  \put(-0.02,1.43){$^{2)}$}
%%\put(0.25,1.0){\parbox[h]{14.2cm}{\small{\\}}
%%\put(-0.25,2.3){\line(1,0){15}}
%%\end{picture}
%\end{small}

%------------------------------------直-接-插-入-文-件--------------------------------------------------------------------------------------
%\textcolor{red}{\textbf{直接插入文件}}:\verbatiminput{/home/jun_jiang/Documents/Latex_art_beamer/Daily_WORKS/Report-2020_model.tex} %为保险:~选用文件名绝对路径
%\textcolor{red}{\textbf{备忘录}}:\verbatiminput{/home/jun_jiang/Documents/备忘录.txt}
%---------------------------------------------------------------------------------------------------------------------------------------------%

%-------------------------------------------------------------------------Thanks------------------------------------------------------------------------------------------------
%\newpage %%
%\newpage %%
%\thispagestyle{fancy}   % 首页插入页眉页脚 
\section{致谢}
致谢内容
%-----------------------------------------------------------------------------------------------------------------------------------------------------------------------%

