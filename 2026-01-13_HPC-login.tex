%---------------------- TEMPLATE FOR REPORT ------------------------------------------------------------------------------------------------------%

%\thispagestyle{fancy}   % 插入页眉页脚                                        %
%%%%%%%%%%%%%%%%%%%%%%%%%%%%% 用 authblk 包 支持作者和E-mail %%%%%%%%%%%%%%%%%%%%%%%%%%%%%%%%%
%\title{More than one Author with different Affiliations}				     %
\title{现有计算资源及登录方式}
\author[ ]{}   %
%\author[ ]{姜~骏\thanks{jiangjun@bcc.ac.cn}}   %
%\affil[ ]{北京市计算中心}    %
%\author[a]{Author A}									     %
%\author[a]{Author B}									     %
%\author[a]{Author C \thanks{Corresponding author: email@mail.com}}			     %
%%\author[a]{Author/通讯作者 C \thanks{Corresponding author: cores-email@mail.com}}     	     %
%\author[b]{Author D}									     %
%\author[b]{Author/作者 D}								     %
%\author[b]{Author E}									     %
%\affil[a]{Department of Computer Science, \LaTeX\ University}				     %
%\affil[b]{Department of Mechanical Engineering, \LaTeX\ University}			     %
%\affil[b]{作者单位-2}			    						     %
											     %
%%% 使用 \thanks 定义通讯作者								     %
											     %
\renewcommand*{\Authfont}{\small\rm} % 修改作者的字体与大小				     %
\renewcommand*{\Affilfont}{\small\it} % 修改机构名称的字体与大小			     %
\renewcommand\Authands{ and } % 去掉 and 前的逗号					     %
\renewcommand\Authands{ , } % 将 and 换成逗号					     %
\date{} % 去掉日期									     %
%\date{2020-12-30}									     %

%%%%%%%%%%%%%%%%%%%%%%%%%%%%%%%%%%%%%%%%%%  不使用 authblk 包制作标题  %%%%%%%%%%%%%%%%%%%%%%%%%%%%%%%%%%%%%%%%%%%%%%
%-------------------------------The Title of The Report-----------------------------------------%
%\title{报告标题:~}   %
%-----------------------------------------------------------------------------

%----------------------The Authors and the address of The Paper--------------------------------%
%\author{
%\small
%Author1, Author2, Author3\footnote{Communication author's E-mail} \\    %Authors' Names	       %
%\small
%(The Address,City Post code)						%Address	       %
%}
%\affil[$\dagger$]{清华大学~材料加工研究所~A213}
%\affil{清华大学~材料加工研究所~A213}
%\date{}					%if necessary					       %
%----------------------------------------------------------------------------------------------%
%%%%%%%%%%%%%%%%%%%%%%%%%%%%%%%%%%%%%%%%%%%%%%%%%%%%%%%%%%%%%%%%%%%%%%%%%%%%%%%%%%%%%%%%%%%%%%%%%%%%%%%%%%%%%%%%%%%%%
\maketitle
%\thispagestyle{fancy}   % 首页插入页眉页脚 

\section{\heiti 汴水超算}
\begin{table}[!h]
\tabcolsep 0pt \vspace*{-12pt}
\caption{\textrm{汴水超算登录信息.}}
\label{Table-Income}
\begin{minipage}{0.95\textwidth}
%\begin{center}
\centering
\def\temptablewidth{0.88\textwidth}
%\rule{\temptablewidth}{2.0pt}
%\begin{tabular*}{\temptablewidth}{|@{\extracolsep{\fill}}c|@{\extracolsep{\fill}}c|@{\extracolsep{\fill}}c|}
\begin{tabular*}{\temptablewidth}{@{\extracolsep{\fill}}c@{\extracolsep{\fill}}c@{\extracolsep{\fill}}c@{\extracolsep{\fill}}c@{\extracolsep{\fill}}c}
	\toprule[2pt]        %% 绘制三线表顶端线,粗细设置
	\multicolumn{2}{c}{\fontsize{10.2pt}{6.2pt}\selectfont{\textrm{VPN}登录信息}}  & \multicolumn{2}{c}{\fontsize{10.2pt}{6.2pt}\selectfont{服务器登录信息}} & \multirow{2}{*}{\fontsize{10.2pt}{6.2pt}\selectfont{备注}}\\
	账号 & 密码 & 账号 & 密码 & \\
%-------------------------------------------------------------------------------------------------------------------------
%&Peak (eV)  & {$\vec k$}-point            &Band{$_v$} to Band{$_c$}  &Transition Orbital
%Components\footnote{波函数主要成分后的括号中,$5s$、$5p$和$5p$、$4f$、$5d$分别指碲和铕的原子轨道。} &Gap (eV)   \\ \hline
%-------------------------------------------------------------------------------------------------------------------------
%&2.35       &(0,0,0)         &33$\rightarrow$34    &$4f$(31.58)$5p$(38.69)$\rightarrow$$5p$      &2.142   \\% \cline{3-7}
%&       &(0,0,0)         &33$\rightarrow$34    &$4f$(31.58)$5p$(38.69)$\rightarrow$$5p$      &2.142   \\% \cline{3-7}
%-------------------------------------------------------------------------------------------------------------------------
	\midrule[1pt]        %% 绘制三线表中间线,粗细设置
	\multicolumn{2}{c}{\fontsize{10.2pt}{6.2pt}\selectfont{\url{https://36.35.236.230:7788}}} &\multicolumn{2}{c}{\fontsize{10.2pt}{6.2pt}\selectfont{\url{https://10.5.10.3:22}}} &\\
	{\fontsize{9.8pt}{6.2pt}\selectfont{bcc\_jiangj}}  &{\fontsize{9.8pt}{6.2pt}\selectfont{BCCbcc123\$\%\^{}}}  &{\fontsize{9.8pt}{6.2pt}\selectfont{bcc\_jiangj}}  &{\fontsize{9.8pt}{6.2pt}\selectfont{@rM6PXCg1IHrTo}} & \\
	{\fontsize{9.8pt}{6.2pt}\selectfont{bcc\_jiangj2}}  &{\fontsize{9.8pt}{6.2pt}\selectfont{bcc\_jiangj2~(已失效)}}  &{\fontsize{9.8pt}{6.2pt}\selectfont{bcc\_jiangj2}}  &{\fontsize{9.8pt}{6.2pt}\selectfont{\#dV69EImgyHvz3}} & \\
	\bottomrule[1.8pt]     %%绘制三线表底部线,粗细设置
\end{tabular*}
%\rule{\temptablewidth}{1.8pt}
%\end{center}
\end{minipage}
\end{table}

\section{\heiti 深圳超算}
\begin{table}[!h]
\tabcolsep 0pt \vspace*{-12pt}
\caption{\textrm{深圳超算登录信息.}}
\label{Table-Income}
\begin{minipage}{0.95\textwidth}
%\begin{center}
\centering
\def\temptablewidth{0.88\textwidth}
%\rule{\temptablewidth}{2.0pt}
%\begin{tabular*}{\temptablewidth}{|@{\extracolsep{\fill}}c|@{\extracolsep{\fill}}c|@{\extracolsep{\fill}}c|}
\begin{tabular*}{\temptablewidth}{@{\extracolsep{\fill}}c@{\extracolsep{\fill}}c@{\extracolsep{\fill}}c@{\extracolsep{\fill}}c@{\extracolsep{\fill}}c}
	\toprule[2pt]        %% 绘制三线表顶端线,粗细设置
	\multicolumn{2}{c}{\fontsize{10.2pt}{6.2pt}\selectfont{\textrm{VPN}登录信息}}  & \multicolumn{2}{c}{\fontsize{10.2pt}{6.2pt}\selectfont{服务器登录信息}} & \multirow{2}{*}{\fontsize{10.2pt}{6.2pt}\selectfont{备注}}\\
	账号 & 密码 & 账号 & 密码 & \\
%-------------------------------------------------------------------------------------------------------------------------
%&Peak (eV)  & {$\vec k$}-point            &Band{$_v$} to Band{$_c$}  &Transition Orbital
%Components\footnote{波函数主要成分后的括号中,$5s$、$5p$和$5p$、$4f$、$5d$分别指碲和铕的原子轨道。} &Gap (eV)   \\ \hline
%-------------------------------------------------------------------------------------------------------------------------
%&2.35       &(0,0,0)         &33$\rightarrow$34    &$4f$(31.58)$5p$(38.69)$\rightarrow$$5p$      &2.142   \\% \cline{3-7}
%&       &(0,0,0)         &33$\rightarrow$34    &$4f$(31.58)$5p$(38.69)$\rightarrow$$5p$      &2.142   \\% \cline{3-7}
%-------------------------------------------------------------------------------------------------------------------------
	\midrule[1pt]        %% 绘制三线表中间线,粗细设置
	\multicolumn{2}{c}{\fontsize{10.2pt}{6.2pt}\selectfont{\url{https://vpn.nsccsz.cn}}} &\multicolumn{2}{c}{\fontsize{10.2pt}{6.2pt}\selectfont{\url{https://10.68.0.121}}} &\\
	{\fontsize{9.8pt}{6.2pt}\selectfont{zkywl\_lhy3}}  &{\fontsize{9.8pt}{6.2pt}\selectfont{lhy\_zkywl3!}}  &{\fontsize{9.8pt}{6.2pt}\selectfont{nsgkx\_lhy3}}  &{\fontsize{9.8pt}{6.2pt}\selectfont{ZC20230155}} & \\
	\midrule[1pt]        %% 绘制三线表中间线,粗细设置
	\multicolumn{2}{c}{\fontsize{10.2pt}{6.2pt}\selectfont{\url{https://dsjvpn.nsccsz.cn:4443}}} &\multicolumn{2}{c}{\fontsize{10.2pt}{6.2pt}\selectfont{\url{https://10.32.48.56:22}}} &\\
	{\fontsize{9.8pt}{6.2pt}\selectfont{gzsc\_jaj}}  &{\fontsize{9.8pt}{6.2pt}\selectfont{BCCbcc@317123}}  &{\fontsize{9.8pt}{6.2pt}\selectfont{nsgm\_jiangj}}  &{\fontsize{9.8pt}{6.2pt}\selectfont{N202507140025}} & \\
	\bottomrule[1.8pt]     %%绘制三线表底部线,粗细设置
\end{tabular*}
%\rule{\temptablewidth}{1.8pt}
%\end{center}
\end{minipage}
\end{table}
在编译节点,加载以下环境,可以使用联网功能:\\
source /home-ssd/Soft/modules/bashrc \\
module load proxy/proxy

\section{\heiti 其余计算资源}
\begin{table}[!h]
\tabcolsep 0pt \vspace*{-12pt}
\caption{\textrm{情报所超算登录信息.}}
\label{Table-Income}
\begin{minipage}{0.95\textwidth}
%\begin{center}
\centering
\def\temptablewidth{0.88\textwidth}
%\rule{\temptablewidth}{2.0pt}
%\begin{tabular*}{\temptablewidth}{|@{\extracolsep{\fill}}c|@{\extracolsep{\fill}}c|@{\extracolsep{\fill}}c|}
\begin{tabular*}{\temptablewidth}{@{\extracolsep{\fill}}c@{\extracolsep{\fill}}c@{\extracolsep{\fill}}c@{\extracolsep{\fill}}c@{\extracolsep{\fill}}c}
	\toprule[2pt]        %% 绘制三线表顶端线,粗细设置
	\multicolumn{2}{c}{\fontsize{10.2pt}{6.2pt}\selectfont{\textrm{VPN}登录信息}}  & \multicolumn{2}{c}{\fontsize{10.2pt}{6.2pt}\selectfont{服务器登录信息}} & \multirow{2}{*}{\fontsize{10.2pt}{6.2pt}\selectfont{备注}}\\
	账号 & 密码 & 账号 & 密码 & \\
%-------------------------------------------------------------------------------------------------------------------------
%&Peak (eV)  & {$\vec k$}-point            &Band{$_v$} to Band{$_c$}  &Transition Orbital
%Components\footnote{波函数主要成分后的括号中,$5s$、$5p$和$5p$、$4f$、$5d$分别指碲和铕的原子轨道。} &Gap (eV)   \\ \hline
%-------------------------------------------------------------------------------------------------------------------------
%&2.35       &(0,0,0)         &33$\rightarrow$34    &$4f$(31.58)$5p$(38.69)$\rightarrow$$5p$      &2.142   \\% \cline{3-7}
%&       &(0,0,0)         &33$\rightarrow$34    &$4f$(31.58)$5p$(38.69)$\rightarrow$$5p$      &2.142   \\% \cline{3-7}
%-------------------------------------------------------------------------------------------------------------------------
	\midrule[1pt]        %% 绘制三线表中间线,粗细设置
	\multicolumn{2}{c}{\fontsize{10.2pt}{6.2pt}\selectfont{\url{https://td.bjast.ac.cn:44433}}} &\multicolumn{2}{c}{\fontsize{10.2pt}{6.2pt}\selectfont{\url{https://10.100.200.149:20016}}} &\\
	{\fontsize{9.8pt}{6.2pt}\selectfont{bkrz02}}  &{\fontsize{9.8pt}{6.2pt}\selectfont{Welcome123\$}}  &{\fontsize{9.8pt}{6.2pt}\selectfont{root}}  &{\fontsize{9.8pt}{6.2pt}\selectfont{Welcome123\$}} & \\
	\bottomrule[1.8pt]     %%绘制三线表底部线,粗细设置
\end{tabular*}
%\rule{\temptablewidth}{1.8pt}
%\end{center}
\end{minipage}
\end{table}
控制台:\\
\textrm{web}访问:~\url{http://10.100.200.149/}\\
用户名:~\textrm{useradmin}
实例名称:~\textrm{Material\_Simulation}

