%---------------------- TEMPLATE FOR REPORT ------------------------------------------------------------------------------------------------------%

%\thispagestyle{fancy}   % 插入页眉页脚                                        %
%%%%%%%%%%%%%%%%%%%%%%%%%%%%% 用 authblk 包 支持作者和E-mail %%%%%%%%%%%%%%%%%%%%%%%%%%%%%%%%%
%\title{More than one Author with different Affiliations}				     %
%\title{\rm{VASP}的电荷密度存储文件\rm{CHGCAR}}
%\title{面向高温合金材料设计的计算模拟软件中的几个主要问题}
\title{化学化工知识图谱的建设与应用}
\author[ ]{北京市计算中心~云平台事业部}   %
%\author[ ]{姜~骏\thanks{jiangjun@bcc.ac.cn}}   %
%\affil[ ]{北京市计算中心}    %
%\author[a]{Author A}									     %
%\author[a]{Author B}									     %
%\author[a]{Author C \thanks{Corresponding author: email@mail.com}}			     %
%%\author[a]{Author/通讯作者 C \thanks{Corresponding author: cores-email@mail.com}}     	     %
%\author[b]{Author D}									     %
%\author[b]{Author/作者 D}								     %
%\author[b]{Author E}									     %
%\affil[a]{Department of Computer Science, \LaTeX\ University}				     %
%\affil[b]{Department of Mechanical Engineering, \LaTeX\ University}			     %
%\affil[b]{作者单位-2}			    						     %
											     %
%%% 使用 \thanks 定义通讯作者								     %
											     %
\renewcommand*{\Authfont}{\small\rm} % 修改作者的字体与大小				     %
\renewcommand*{\Affilfont}{\small\it} % 修改机构名称的字体与大小			     %
\renewcommand\Authands{ and } % 去掉 and 前的逗号					     %
\renewcommand\Authands{ , } % 将 and 换成逗号					     %
\date{} % 去掉日期									     %
%\date{2020-12-30}									     %

%%%%%%%%%%%%%%%%%%%%%%%%%%%%%%%%%%%%%%%%%%  不使用 authblk 包制作标题  %%%%%%%%%%%%%%%%%%%%%%%%%%%%%%%%%%%%%%%%%%%%%%
%-------------------------------The Title of The Report-----------------------------------------%
%\title{报告标题:~}   %
%-----------------------------------------------------------------------------

%----------------------The Authors and the address of The Paper--------------------------------%
%\author{
%\small
%Author1, Author2, Author3\footnote{Communication author's E-mail} \\    %Authors' Names	       %
%\small
%(The Address,City Post code)						%Address	       %
%}
%\affil[$\dagger$]{清华大学~材料加工研究所~A213}
%\affil{清华大学~材料加工研究所~A213}
%\date{}					%if necessary					       %
%----------------------------------------------------------------------------------------------%
%%%%%%%%%%%%%%%%%%%%%%%%%%%%%%%%%%%%%%%%%%%%%%%%%%%%%%%%%%%%%%%%%%%%%%%%%%%%%%%%%%%%%%%%%%%%%%%%%%%%%%%%%%%%%%%%%%%%%
\maketitle
%\thispagestyle{fancy}   % 首页插入页眉页脚 
\section{引言}
知识图谱\textrm{(Knowledge~Graph)}是一种用于组织、表示和存储知识的图形化数据结构,目的是仿照人类对于知识的认知、理解方式,将实体\textrm{(Entities)}、关系\textrm{(Relationship)}和属性\textrm{(Attributes)}以图形的形式呈现出来,使计算机能够更好地认知、理解和推理知识。早在1960年代,就曾出现过以模拟仿真人类思维-决策流程的专家系统,称为知识工程\textrm{(Knowledge Engineering, KE)}\upcite{Knowledge_Engineering}。此后不久涌现了多种商务的专家与知识管理系统,一般统称为知识库系统\textrm{(knowledge-based system)}。在知识库系统基础上,通过用户人机问答式互动,可以构建结构化的机器可读与生成式的文本知识,而运维知识库系统的通常是行业和领域专家(如图\ref{Fig:Knowledge-based_system}所示)。
\begin{figure}[h!]
\centering
\includegraphics[height=1.85in,width=5.85in,viewport=0 0 1500 475,clip]{Knowledge-based_system.png}
\caption{\small\textrm{The minimal components of a knowledge-based system. cite from\cite{ACR56-128_2023}}}%(与文献\cite{EPJB33-47_2003}图1对比)
\label{Fig:Knowledge-based_system}
\end{figure}

进入21世纪以来,得益于计算能力的大幅度提升和语义网\textrm{Semantic Web}\upcite{SA284-34_2001}技术与开源软件的发展,知识图谱作为继知识工程后的计算机辅助学习和智能数据支撑工具得到了长足发展\upcite{SA141-112948_2020,IEEETNNLS33-494_2022},特别是在%以下是一些关于知识图谱在近年发展趋势:
自然语言处理\textrm{(Natural Language Processing,~NLP)}、%中语言处理任务的改进与问答系统的文档与摘要生成,搜索引擎优化中的知识卡片及相关结果的丰富: 知识图谱被广泛用于改进自然语言处理任务,如问答系统和文档摘要生成。此外,搜索引擎如Google使用知识图谱来提供更丰富的搜索结果,例如知识卡片和相关问题。 
医疗和生命科学中的辅助精准诊疗与决策、%: 知识图谱在医疗诊断、药物研发和疾病管理方面的应用迅速增长。它们用于。 
智能城市和物联网\textrm{(Internet of Things,~IoT)}、%: 知识图谱在智能城市项目中发挥关键作用,帮助城市管理者优化资源分配、交通管理和环境监测。此外,它们在IoT中用于设备和传感器之间的数据整合和智能控制。 
金融和风险管理中的复杂金融关系与风险识别、个性化投资建议等领域,都起到了重要的业务提升和精准服务的作用。%此外在: 在金融领域,知识图谱用于分析复杂的金融关系、识别风险和欺诈行为,以及提供个性化的投资建议。它们有助于机构更好地理解市场动态和客户行为。 
%智能助手和虚拟助手: 知识图谱被广泛用于智能助手和虚拟助手,如Siri、Alexa和Google助手。这些助手使用知识图谱来回答用户的问题、提供建议和执行任务。 
%教育和培训: 在教育领域,知识图谱被用于个性化学习路径的设计,帮助学生更有效地学习和掌握知识。 
%企业知识管理: 企业越来越多地使用知识图谱来管理内部知识资源,以促进知识共享、问题解决和决策支持。 
%可持续发展和环境保护、: 在可持续发展领域,知识图谱有助于整合环境数据、能源消耗和碳排放等信息,以支持可持续发展决策。 
%卫生和流行病学、: 知识图谱在流行病学研究中有应用潜力,可以帮助疾病控制机构追踪疾病传播、预测爆发并采取预防措施。 
%多模态知识图谱等不同领域,: 近年来,研究者开始探索将文本数据、图像、声音和其他多媒体数据整合到知识图谱中,以更全面地表示世界的知识。
知识图谱是基于语义网技术发展起来的,网格化结构的整合能力强,可实现异质数据的相互链接,确保了知识图谱可以被软件自动接受。知识图谱对人类专家系统的决策仿真,除了传统的问答式输出外,在软件支持下,拥有了学习和推理能力,也具备初级的创造知识的能力。因此非常适合跨专业领域、跨空间的应用场景。在化学和化工研究领域,具有专业深度知识背景的图谱还处于起步建设阶段。%数据当前的科学数据来源,中文文献检索数据主要源于知网\textrm{(CNKI)},英文文献则主要来源\textrm{(Web of science,~WOS)}核心库

随着人工智能、数据科学和信息管理技术的不断进步和应用领域的扩展,可以预见知识图谱也有望发挥更大的作用。未来化学和化工知识图谱的发展有望进一步推动科学研究、新材料开发和绿色化学等领域的创新。

%煤化工技术是指以煤为原料生产各种能源或化工产品的工艺过程,一般包括煤炭转化和后续加工
\section{化学-化工知识图谱}
机器仿真接受和感知知识,是对数据赋予一定的意义,并寻找数据间的关系。传统的关系型文本数据间是通过表格直接相互关联的,一旦知识结构发生变化,这种表达形式成为极大的制约。知识图谱以图形结构的方式,通常使用节点(实体)和连线/边(关系)的形式来表示知识。这种结构使得知识点之间的关联关系更加清晰和可视化。当有新的数据和关系加入知识图谱时,不会影响原有的知识结构。它主要的关键词为:
\begin{itemize}
	\item 实体\textrm{(Entities)}:~知识图谱中的实体是指具体的事物、概念、人物、地点等,每个实体都有一个唯一的标识符。例如,在一个化学知识图谱中,分子、合成体、密度等都可以是实体。
	\item 关系\textrm{(Relationships)}:~实体之间的关系表示不同实体之间的连接或互动。这些关系可以是有向的或无向的,用于描述实体之间的各种联系,如"具有"、"属于"、"值为"等。
	\item 属性\textrm{(Attributes)}:~实体可以有一些描述性的属性,这些属性是与实体相关的额外信息。例如,一个分子实体可以有属性包括分子量、化合价、密度等。
\end{itemize}

知识图谱是一种强大的工具,用于组织、存储和分析知识,它在推动人工智能、自然语言处理和数据科学领域的发展中发挥着重要作用。知识图谱的应用领域在不断扩展,对于解决复杂的问题和推动科学研究具有巨大潜力。其主要的特点是:
\begin{itemize}
	\item 语义丰富:~知识图谱不仅仅是简单的数据存储结构,它还具有语义信息,可以帮助计算机更好地理解知识。例如,通过关系的定义,计算机可以知道"父亲"关系是指一个人与另一个人之间的亲属关系。
	\item 查询和推理:~知识图谱支持复杂的查询和推理操作,允许用户通过查询知识图谱来获取有关实体和关系的信息,并进行逻辑推理以回答复杂的问题。
	\item 应用领域丰富:~知识图谱在多个领域有广泛的应用,包括自然语言处理、搜索引擎优化、智能助手、数据挖掘、推荐系统、医疗诊断、金融风险分析等。
	\item 标准化:~知识图谱的创建和管理通常遵循特定的标准和规范,以确保数据的一致性和互操作性。例如,资源描述框架\textrm{(Resource Description Framework, RDF)}和编程语言\textrm{WOL~(Web Ontology Language)},是常用于表示知识图谱的标准。
	\item 可持续更新:~知识图谱需要不断更新和维护,以反映知识的最新发展和变化。需要自动化的数据抓取和人工编辑的结合。
\end{itemize}

具体到化学-化工知识图谱,是一种用于组织、表示和存储化学科学、化学工程和领域知识的特定类型的知识图谱。它的目标是捕捉和整合与化学科学、化学工程和领域相关的实体(如化合物、元素、分子、反应等)、关系(如化学反应、化学结构、性质等)以及属性(如物理性质、化学性质、命名等),以便计算机可以更好地理解和分析化学与应用化学的信息。图\ref{Fig:Mapping-relationship-molecule-synthon}给出分子与合成体的图谱表示。以下是有关化学-化工知识图谱的一些关键方面: 
\begin{figure}[h!]
\centering
\includegraphics[height=4.90in,width=5.85in,viewport=0 0 950 790,clip]{Mapping-the-relationship-between-molecule-and-synthon.png}
\caption{\small\textrm{Mapping the relationship molecule (chemical) and synthon (abstract) concepts and illustrating them with instrances. cite from\cite{ACR56-128_2023}}}%(与文献\cite{EPJB33-47_2003}图1对比)
\label{Fig:Mapping-relationship-molecule-synthon}
\end{figure}
\begin{itemize}
	\item 实体:~知识图谱中的化学科学的实体包括元素、化合物、分子、离子、反应、化学方程式、化学领域的科学家和研究机构等;化学工程的实体包括化工工艺、化工设备、化学反应、化工原料、化工产品、化工企业、化工工程师等。每个实体都具有唯一的标识符。
	\item 关系:~关系用于描述不同实体之间的连接或互动,例如,化合物之间的化学反应、元素之间的关系、实验数据与化学实体之间的关联等;化工工艺中的流程步骤、设备之间的连接、原料与产品之间的转化等。
	\item 属性:~化学知识图谱中的实体可以具有各种属性,包括物理性质(如密度、熔点、沸点)、化学性质(如酸碱性、溶解性)、结构信息(如分子式、分子结构)、命名规范、工艺参数(如产量、效率)等。
	\item 图形结构:化学知识图谱通常以图形结构的形式表示,其中节点表示化学-化工实体,边表示实体之间的关系。这种结构有助于可视化和查询化学科学与化学工程的信息。
	\item 领域知识的表示:化学知识图谱可以捕捉领域专家的知识,包括领域特定的术语、概念、分类体系、工艺流程、设备设计规范、化工安全标准和规则等。
	\item 查询和推理:化学知识图谱支持各种查询操作,使用户能够检索与特定化学实体或关系相关的信息。还可以进行推理操作,以推断新的化学关系或性质。
	\item 应用领域:化学知识图谱在药物研发、材料科学、环境科学、化学教育、化学信息检索、化学工程、工业生产、工艺优化、化工安全、环境保护等领域有广泛的应用。它有助于提高科学研究与工业生产效率、减少环境风险,并促进化学科学研究者与化工工程师的决策制定。
	\item 数据来源:构建化学知识图谱需要整合来自各种数据源的信息,包括文献、化学软件、工艺设计文档、计算与实验数据、安全手册、化工数据库等等。
	\item 标准和本体:化学-化工知识图谱的创建和管理通常需要依靠化学-化工领域的标准和本体,以确保数据的一致性和互操作性。
\end{itemize} 
总的来说,化学知识图谱有助于推动化学领域的研究和应用,使科学家和工程师能够更好地利用化学知识来解决问题、设计新材料和药物,以及改进和加速化学与化工工程的创新和发展,提高生产效率,并确保工业过程的可持续性和安全性。这种图谱的发展有助于研究人员、工程师和决策者更好地利用和分享化工知识,也将加速化学领域的创新和发展。

\section{化学-化工知识图谱建设}
煤化工知识图谱是一个用于组织和表示煤化工领域知识的图形化工具,它有助于科研人员、工程师和学生更好地理解和应用煤化工学科的相关信息。下面是关于煤化工知识图谱的一些评论和重要方面的回顾:

知识整合和可视化:煤化工知识图谱通过将各种煤化工学科的信息整合到一个可视化图谱中,为用户提供了一个清晰的知识结构。这有助于研究人员更好地理解不同领域之间的联系,从而更有效地解决问题。

知识发现:知识图谱不仅可以帮助用户获取已知信息,还可以用于知识发现。通过图谱的搜索和浏览功能,用户可以发现他们之前可能不知道的相关信息,从而激发新的研究方向。

跨学科研究:煤化工知识图谱有助于促进跨学科研究,因为它将多个学科领域的知识整合在一起,鼓励不同领域的专家共同探讨和解决复杂的煤化工问题。

教育工具:煤化工知识图谱可以作为教育工具使用,帮助学生更好地理解煤化工学科的基本概念和关键概念。它可以用于制定教学材料和教学计划。

持续更新和维护:知识图谱需要不断更新和维护,以反映煤化工领域的最新研究和发展。这需要持续的努力和资源投入。

隐私和数据安全:在创建和使用煤化工知识图谱时,需要注意隐私和数据安全问题,特别是涉及敏感信息或专利数据的情况。

总的来说,煤化工知识图谱是一个有潜力的工具,可以促进煤化工领域的研究和应用。然而,它的有效性取决于数据的质量和持续的维护工作。随着技术的进步和研究的不断发展,煤化工知识图谱将继续演化和改进,为该领域的相关工作和研究提供更大的帮助。
中国的煤化学化工知识图谱建设是一个具有战略性重要性的项目,旨在整合、组织和推广中国在煤化学和化工领域的丰富知识资源。以下是有关中国煤化学化工知识图谱建设的一些关键方面:

数据整合:首要任务是整合来自不同数据源的煤化学和化工领域的信息,包括煤矿资源、煤的化学成分、化工工艺、产品制造、环保技术、市场趋势等。这些数据可以来自政府报告、研究机构、大学研究、化工企业和专业数据库。

知识结构:建设知识图谱需要建立适当的知识结构,包括实体、关系和属性的定义,以便能够清晰地表示和连接各种煤化学和化工领域的知识。

标准化和本体:采用标准化的数据模型和领域本体是确保知识图谱数据一致性和互操作性的关键。这可以帮助不同数据源的信息进行有效的整合。

自动化数据采集:为了保持知识图谱的及时性,可以使用自动化工具来定期收集和更新数据。这可能涉及到网络爬虫、自然语言处理技术和数据挖掘方法。

知识图谱构建工具:选择合适的知识图谱构建工具和技术,例如图数据库,以存储和查询知识图谱数据。

领域专家参与:引入煤化学和化工领域的专家参与知识图谱的构建和维护,以确保数据的准确性和相关性。

应用开发:知识图谱的价值在于其应用。开发应用程序和工具,使煤化学和化工领域的研究人员、工程师和决策者能够更好地访问和利用知识图谱中的信息。

教育和培训:提供关于知识图谱的培训和教育,以确保人才具备使用和贡献知识图谱的能力。

隐私和安全:在数据收集和共享过程中,必须遵守隐私法规和确保数据的安全性。

中国的煤化学化工知识图谱建设有望推动煤化学产业的创新和可持续发展,促进资源利用效率的提高,减少环境污染,提供决策支持,以及培养煤化学和化工领域的人才。这是一个长期而复杂的任务,需要政府、学术界、工业界和研究机构的合作来实现。

截止到我的知识截止日期(2021年9月),煤化学和化工知识图谱的建设仍处于不断发展和演进的阶段,以下是一些在煤化学和化工领域知识图谱方面的主要现状和趋势:

数据整合和标准化:许多煤化学和化工领域的知识图谱项目正在努力整合来自不同数据源的信息,包括煤矿资源、化学反应、工艺流程、环保技术和市场数据。标准化数据模型和本体的使用有助于确保数据的一致性和互操作性。

领域特定知识图谱:一些项目专注于创建领域特定的知识图谱,例如,关注煤气化工程的知识图谱、煤制油品的知识图谱等。这些图谱有助于深入研究特定领域的问题。

知识图谱的应用:知识图谱在煤化学和化工领域的应用正在扩展。它们用于化学工程流程的优化、煤制品的研发、环境保护和化工安全管理等方面。

机器学习和人工智能:知识图谱与机器学习和人工智能的结合是一个重要趋势。这种组合使得知识图谱可以更好地用于自动化数据分析、预测和决策支持。

国际合作:一些项目涉及国际合作,吸引了来自不同国家的研究人员和机构,以共同构建和维护跨国界的煤化学和化工知识图谱。

教育和培训:越来越多的教育和培训机构开始将知识图谱技术纳入课程,培养煤化学和化工领域的人才以利用知识图谱进行研究和实践。

需要注意的是,煤化学和化工知识图谱的建设是一个复杂的任务,需要不断的数据更新、质量控制和领域专家的参与。随着科技的不断发展,这一领域的知识图谱也会不断演化和完善,为煤化学和化工领域的研究和应用提供更大的支持。因此,了解最新进展需要查阅最新的学术文献和项目报道。

请注意,这些文献的可用性和具体内容可能会因时间和您所在地区的访问权限而有所不同。您可以使用学术搜索引擎或图书馆资源来查找并获取这些文献的全文或详细信息。

\section{小结}
