%---------------------- TEMPLATE FOR REPORT ------------------------------------------------------------------------------------------------------%

%\thispagestyle{fancy}   % 插入页眉页脚                                        %
%%%%%%%%%%%%%%%%%%%%%%%%%%%%% 用 authblk 包 支持作者和E-mail %%%%%%%%%%%%%%%%%%%%%%%%%%%%%%%%%
%\title{More than one Author with different Affiliations}				     %
%\title{\rm{VASP}的电荷密度存储文件\rm{CHGCAR}}
%\title{面向高温合金材料设计的计算模拟软件中的几个主要问题}
\title{化学-化工知识图谱的建设与应用}
\author[ ]{北京市计算中心~云平台事业部}   %
%\author[ ]{姜~骏\thanks{jiangjun@bcc.ac.cn}}   %
%\affil[ ]{北京市计算中心}    %
%\author[a]{Author A}									     %
%\author[a]{Author B}									     %
%\author[a]{Author C \thanks{Corresponding author: email@mail.com}}			     %
%%\author[a]{Author/通讯作者 C \thanks{Corresponding author: cores-email@mail.com}}     	     %
%\author[b]{Author D}									     %
%\author[b]{Author/作者 D}								     %
%\author[b]{Author E}									     %
%\affil[a]{Department of Computer Science, \LaTeX\ University}				     %
%\affil[b]{Department of Mechanical Engineering, \LaTeX\ University}			     %
%\affil[b]{作者单位-2}			    						     %
											     %
%%% 使用 \thanks 定义通讯作者								     %
											     %
\renewcommand*{\Authfont}{\small\rm} % 修改作者的字体与大小				     %
\renewcommand*{\Affilfont}{\small\it} % 修改机构名称的字体与大小			     %
\renewcommand\Authands{ and } % 去掉 and 前的逗号					     %
\renewcommand\Authands{ , } % 将 and 换成逗号					     %
\date{} % 去掉日期									     %
%\date{2020-12-30}									     %

%%%%%%%%%%%%%%%%%%%%%%%%%%%%%%%%%%%%%%%%%%  不使用 authblk 包制作标题  %%%%%%%%%%%%%%%%%%%%%%%%%%%%%%%%%%%%%%%%%%%%%%
%-------------------------------The Title of The Report-----------------------------------------%
%\title{报告标题:~}   %
%-----------------------------------------------------------------------------

%----------------------The Authors and the address of The Paper--------------------------------%
%\author{
%\small
%Author1, Author2, Author3\footnote{Communication author's E-mail} \\    %Authors' Names	       %
%\small
%(The Address,City Post code)						%Address	       %
%}
%\affil[$\dagger$]{清华大学~材料加工研究所~A213}
%\affil{清华大学~材料加工研究所~A213}
%\date{}					%if necessary					       %
%----------------------------------------------------------------------------------------------%
%%%%%%%%%%%%%%%%%%%%%%%%%%%%%%%%%%%%%%%%%%%%%%%%%%%%%%%%%%%%%%%%%%%%%%%%%%%%%%%%%%%%%%%%%%%%%%%%%%%%%%%%%%%%%%%%%%%%%
\maketitle
%\thispagestyle{fancy}   % 首页插入页眉页脚 
\section{引言}
知识图谱\textrm{(Knowledge~Graph)}是一种用于组织、表示和存储知识的图形化数据结构形式,目的是仿照人类对于知识的认知、理解方式,将实体\textrm{(Entities)}、关系\textrm{(Relationship)}和属性\textrm{(Attributes)}以图形的形式呈现出来,使计算机能够更好地认知、理解和推理知识。早在1960年代,就曾出现过以模拟仿真人类思维-决策流程的专家系统,称为知识工程\textrm{(Knowledge Engineering, KE)}\upcite{Knowledge_Engineering}。此后不久涌现了多种商务的专家与知识管理系统,一般统称为知识库系统\textrm{(knowledge-based system)}。在知识库系统基础上,通过用户人机问答式互动,可以构建结构化的机器可读与生成式的文本知识,而运维知识库系统的通常是行业和领域专家(如图\ref{Fig:Knowledge-based_system}所示)。
\begin{figure}[h!]
\centering
\includegraphics[height=1.85in,width=5.85in,viewport=0 0 1500 475,clip]{Knowledge-based_system.png}
\caption{\small\textrm{The minimal components of a knowledge-based system. cite from\cite{ACR56-128_2023}}}%(与文献\cite{EPJB33-47_2003}图1对比)
\label{Fig:Knowledge-based_system}
\end{figure}

进入21世纪以来,得益于计算能力的大幅度提升和语义网\textrm{(Semantic Web)}\upcite{SA284-34_2001}技术与开源软件的发展,知识图谱作为继知识工程后的计算机辅助学习和智能数据支撑工具得到了长足发展\upcite{SA141-112948_2020,IEEETNNLS33-494_2022},特别是在%以下是一些关于知识图谱在近年发展趋势:
自然语言处理\textrm{(Natural Language Processing,~NLP)}、%中语言处理任务的改进与问答系统的文档与摘要生成,搜索引擎优化中的知识卡片及相关结果的丰富: 知识图谱被广泛用于改进自然语言处理任务,如问答系统和文档摘要生成。此外,搜索引擎如Google使用知识图谱来提供更丰富的搜索结果,例如知识卡片和相关问题。 
医疗和生命科学中的辅助精准诊疗与决策、%: 知识图谱在医疗诊断、药物研发和疾病管理方面的应用迅速增长。它们用于。 
智能城市和物联网\textrm{(Internet of Things,~IoT)}、%: 知识图谱在智能城市项目中发挥关键作用,帮助城市管理者优化资源分配、交通管理和环境监测。此外,它们在IoT中用于设备和传感器之间的数据整合和智能控制。 
金融和风险管理中的复杂金融关系与风险识别、个性化投资建议等领域,都起到了重要的业务提升和精准服务的作用。%此外在: 在金融领域,知识图谱用于分析复杂的金融关系、识别风险和欺诈行为,以及提供个性化的投资建议。它们有助于机构更好地理解市场动态和客户行为。 
%智能助手和虚拟助手: 知识图谱被广泛用于智能助手和虚拟助手,如Siri、Alexa和Google助手。这些助手使用知识图谱来回答用户的问题、提供建议和执行任务。 
%教育和培训: 在教育领域,知识图谱被用于个性化学习路径的设计,帮助学生更有效地学习和掌握知识。 
%企业知识管理: 企业越来越多地使用知识图谱来管理内部知识资源,以促进知识共享、问题解决和决策支持。 
%可持续发展和环境保护、: 在可持续发展领域,知识图谱有助于整合环境数据、能源消耗和碳排放等信息,以支持可持续发展决策。 
%卫生和流行病学、: 知识图谱在流行病学研究中有应用潜力,可以帮助疾病控制机构追踪疾病传播、预测爆发并采取预防措施。 
%多模态知识图谱等不同领域,: 近年来,研究者开始探索将文本数据、图像、声音和其他多媒体数据整合到知识图谱中,以更全面地表示世界的知识。
知识图谱是基于语义网技术发展起来的,网格化结构的整合能力强,可实现异质数据的相互链接,确保了知识图谱可以被软件自动接受。知识图谱对人类专家系统的决策仿真,除了传统的问答式输出外,在软件支持下,拥有了学习和推理能力,也具备初级的创造知识的能力。因此非常适合跨专业领域、跨空间的应用场景。在化学和化工研究领域,具有专业深度知识背景的图谱还处于起步建设阶段。%数据当前的科学数据来源,中文文献检索数据主要源于知网\textrm{(CNKI)},英文文献则主要来源\textrm{(Web of science,~WOS)}核心库

随着人工智能\textrm{(Artificial Intelligence AI)}、数据科学和信息管理技术的不断进步和应用领域的扩展,可以预见知识图谱也有望发挥更大的作用。未来化学和化工知识图谱的发展有望进一步推动科学研究、新材料开发和绿色化学等领域的创新。

%煤化工技术是指以煤为原料生产各种能源或化工产品的工艺过程,一般包括煤炭转化和后续加工
\section{化学-化工知识图谱}
机器仿真接受和感知知识,是对数据赋予一定的意义,并寻找数据间的关系。传统的关系型文本数据间是通过表格直接相互关联的,一旦知识结构发生变化,这种表达形式成为极大的制约。知识图谱以图形结构的方式,通常使用节点(实体)和连线/边(关系)的形式来表示知识。这种结构使得知识点之间的关联关系更加清晰和可视化。当有新的数据和关系加入知识图谱时,不会影响原有的知识结构。它主要的关键词为:
\begin{itemize}
	\item 实体\textrm{(Entities)}:~知识图谱中的实体是指具体的事物、概念、人物、地点等,每个实体都有一个唯一的标识符。例如,在一个化学知识图谱中,分子、合成体、密度等都可以是实体。
	\item 关系\textrm{(Relationships)}:~实体之间的关系表示不同实体之间的连接或互动。这些关系可以是有向的或无向的,用于描述实体之间的各种联系,如"具有"、"属于"、"值为"等。
	\item 属性\textrm{(Attributes)}:~实体可以有一些描述性的属性,这些属性是与实体相关的额外信息。例如,一个分子实体可以有属性包括分子量、化合价、密度等。
\end{itemize}

知识图谱是一种强大的工具,用于组织、存储和分析知识,它在推动人工智能、自然语言处理和数据科学领域的发展中发挥着重要作用。知识图谱的应用领域在不断扩展,对于解决复杂的问题和推动科学研究具有巨大潜力。其主要的特点是:
\begin{itemize}
	\item 语义丰富:~知识图谱不仅仅是简单的数据存储结构,它还具有语义信息,可以帮助计算机更好地理解知识。例如,通过关系的定义,计算机可以知道"父亲"关系是指一个人与另一个人之间的亲属关系。
	\item 查询和推理:~知识图谱支持复杂的查询和推理操作,允许用户通过查询知识图谱来获取有关实体和关系的信息,并进行逻辑推理以回答复杂的问题。
	\item 应用领域丰富:~知识图谱在多个领域有广泛的应用,包括自然语言处理、搜索引擎优化、智能助手、数据挖掘、推荐系统、医疗诊断、金融风险分析等。
	\item 标准化:~知识图谱的创建和管理通常遵循特定的标准和规范,以确保数据的一致性和互操作性。例如,资源描述框架\textrm{(Resource Description Framework, RDF)}和编程语言\textrm{WOL~(Web Ontology Language)},是常用于表示知识图谱的标准。
	\item 可持续更新:~知识图谱需要不断更新和维护,以反映知识的最新发展和变化。需要自动化的数据抓取和人工编辑的结合。
\end{itemize}

具体到化学-化工知识图谱,是一种用于组织、表示和存储化学科学、化学工程和领域知识的特定类型的知识图谱。它的目标是捕捉和整合与化学科学、化学工程和领域相关的实体(如化合物、元素、分子、反应等)、关系(如化学反应、化学结构、性质等)以及属性(如物理性质、化学性质、命名等),以便计算机可以更好地理解和分析化学与应用化学的信息。以下是有关化学-化工知识图谱的一些关键方面: 
\begin{itemize}
	\item 实体:~化学科学的实体包括元素、化合物、分子、离子、反应、化学方程式、化学领域的科学家和研究机构等;化学工程的实体包括化工工艺、化工设备、化学反应、化工原料、化工产品、化工企业、化工工程师等。每个实体都具有唯一的标识符。
	\item 关系:~关系用于描述不同实体之间的连接或互动,例如,化合物之间的化学反应、元素之间的关系、实验数据与化学实体之间的关联等;化工工艺中的流程步骤、设备之间的连接、原料与产品之间的转化等。
	\item 属性:~化学知识图谱中的实体可以具有各种属性,包括物理性质(如密度、熔点、沸点)、化学性质(如酸碱性、溶解性)、结构信息(如分子式、分子结构)、命名规范、工艺参数(如产量、效率)等。
	\item 图形结构:化学知识图谱通常以图形结构的形式表示,其中节点表示化学-化工实体,边表示实体之间的关系。这种结构有助于可视化和查询化学科学与化学工程的信息。
	\item 领域知识的表示:化学知识图谱可以捕捉领域专家的知识,包括领域特定的术语、概念、分类体系、工艺流程、设备设计规范、化工安全标准和规则等。
	\item 查询和推理:化学知识图谱支持各种查询操作,使用户能够检索与特定化学实体或关系相关的信息。还可以进行推理操作,以推断新的化学关系或性质。
	\item 应用领域:化学知识图谱在药物研发、材料科学、环境科学、化学教育、化学信息检索、化学工程、工业生产、工艺优化、化工安全、环境保护等领域有广泛的应用。它有助于提高科学研究与工业生产效率、减少环境风险,并促进化学科学研究者与化工工程师的决策制定。
	\item 数据来源:构建化学知识图谱需要整合来自各种数据源的信息,包括文献、化学软件、工艺设计文档、计算与实验数据、安全手册、化工数据库等等。
	\item 标准和本体:化学-化工知识图谱的创建和管理通常需要依靠化学-化工领域的标准和本体,以确保数据的一致性和互操作性。
\end{itemize} 
总的来说,化学知识图谱有助于推动化学领域的研究和应用,使科学家和工程师能够更好地利用化学知识来解决问题、设计新材料和药物,以及改进和加速化学与化工工程的创新和发展,提高生产效率,并确保工业过程的可持续性和安全性。这种图谱的发展有助于研究人员、工程师和决策者更好地利用和分享化工知识,也将加速化学领域的创新和发展。

\section{知识图谱建设}
知识图谱具有知识构建的能力,借助机器学习\textrm{(Machine Learning,~ML)},可以在现有知识基础上产生(如外推)新的知识。必须解决化学-化工知识的格式化表示\textrm{(Formal Representation)}、基本逻辑推理和化学知识的产生等相关问题。
\subsection{化学知识的格式化表示}
知识图谱的知识点表示的逻辑结构\textrm{(schema)},将术语(或实体)相关的整体表示成\textrm{TBox}\footnote{引入\textit{Box}的概念后,实体依然被视作点,而关系则被视作$n$维空间的中的边,点和边组成的区域就被称为\textit{Box}。}\textrm{(terminological box)},这种实体-关系的描述习惯上称为\textrm{Ontology}\footnote{\textrm{Ontology}直译为本体论,这里是指使用逻辑形式化的方法规定概念与关系,也就是“主\textrm{(subject)}-谓\textrm{predicate}-宾\textrm{(object)}”的谓词逻辑,使人或机器可以用统一的、准确的推理方法处理数据。}由\textrm{Ontology}出发表示的实体-实体关系为\textrm{ABox~(Assertion component)}。图\ref{Fig:Mapping-relationship-molecule-synthon}给出分子与合成体的图谱表示。 
\begin{figure}[h!]
\centering
\includegraphics[height=4.90in,width=5.85in,viewport=0 0 950 790,clip]{Mapping-the-relationship-between-molecule-and-synthon.png}
\caption{\small\textrm{Mapping the relationship molecule (chemical) and synthon (abstract) concepts and illustrating them with instrances. cite from\cite{ACR56-128_2023}}}%(与文献\cite{EPJB33-47_2003}图1对比)
\label{Fig:Mapping-relationship-molecule-synthon}
\end{figure}

\subsection{知识的逻辑-推理}
人类基于逻辑对知识的推理形式为演绎\textrm{(deductive)}、归纳\textrm{(inductive)}和溯因\textrm{(abductive)}。在人工智能系统中,溯因已经是广为接受但也是最优挑战的推理形式。从人类认知的角度看,逻辑推理减少的脑力思维称为启发。但是启发也模糊了“规则”和“类似”。近年来,机器学习广泛应用于化学知识的提取,特别是对于有足够的关系清晰的数据,效果非常显著。原则上,机器学习基于离散数据回归来处理数据,并不需要理解数据间的行为和关联。如果数据关联是基于统计的相关性,机器学习也可以视为归纳推理。另一方面,知识图谱是基于知识和领域专家经验构建的,当数据稀少时,就可以应用图谱算法而不必清洗或处理数据。\upcite{JACS144-11713_2022}

虽然归纳和演绎方法论上的差别,由于其推理对已有前提深度依赖,所以即使前提数据不完备(比如存在个别例外),认知的结果可能仍然是一致的。但是在溯因推理中,这种情形会需要更高层次的抽象,这对于人类专家也是非常大的挑战。

\subsection{认知的三个阶段}
知识图谱仿真人类认知解决和认知问题的三个阶段如图\ref{Fig:Three-main-stages-in-KE-pro-dev}所示:~标准化\textrm{(specification)}-概念化\textrm{(conceptualization)}-应用\textrm{(implementation)}
\begin{figure}[h!]
\centering
\includegraphics[height=1.85in,width=5.85in,viewport=0 0 1500 430,clip]{Three-main-stages-in-KE-project-development.png}
\caption{\small\textrm{The three main stages in Knowledge Graph project development. cite from\cite{ACR56-128_2023}}}%(与文献\cite{EPJB33-47_2003}图1对比)
\label{Fig:Three-main-stages-in-KE-pro-dev}
\end{figure}
在标准化阶段,领域专家根据认知反馈,确定其理解的知识点和如何认知知识点的规范过程,支持团队定义一套投喂知识图谱认知所需的完整问答/判断。这两项操作将有效聚焦知识图谱的认知对象,为概念化提供坚实的基础。概念映射将是知识的半正式表示,并对关联的实体有基本的确定,领域专家可以根据概念映射定义或设计适合问答/判断的算法。

在应用阶段,概念映射的实体实现本体化\textrm{(ontologized)},专家在本体化形式基础上梳理信息并实例化,由此完成\textrm{ABox}。然后,软件在设计的算法基础上程式化地传递知识并作出推断。整个系统经过验证后运行,遍历三阶段反复迭代也并不罕见,可以得到更好的输出结果\upcite{Handbook-Ontology}。

\section{化学-化工的知识图谱}
语义网概念出现不久,化学信息学研究者就考虑如何助力化学家\upcite{JCIM46-939_2006,Nature451-648_2008}。\textrm{M.~Kraft}等\upcite{TRSA368-3633_2010}从更广泛的层面考虑化学语义实例的应用,在该文中,作者通过讨论两个专门主题(化工复合物和燃烧的环境影响)的语义实例,试图建立分子尺度化学对宏观现象中复杂性、环境和健康的影响。他们构建了基本的化学-化工图谱\textrm{J-Park Simulator~(JPS)},\textrm{JPS}包含了除化工过程对产品和环境的影响之外的很多内容,如物流、基础设施和能耗与废物处理等\upcite{EP75-1536_2015,AE175-305_2016,CCE118-49_2018,CCE131-106586_2019,CCE130-106577_2019}。通过创建此数字化工具,产生了一个更宏大的规划,即\textrm{The World Avatar~(TWA)}项目\upcite{DCE1-e6_2020,DCE2-e10_2021}。\textrm{TWA}项目的层次规划如图\ref{Fig:Three_Layers-of-TWA}所示。
\begin{figure}[h!]
\centering
\includegraphics[height=3.55in,width=5.85in,viewport=0 0 1380 800,clip]{Three_Layers-of-TWA-digital-twin-real_world.png}
\caption{\small\textrm{Three layer of TWA (\url{www.theworldavatar.com}) digital twin of the real world. cite from\cite{ACR56-128_2023}}}%(与文献\cite{EPJB33-47_2003}图1对比)
\label{Fig:Three_Layers-of-TWA}
\end{figure}
\textrm{TWA}可以看成基于语义网技术的通用数字孪生\textrm{(Digital twins)}技术\footnote{数字孪生系统可以根据人员、设备或系统的基础,在信息化平台上创造一个数字版的“克隆体”,“克隆体”可以模拟实际设备或系统的发展走向。数字孪生的本质是信息建模,旨在为现实世界中的实体对象在数字虚拟世界中构建完全一致的数字模型,但数字孪生涉及的信息建模已不再是基于传统的底层信息传输格式的建模。}知识图谱映射到整个现实世界,\textrm{TWA}的数据满足\textrm{FAIR}原理\footnote{\textrm{FAIR}原理表示数据\textrm{Findable}、\textrm{Accessible}、\textrm{Interoperable}、\textrm{Reusable}\upcite{SD3-160018_2016}。}。

当前很多高科技公司,包括\textrm{Google}、\textrm{IBM}、\textrm{Microsoft}和\textrm{eBay}等都在应用企业级的知识图谱\upcite{Queue17-48_2019}。制药公司\textrm{AstraZeneca}是新药制备领域应用知识图谱的领先者\upcite{NC13-1667_2022}。

\section{作为知识生态组成的化学知识图谱}
文献\cite{ACR56-128_2023}详细讨论了作为\textrm{TWA}知识生态\textrm{(knowledge ecosystem)}构成的中化学知识图谱及相关软件。
\subsection{化学物种}
在\textrm{TWA}中,化学物种及其性质由\textrm{ontology~OntoSpecies}表示,如图\ref{Fig:OntoSpecies-to-segments-TWA}所示。\textrm{OntoSpecies}中的化学物种主要纪录分子式、电荷、分子量和自旋多重度。不同同位素、电荷和自旋态表示的化学物种不相同。因为\textrm{IRIs~(Internationalized Resource Identifiers)}包含\textrm{UUIDs~(universally unique identifiers)},因此可通过对化学物种赋不同的\textrm{IRIs},\textrm{OntoSpecies}就可以数字表示同位素,以区分实验的、氧化-还原与电化学驱动过程和光化学下的物种。对于反应物模拟,\textrm{OntoSpecies}纪录了特定反应条件下的反应标准生成焓\textrm{(standrd enthalpy)}。\upcite{CCE137-106813_2020}
\begin{figure}[h!]
\centering
\includegraphics[height=3.40in,width=5.85in,viewport=0 0 1170 700,clip]{Connection-of-OntoSpecies-to-segments-of-KG.png}
\caption{\small\textrm{Connection of OntoSpecies to other segments of TWA KG. cite from\cite{ACR56-128_2023}}}%(与文献\cite{EPJB33-47_2003}图1对比)
\label{Fig:OntoSpecies-to-segments-TWA}
\end{figure}

用\textrm{IRIs}标签化学物种是机器可操作的,但不方便化学家识别,因此\textrm{TWA}中的化学物种需要进一步添加通用化学信息学标签\upcite{COC26-33_2019},如\textrm{\textrm{InChI}}、\textrm{InChIKey}和\textrm{CAS}注册码、\textrm{PubChemCID}和\textrm{SMILES}等,如图\ref{Fig:Key-OntoSpecies-and-external-concepts}所示。
\begin{figure}[h!]
\centering
\includegraphics[height=3.20in,width=4.35in,viewport=0 0 990 750,clip]{Key_OntoSpecies-and-external_concepts.png}
\caption{\small\textrm{Key OntoSpecies (black) and external (blue) concepts, along with a number of properties (green) used to describe chemical species in TWA KG. cite from\cite{ACR56-128_2023}}}%(与文献\cite{EPJB33-47_2003}图1对比)
\label{Fig:Key-OntoSpecies-and-external-concepts}
\end{figure}

\subsection{反应复杂性研究}
很多化学反应和自组装过程都是由亚稳的简单分子前驱体(化学物种)引发的,对化学反应过程的模拟和反应机理的理解需要考虑热力学和动力学因素。为通过语义学研究化学反应的物种,\textrm{M.~Kraft}开发了\textrm{OntoKin}和\textrm{OntoCompChem}\upcite{JCIM59-3154_2019},如图\ref{Fig:Automated-linking-between-OntoSpecies-Kin-CompChem}所示。\textrm{OntoKin}是用于计算机辅助设计的标准命名下表示反应机理的算法的\textrm{ontology},化学反应过程中,反应机理由成比例的化学物种表示。在\textrm{OntoKin}中,化学反应由反应物和产物描述,反应物和产物由\textrm{OntoSpecoes~IRIs}的热力学和输运模型标签。在\textrm{OntoKin}中还会标记反应发生的形式(如气相、表面等)。每个化学反应的速度由\textrm{Arrhenius}模型表示,可通过调节温度、压力来控制气相反应动力学(即计算反应速率)。描述一个化学反应可能涉及很多反应步骤,所以\textrm{OntoKin}结合\textrm{OntoSpecies}就可以提供数据和模型,并可与文献中报导的动力学、热力学或输运模型对比\upcite{ACSO5-18342_2020}。所以这样的知识图谱框架可用于评估专家经验判断的可靠程度。\textrm{OntoCompChem}目前主要关注分子,图谱用密度泛函理论\textrm{(DFT)}的输入输出表示\upcite{JCIM59-3154_2019}。\textrm{OntoCompChem}是基于由\textrm{CompChem}规定的\textrm{CML~(Chemical Markup Language)}\upcite{JCI4-15_2012}的语义概念发展起来的。\textrm{OntoCompCHem}中对计算的描述包括:
\begin{itemize}
	\item 计算对象(单点计算,几何结构优化和频率计算)
	\item 使用的计算软件(如\textrm{Gaussian~16})
	\item 计算中采用的方法,包括泛函和基组(如\textrm{B3LYP,6-31G(d)})
	\item 电荷与自旋的极化
\end{itemize}
图谱中也纪录了前线轨道\textrm{frontier orbitals}和收敛的自洽迭代能量。对几何结构优化,纪录最终优化的几何结构;而频率计算,纪录的是零点能的校正和几何结构对应的完整的振动频率。

利用组织衔接软件\footnote{组织衔接软件是能够感知环境、进行决策和执行动作的智能处理软件,也称为\textrm{Agent}。该软件的设计和训练,需要结合机器学习和人工智能技术,如强化学习、深度学习等。},可以将化合物、化学反应和\textrm{DFT}衔接在一起\upcite{CCE137-106813_2020}。因为\textrm{OntoKin}中可能涉及到成千上万的化学物种和化学反应,所以衔接软件\textrm{Linking}是非常必要的\upcite{ACSO5-18342_2020}。组织衔接软件工作方式类似于人类代理:~能接收输入数据(如传感器信息、文本、图像等),通过分析和处理数据,理解环境和任务要求,并做出相应的决策和行动。通过与环境的交互和反馈,组织衔接软件可以逐步改进性能和表现,实现好的任务执行能力。组织衔接软件的核心功能是感知、推理和决策:
\begin{itemize}
	\item 感知:~通过传感器等方式获取环境信息的能力,例如通过摄像头获取图像或通过麦克风获取声音
	\item 推理:~基于获取的信息进行逻辑推理和分析的能力,以了解环境和任务需求
	\item 决策:~根据推理结果做出相应的决策,并执行相应的动作 
\end{itemize}
图\ref{Fig:Automated-linking-between-OntoSpecies-Kin-CompChem}给出火箭推进的氢燃料燃烧反应涉及的10种化学物种和40个基元反应,\textrm{Linking}软件可以追索每个反应,有一个反应是\ch{H2O2}~+\ch{*OH}~$\rightleftharpoons$~\ch{H2O}~+\ch{*OOH}。
\begin{figure}[h!]
\centering
\includegraphics[height=3.40in,width=4.05in,viewport=0 0 1010 750,clip]{Automated-linking-between-OntoSepcies-Kin-CompChem.png}
\caption{\small\textrm{Automated linking between OntoSpecies, OntoKin and OntoCompChem. cite from\cite{ACR56-128_2023}}}%(与文献\cite{EPJB33-47_2003}图1对比)
\label{Fig:Automated-linking-between-OntoSpecies-Kin-CompChem}
\end{figure}
\subsection{合理的自组装材料的自动化设计}
金属有机多面体\textrm{(Metal-organic polyhedra, MOPs)}是由有机的和金属的化学单元\textrm{(Chemical Building Units, CBUs)}组装,并重构成规则的多面体\upcite{CR120-8987_2020}。金属有机多面体和其它的笼状结构过去主要由领域行业专家设计的,为了设计新的金属有机多面体,专家必须考虑化学的和空间组成的影响,但小朋友搭玩具模型时并不具备多面体的几何知识背景,可见直觉想象力是参与推理的。如图\ref{Fig:OntoMOPs-MOPs}所示,在\textrm{OntoMOPs}中,将组装模型\textrm{(Assembly Models, AMs)}的概念和通用组装单元\textrm{(Generic Building Units, GBUs)}引入到\textrm{MOPs}的合理设计。而\textrm{MOPs}的化学单元也用\textrm{OntoSpecies}标注,并在图谱中标注该\textrm{MOPs}最初的出处文献。而\textrm{MOP}发现软件的算法思路是设置规定\textrm{CBUs}的允许组合,又不能有太大的应力。
\begin{figure}[h!]
\centering
\includegraphics[height=2.70in,width=5.05in,viewport=0 0 1330 700,clip]{Key_concepts-in-OntoMOPs-and-designed-MOPs.png}
\caption{\small\textrm{Key concepts in OntoMOPs (left) and examples of newly rationally designed MOPs (right). cite from\cite{ACR56-128_2023}}}%(与文献\cite{EPJB33-47_2003}图1对比)
\label{Fig:OntoMOPs-MOPs}
\end{figure}
\subsection{界面友好的\rm{TWA}知识图谱}
问答式知识图谱用的查询语言(如\textrm{SPARQL})并需要了解知识图谱中如何组织知识,因此对于缺少知识图谱基本了解的用户,缺乏使用知识图谱的动力,所以界面友好的知识图谱的需求非常迫切。在\textrm{TWA}项目中,化学-化工类的知识图谱中,问答模块\textrm{Marie}就是允许用户用自然语言提问,在后台问题描述的场景会转换成机器可读的问答形式\upcite{DCE3-100032_2022}。为了实现这些功能,\textrm{Marie}使用自然语言处理和网络软件来确定主题、问题类型和用户提问的实体。一旦问题被理清,软件会将相关信息提交给查询软件,查询软件会将信息转递到\textrm{SPARQL}软件,\textrm{SPARQL}软件查询知识图谱并向用户返回信息。图\ref{Fig:TWA-KG-Marie}给出典型的用户提问\textrm{Marie}要求展示芳香\ch{C-H}化合物模型。
\begin{figure}[h!]
\centering
\includegraphics[height=3.70in,width=3.35in,viewport=0 0 750 790,clip]{TWA-KG-Marie.png}
\caption{\small\textrm{Marie's back-end operations involved in querying information that is in TWA KG and one that it is generated through agent operation. Printed results of queries are adapted with permission from (i) ref \cite{JCIM61-3868_2021}. cite from\cite{ACR56-128_2023}}}%(与文献\cite{EPJB33-47_2003}图1对比)
\label{Fig:TWA-KG-Marie}
\end{figure}
为了节约知识存储的成本,很多知识可通过推演或计算得到。所以如果\textrm{Marie}在知识图谱中没有找到答案,它会采取另外的技术路线:~通过激活发现软件,该软件会寻找适合问题的软件,合适的软件再向知识图谱查询并计算相关结果。一个典型的例子是用户提问查询\ch{CO2}的热容时,软件\textrm{Thermo}会在诸如\textrm{OntoCompChem}中计算\ch{CO2}的热容。
\section{小结}
知识图谱结合处理软件可以实现复杂决策并通过计算、新的属性和自主的实验产生新的知识\upcite{JACSAu2-292_2022}。因此可以预见,在不远的将来,通过软件和知识图谱的复杂的组合会发现和创造新的分子或材料。因此如果化学知识形成的生态系统,将会通过化学知识空间的探索,有效地发现更多的新知识。
