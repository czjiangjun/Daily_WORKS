%---------------------- TEMPLATE FOR REPORT ------------------------------------------------------------------------------------------------------%

%\thispagestyle{fancy}   % 插入页眉页脚                                        %
%%%%%%%%%%%%%%%%%%%%%%%%%%%%% 用 authblk 包 支持作者和E-mail %%%%%%%%%%%%%%%%%%%%%%%%%%%%%%%%%
%\title{More than one Author with different Affiliations}				     %
%\title{\rm{VASP}的电荷密度存储文件\rm{CHGCAR}}
%\title{面向高温合金材料设计的计算模拟软件中的几个主要问题}
\title{\rm{VASP}中的\rm{FFT}处理}
\author[ ]{姜~骏}   %
%\author[ ]{姜~骏\thanks{jiangjun@bcc.ac.cn}}   %
%\affil[ ]{北京市计算中心}    %
%\author[a]{Author A}									     %
%\author[a]{Author B}									     %
%\author[a]{Author C \thanks{Corresponding author: email@mail.com}}			     %
%%\author[a]{Author/通讯作者 C \thanks{Corresponding author: cores-email@mail.com}}     	     %
%\author[b]{Author D}									     %
%\author[b]{Author/作者 D}								     %
%\author[b]{Author E}									     %
%\affil[a]{Department of Computer Science, \LaTeX\ University}				     %
%\affil[b]{Department of Mechanical Engineering, \LaTeX\ University}			     %
%\affil[b]{作者单位-2}			    						     %
											     %
%%% 使用 \thanks 定义通讯作者								     %
											     %
\renewcommand*{\Authfont}{\small\rm} % 修改作者的字体与大小				     %
\renewcommand*{\Affilfont}{\small\it} % 修改机构名称的字体与大小			     %
\renewcommand\Authands{ and } % 去掉 and 前的逗号					     %
\renewcommand\Authands{ , } % 将 and 换成逗号					     %
\date{} % 去掉日期									     %
%\date{2020-12-30}									     %

%%%%%%%%%%%%%%%%%%%%%%%%%%%%%%%%%%%%%%%%%%  不使用 authblk 包制作标题  %%%%%%%%%%%%%%%%%%%%%%%%%%%%%%%%%%%%%%%%%%%%%%
%-------------------------------The Title of The Report-----------------------------------------%
%\title{报告标题:~}   %
%-----------------------------------------------------------------------------

%----------------------The Authors and the address of The Paper--------------------------------%
%\author{
%\small
%Author1, Author2, Author3\footnote{Communication author's E-mail} \\    %Authors' Names	       %
%\small
%(The Address,City Post code)						%Address	       %
%}
%\affil[$\dagger$]{清华大学~材料加工研究所~A213}
%\affil{清华大学~材料加工研究所~A213}
%\date{}					%if necessary					       %
%----------------------------------------------------------------------------------------------%
%%%%%%%%%%%%%%%%%%%%%%%%%%%%%%%%%%%%%%%%%%%%%%%%%%%%%%%%%%%%%%%%%%%%%%%%%%%%%%%%%%%%%%%%%%%%%%%%%%%%%%%%%%%%%%%%%%%%%
\maketitle
%\thispagestyle{fancy}   % 首页插入页眉页脚 
\textrm{VASP}软件中
\begin{itemize}
	\item 波函数(\textrm{wave function})的\textrm{Fouriier}变换主要调用子程序\textrm{FFTWAV}完成(根据文件\textrm{symbol.inc}的宏定义,源代码\textrm{.F}中,\textrm{FFTWAV}在串行版表示\textrm{FFTWAV},并行版表示\textrm{FFTWAV\_MPI})。
	\item 电荷密度(\textrm{charge density})和势函数(\textrm{potential})的\textrm{Fouriier}变换主要调用子程序\textrm{FFT3D}完成(根据文件\textrm{symbol.inc}的宏定义,源代码\textrm{.F}中,\textrm{FFT3D}在串行版表示\textrm{FFT3D},并行版表示\textrm{FFT3D\_MPI})。
\end{itemize}
各部分代码的依赖关系如图\ref{Fig:MPI_FLOW}所示。在\textrm{VASP}软件中的\textrm{FFT}变换(串行和并行,主要是并行)所需的网格点由模块\textrm{mgrid}生成,变换过程则由\textrm{FFTD}和\textrm{FFTWAV}子程序完成。

\begin{figure}
%\tikzstyle{startstop} = [rectangle,rounded corners, minimum width=3cm,minimum height=1cm,text centered,text width =3cm, draw=black,fill=red!30]
%\tikzstyle{io} = [trapezium, trapezium left angle = 70,trapezium right angle=110,minimum width=3cm,minimum height=1cm,text centered,text width =3cm,draw=black,fill=blue!30]
\tikzstyle{process1} = [rectangle,minimum width=5cm,minimum height=1cm,text centered,text width =5cm,draw=black,fill=orange!30]
\tikzstyle{process2} = [rectangle,minimum width=4cm,minimum height=1cm,text centered,text width =4cm,draw=black,fill=blue!30]
\tikzstyle{process3} = [rectangle,minimum width=5cm,minimum height=1cm,text centered,text width =5cm,draw=black,fill=red!30]
%\tikzstyle{decision} = [diamond,aspect = 3,text centered,draw=black,fill=green!30]
\tikzstyle{arrow} = [thick,->,>=stealth]
\tikzstyle{straightline} = [line width = 1pt,-]
\tikzstyle{dashedline} = [dashed, line width = 1pt,-]
\tikzstyle{point}=[coordinate]

\begin{tikzpicture}[node distance=2cm]
%\node (start) [startstop] {开始};
%\node (input1) [io,below of=start] {输入聚类的个数 $k$ 和最大迭代次数 $n$ };
	\node (process1) [process1] {电荷密度、势函数的\textrm{FFT}变换:\\调用子程序\textrm{FFT3D}完成};
	\node (process2) [process2,right of=process1, xshift= 3cm] {波函数的\textrm{FFT}变换:\\调用子程序\\\textrm{FFTWAV}完成};
	\node (process3) [process3,below of=process1, xshift= -2cm,yshift=-2cm] {模块\textrm{mgrid}定义了\textrm{VASP}的\textrm{FFT}变换的网格规范,主要功能\\由子程序\textrm{GEN\_RC\_GRID}、\textrm{GEN\_SUB\_GRID\_}和\textrm{SET\_RL\_GRID}完成\\\textcolor{blue}{\textbf{是解析的重点}}(文件\textrm{mgrid.f90})};
	\node (process4) [process3,below of=process3,yshift=-1.0cm] {子程序\textrm{INIT\_MPI}确定\textrm{VASP}并行网格通信的初始构造(文件\textrm{main\_mpi.f90})};
	\node (process5) [process3,below of=process4] {\textrm{MODULE~mympi}定义\textrm{MPI}的网格通信模式(文件\textrm{mpi.f90})};
	\node (process6) [process1,below of=process1, xshift= 4cm] {子程序\textrm{FFT3D\_MPI}(文件\textrm{fftmpi.f90}和\textrm{fftmpiw.f90})};
	\node (process7) [process2,below of=process6, xshift= 5cm] {文件\textrm{fftw3d.F}提供\textrm{VASP}和\textrm{FFTW}的等价接口};
	\node (process8) [process2,below of=process6, yshift=-2cm] {\textrm{VASP}自带\textrm{FFT}库\\(文件\textrm{fftlib.f90})};
	\node (process9) [process1,below of=process7, yshift=-2cm] {通用\textrm{FFTW}库,提供诸如\\\textrm{dfftw\_plan\_many\_dft}等\\通用\textrm{FFTW}库函数};

%\node (decision1) [decision,below of=process2,yshift=-0.5cm] {是否收敛或迭代次数达到 $n$ };
%\node (stop) [startstop,below of=decision1,node distance=3cm] {输出聚类结果};
	\node(point1)[point,left of=process3 ,xshift =-1cm]{};
	\node(point2)[point,above of=process2]{};
	\node(point3)[point,left of=process1, xshift =-3cm]{};
	\node(point4)[point,left of=process6, xshift =-4cm]{};
	\node(point5)[point,left of=process9, xshift =-6cm]{};

%\draw [arrow] (start) -- (input1);
%\draw [arrow] (input1) -- (process1);
\draw [arrow] (process1) -- (process6);
\draw [arrow] (process2) -- (process6);
\draw [arrow] (process3) -- (process4);
\draw [dashedline] (process3.west) -- (point1);
\draw [dashedline] (point1) -- (point3);
\draw [dashedline] (point3) -- (process1.west);
\draw [dashedline] (process3) -- (point1);
\draw [dashedline] (point1) |- (point2);
\draw [dashedline] (point2) -- (process2.north);
\draw [dashedline] (process7.west) -- (process3.east);
\draw [dashedline] (process3.north) -- (point4);
\draw [dashedline] (point4) -- (process6.west);
\draw [arrow] (process4) -- (process5);
\draw [arrow] (process2.east) -| (process7.north);
\draw [arrow] (process6) -- (process8);
\draw [arrow] (process6) |- (process7);
\draw [arrow] (process7) |- (process8.east);
%\draw [arrow] (process8) -- node[anchor=south] {否} (process9);
\draw [arrow] (process7) -- (process9);
%\draw [arrow] (decision1) -- node[anchor=east] {是} (stop);
%\draw [straightline] (decision1) -|  (point1);
\draw [straightline] (process6) -|  (point5);
%\draw [arrow] (point1) -- node[anchor=south] {否} (input1);
\draw [arrow] (point5) -- (process9);
\end{tikzpicture}
\caption{\textrm{The Flow of FFT in VASP.}}
\label{Fig:MPI_FLOW}
\end{figure}
%%%%%%%%%%%%%%%%%%%%%%%%%%%%%%%%%%%%%%%%  绘制流程图  %%%%%%%%%%%%%%%%%%%%%%%%%%%%%%%%%%%%%%%%%%%%%%%%%%%%%%%%%%%%%%%%%
%\tikzstyle{startstop} = [rectangle,rounded corners, minimum width=3cm,minimum height=1cm,text centered,text width =3cm, draw=black,fill=red!30]
%\tikzstyle{io} = [trapezium, trapezium left angle = 70,trapezium right angle=110,minimum width=3cm,minimum height=1cm,text centered,text width =3cm,draw=black,fill=blue!30]
%\tikzstyle{process} = [rectangle,minimum width=3cm,minimum height=1cm,text centered,text width =3cm,draw=black,fill=orange!30]
%\tikzstyle{decision} = [diamond,aspect = 3,text centered,draw=black,fill=green!30]
%\tikzstyle{arrow} = [thick,->,>=stealth]
%\tikzstyle{straightline} = [line width = 1pt,-]
%\tikzstyle{point}=[coordinate]
%
%\begin{tikzpicture}[node distance=2cm]
%\node (start) [startstop] {开始};
%\node (input1) [io,below of=start] {输入聚类的个数 $k$ 和最大迭代次数 $n$ };
%\node (process1) [process,below of=input1] {初始化 $k$ 个聚类中心};
%\node (process2) [process,below of=process1] {分配各数据对象到距离最近的类中};
%\node (decision1) [decision,below of=process2,yshift=-0.5cm] {是否收敛或迭代次数达到 $n$ };
%\node (stop) [startstop,below of=decision1,node distance=3cm] {输出聚类结果};
%\node(point1)[point,left of=input1,node distance=5cm]{};
%
%\draw [arrow] (start) -- (input1);
%\draw [arrow] (input1) -- (process1);
%\draw [arrow] (process1) -- (process2);
%\draw [arrow] (process2) -- (decision1);
%\draw [arrow] (decision1) -- node[anchor=east] {是} (stop);
%\draw [straightline] (decision1) -|  (point1);
%\draw [arrow] (point1) -- node[anchor=south] {否} (input1);
%\end{tikzpicture}

%%%%%%%%%%%%%%%%%%%%%%%%%%%%%%%%%%%%%%%%  绘制流程图  %%%%%%%%%%%%%%%%%%%%%%%%%%%%%%%%%%%%%%%%%%%%%%%%%%%%%%%%%%%%%%%%%
%\begin{figure}
%\scriptsize
%\tikzstyle{format}=[rectangle,draw,thin,fill=white]
%%定义语句块的颜色,形状和边
%\tikzstyle{test}=[diamond,aspect=2,draw,thin]
%%定义条件块的形状,颜色
%\tikzstyle{point}=[coordinate,on grid,]
%%像素点,用于连接转移线
%\begin{tikzpicture}%[node distance=10mm,auto,>=late',thin,start chain=going below,every join/.style={norm},]
%%start chain=going below指明了流程图的默认方向,node distance=8mm则指明了默认的node距离。这些可以在定义node的时候更改,比如说
%%\node[point,right of=n3,node distance=10mm] (p0){};
%%这里声明了node p0,它在node n3 的右边,距离是10mm。
%\node[format] (start){Start};
%\node[format,below of=start,node distance=7mm] (define){Some defines};
%\node[format,below of=define,node distance=7mm] (PCFinit){PCF8563 Initialize};
%\node[format,below of=PCFinit,node distance=7mm] (DS18init){DS18 Initialize};
%\node[format,below of=DS18init,node distance=7mm] (LCDinit){LCD Initialize};
%\node[format,below of=LCDinit,node distance=7mm] (processtime){Processtime};
%\node[format,below of=processtime,node distance=7mm] (keyinit){Key Initialize};
%\node[test,below of=keyinit,node distance=15mm](setkeycheck){Check Set Key};
%\node[point,left of=setkeycheck,node distance=18mm](point3){};
%\node[format,below of=setkeycheck,node distance=15mm](readtime){Read Time};
%\node[point,right of=readtime,node distance=15mm](point4){};
%\node[format,below of=readtime](processtime1){Processtime};
%\node[format,below of=processtime1](gettemp){Get Temperature};
%\node[format,below of=gettemp](display){Display All Data};
%\node[format,right of=setkeycheck,node distance=40mm](setsetflag){Set SetFlag=1};
%\node[format,below of=setsetflag](setinit){Set Mode Initialize};
%\node[format,below of=setinit](checksetting){Checksetting()};
%\node[test,below of=checksetting,node distance=15mm](savecheck){Check Save Key};
%\node[format,below of=savecheck,node distance=15mm](clearsetflag){Clear SetFlag=0};
%\node[format,below of=clearsetflag](settime){Set Time};
%\node[point,below of=display,node distance=7mm](point1){};
%\node[point,below of=settime,node distance=7mm](point2){};
%%\node[format] (n0) at(4,4){A}; 直接指定位置
%%定义完node之后进行连线,
%%\draw[->] (n0.south) -- (n1); 带箭头实线
%%\draw[-] (n0.south) -- (n1); 不带箭头实线
%%\draw[&lt;->] (n0.south) -- (n1.north);   双箭头
%%\draw[&lt;-,dashed] (n1.south) -- (n2.north); 带箭头虚线 
%%\draw[&lt;-] (n0.south) to node{Yes} (n1.north);  带字,字在箭头方向右边
%%\draw[->] (n1.north) to node{Yes} (n0.south);  带字,字在箭头方向左边
%%\draw[->] (n1.north) to[out=60,in=300] node{Yes} (n0.south);  曲线
%%\draw[->,draw=red](n2)--(n1);  带颜色的线
%\draw[->] (start)--(define);
%\draw[->] (define)--(PCFinit);
%\draw[->](PCFinit)--(DS18init);
%\draw[->](DS18init)--(LCDinit);
%\draw[->](LCDinit)--(processtime);
%\draw[->](processtime)--(keyinit);
%\draw[->](keyinit)--(setkeycheck);
%\draw[->](setkeycheck)--node[above]{Yes}(setsetflag);
%\draw[->](setkeycheck) --node[left]{No} (readtime);
%\draw[->](readtime)--(processtime1);
%\draw[->](processtime1)--(gettemp);
%\draw[->](gettemp)--(display);
%\draw[-](display)--(point1);
%\draw[-](point1)-|(point3);
%\draw[->](point3)--(setkeycheck.west);
%\draw[->](setsetflag)--(setinit);
%\draw[->](setinit)--(checksetting);
%\draw[->](checksetting)--(savecheck);
%\draw[->](savecheck)--node[left]{Yes}(clearsetflag);
%\draw[->](savecheck.west)|-node[left]{No}(checksetting);
%\draw[->](clearsetflag)--(settime);
%\draw[-](settime)--(point2);
%\draw[-](point2)-|(point4);
%\draw[->](point4)--(readtime.east);
%\end{tikzpicture}
%\end{figure}
\section{\rm{FFT}变换的实现过程}\label{label:Section-1}
1.~子程序\textrm{FFT3D}\\
子程序格式为:~\textrm{FFT3D~(CR,~GRID,~ISIGN)},其参数含义\\
\begin{itemize}
	\item \textrm{CR}:~核心输入/输出参数,\textrm{FFT}变换对象
	\item \textrm{GRID}:~执行\textrm{FFT}变换的网格
	\item \textrm{ISIGN}:~-1实空间$\rightarrow$倒空间-变换/+1倒空间$\rightarrow$实空间-变换
\end{itemize}
\textrm{}

2.~子程序\textrm{FFTWAV}\\
子程序格式为:~\textrm{FFTWAV~(NPL,~NINDPW,~CR,~C,~GRID)},其参数含义\\
\begin{itemize}
	\item \textrm{NPL}:~用来展开波函数的平面波总数
	\item \textrm{NINDPW}:~用以波函数的平面波在\textrm{FFT}变换中对应关系
	\item \textrm{CR}:~实空间波函数
	\item \textrm{C}:~倒空间波函数
	\item \textrm{GRID}:~执行\textrm{FFT}变换的网格
\end{itemize}
\textcolor{red}{\textrm{FFTWAV}最终还是通过调用\textrm{FFT3D}完成变换}

因此解析\textrm{VASP}的\textrm{FFT}变换的核心就是解析\textrm{FFT3D}子程序(见\textrm{fftw3d.F}文件——文件编译中产生\textrm{fftw3d.f90}文件,该\textrm{.f90}文件包含了\textrm{fftw3d.F}和\textrm{fftw3dsimple.F}两部分)。

\begin{itemize}
	\item \textrm{FFT3D}通过调用子程序\textrm{FFTBAS},达到调用通用\textrm{FFT}库接口\textrm{dfftw\_plan\_dft\_3d}等的目的
	\item \textrm{FFT3D}通过调用\textrm{FFTBRC}达到调用\textrm{VASP}自带\textrm{FFT}库的目的
\end{itemize}

此外,\textrm{VASP}的主程序(\textrm{main.F}文件)中通过调用子程序\textrm{FFTMAKEPLAN}(类似地,有宏定义为\textrm{FFTMAKEPLAN\_MPI},分别在\textrm{fftmpi.F}文件和\textrm{fftmpiw.F}文件),可以调用子程序\textrm{FFTMAKEPLAN}(\textrm{fftw3d.f90}文件),达到通用\textrm{FFT}库接口\textrm{dfftw\_plan\_dft\_3d}等的目的

从上述梳理不难看出,图\ref{Fig:MPI_FLOW}右侧给出了\textrm{VASP}的\textrm{FFT}变换的主要函数(模块)调用关系。具体的并行\textrm{FFT}变换实施,则将涉及到\textrm{FFT}网格点在各节点的分配,这部分主要由图\ref{Fig:MPI_FLOW}的左侧完成。以下作必要说明。

\section{\rm{FFT}变换的网格分配与通信}
\textrm{VASP}软件除了在模块\textrm{mpimy}(\textrm{mpi.F}文件)中规定了一般并行节点的通信基本规则(包括主节点和各节点的通信,以及定义的二维\textrm{topology}~$\mathrm{NCPU}\times\mathrm{NPARA}$~通信),还对\textrm{FFT}网格作了一些特别的处理,保证了\textrm{FFT}变换过程中网格点的负载均衡,主要实现过程参见子程序\textrm{GEN\_RC\_GRID}、\textrm{GEN\_SUB\_GRID\_}和\textrm{SET\_RL\_GRID}(\textrm{mgrid.F}文件),这一网格分布以模块\textrm{mgrid}形式贯穿\textrm{FFT}变换(图\ref{Fig:MPI_FLOW}的虚线)主要思路包括两个方面
\begin{itemize}
	\item 倒空间-实空间的\textrm{FFT}变换
	\item 实空间的数据与网格分配
\end{itemize}

\textrm{VASP}计算中有大量的实空间-倒空间变换和运算,为了维持计算中的高精度,\textrm{VASP}的\textrm{FFT}变换使用了两套网格:~密网格(\textrm{supergrid})和疏网格(\textrm{subgrid})。两套网格的三维网格点(\textrm{GX},~\textrm{GY},~\textrm{GZ})遵守的格点分布模式如图\ref{Fig:MPI-FFT}的上图所示,即粗网格和密网格向对应的平面映射到同一计算节点上,并保留变换列表(轮询对称算法),由此保证了倒空间-实空间\textrm{FFT}变换的负载均衡,这一网格分布形式由子程序\textrm{GEN\_RC\_GRID}和子程序\textrm{GEN\_SUB\_GRID\_}实现。
\begin{figure}[h!]
\centering
%\includegraphics[height=2.7in,width=4.0in,viewport=0 0 1180 875,clip]{Figures/dual_grid.png}
\includegraphics[height=1.5in,width=6.0in,viewport=0 0 1500 450,clip]{VASP_FFT-MPI_Reciprocal.png}
\vskip 0.5pt
\includegraphics[height=1.2in,width=5.0in,viewport=0 0 730 150,clip]{VASP_FFT-MPI_Real.png}
\caption{\textrm{VASP:~ Reciprocal-Real space layout for grids in MPI.}}%(与文献\cite{EPJB33-47_2003}图1对比)
\label{Fig:MPI-FFT}
\end{figure} 
针对\textrm{VASP}计算中的二维\textrm{topology}网格,在完成倒空间-实空间变换后,数据将会存于主节点带内通信节点内(此处称为原始网格),然后经过子程序\textrm{SET\_RL\_GRID},将数据映射到并行各节点内(此处称为目标网格),允许全部节点内的通信,如图\ref{Fig:MPI-FFT}的下图所示,这样就是实现了变换后实空间数据在计算节点上的均衡负载。

\section{小结}
\textrm{VASP}中的基函数是平面波,但主要计算都是在实空间完成的,所以程序中充满了大量的倒空间-实空间的\textrm{FFT}变换,程序在实现过程中为了兼顾计算软件的特点,在实现\textrm{FFT}变换中应用了大量的独特设计,而不是简单地利用通用\textrm{FFTW}库中的函数支持。在解析\textrm{VASP}的\textrm{FFT}变换实现代码中,重点解析的代码包括
\begin{itemize}
	\item \textrm{MPI}通信规则主要由\textrm{mpimy}模块(\textrm{mpi.f90}文件)确定,二维\textrm{topology}通信规范的设计,主要在\textrm{main\_mpi.F}文件中设计完成
	\item \textrm{FFT}网格分布和通信模式设计,主要是\textrm{mgrid.F}文件,特别是其中的疏密网格分配及对应的映射关系,主要是子程序:~\textrm{GEN\_SUB\_GRID\_}
	\item \textrm{FFT}变换的实现,主要是\textrm{fftw3d.f90}文件(涵盖了\textrm{fft3dsimple.F}和\textrm{ffw3d.F}文件)
\end{itemize}
\textcolor{red}{注意}:~因为这部分子程序传递的变量和参数基本上都是描述网格/子网格的映射和变换关系,与\textrm{VASP}的物理变量关系不大,所以未作专门说明。在主体程序中,\textrm{FFT}变换是通过调用\textrm{FFT3D/FFT3D\_MPI}来完成的,\textrm{FFT3D}的变量和参数解析见\ref{label:Section-1}部分。

以上,是\textrm{VASP}中\textrm{FFT}变换的总体实现思想,具体细节后续讨论
