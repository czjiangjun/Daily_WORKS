\documentclass[14pt]{article}      % Specifies the document class

%%%%%%%%%%%%%%%%% CJK 中文版面控制  %%%%%%%%%%%%%%%%%%%%%%%%%%%%%%
%\usepackage{CJK} % CTEX-CJK 中文支持                            %
\usepackage{xeCJK} % seperate the english and chinese		 %
%\usepackage{CJKutf8} % Texlive 中文支持                         %
\usepackage{CJKnumb} %中文序号                                   %
\usepackage{indentfirst} % 中文段落首行缩进                      %
%\setlength\parindent{22pt}       % 段落起始缩进量               %
\renewcommand{\baselinestretch}{1.2} % 中文行间距调整            %
\setlength{\textwidth}{16cm}                                     %
\setlength{\textheight}{24cm}                                    %
\setlength{\topmargin}{-1cm}                                     %
\setlength{\oddsidemargin}{0.1cm}                                %
\setlength{\evensidemargin}{\oddsidemargin}                      %
%%%%%%%%%%%%%%%%%%%%%%%%%%%%%%%%%%%%%%%%%%%%%%%%%%%%%%%%%%%%%%%%%%

\usepackage{amsmath,amsthm,amsfonts,amssymb,bm}          %数学公式
\usepackage{mathrsfs}                                    %英文花体
\usepackage{xcolor}                                        %使用默认允许使用颜色
%\usepackage{hyperref} 
\usepackage{graphicx}
\usepackage{subfigure}           %图片跨页
\usepackage{animate}		 %插入动画
\usepackage{caption}
\captionsetup{font=footnotesize}

%\usepackage[version=3]{mhchem}		%化学公式
\usepackage{chemfig}		%化学公式

\usepackage{fontspec} % use to set font
\setCJKmainfont{SimSun}
\XeTeXlinebreaklocale "zh"  % Auto linebreak for chinese
\XeTeXlinebreakskip = 0pt plus 1pt % Auto linebreak for chinese

\usepackage{longtable}                                   %使用长表格
\usepackage{multirow}
\usepackage{makecell}		%允许单元格内换行

%%%%%%%%%%%%%%%%%%%%%%%%%  参考文献引用 %%%%%%%%%%%%%%%%%%%%%%%%%%%
%%尽量使用 BibTeX(含有超链接,数据库的条目URL即可)                %
%%%%%%%%%%%%%%%%%%%%%%%%%%%%%%%%%%%%%%%%%%%%%%%%%%%%%%%%%%%%%%%%%%%

\usepackage[numbers,sort&compress]{natbib} %紧密排列             %
\usepackage[sectionbib]{chapterbib}        %每章节单独参考文献   %
\usepackage{hypernat}                                                                         %
\usepackage[bookmarksopen=true,pdfstartview=FitH,CJKbookmarks]{hyperref}              %
\hypersetup{bookmarksnumbered,colorlinks,linkcolor=green,citecolor=blue,urlcolor=red}         %
%参考文献含有超链接引用时需要下列宏包,注意与natbib有冲突        %
%\usepackage[dvipdfm]{hyperref}                                  %
%\usepackage{hypernat}                                           %
\newcommand{\upcite}[1]{\hspace{0ex}\textsuperscript{\cite{#1}}} %

%%%%%%%%%%%%%%%%%%%%%%%%%%%%%%%%%%%%%%%%%%%%%%%%%%%%%%%%%%%%%%%%%%%%%%%%%%%%%%%%%%%%%%%%%%%%%%%
%\AtBeginDvi{\special{pdf:tounicode GBK-EUC-UCS2}} %CTEX用dvipdfmx的话,用该命令可以解决      %
%						   %pdf书签的中文乱码问题		      %
%%%%%%%%%%%%%%%%%%%%%%%%%%%%%%%%%%%%%%%%%%%%%%%%%%%%%%%%%%%%%%%%%%%%%%%%%%%%%%%%%%%%%%%%%%%%%%%

%%%%%%%%%%%%%%%%%%%%%  % EPS 图片支持  %%%%%%%%%%%%%%%%%%%%%%%%%%%
\usepackage{graphicx}                                            %
%%%%%%%%%%%%%%%%%%%%%%%%%%%%%%%%%%%%%%%%%%%%%%%%%%%%%%%%%%%%%%%%%%


\begin{document}
%\CJKindent     %在CJK环境中,中文段落起始缩进2个中文字符
%\indent
\graphicspath{{Figures/}}
%
\renewcommand{\abstractname}{\small{\CJKfamily{hei} 提\quad 要}} %\CJKfamily{hei} 设置中文字体,字号用\big \small来设
\renewcommand{\refname}{\centering\CJKfamily{hei} 参考文献}
%\renewcommand{\figurename}{\CJKfamily{hei} 图.}
\renewcommand{\figurename}{{\bf Fig}.}
%\renewcommand{\tablename}{\CJKfamily{hei} 表.}
\renewcommand{\tablename}{{\bf Tab}.}

%将图表的Caption写成 图(表) Num. 格式
\makeatletter
\long\def\@makecaption#1#2{%
  \vskip\abovecaptionskip
  \sbox\@tempboxa{#1. #2}%
  \ifdim \wd\@tempboxa >\hsize
    #1. #2\par
  \else
    \global \@minipagefalse
    \hb@xt@\hsize{\hfil\box\@tempboxa\hfil}%
  \fi
  \vskip\belowcaptionskip}
\makeatother

\newcommand{\keywords}[1]{{\hspace{0\ccwd}\small{\CJKfamily{hei} 关键词:}{\hspace{2ex}{#1}}\bigskip}}

%%%%%%%%%%%%%%%%%%中文字体设置%%%%%%%%%%%%%%%%%%%%%%%%%%%
%默认字体 defalut fonts \TeX 是一种排版工具 \\		%
%{\bfseries 粗体 bold \TeX 是一种排版工具} \\		%
%{\CJKfamily{song}宋体 songti \TeX 是一种排版工具} \\	%
%{\CJKfamily{hei} 黑体 heiti \TeX 是一种排版工具} \\	%
%{\CJKfamily{kai} 楷书 kaishu \TeX 是一种排版工具} \\	%
%{\CJKfamily{fs} 仿宋 fangsong \TeX 是一种排版工具} \\	%
%%%%%%%%%%%%%%%%%%%%%%%%%%%%%%%%%%%%%%%%%%%%%%%%%%%%%%%%%

%\addcontentsline{toc}{section}{Bibliography}

%-------------------------------The Title of The Paper-----------------------------------------%
\title{文献Comp. Mat. Sci., 6, 15-50, (1996)概要}
%----------------------------------------------------------------------------------------------%

%----------------------The Authors and the address of The Paper--------------------------------%
\author{
\small
%Author1, Author2, Author3\footnote{Communication author's E-mail} \\    %Authors' Names	       %
\small
%(The Address,City Post code)						%Address	       %
}
\date{}					%if necessary					       %
%----------------------------------------------------------------------------------------------%
\maketitle

%-------------------------------------------------------------------------------The Abstract and the keywords of The Paper----------------------------------------------------------------------------%
\begin{abstract}
%The content of the abstract
	本文详细描述和对比了基于赝势-平面波的第一原理量子力学计算中的有关算法。主要讨论了(\textrm{a})(金属-导体体系)的能带中电子分数占据处理方案:四面体方法和有限温度的\textrm{DFT};~(\textrm{b})\textrm{Kohn-Sham}方程求解的矩阵迭代对角化方法和基于\textrm{Pulay}的残矢最小化(\textrm{Residual minimization, RMM})的高效迭代方法;~(\textrm{c})高效的\textrm{Broyden}和\textrm{Pulay}电荷密度混合方案,并提出了针对平面波基组的具体的预处理优化方法;~(\textrm{d})基于共轭梯度(\textrm{Conjugate gradient, CG})的直接全自由度最小化电子自由能。

	本文的结构严谨,章节关联逻辑清晰,建议对第一原理计算方法有兴趣的同志们仔细阅读。
\end{abstract}

%\keywords {Keyword1; Keyword2; Keyword3}
%-----------------------------------------------------------------------------------------------------------------------------------------------------------------------------------------------------%

根据本文结构,以下将本文中涉及算法章节部分概述如下
%----------------------------------------------------------------------------------------The Body Of The Paper----------------------------------------------------------------------------------------%
%Introduction
\section{Introduction}
\subsection{General}
第一段主要介绍了第一原理计算的优势,特别是注意到结合\textrm{Hellmann-Feynman}定理,考虑原子的受力,可以将第一原理应用到电子和(分子)动力学性质的研究中。文中评论指出\textrm{Car-Parrinello(CP)}方法的重要地位,也指出其局限:~电子态必须处于基态附近,电子态的激发要高于离子运动状态(确保电子与离子的运动绝热解耦(\textrm{decouple adiabatically}))

另一种求解思路,即每次离子移动后精确求解电子基态(基于\textrm{Born-Oppenheimer}近似)。作者指出该方法可行的前提是电子基态计算的算法足够高效。这些方法为了保持离子受力计算的准确与高效,保留了平面波基组,因此电子计算~\textrm{Hamiltonian}~可以快速处理(分别在实空间和倒空间中),这也使得迭代算法比起直接对角化方法更为高效:~这里明确了讨论求解电子态\textrm{KS}能量泛函的量中方式
\begin{itemize}
	\item 直接法:~直接最小化\textrm{KS}能量泛函
	\item 自洽法:~迭代求解\textrm{KS}方程并要求密度混合直至完全自洽
\end{itemize}

直接法主要思想是电子能量泛函在基态时取极小,因此能量泛函对全部自由度变分可以方便地计算电子基态,唯一需要考虑的是电子波函数的正交,在原始的\textrm{CP}方法中,该约束条件是通过\textrm{Lagrange}乘子引入。总的来说,单纯计算电子态,标准的\textrm{CP}算法并不快,因此有了一些改进,作者简要的\textrm{review}了有关改进方案,见文献\textrm{[7-9]}。

比\textrm{CP}方法更有希望的是\textrm{CG}方法,根据\textrm{CG}方法,\textrm{KS}能量泛函首先沿指定搜索方向作最小化,随后新的最小化方向保持与前次搜索方向共轭。\textrm{CG}方法的主要问题正交归一条件不容易满足。文献[10]针对半导体和绝缘体,给出每个能带电子能量逐个优化的算法,在该方案中,总能对子空间中单个能带最小化。该方法耗内存少但相对较慢,因为每个轨道只执行有限步数的\textrm{CG},但每个轨道每次更新后都需要重新计算电荷密度和势。因此同时\textrm{update}全部轨道的算法更优越。这类算法详见文献[11-12],文献[13]包含正交化条件最系统亦最漂亮。

本文的计算表明,自洽法(对角化矩阵结合密度混合)比直接法高效得多。特别是对于金属体系,效果更明显。乍看起来自洽方法的数学上似乎更粗糙,而且\textrm{KS}能量泛函最小化变成体系密度和能量本征值的分别迭代\textrm{update}。但是考虑以下几点,情况则不同了
\begin{enumerate}
	\item 迭代对角化\textrm{KS-Hamilitonian}算法比起最小化总能泛函要成熟得多
	\item 更重要的是:~电荷密度自洽混合可以保留此前各步的混合信息。这是自洽法与直接法的重要区别。\\
		直接法只保留了前面一到两步的信息,即使\textrm{CG}方法可以通过保持两次搜索方向共轭,部分克服直接法的不足,但\textrm{CG}方法的计算效率收到限制,因为需要确定精确的搜索方向。对于金属-导体体系,这是非常困难的,因此影响了收敛速度。
\end{enumerate}
作者将自洽法成功地应用于各种体系。高效电子能量最小化的一个优势是可以很快将离子弛豫到平衡态。作者提及了相关文献[19-21],证明该方法应用离子弛豫也是可行的。最后作者计算了绝缘体和金属的体相-声子,表明用迭代法可以快速准确地计算离子受力。

\subsection{Outline of the paper}
本文重点讨论\textrm{KS-Hamiltonian}的迭代对角化和电荷密度的混合(自洽法)。首先概要介绍\textrm{Kohn-Sham}能量泛函(\textrm{section~2.1})和分数点着占据对能量泛函的影响(\textrm{section~2.2}),介绍了改进四面体方法,并将四面体方法与有限温度\textrm{DFT}方法作对比。\textrm{section~2.3}给出自洽迭代循环框图,随后\textrm{section~2.4}简要讨论了\textrm{Hellmann-Feyman}定理,并给出受力计算的重要技术细节。

\textrm{Section~3}给出多种不同迭代矩阵对角化方法的更深入的讨论和对比。作者除了\textrm{review}各种方法,还指出了其在应用中的重要技术问题。此外作者特别注意到了\textrm{DIIS}方法。

\textrm{Section~4}主要讨论了电荷密度混合问题,作者特别讨论了\textrm{Broyden}混合方案(特别是\textrm{Broyden}第二混合法,即\textrm{Jacobian}求逆update),\textrm{Pulay}混合方案,并指出两种方法的密切关联。此外,对于平面波基,作者提出了一个具体的优化标准。

最后在\textrm{Section~5}讨论了直接最小化能量泛函的方法,特别是用共轭梯度法的实现。

在\textrm{Section~6}通过算例对比了各种方法。(略)

\section{分数占据与Kohn-Sham能量泛函}
\subsection{The Kohn-Sham~能量泛函}
概要介绍了\textrm{Kohn-Sham}能量泛函的定义和基于超软赝势\textrm{US-PP}的\textrm{Kohn-Sham}方程表达式和能量泛函。作者指出,因为电子分数占据的存在,电子布居数$f_n$成为计算\textrm{Kohn-Sham}能量本征值方程时的额外变分自由度,对于金属-导体来说,因为考虑电子分数占据,基态计算变得更加复杂。
\subsection{Metallic systems and partial occupancies}
简要介绍了金属-导体的\textrm{Fermi~}面分布和离散逼近计算的问题
\subsubsection{Linear tetrahedron methon}
概要介绍了线性四面体插值积分方法的基本思想,讨论了四面体插值积分的优势,以及对离子受力计算的问题(\textrm{Bl\"ochl}改进的四面体方法\textrm{v.s.}传统的四面体方法),改进四面体方法因为引入占据数校正破坏了对占据数变分的一些特性,导致在\textrm{US-PP}下计算受力非常不便。
\subsubsection{Finite-temperature approaches~-~'smearing methods'}
分别介绍了用\textrm{Fermi-Dirac}分布函数和\textrm{Gaussian}分布函数代替阶跃函数(\textrm{step function}),引入形式的“有限电子温度”计算总能泛函。注意到因为温度的影响,变分函数由总能变成自由能(\textrm{Free energy functional})\textbf{F}。在此处讨论了部分分布函数下熵\textbf{S}的定义,以及熵\textbf{S}($\sigma$)随展宽参数$\sigma$的变化
\subsubsection{Improved functional form for $f(\epsilon)$~-~Method of Methfessel and Paxton}
考虑引入\textrm{Hermite}多项式逼近分布函数(\textrm{Methfessel-Paxton},MP),更精确地考虑温度的影响和力的计算,给出自由能\textbf{F}($\sigma$)随展宽系数的表达式
\subsubsection{Convergence of the total energy with number of k-points}
用算例验证了四面体方法\textrm{v.s.}“有限温度展宽”方法的能量泛函$E(\vec k)$随$\vec k$点数变化的收敛情况
\subsubsection{Calculation of phonon frequencies for metals}
说明改进的四面体方法因为力的计算存在问题,不适合用于有关声子频率的计算,传统四面体方法算声子频率需要的$\vec k$点比较多,所以总的来说,声子计算建议用\textrm{MP}方法最合适
\subsection{Self-consistency loop and iterative methods}
\textrm{KS-Hamiltonian~}迭代对角化结合密度混合的主要流程列于\textrm{Fig.3}。初始电荷密度采用原子电荷密度叠加,应用\textrm{US-PP}计算公式得到总自由能公式\textcolor{blue}{\textrm{(30)}}。作者指出,据此方案得到的总自由能应与\textcolor{magenta}{\textrm{Harris-Foulkes(HF)}非自洽\textrm{DFT}}在给定电荷密度时的结果一致。

为了得到自洽的精确\textrm{KS~}基态能量,电荷密度残矢$R[\rho_{\mathrm in}]$\textcolor{blue}{\textrm{(31)}}
\begin{displaymath}
	R[\rho_{\mathrm in}]=\rho_{\mathrm{out}}-\rho_{\mathrm{in}}
\end{displaymath}
为零。原则上,迭代计算中,为了得到使$\rho_{\mathrm{out}}$唯一(确定),本征波函数$\phi_n$必须精确计算,但实际上没有必要这样做,只要确保前次混合-迭代得到的波函数作为新的迭代的尝试波函数,只要(少数)几次矩阵迭代对角化即可得到可靠的电荷密度残矢$R[\rho]$。在\textrm{Section~3}将集中介绍不同的\textrm{KS-Hamiltonian}矩阵迭代对角化方法,\textrm{Section~4}姜讨论电荷密度混合方法。
\subsection{Forces}
推导\textrm{US-PP}下离子受力计算公式(略)

\section{Iterative method for the diagonalization of the KS-Hamiltonian}
本节作者讨论了多种\textrm{KS-Hamiltonian~}矩阵迭代对角化方法,包括适用于大体系的\textrm{RMM-DIIS}方法。对于超软赝势,求解广义本征值问题\textcolor{blue}{(42)}
\begin{displaymath}
	\mathbf{H}|\phi_n\rangle=\epsilon_n\mathbf{S}|\phi_n\rangle
\end{displaymath}
如果基组很小,本征值可以通过直接对角化\textrm{Hamiltonian}得到,但对于大的矩阵,直接对角化的计算量将是$N_{\mathrm{plw}}^3$,根本没法处理。其中$N_{\mathrm{plw}}$是平面波基组的数目。迭代对角化比直接对角化要快得多,原因是
\begin{itemize}
	\item 只需计算$N_{\mathrm{b}}<<N_{\mathrm{plw}}$个轨道
	\item 对平面波基组,$\mathrm{H}|\phi_n$的计算特别方便
	\item 矩阵迭代对角化结合电荷密度优化非常高效,因为波函数和电荷密度几乎是同时优化的
\end{itemize}
文献[44]对于迭代对角化方法做了很好的\textrm{review},本文将采用文献[44]的记号。此外还介绍了新的迭代方法(逐个能带\textrm{CG}方法),并将不同的方法对比。

正如文献[44]所指出,绝大部分迭代算法开始时有一个展开基$\{|b_i\rangle,i=1,2,\cdots N_{\mathrm{a}}\}$,得到最佳近似波函数和本征值,展开基$N_{\mathrm{a}}<<N_{\mathrm{plw}}$,并且$N_{\mathrm a}$相比于能带数目$N_{\mathrm{b}}$可大可小。\textcolor{red}{每一次迭代,都会有新的矢量加入到展开基中}。迭代对角化可分为块迭代、非块迭代对角化(能带序列)。非块迭代对角化通常从一个矢量$|b_0\rangle$(用于近似本征矢$|\phi_n\rangle$)开始,每经过一次迭代$i$,便将一个修正矢量$|b_i\rangle$加入到展开基中。块方法在每次迭代中加入$N$个矢量,每次同时优化$N$个轨道。

在绝大部分块和非块方法中,首先必须从初步展开基组中得到本征矢和本征值的好的近似,一般可通过\textrm{Rayleigh-Ritz}变分法得到:~即\textrm{Hamiltonian~}在展开基张成的空间中展开为$N_{\mathrm{a}}\times N_{\mathrm{a}}$本征值问题\textcolor{blue}{(43)}
\begin{displaymath}
	\sum_m\bar{H}_{nm}B_{mk}=\sum_m\epsilon_k^{\mathrm{app}}\bar{S}_{nm}B_{mk}
\end{displaymath}
这里$\bar{H}_{nm}$和$\bar{S}_{nm}$按下式\textcolor{blue}{(44)}计算
\begin{displaymath}
	\bar{H}_{nm}=\langle b_n|\mathbf{H}|b_m\rangle,\qquad\bar{S}_{nm}=\langle b_n|\mathbf{S}|b_m\rangle
\end{displaymath}
因此在子空间中与$m$个精确本征值和本征态对应的近似$m$本征值/本征矢对由下式\textcolor{blue}{(45)}计算
\begin{displaymath}
	\epsilon_k^{\mathrm{app}},|\bar{b}_k\rangle=\sum_mB_{mk}|b_m\rangle
\end{displaymath}
\subsection{Residual vector and preconditioning}
\textcolor{red}{所有迭代对角化方法的关键步骤是计算每次迭代引入展开基中的修正矢量},所有迭代计算都为的是计算\textrm{Rayleigh~}商,\textrm{Rayleigh~}商的定义为\textcolor{blue}{(46)}
\begin{displaymath}
	\epsilon_{\mathrm{app}}=\dfrac{\langle\phi_n|\mathbf{H}|\phi_n\rangle}{\langle\phi_n|\mathbf{S}|\phi_n\rangle}
\end{displaymath}
该值在精确本征矢处取极值(鞍点),因此在约束条件$\langle\phi_n|\mathbf{S}|\phi_n\rangle=1$下,求\textrm{Rayleigh~}商对$\langle\phi_n|$的变分,可得残矢的定义\textcolor{blue}{(47)}
\begin{displaymath}
	|R(\phi_n)\rangle=(\mathbf{H}-\epsilon_{\mathrm{app}}\mathbf{S})|\phi_n\rangle
\end{displaymath}
残矢模量$\langle R|R\rangle$可作为本征矢误差的一种衡量指标。一般地,近似本征矢$|\phi_n\rangle$对精确本征矢的差分可由残矢计算\textcolor{blue}{(48)}
\begin{displaymath}
	|\delta\phi_n\rangle=-\dfrac1{\mathbf{H}-\epsilon_{\mathrm{app}}\mathbf{S}}|R\rangle
\end{displaymath}
显然,考虑到精确本征矢满足\textcolor{blue}{(49)}
\begin{displaymath}
	0=(\mathbf{H}-\epsilon_{\mathrm{app}}\mathbf{S})|\bar{\phi}_n\rangle
\end{displaymath}
根据差分关系$|\bar{\phi}_n\rangle=|\phi_n\rangle+|\delta\phi_n\rangle$,可实现残矢最小化。

要命的是矩阵$\mathbf{H}-\epsilon_{\mathrm{app}}\mathbf{S}$求逆并不比对角化矩阵$\mathbf{H}$简单,对于大的矩阵也非常麻烦,因此有必要作进一步的近似。以下\textcolor{magenta}{将由残矢计算近似差分$|\delta\phi_n]\rangle$称为\textrm{preconditioning}}。本征矢差分$|\delta\phi_n\rangle$定义为矩阵$\mathbf{K}$乘残矢得到\textcolor{blue}{(50)}
\begin{displaymath}
	|\delta\phi_n\rangle=\mathbf{K}|R\rangle
\end{displaymath}
这里矩阵$\mathbf{K}$称为预处理矩阵(\textrm{preconditioning matrix})。

一般来说,式\textcolor{blue}{(49)}只考虑对角元,因此\textcolor{blue}{(51)}
\begin{displaymath}
	\mathbf{K}=-\sum_q\dfrac{|\mathbf{q}\rangle\langle\mathbf{q}|}{\langle\mathbf{q}|\mathbf{H}-\epsilon_{\mathrm{app}}\mathbf{S}|\mathbf{q}\rangle}
\end{displaymath}
这里$q$遍历全部平面波基组。式\textcolor{blue}{(51)}也可以选择另一套完备基组,注意到如果$|\mathbf{q}\rangle$用\textrm{Hamiltonian~}$\mathbf{H}$的精确本征矢代替,式\textcolor{blue}{(51)}与\textcolor{blue}{(48)}等价。作者评价了文献[44]中的\textrm{preconditioning matrix}\textcolor{blue}{(52)}的优点与不足:~适用于中小体系($N_{\mathrm{plw}}<1000$),但对于大体系($N_{\mathrm{plw}}>1000$)则明显没有优势。在此基础上,作者将文献[10]提出的经验\textrm{preconditioning function}稍作改动\textcolor{blue}{(53)}
\begin{displaymath}
	\mathbf{K}=-\sum_q\dfrac{2|\mathbf{q}\rangle\langle\mathbf{q}|}{\frac32E^{\mathrm{kin}}(R)}\times\dfrac{27+18x+12x^2+8x^3}{27+18x+12x^2+8x^3+16x^4}
\end{displaymath}
这里$x=\dfrac{\hbar^2}{2m_{\mathrm{e}}}\dfrac{q^2}{\frac32E^{\mathrm{kin}}(R)}$,$E^{\mathrm{kin}}(R)$是残矢动能。作者补充说明了有关改动的原因,此外也提到了存在的其他类似的\textrm{preconditioning function}形式\textcolor{blue}{(55)}。

\subsection{Blocked Davidson scheme}
该方法最初由\textrm{Davidson~}提出,最近改造成可同时\textrm{update}全部能带的方案。展开基经过$M$步迭代,会增大$N_{\mathrm{b}}$个基函数\textcolor{blue}{(56)}
\begin{displaymath}
	\{|b_i\rangle\cdots,i=1,N_{\mathrm{b}}(M+1)\}=\{|\phi_i^0\rangle,i=1,\cdots,N_{\mathrm{b}}/|P_i^0\rangle,i=1,\cdots,N_{\mathrm{b}}/|P_i^1\rangle,i=1,\cdots,N_{\mathrm{b}}/\cdots\}
\end{displaymath}
$|P_i^0\rangle=\mathbf{M}|R(\phi_i^0\rangle)$是对于初始波函数的预处理残矢(\textrm{preconditioned residual vectors})。第$M$步迭代,通过\textrm{Rayleigh-Ritz}变分得到最低本征值/本征态$\epsilon_i^M,|\phi_i^M$。由这些本征矢计算得到预处理残矢$|P_i^M\rangle=\mathbf{M}R(\phi_i^M)$,并将残矢加入展开基中。对于大体系计算,存储这样的展开基也会是很大的开销,\textcolor{magenta}{因此$M$的值不能太大}。作者一般取$M=1$:~每一步最后的展开基组都是由2$N_{\mathrm{b}}$个矢量(初始基组$\{|\phi_i^0\rangle\}$和$\{|P_i^0\rangle\}$)构成,对角化$2N_{\mathrm{b}}$维子空间并计算$N_{\mathrm{b}}$个最低本征态;~下一步,这些新的本征矢和新的残矢又构成新的展开基。对于自洽计算来说,两次电荷密度混合之间,只需要有两步迭代即可满足要求。作者将此算法命名为\textrm{DAV2}。

\subsection{Unblocked algorithms}
一般认为,非块或序列算法比块对角化要慢,但由于块对角需要存储至少$2N_{\mathrm{b}}$个矢量,对大体系来说,这是比较麻烦的。每个时间步内优化一个能带需要的存储开销小得多,对于大体系这是更高效的处理方式,因为迭代次数可以比\textrm{Davidson}块对角化方法可以多。一般地,在非块对角化方法中,将变分方向约束在与当前波函数构成的子空间正交的方向会更有效,\textcolor{magenta}{经过一个能带、一个能带的顺序\textrm{update}全部波函数后,必须采用\textrm{Rayleigh-Ritz}变分,在最终确定的$N_{\mathrm{b}}$个尝试波函数$\{|\phi_n\rangle,n=1,\cdots,N_{\mathrm{b}}\}$构成的子空间中得到优化的波函数,这一步称为子空间对角化(\textrm{sub-space diagonalization})或子空间旋转(\textrm{sub-space rotation})}。在高效的序列算法中,子空间旋转和序列\textrm{update}是轮替进行的。

\subsection{Correction vector for sequential, band-by-band methods}
在序列算法中修正矢量(correction vector)的选择必须更谨慎。具体地,考虑某个待优化的能带$m$:~序列算法中,为方便起见,将优化能带$m$的搜索方向选为与当前尝试基组$\{|\phi_n\rangle,n=1,\cdots,N_{\mathrm{b}}\}$正交的方向上。根据\textrm{Lagrange}乘子法,在有额外正交条件约束,有\textcolor{blue}{(57)}
\begin{displaymath}
	\sum_n\gamma_{mn}(\langle\phi_m|\mathbf{S}|\phi_n\rangle-\delta_{mn})=0\qquad\forall n=1,\cdots,N_{\mathrm{b}}
\end{displaymath}
因此可得含\textrm{Lagrange}乘子表示的梯度矢量\textcolor{blue}{(58)}
\begin{displaymath}
	|g(\phi_m\rangle)=|g_m\rangle=\mathbf{H}|\phi_m\rangle-\sum_n\mathbf{S}\gamma_{mn}|\phi_n\rangle
\end{displaymath}
为确定\textrm{Lagrange}乘子,要求梯度矢量与尝试基组正交\textcolor{blue}{(59)}
\begin{displaymath}
	\langle\phi_n|g_m\rangle=0\qquad\forall n=1,\cdots,N_{\mathrm{b}}
\end{displaymath}
因此可有\textcolor{blue}{(60)}
\begin{displaymath}
	|g(\phi_m)\rangle=|g_m\rangle=\bigg(1-\sum_n\mathbf{S}|\phi_n\rangle\langle\phi_n|\bigg)\mathbf{H}|\phi_m\rangle
\end{displaymath}
如果尝试波函数满足\textrm{Hamiltonian}对角化条件,即$\langle\phi_n|\mathbf{H}|\phi_m\rangle=\delta_{mn}\epsilon_{m}^{\mathrm{app}}$,式\textcolor{blue}{(60)}还原为残矢定义式\textcolor{blue}{(47)}。因此也可用梯度计算\textrm{Rayleigh~}商的一阶改变\textcolor{blue}{(61)}
\begin{displaymath}
	\mathrm{d}\epsilon_m^{\mathrm{app}}=\langle\delta\phi_m|g_m\rangle+\mathrm{c.c}
\end{displaymath}
作者指出,之所以选择式\textcolor{blue}{(59)}作为确定\textrm{Lagrange}乘子的约束条件,就是为了保证能量变化精确到一阶:~波函数的改变$|\delta\phi_m\rangle$对于任意基函数$\{\phi_n\}$,由式\textcolor{blue}{(61)}给出的能量变化都是零。

考虑到作者给出的序列算法总是与自空间旋转轮替,因此式\textcolor{blue}{(60)}的精确梯度求解可以用式\textcolor{blue}{(47)}的“对角元残矢”近似。预处理后的残矢与基组$\{\phi_n\}$关于$\mathbf{S}$正交\textcolor{blue}{(62)}
\begin{displaymath}
	|p(\phi_m)\rangle=|p_m\rangle=\bigg(1-\sum_n|\phi_n\rangle\langle\phi_n|\mathbf{S}\bigg)\times\mathbf{K}(\mathbf{H}-\epsilon_{\mathrm{app}}\mathbf{S})|\phi_m\rangle
\end{displaymath}
预处理的搜索矢量满足\textcolor{blue}{(63)}
\begin{displaymath}
	\langle\phi_n|\mathbf{S}|p_m\rangle=0\qquad\forall n=1,\cdots,N_{\mathrm{b}}
\end{displaymath}
不同的序列算法的区别在于:~加入到波函数$\{\phi_n\}$的修正矢量的方式不同。

\subsection{Unblocked Davidson-like update}
传统的非块\textrm{Davidson}方法,每次迭代后向扩展基$\{b_i\}$中加入一个预处理修正矢量$\mathbf{K}|R(\phi_m)\rangle$,如果初始展开基组是由一套尝试波函数,因此每一次迭代的展开基组为\textcolor{blue}{(64)}
\begin{displaymath}
	\{|b_i\rangle,i=1,\cdots,N_{\mathrm{b}}+M\}=\{|\phi_n^0\rangle,n=1,\cdots,N_{\mathrm{b}/|P_m^0\rangle/|P_m^1/\cdots}\}
\end{displaymath}
如果优化能带$m$,第一次迭代时,残矢可由初始尝试波函数计算$|P_m^0\rangle=\mathbf{K}R(\phi_m^0)$,残矢加入到展开基组后,利用\textrm{Rayleigh-Ritz}变分法得到新的优化函数$|\phi_m^1\rangle$,下一步残矢由波函数$|\phi_m^1\rangle$计算得到:~$|P_m^1=\mathbf{K}R(\phi_m^1)$,依次类推。

这样的计算效率比较低,每一步迭代都要对角化一个大的矩阵。为简化该方法,\textcolor{magenta}{首先将残矢$|P_m^M\rangle$用预处理梯度$|p_m^M\rangle=|p(\phi_m^M)\rangle$(式\textcolor{blue}{(62)})代替}。作者指出,这样做并不改变迭代方法的结果,但\textrm{Rayleigh-Ritz}变分中的重叠矩阵$\bar{S}_{ij}$(式\textcolor{blue}{(44)})计算大为简化;~\textcolor{magenta}{其次,在$\bar{H}_{ij}$(式\textcolor{blue}{(44)})中忽略某些非对角元}。这样的处理使得基组中矢量$\{|\phi_i^0\rangle,i=1,\cdots,N_{\mathrm{b}}:i\ne m\}$和$\{|\phi_m^0\rangle/|p_m^0\rangle/\cdots\}$解耦,因此对角化的矩阵大大减小。

总之,在非块\textrm{Davidson-like update},初始基组是尝试波函数$\phi_m^0$,连续$M$次迭代后,经预处理并正交化的梯度$|p_m^M\rangle=|p(\phi_m^M)\rangle$加入到展开基组中\textcolor{blue}{(65)}
\begin{displaymath}
	\{b_i\rangle,i=1,\cdots,M\}=\{|\phi_m^0\rangle/|p_m^0\rangle/|p_m^1\rangle/\cdots\}
\end{displaymath}
每次迭代中,优化波函数$|\phi_m^M$由该展开基组通过\textrm{Rayleigh-Ritz}变分法确定。每个能带\textrm{update}若干次,再处理下一个能带,最后再在子空间$\{\phi_n^M\rangle,n=1,\cdots,N_{\mathrm{b}}\}$中作子空间旋转。作者指出:~\textcolor{red}{为了得到精确的基态波函数,最后的子空间旋转是必须的;~因为没有子空间旋转,该方法有可能收敛到线性相关的最低本征态}。

\subsection{Conjugate gradient minimization}
采用共轭梯度\textrm{(CG)}的思想,可以进一步减少各类矩阵操作。文献[10]和[45]分别在序列最小化能量和迭代对角化\textrm{KS-Hamiltonian}时用了\textrm{CG}思想。

除了存储所有此前迭代的前处理梯度,也可以采用标准的\textrm{CG}方法,使每一个搜索方向与前一次的方向共轭,即第$M$次迭代的搜索方向$|f^M\rangle$可以是\textcolor{blue}{(66)}
\begin{displaymath}
	|f^M\rangle=|p_m^M\rangle+\dfrac{\langle p_m^M|g_m^M\rangle}{\langle p_m^{M-1}|g_m^{M-1}\rangle}|f^{M-1}\rangle
\end{displaymath}
上式中$|g_m^M\rangle=|g(\phi_m^M)\rangle$是式\textcolor{blue}{(60)}定义的梯度矢量,$|p_m^M\rangle=|p(\phi_m^M)\rangle$是式\textcolor{blue}{(62)}定义的预处理梯度。每次迭代中,优化的新波函数$|\phi_m^{M+1}\rangle$由基组$\{|\phi_m^M\rangle/f^M\rangle\}$经\textrm{Rayleigh-Ritz}变分得到。作者指出,考虑到预处理梯度与所有波函数正交,因此式\textcolor{blue}{(66)}中可将梯度矢量$|g_m^M\rangle$用残矢$|R_m^M\rangle$代替,即$\langle p_m^M|g_m^M\rangle=\langle p_m^M|R_m^M\rangle$。除了数值误差,\textrm{CG}优化每次迭代与上一小节式\textcolor{blue}{(65)}的优化效果相同。以下讨论更高效的\textrm{CG}算法。

\subsection{Residual minimization method -direct inversion in the iterative subspace (RMM-DIIS)}
上述序列共轭梯度方法是相对快速和稳定的方法,其唯一的缺点是必须使预处理残矢$\mathbf{K}R(\phi_m)$与当前尝试波函数\textcolor{blue}{(62)}正交化。\textrm{Rayleigh-Ritz}变分法在展开基组张成的空间中求能量本征值极小值,即对给定波函数最小化\textrm{Rayleigh~}商,对于每个本征态,定态\textrm{Rayleigh~}商并不一定是极小值。如果没有正交化约束(式\textcolor{blue}{(62)}),类似于文献[44,52]讨论的\textrm{Lanczos}方法,变分的结果有可能全部矢量都收敛到\textrm{Hamiltonian~}的最小本征矢。只有正交化才可能使之有效地收敛到特定的本征态。

所幸文献[44]首先提出了一个解决的方案:~\textcolor{red}{最小化残矢模量而并非是\textrm{Rayleigh~}商,因此正交归一条件不再是必要条件,因为残矢模对于每个本征矢都有非约束的极小值}。

作者采用\textrm{Pulay}的原始文献[24]而非文献[44]的推导,文献[44]的方法要求计算并存储$\mathbf{S}|\phi\rangle$,因此比当前算法慢。对于给定能带$m$,先计算预处理残矢$\mathbf{K}|R_m^0\rangle=\mathbf{K}R(\phi_m^0)\rangle$,沿该方向的尝试步选为\textcolor{blue}{(67)}
\begin{displaymath}
	|\phi_m^1\rangle=|\phi_m^0\rangle+\lambda\mathbf{K}|R_m^0\rangle
\end{displaymath}
由此可以确定新的残矢$|R_m^1\rangle=|R(\phi_m^1)\rangle$(注意,通过计算$R(\phi_m^1)$可\textrm{update}~$\epsilon^{\mathrm{app}}$,见式\textcolor{blue}{(47)})。因此线性组合初始波函数$|\phi_m^0\rangle$和尝试波函数$|\phi_m^1\rangle$\textcolor{blue}{(68)}
\begin{displaymath}
	|\bar{\phi}^M\rangle=\sum_{i=0}^M\alpha_i|\phi_m^i\rangle\qquad M=1
\end{displaymath}
最小化残矢模量,可确定组合系数。进一步假设残矢也满足该线性化要求\textcolor{blue}{(69)}
\begin{displaymath}
	|\bar{R}^M\rangle=|R(\bar{\phi}^M)\rangle=\sum_{i=0}^M\alpha_i|R_m^i\rangle
\end{displaymath}
因此最小化\textcolor{blue}{(70)}
\begin{displaymath}
	\dfrac{\sum_{ij}\alpha_i^{\ast}\alpha_j\langle R_m^i|R_m^j\rangle}{\sum_{ij}\alpha_i^{\ast}\alpha_j\langle\phi_m^i|\mathbf{S}\phi_m^j\rangle}
\end{displaymath}
\textcolor{red}{这一步一般称为\textrm{DIIS}},式\textcolor{blue}{(70)}等价于求解\textrm{Hermitian}本征值问题\textcolor{blue}{(71)}
\begin{displaymath}
	\sum_{j=0}^M\langle R_m^i|R_m^j\rangle\alpha_j=\epsilon\sum_{j=0}^M\langle\phi_m^i|\mathbf{S}|\phi_m^j\rangle\alpha_j
\end{displaymath}
下一尝试步始自$|\bar{\phi}^M\rangle$,沿方向$\mathbf{K}|\bar{R}^M\rangle$。在每个迭代步$M$中,得到新的波函数$|\phi_m^M\rangle=|\bar{\phi}^{M-1}\rangle+\lambda\mathbf{K}|\bar{R}^{M-1}\rangle$,新的残矢$|R(\phi_m^M)\rangle$加入到迭代子空间。上述计算中,尝试步长参数$\lambda$的大小非常重要。作者指出,合理的$\lambda$可通过第一步最小化\textrm{Rayleigh~}商确定,每个能带有一个$\lambda$值,该值一般在$0.1\sim1$。

本小节描述的方法和\textrm{CG}方法有相同的迭代次数,但因为避免了正交化要求,因此比\textrm{CG}方法快得多。更重要的是,残矢最小化方法更容易并行实现。

\textrm{RMM}方法的一个缺点是,该方法总是在初始本征矢非常近邻的空间寻找新矢量,因此初始本征矢的确定非常关键,还会有找不到本征矢的情况出现。初始本征矢的确定要异常小心,\textcolor{magenta}{一般由一套随机初始本征矢开始,先完成全部能带空间三次迭代计算(初始化),对每个能带,每次初始化都包括三次子空间旋转和两次沿预处理残矢方向的最陡下降。注意在此初始化过程中,\textrm{Hamiltonian~}保持不变,然后才开启\textrm{RMM}迭代}。

子空间旋转和序列能带\textrm{update}是轮替的,在\textrm{RMM}中,最终的本征矢是不正交的,应用\textrm{Rayleigh-Ritz}变分法可保证本征矢正交。作者强调指出,原则上\textrm{RMM}方法不需要子空间旋转或正交归一处理,也可以收敛。但对于当前的计算体系,子空间旋转虽然其计算量是$\mathrm{O}(N^3)$,但可以加速计算。因为直接\textrm{RMM}优化中,有可能出现两个相邻矢量,能量本征值是$\epsilon$和$\epsilon+\mathrm{d}\epsilon$,此时残矢模的差$\sim\delta\epsilon$,这种情况是很难收敛的最小化问题。通过子空间旋转,残矢与尝试波函数正交,恰好能避免这一问题的出现。

\subsection{The complete algorithm}
完全自洽迭代循环包括若干步(见\textrm{Fig.~3})
\begin{itemize}
	\item 子空间旋转 (3)
	\item \textrm{DAV2(3.2)},\textrm{CG(3.6)}或\textrm{RMM(3.7)}算法实现
	\item 利用\textrm{Gram-Schmidt}方法正交归一化(只针对\textrm{RMM}方法)
	\item 对每个自洽计算,\textrm{update}分数占据和电荷密度
\end{itemize}
每次迭代中,前次迭代最终得到的波函数作为本次迭代的初始尝试波函数$\{|\phi_n\rangle,n=1,\cdots,N_{\mathrm{b}}\}$,初次迭代的波函数则由通过程序随机生成,迭代-自洽计算过程直到完全自洽后停止,对于非自洽计算,只有矩阵迭代对角化计算,没有电荷密度的迭代。\textrm{Fig.~3}中只有\textrm{RMM}最后需要波函数正交归一化,\textrm{DAV2}则不需要子空间旋转。

作者指出,在自洽计算之初,电荷密度混合后采用\textrm{RMM}或\textrm{CG}方法迭代对角化时,由于计算的残矢$|R(\phi_m)\rangle$与梯度$|g(\phi_m)\rangle$一致,必须引入子空间旋转。这是考虑到电荷密度混合后波函数仍必须正交,因此通过\textrm{Rayleigh-Ritz}变分法实现子空间的\textrm{Hamiltonian}对角化,同时也确保波函数正交归一。

此外,对于\textrm{RMM}和\textrm{CG}迭代优化,必须有合适的收敛标准。采用静态标准,比如每个能带迭代两次并非好的选择,因为能量低的能带收敛比能量高的的能带收敛快得多。因此采用如下动态收敛标准:
\begin{enumerate}
	\item 总的能量本征值的改变小于$E_{\mathrm{accuracy}}/N_{\mathrm{b}}/4$,此处$E_{\mathrm{accuracy}}$是能量收敛精度,$N_{\mathrm{b}}$是占据能带数目
	\item 如果残矢模量比初始值小于\textrm{30\%},\textrm{RMM}收敛
	\item 如果本征矢的改变比初始值最陡下降改变小于\textrm{30\%},\textrm{CG}收敛
	\item 最大迭代次数设为\textrm{4},对于\textrm{RMM},残矢最小化\textrm{3}次,至多做到第\textrm{4}尝试步
	\item 空带最大优化次数为\textrm{2}
\end{enumerate}

作者指出,该标准对于大量体系测试计算表现非常稳定。大部分情况下,每个能带执行\textrm{2}次\textrm{CG}和\textrm{RMM}迭代步,有些本征矢/本征值对需要更多的迭代。一般来说,对于更高的能带,需要更多的迭代次数,这样总的能带的收敛才会有一个好的速度。

\subsection{Computational considerations}
为了对不同的计算技术作出合理的对比,必须估计一下每种算法的矩阵运算开销。与\textrm{RMM}相比,\textrm{CG}最小化\textrm{Rayleigh~}商的\textrm{Hamiltonian}矩阵-向量乘次数更少一些,但对于大体系,波函数的正交归一开销更大。作者估计$(\mathbf{H}-\epsilon_n\mathbf{S})$的计算开销是\textcolor{blue}{(72)}
\begin{displaymath}
	T^{\mathbf{H}}=N_{\mathrm{b}}N_{\mathrm{plw}}\ln N_{\mathrm{plw}}\propto N^2\ln N
\end{displaymath}
这里$N$是矩阵维度。这里的限速因素包括快速\textrm{Fourier}变换($N_{\mathrm{b}}N_{\mathrm{plw}}\ln N_{\mathrm{plw}}\propto N^2\ln N$)和非局域投影操作。对于大体系,可以在实空间完成非局域投影,其计算开销正比于体系大小($CN_{\mathrm{ion}}$),对于全部能带,这部分计算量$N^2$。\textrm{Gram-Schmidt}正交化的计算开销\textcolor{blue}{(73)}
\begin{displaymath}
	T^{\mathrm{GS}}=N_{\mathrm{b}}^2\times N_{\mathrm{plw}}\propto N^3
\end{displaymath}
而式\textcolor{blue}{(62)}要求每个能带的梯度与其他能带精确正交化,计算开销为\textcolor{blue}{(74)}
\begin{displaymath}
	T^{\mathrm{ort}}=2N_{\mathrm{b}}^2\times N_{\mathrm{plw}}\propto 2N^3
\end{displaymath}
更糟糕的是,这种精确对角化无法做任何有效的内存\textrm{catching}。\textrm{CG}是严格的序列算法,每次迭代都要求梯度与其他能带波函数正交,对内存带宽要求很高。作者发现在类似\textrm{Sillcon Graphics Power Challenge}一类共享大主内存架构机器上,这个问题很严重(在矢量机上,这个问题好一些)。对于\textrm{Gram-Schmidt}正交化,因为算法路线中数据局域化程度高,因此在标量机上,$T^{\mathrm{ort}}$一般是$T^{\mathrm{GS}}$的$3\sim10$倍。子空间旋转的实现方式和数据局域化程度也比较好,计算开销为\textcolor{blue}{(75)}
\begin{displaymath}
	T^{\mathrm{diag}}=T^{\mathbf{H}}+2N_{\mathrm{b}}^2\times N_{\mathrm{plw}}
\end{displaymath}
对于块\textrm{Davidson}方法,第一次迭代的计算开销是\textcolor{blue}{(76)}
\begin{displaymath}
	T^{\mathrm{dav}}=2T^{\mathbf{H}}+5\frac12N_{\mathrm{b}}^2\times N_{\mathrm{plw}}
\end{displaymath}
如果势能不变,剩下的迭代计算开销\textcolor{blue}{(77)}
\begin{displaymath}
	T^{\mathrm{dav}}=1T^{\mathbf{H}}+4N_{\mathrm{b}}^2\times N_{\mathrm{plw}}
\end{displaymath}
作者对比了各种组合的计算开销,详见\textrm{Section~6}。

\section{Charge density mixing}
本文第二个关键步骤是输入电荷密度与输出电荷密度的有效混合。文中采用文献[56]\textrm{Johnson}提出的改进\textrm{Broyden}方法。该方法灵活性高,给定某些特定参数,可以得到\textrm{Pulay}的电荷密度混合方案和\textrm{Srivastava}和\textrm{Bl\"ugel}的改进方案。为进一步提高收敛性,作者采用特定的初始化混合矩阵和评定标准,同时对平面波基组优化。以下将讨论简单混合、\textrm{Pulay}和\textrm{Johnosn}的混合方案。
\subsection{Simple mixing}
电荷密度混合的核心量是电荷密度残矢$R[\rho_{\mathrm{in}}]$\textcolor{blue}{(78)}
\begin{displaymath}
	R[\rho_{\mathrm{in}}]=\rho_{\mathrm{out}}[\rho_{\mathrm{in}}]-\rho_{\mathrm{in}}
\end{displaymath}
残矢模量为\textcolor{blue}{(79)}
\begin{displaymath}
	\langle R[\rho_{\mathrm{in}}]|R[\rho_{\mathrm{in}}]\rangle
\end{displaymath}
在体系达到自洽时,残矢模量必须为零。所谓简单混合,就是每次只利用当前迭代的电荷密度信息。以线性混合为例,新的电荷密度与当前电荷密度关系为\textcolor{blue}{(80)}
\begin{displaymath}
	\rho_{\mathrm{in}}^{\mathrm{m+1}}=\rho_{\mathrm{in}}^{\mathrm{m}}+\gamma R[\rho_{\mathrm{in}}^{\mathrm{m}}]
\end{displaymath}
与矩阵迭代优化(见\textrm{Section~3.1})类似,通过有关\textrm{Jacobian~}矩阵预处理密度残矢,可实现简单混合密度的优化。因此混合密度应满足\textcolor{blue}{(81)}
\begin{displaymath}
	\rho_{\mathrm{in}}^{\mathrm{m+1}}=\rho_{\mathrm{in}}^{\mathrm{m}}+\mathbf{G}^1R[\rho_{\mathrm{in}}^{\mathrm{m}}]
\end{displaymath}
这里$\mathbf{G}^1$是具体的前处理矩阵。在文献[60]\textrm{Kerker}曾针对平面波基组,提出一个简单却非常有效的电荷密度混合方案。作者用该方法也取得了一些成功。\textrm{Kerker}方案前处理矩阵在到空间中是对角化的\textcolor{blue}{(82)}
\begin{displaymath}
	G_q^1=A\dfrac{q^2}{q^2+q_0^2}
\end{displaymath}
\textcolor{magenta}{该方法的优点是压制了电荷密度在倒空间低波数处的振荡}。对于小的波矢,函数\textcolor{blue}{(82)}随$Aq^2/q_0^2$变化,低波数部分的输出电荷密度残矢对新的输入电荷密度贡献少;~对于高波数$q$,电荷密度混合只与线性混合参数$A$有关。总的说参数$A$可以很大,作者指出$A=0.8$是通用的选择,$q_0$可根据实际体系优化。

\subsection{Pulay mixing}
在文献[24]的\textrm{Pulay}方案中,每一步混合的输入电荷密度和密度残矢都需要存储,每一步新的优化输入电荷密度来自前面所有步的输入电荷密度的线性组合\textcolor{blue}{(83)}
\begin{displaymath}
	\rho_{\mathrm{in}}^{\mathrm{opt}}=\sum_i\alpha_i\rho_{\mathrm{in}}^i
\end{displaymath}
假设密度残矢与输入电荷密度有相同的线性关系,可有优化密度的残矢\textcolor{blue}{(84)}
\begin{displaymath}
	R[\rho_{\mathrm{in}}^{\mathrm{opt}}]=R[\sum_i\alpha_i\rho_{\mathrm{in}}^i]=\sum_i\alpha_iR[\rho_{\mathrm{in}}^i]
\end{displaymath}
优化新的电荷密度,即通过系数$\alpha_i$最小化其残矢模\textcolor{blue}{(85)}
\begin{displaymath}
	\langle R[\rho_{\mathrm{in}}^{\mathrm{opt}}]|R[\rho_{\mathrm{in}}^{\mathrm{opt}}]\rangle
\end{displaymath}
系数$\alpha_i$的约束条件为电子数守恒\textcolor{blue}{(86)}
\begin{displaymath}
	\sum_i\alpha_i=1
\end{displaymath}
除了约束条件不同,上述方程与\textrm{Section~3.7}非常相似,因此可有优化系数$\alpha_i$为\textcolor{blue}{(87)}
\begin{displaymath}
	\alpha_i=\dfrac{\sum_jA_{ji}^{-1}}{\sum_{kj}A_{kj}^{-1}}\qquad\mbox{这里} A_{ij}=\langle R[\rho_{\mathrm{in}}^j]|R[\rho_{\mathrm{in}}^i]\rangle
\end{displaymath}
为提高数值稳定性,也为了和下一节\textrm{Broyden}方案对比,可将第$m$次迭代用新的独立变量表示\textcolor{blue}{(88)}
\begin{displaymath}
	\begin{aligned}
		&\rho^m=\rho_{\mathrm{in}}^m,&\Delta\rho^i=\rho_{\mathrm{in}}^{i+1}-\rho_{\mathrm{in}}^{i}\\
		&R^m=R[\rho_{\mathrm{in}}^m],\quad&\Delta R^i=R[\rho_{\mathrm{in}}^{i+1}]-R[\rho_{\mathrm{in}}^{i}]
	\end{aligned}
\end{displaymath}
这里$i<m$。新的优化输入电荷密度可以是线性组合\textcolor{blue}{(89)}
\begin{displaymath}
	\rho_{\mathrm{in}}^{\mathrm{opt}}=\rho^m+\sum_{i=1}^{m-1}\bar{\alpha}_i\Delta\rho^i
\end{displaymath}
显然$\alpha_i$与$\bar{\alpha}_i$存在一一对应关系,考虑$\alpha_i$电子数守恒条件,因此有\textcolor{blue}{(90)}
\begin{displaymath}
	\bar{\alpha}_i=-\sum_{j=1}^{m-1}\bar{A}_{ji}^{-1}\langle\Delta R^j|R^m\rangle
\end{displaymath}
其中\textcolor{blue}{(91)}
\begin{displaymath}
	\bar{A}_{ij}=\langle\Delta R^j|\Delta R^i\rangle
\end{displaymath}
下一步电荷密度可以通过以下方程计算\textcolor{blue}{(92)}
\begin{displaymath}
	\rho_{\mathrm{in}}^{m+1}=\rho_{\mathrm{in}}^{\mathrm{opt}}+\mathbf{G}^1R[\rho_{\mathrm{in}}^{\mathrm{opt}}]=\rho^m+\mathbf{G}^1R^m+\sum_{i=1}^{m-1}\bar{\alpha}_i(\Delta\rho^i+\mathbf{G}^1\Delta R^i)
\end{displaymath}
这里$\mathbf{G}^1$可以对应简单混合中的常数或式\textcolor{blue}{(82)}中给定的混合系数。

\subsection{Broyden mixing}
在文献[23]中\textrm{Broyden}提出来了一类最复杂的用\textrm{quasi-Newton}算法自洽求解\textrm{KS~}方程方案。\textrm{quasi-Newton}算法中,通过每次迭代\textrm{update}~\textrm{Jacobian}矩阵来逼近\textrm{Jacobian}矩阵或其逆阵。对于大体系自洽问题,存储全部$N\times N$~\textrm{Jacobian}矩阵是非常困难的。作者指出,最近些年,经过一些改进算法努力,这类算法每次迭代中只需要存储一些$N-$维矢量。有关改进方法,作者重点提了文献[57-59]。本文围绕文献[56]\textrm{Johnson}方法讨论。

\textrm{quasi-Newton}方法的关键是\textcolor{magenta}{假设残矢在临近最小值处是线性变化的}\textcolor{blue}{(93)}
\begin{displaymath}
	R[\rho]=R[\rho_{\mathrm{in}}^m]-\mathbf{J}^m(\rho-\rho^m)
\end{displaymath}
这里$\mathbf{J}^m$是近似\textrm{Jacobian}矩阵。如果能找到$\rho^{\ast}$满足$R[\rho^{\ast}]=0$,则$\rho^{\ast}$是优化电荷密度,$\rho^{\ast}$满足\textcolor{blue}{(94)}
\begin{displaymath}
	\rho^{\ast}=\rho_{\mathrm{in}}^m+(\mathbf{J}^m)^{-1}R[\rho_{\mathrm{in}}^m]
\end{displaymath}

进过若干次连续迭代,可构造近似\textrm{Jacobian}矩阵$\mathbf{J}^m$或其逆阵$(\mathbf{J})^{-1}$,由该\textrm{Jacobian}逆阵、当前电荷$\rho_{\mathrm{in}}$和当前电荷密度残矢$R[\rho_{\mathrm{in}}^m]$计算新的电荷密度\textcolor{blue}{(95)}
\begin{displaymath}
	\rho_{\mathrm{in}}^{m+1}=\rho_{\mathrm{in}}^{m}+(\mathbf{J}^m)^{-1}R[\rho_{\mathrm{in}}^{m}]
\end{displaymath}

这类算法的差别主要体现在每次迭代中$\mathbf{J}^m$的变化和\textrm{update}方式上。为与文献[56]一致,令$\mathbf{G}^m=(\mathbf{J}^m)^{-1}$。文献[56]中\textrm{Johnson}方案,当前$\mathbf{G}^m$的计算用到了前面所有迭代的信息。对于第$m$次迭代,\textcolor{red}{$\mathbf{G}^m$的计算是通过最小二乘法最小误差函数得到}\textcolor{blue}{(96)}
\begin{displaymath}
	E=w_0\lVert\mathbf{G}^{m+1}-\mathbf{G}^m\lVert^2+\sum_{i=1}^mw_i\lVert\Delta\rho^i+\mathbf{G}^{m+1}\Delta R^i\lVert^2
\end{displaymath}
这里$\lVert A\lVert^2=\langle A|A\rangle$,$\Delta\rho^i$由上一节式\textcolor{blue}{(88)}定义,$w_i$是权重因子。误差函数主要从以下几方面考虑:
\begin{enumerate}
	\item 误差函数第一项对应于每次迭代近似\textrm{Jacobian}矩阵的逆阵变换不应太大:\\
		实际上,这一项的要求并不高,因为重点考虑的是$w_0\rightarrow0$的情况
	\item 误差函数第二项要求模量\textcolor{blue}{(97)}
		\begin{displaymath}
			\Delta\rho^i+\mathbf{G}^{m+1}\Delta R^i
		\end{displaymath}
		足够小。如果$R[\rho]$随$\rho$线性变化,当得到精确的\textrm{Jacobian}矩阵$\mathbf{G}^{m+1}=\mathbf{G}^{\mathrm{exact}}$时,式\textcolor{blue}{(97)}为零(对比式\textcolor{blue}{(94)})
\end{enumerate}

由式\textcolor{blue}{(96)}开始,可以导出$\mathbf{G}^{m+1}$的精确表达式,由于文献[56]有太多的印刷错误,作者重新推导了有关公式\textcolor{blue}{(98)}
\begin{displaymath}
	\mathbf{G}^{m+1}=\mathbf{G}^1-\sum_{k=1}^m|Z_k^m\rangle\langle\Delta R^k|
\end{displaymath}
这里\textcolor{blue}{(99)}
\begin{displaymath}
	|Z_k^m\rangle=\sum_{n=1}^m\beta_{kn}w_kw_n|u^n\rangle+\sum_{n=1}^{m-1}\bar{\beta}_{kn}|Z_n^{m-1}\rangle
\end{displaymath}
\textcolor{blue}{(100)}
\begin{displaymath}
	|u^n\rangle=\mathbf{G}^1|\Delta R^n\rangle+|\Delta\rho^n\rangle
\end{displaymath}
$\beta_{kn}$和$\bar{\beta}_{kn}$的定义\textcolor{blue}{(101)}
\begin{displaymath}
	\beta_{kn}=(w_0^2+\bar{A})_{kn}^{-1},\qquad\bar{A}_{kn}=w_kw_n\langle\Delta R^n|\Delta R^k\rangle
\end{displaymath}
\textcolor{blue}{(102)}
\begin{displaymath}
	\bar{\beta}_{kn}=\delta_{kn}-\sum_{j=1}^mw_kw_j\beta_{kj}\langle\Delta R^n|\Delta R^j\rangle
\end{displaymath}
当全部迭代权重$w_n$与$\bar{\beta}_{kn}=w_0^2\beta_{kn}$相等,上述方程与文献[56]一致。

不难看出,当$w_0\rightarrow0$并且$w_0<<w_n$,上述方程可导出\textrm{Pulay}方案。有意思的是,当$w_0\rightarrow0$,改变$w_n$的值,完全不影响$\mathbf{G}^{m+1}$和$w_kw_n\beta_{kn}$的值。因此一般地,可令$w_n=1$,并有\textrm{Jacobian}矩阵的逆阵\textcolor{blue}{(103)}
\begin{displaymath}
	\mathbf{G}^m=\mathbf{G}^1-\sum_{k,n=1}^{m-1}\beta_{kn}|u^n\rangle\langle\Delta R^k|
\end{displaymath}

对新的输入电荷密度(式\textcolor{blue}{(95)})$\rho_{\mathrm{in}}^{m+1}=\rho_{\mathrm{in}}^m+\mathbf{G}^mR[\rho_{\mathrm{in}}^m]$直接处理可以得到\textrm{Pulay}方案的式\textcolor{blue}{(92)}。由此得到的\textrm{Jacobian}矩阵逆阵要求式\textcolor{blue}{(97)}满足$i<m$时为零,由此$\mathbf{G}^m$可以看成到到此为止对\textrm{Jacobian}矩阵求逆最好的近似。

第二种情况,对上述方程当$i<m$,令$w_i=0$并要求$w_0<<w_m$,可导出\textrm{Broyden}的第二方案。在这种情况下,\textrm{update}方程\textcolor{blue}{(104)}
\begin{displaymath}
	|Z_k^m\rangle=|Z_k^{m-1}\rangle\qquad k<m
\end{displaymath}
并有\textcolor{blue}{(105)}
\begin{displaymath}
	|Z_m^m\rangle=\dfrac1{\lVert\Delta R^m\lVert^2}\bigg(|u^m\rangle-\sum_{k=1}^{m-1}\langle\Delta R^k|\Delta R^m\rangle|Z_k^{m-1}\rangle\bigg)
\end{displaymath}
与文献[58]\textrm{Bl\"ugel}的公式一致。在\textrm{Broyden}第二方案中,允许用当前迭代信息覆盖前面所有迭代的信息。

作者指出,\textrm{Broyden}第二方案的电荷密度混合比\textrm{Pulay}方案慢。\textcolor{magenta}{\textrm{Pulay}方案的唯一问题可能是连续两次搜索的优化方向相关度太强}。在电荷密度混合时,这种情况一般不会出现。作者尝试将\textrm{Pulay}和\textrm{Broyden}第二优化方案结合,用于离子自由度的弛豫。当离子构型出现少量不同位置的力和离子自由度线性相关时,\textrm{Pulay}方案出现不稳定,在这种情况下,\textrm{Broyden}第二方案更适用。对于离子弛豫,另一个更方便的选择是只考虑此前有限$n$步的弛豫信息(如对$k<m-n$,取$w_k=0$;~对$m-n\leqslant k<m$,取$w_k>>w_0$)

最后,考虑$w_0\approx w_n$:~这样取值,限制了两次迭代的$\mathbf{G}$的变化,结果是破坏了\textrm{Broyden}方法的优势,$\mathbf{G}$的\textrm{update}也不像预期那样有效。这种情况下,要想实现合理的收敛,$\mathbf{G}^1$必须与\textrm{Jacobian}矩阵的逆阵足够接近。上述讨论表明,\textrm{Johnson}建议的动态选择$w_n$一般是不可行的。有用的设置只有$w_n=0$或$w_n>>w_0$,作者已经表明,对于$w_n>>w_0$,实际上$w_n$的选择完全不会影响$\mathbf{G}$。
\subsection{Preconditioning and metric}
到目前为止,还有两个问题没有回答,首先是初始$\mathbf{G}^1$如何选,第二是是否存在用于估计标量积$\langle\cdot|\cdot\rangle$的优化的标准。

初始混合并不重要,但为了方便起见,作者计算$\mathbf{G}^1$采用\textrm{Kerker}矩阵(式\textcolor{blue}(82)),因为在前几步计算中,这样的$\mathbf{G}^1$能够很好收敛。作者在\textrm{Section~6.2.2}给出例子证明,对不同体系,优化参数$A=0.8$和$q_0=1.5\AA^{-1}$一般总能满足需求,无需改变,对于磁性和某些表面计算,初始线性混合系数可取$A=0.1$。

对于第二个问题,选一个合理的标准,可以减少迭代次数。作者发现,对于复杂的金属体系,引入权重因子\textcolor{blue}{(106)}
\begin{displaymath}
	f_q=\dfrac{q^2+q_1^2}{q^2}
\end{displaymath}
再计算标量积\textcolor{blue}{(107)}
\begin{displaymath}
	\langle A|B\rangle=\sum_qf_qA_q^{\ast}B_q
\end{displaymath}
可以大大加快收敛。引入权重函数是考虑到对于金属体系,短波长矢量比长波长矢量的贡献更大。$q_1$的选择并不重要,因此作者选择$q_1$的判据是最短波长的权重强度是最长波长权重的20倍,在此,作者提醒注意电荷密度混合与势混合的差别,比如\textrm{Hartree}势可表示为
\begin{displaymath}
	V(q)\propto\dfrac1{q^2}\rho(\vec q)
\end{displaymath}
因此对于电荷密度和势,标度积的优化标准可能差一个因子$1/q^4$。

第三,作者经常遇到大体系中会有\textrm{FFT}网格点高达$64\times64\times64$的情况,用于计算过渡金属补偿电荷。对于高效电荷密度混合来说,这样高的网格点超过存储能力上限。这个问题有一个简单的解决方案:~作者发现,对于大的波矢$q$,可以令\textcolor{blue}{(108)}
\begin{displaymath}
	\rho_{\mathrm{in}\;q}^{m+1}=\rho_{\mathrm{out}\;q}^m
\end{displaymath}
而不会损失效率,而只对相对少量的网格点应用\textrm{Broyden}方案;~一般情况下,作者\textcolor{magenta}{取落在平面波基组内包含的网格点($\hbar^2|q|^2/(2m_{\mathrm{e}})<E_{\mathrm{cut}}$)}

综上所述,对于电荷密度混合,作者建议一般使用\textrm{Pulay}混合方案并根据\textrm{Kerker}方案,取参数$A=0.8$$q_0=1.5\AA^{-1}$计算$\mathbf{G}^1$。截止到目前为止,这样的参数选择,在自洽过程中,都能得到良好的收敛结果。即使再进行参数优化,收敛速度改善也不超过10\%。不同混合方法的计算对比见\textrm{Section~6.2}。

\section{Direct minimization of the KS-functional}
本节介绍\textrm{CP}方法的直接求解方案,与迭代-自洽计算流程算法关系不大,从略
\section{Comparison of different techniques}
(略)
\section{Conclusion}
(略)
%-------------------The Figure Of The Paper------------------
%\begin{figure}[h!]
%\centering
%\includegraphics[height=3.35in,width=2.85in,viewport=0 0 400 475,clip]{PbTe_Band_SO.eps}
%\hspace{0.5in}
%\includegraphics[height=3.35in,width=2.85in,viewport=0 0 400 475,clip]{EuTe_Band_SO.eps}
%\caption{\small Band Structure of PbTe (a) and EuTe (b).}%(与文献\cite{EPJB33-47_2003}图1对比)
%\label{Pb:EuTe-Band_struct}
%\end{figure}

%-------------------The Equation Of The Paper-----------------
%\begin{equation}
%\varepsilon_1(\omega)=1+\frac2{\pi}\mathscr P\int_0^{+\infty}\frac{\omega'\varepsilon_2(\omega')}{\omega'^2-\omega^2}d\omega'
%\label{eq:magno-1}
%\end{equation}

%\begin{equation} 
%\begin{split}
%\varepsilon_2(\omega)&=\frac{e^2}{2\pi m^2\omega^2}\sum_{c,v}\int_{BZ}d{\vec k}\left|\vec e\cdot\vec M_{cv}(\vec k)\right|^2\delta [E_{cv}(\vec k)-\hbar\omega] \\
% &= \frac{e^2}{2\pi m^2\omega^2}\sum_{c,v}\int_{E_{cv}(\vec k=\hbar\omega)}\left|\vec e\cdot\vec M_{cv}(\vec k)\right|^2\dfrac{dS}{\nabla_{\vec k}E_{cv}(\vec k)}
% \end{split}
%\label{eq:magno-2}
%\end{equation}

%-------------------The Table Of The Paper----------------------
%\begin{table}[!h]
%\tabcolsep 0pt \vspace*{-12pt}
%%\caption{The representative $\vec k$ points contributing to $\sigma_2^{xy}$ of interband transition in EuTe around 2.5 eV.}
%\label{Table-EuTe_Sigma}
%\begin{minipage}{\textwidth}
%%\begin{center}
%\centering
%\def\temptablewidth{0.84\textwidth}
%\rule{\temptablewidth}{1pt}
%\begin{tabular*} {\temptablewidth}{|@{\extracolsep{\fill}}c|@{\extracolsep{\fill}}c|@{\extracolsep{\fill}}l|}

%-------------------------------------------------------------------------------------------------------------------------
%&Peak (eV)  & {$\vec k$}-point            &Band{$_v$} to Band{$_c$}  &Transition Orbital
%Components\footnote{波函数主要成分后的括号中,$5s$、$5p$和$5p$、$4f$、$5d$分别指碲和铕的原子轨道。} &Gap (eV)   \\ \hline
%-------------------------------------------------------------------------------------------------------------------------
%&2.35       &(0,0,0)         &33$\rightarrow$34    &$4f$(31.58)$5p$(38.69)$\rightarrow$$5p$      &2.142   \\% \cline{3-7}
%&       &(0,0,0)         &33$\rightarrow$34    &$4f$(31.58)$5p$(38.69)$\rightarrow$$5p$      &2.142   \\% \cline{3-7}
%-------------------------------------------------------------------------------------------------------------------------
%\end{tabular*}
%\rule{\temptablewidth}{1pt}
%\end{minipage}{\textwidth}
%\end{table}

%-------------------The Long Table Of The Paper--------------------
%\begin{small}
%%\begin{minipage}{\textwidth}
%%\begin{longtable}[l]{|c|c|cc|c|c|} %[c]指定长表格对齐方式
%\begin{longtable}[c]{|c|c|p{1.9cm}p{4.6cm}|c|c|}
%\caption{Assignment for the peaks of EuB$_6$}
%\label{tab:EuB6-1}\\ %\\长表格的caption中换行不可少
%\hline
%%
%--------------------------------------------------------------------------------------------------------------------------------
%\multicolumn{2}{|c|}{\bfseries$\sigma_1(\omega)$谱峰}&\multicolumn{4}{c|}{\bfseries部分重要能带间电子跃迁\footnotemark}\\ \hline
%\endfirsthead
%--------------------------------------------------------------------------------------------------------------------------------
%%
%\multicolumn{6}{r}{\it 续表}\\
%\hline
%--------------------------------------------------------------------------------------------------------------------------------
%标记 &峰位(eV) &\multicolumn{2}{c|}{有关电子跃迁} &gap(eV)  &\multicolumn{1}{c|}{经验指认} \\ \hline
%\endhead
%--------------------------------------------------------------------------------------------------------------------------------
%%
%\multicolumn{6}{r}{\it 续下页}\\
%\endfoot
%\hline
%--------------------------------------------------------------------------------------------------------------------------------
%%
%%\hlinewd{0.5$p$t}
%\endlastfoot
%--------------------------------------------------------------------------------------------------------------------------------
%%
%% Stuff from here to \endlastfoot goes at bottom of last page.
%%
%--------------------------------------------------------------------------------------------------------------------------------
%标记 &峰位(eV)\footnotetext{见正文说明。} &\multicolumn{2}{c|}{有关电子跃迁\footnotemark} &gap(eV) &\multicolumn{1}{c|}{经验指认\upcite{PRB46-12196_1992}}\\ \hline
%--------------------------------------------------------------------------------------------------------------------------------
%
%     &0.07 &\multicolumn{2}{c|}{电子群体激发$\uparrow$} &--- &电子群\\ \cline{2-5}
%\raisebox{2.3ex}[0pt]{$\omega_f$} &0.1 &\multicolumn{2}{c|}{电子群体激发$\downarrow$} &--- &体激发\\ \hline
%--------------------------------------------------------------------------------------------------------------------------------
%
%     &1.50 &\raisebox{-2ex}[0pt][0pt]{20-22(0,1,4)} &2$p$(10.4)4$f$(74.9)$\rightarrow$ &\raisebox{-2ex}[0pt][0pt]{1.47} &\\%\cline{3-5}
%     &1.50$^\ast$ & &2$p$(17.5)5$d_{\mathrm E}$(14.0)$\uparrow$ & &4$f$$\rightarrow$5$d_{\mathrm E}$\\ \cline{3-5}
%     \raisebox{2.3ex}[0pt][0pt]{$a$} &(1.0$^\dagger$) &\raisebox{-2ex}[0pt][0pt]{20-22(1,2,6)} &\raisebox{-2ex}[0pt][0pt]{4$f$(89.9)$\rightarrow$2$p$(18.7)5$d_{\mathrm E}$(13.9)$\uparrow$}\footnotetext{波函数主要成分后的括号中,2$s$、2$p$和5$p$、4$f$、5$d$、6$s$分别指硼和铕的原子轨道;5$d_{\mathrm E}$、5$d_{\mathrm T}$分别指铕的(5$d_{z^2}$,5$d_{x^2-y^2}$和5$d_{xy}$,5$d_{xz}$,5$d_{yz}$)轨道,5$d_{\mathrm{ET}}$(或5$d_{\mathrm{TE}}$)则指5个5$d$轨道成分都有,成分大的用脚标的第一个字母标示;2$ps$(或2$sp$)表示同时含有硼2$s$、2$p$轨道成分,成分大的用第一个字母标示。$\uparrow$和$\downarrow$分别标示$\alpha$和$\beta$自旋电子跃迁。} &\raisebox{-2ex}[0pt][0pt]{1.56} &激子跃迁。 \\%\cline{3-5}
%     &(1.3$^\dagger$) & & & &\\ \hline
%--------------------------------------------------------------------------------------------------------------------------------

%     & &\raisebox{-2ex}[0pt][0pt]{19-22(0,0,1)} &2$p$(37.6)5$d_{\mathrm T}$(4.5)4$f$(6.7)$\rightarrow$ & & \\\nopagebreak %\cline{3-5}
%     & & &2$p$(24.2)5$d_{\mathrm E}$(10.8)4$f$(5.1)$\uparrow$ &\raisebox{2ex}[0pt][0pt]{2.78} &a、b、c峰可能 \\ \cline{3-5}
%     & &\raisebox{-2ex}[0pt][0pt]{20-29(0,1,1)} &2$p$(35.7)5$d_{\mathrm T}$(4.8)4$f$(10.0)$\rightarrow$ & &包含有复杂的\\ \nopagebreak%\cline{3-5}
%     &2.90 & &2$p$(23.2)5$d_{\mathrm E}$(13.2)4$f$(3.8)$\uparrow$ &\raisebox{2ex}[0pt][0pt]{2.92} &强激子峰。$^{\ast\ast}$\\ \cline{3-5}
%$b$  &2.90$^\ast$ &\raisebox{-2ex}[0pt][0pt]{19-22(0,1,1)} &2$p$(33.9)4$f$(15.5)$\rightarrow$ & &B2$s$-2$p$的价带 \\ \nopagebreak%\cline{3-5}
%     &3.0 & &2$p$(23.2)5$d_{\mathrm E}$(13.2)4$f$(4.8)$\uparrow$ &\raisebox{2ex}[0pt][0pt]{2.94} &顶$\rightarrow$B2$s$-2$p$导\\ \cline{3-5}
%     & &12-15(0,1,2) &2$p$(39.3)$\rightarrow$2$p$(25.2)5$d_{\mathrm E}$(8.6)$\downarrow$ &2.83 &带底跃迁。\\ \cline{3-5}
%     & &14-15(1,1,1) &2$p$(42.5)$\rightarrow$2$p$(29.1)5$d_{\mathrm E}$(7.0)$\downarrow$ &2.96 & \\\cline{3-5}
%     & &13-15(0,1,1) &2$p$(40.4)$\rightarrow$2$p$(28.9)5$d_{\mathrm E}$(6.6)$\downarrow$ &2.98 & \\ \hline
%--------------------------------------------------------------------------------------------------------------------------------
%%\hline
%%\hlinewd{0.5$p$t}
%\end{longtable}
%%\end{minipage}{\textwidth}
%%\setlength{\unitlength}{1cm}
%%\begin{picture}(0.5,2.0)
%%  \put(-0.02,1.93){$^{1)}$}
%%  \put(-0.02,1.43){$^{2)}$}
%%\put(0.25,1.0){\parbox[h]{14.2cm}{\small{\\}}
%%\put(-0.25,2.3){\line(1,0){15}}
%%\end{picture}
%\end{small}

%-----------------------------------------------------------------------------------------------------------------------------------------------------------------------------------------------------%


%--------------------------------------------------------------------------The Biblography of The Paper-----------------------------------------------------------------%
%\newpage																				%
%-----------------------------------------------------------------------------------------------------------------------------------------------------------------------%
%\begin{thebibliography}{99}																		%
%%\bibitem{PRL58-65_1987}H.Feil, C. Haas, {\it Phys. Rev. Lett.} {\bf 58}, 65 (1987).											%
%\end{thebibliography}																			%
%-----------------------------------------------------------------------------------------------------------------------------------------------------------------------%
%																					%
%\phantomsection\addcontentsline{toc}{section}{Bibliography}	 %直接调用\addcontentsline命令可能导致超链指向不准确,一般需要在之前调用一次\phantomsection命令加以修正	%
%\bibliography{ref/Myref}																			%
%\bibliographystyle{ref/mybib}																		%
%  \nocite{*}																				%
%-----------------------------------------------------------------------------------------------------------------------------------------------------------------------%

\clearpage     %\end{CJK} 前加上\clearpage是CJK的要求
\end{document}
