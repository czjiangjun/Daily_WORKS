%---------------------- TEMPLATE FOR REPORT ------------------------------------------------------------------------------------------------------%

%\thispagestyle{fancy}   % 插入页眉页脚                                        %
%%%%%%%%%%%%%%%%%%%%%%%%%%%%% 用 authblk 包 支持作者和E-mail %%%%%%%%%%%%%%%%%%%%%%%%%%%%%%%%%
%\title{More than one Author with different Affiliations}				     %
%\title{\rm{VASP}的电荷密度存储文件\rm{CHGCAR}}
%\title{面向高温合金材料设计的计算模拟软件中的几个主要问题}
\title{数据驱动视野下的材料设计}
\author[ ]{}   %
%\author[ ]{姜~骏\thanks{jiangjun@bcc.ac.cn}}   %
%\affil[ ]{北京市计算中心}    %
%\author[a]{Author A}									     %
%\author[a]{Author B}									     %
%\author[a]{Author C \thanks{Corresponding author: email@mail.com}}			     %
%%\author[a]{Author/通讯作者 C \thanks{Corresponding author: cores-email@mail.com}}     	     %
%\author[b]{Author D}									     %
%\author[b]{Author/作者 D}								     %
%\author[b]{Author E}									     %
%\affil[a]{Department of Computer Science, \LaTeX\ University}				     %
%\affil[b]{Department of Mechanical Engineering, \LaTeX\ University}			     %
%\affil[b]{作者单位-2}			    						     %
											     %
%%% 使用 \thanks 定义通讯作者								     %
											     %
\renewcommand*{\Authfont}{\small\rm} % 修改作者的字体与大小				     %
\renewcommand*{\Affilfont}{\small\it} % 修改机构名称的字体与大小			     %
\renewcommand\Authands{ and } % 去掉 and 前的逗号					     %
\renewcommand\Authands{ , } % 将 and 换成逗号					     %
\date{} % 去掉日期									     %
\date{2024-07-15}									     %

%%%%%%%%%%%%%%%%%%%%%%%%%%%%%%%%%%%%%%%%%%  不使用 authblk 包制作标题  %%%%%%%%%%%%%%%%%%%%%%%%%%%%%%%%%%%%%%%%%%%%%%
%-------------------------------The Title of The Report-----------------------------------------%
%\title{报告标题:~}   %
%-----------------------------------------------------------------------------

%----------------------The Authors and the address of The Paper--------------------------------%
%\author{
%\small
%Author1, Author2, Author3\footnote{Communication author's E-mail} \\    %Authors' Names	       %
%\small
%(The Address,City Post code)						%Address	       %
%}
%\affil[$\dagger$]{清华大学~材料加工研究所~A213}
%\affil{清华大学~材料加工研究所~A213}
%\date{}					%if necessary					       %
%----------------------------------------------------------------------------------------------%
%%%%%%%%%%%%%%%%%%%%%%%%%%%%%%%%%%%%%%%%%%%%%%%%%%%%%%%%%%%%%%%%%%%%%%%%%%%%%%%%%%%%%%%%%%%%%%%%%%%%%%%%%%%%%%%%%%%%%
\maketitle
%\thispagestyle{fancy}   % 首页插入页眉页脚 
20世纪90年代以来,随着理论化学、计算物理方法的快速发展以及计算机技术不断升级和更新,计算材料科学获得空前发展,它是传统材料科学与物理、化学、工程力学以及应用数学等诸多基础和应用学科交叉与融合形成的一门新兴学科,在材料研究中发挥越来越重要的作用\upcite{NatMat3-429_2004,App-CataA254-5_2003,JACS125-4306_2003,JCombChem5-472_2003,Meas_Sci-Tech16-1_2005,Nature392-694_1998}。在新材料研究中,借鉴药物组合合成与筛选\textrm{(combinatorial synthesis and screening)}思想\upcite{Science268-1738_1995}形成的高通量(\textrm{high~throughput})材料计算\upcite{Nature410-643_2001}模式,在21世纪最初十年,成为创新发展新材料、发现新现象的重要手段。

材料的结构、物性数据是材料研发和加速材料性能提升的重要依据,在计算材料学科发展之前,人类主要通过实验手段积累材料数据,受客观实验条件的制约,新材料的研发成本非常高。得益于高精度的多尺度计算方法和高性能并行计算技术的突破\upcite{PRL88-255506_2002,Nano-Lett3-1183_2003},计算材料可以获得大量材料数据,大大降低了材料研发门槛。进入21世纪以后,欧洲、日本、美国等发达国家先后启动了以多尺度模拟、研发和科学设计为手段,提升材料设计成功率,缩短材料开发周期为目的的多种类型的国家级科研专项,其中以美国的“材料基因组计划\textrm{(Materials Genome Initiative, MGI)}”\upcite{MGI_USA}最为著名。该计划的根本目的是通过增效集成各个尺度的材料模拟工具、高效实验手段和数据库,把材料研发从传统 经验式提升到科学设计,从而大大加快材料研发速度。我国也于2016年起启动了国家重点研发计划“材料基因工程关键技术与支撑平台”。
随着``材料基因组计划''的实施,为计算材料科学提供了越来越广阔的应用空间,高性能计算与\textrm{DFT}相结合推动了第一原理材料模拟数据生成和结果分析的自动化,涌现了越来越多的计算材料数据库,特别是在功能材料领域,高通量\textrm{DFT}计算大大推进了新材料合成的进步。\upcite{JCED59-3232_2014, IC53-11849_2014,JPCL4-3607_2013,PCCP16-22073_2014}


但是限于实验条件的限制,很多数据库的材料物性数据并不完整。这些数据也是产生和检验第一原理模拟结果的重要来源和依据。

显现出强大的能力,借助机器学习技术进行材料性能预测,加速材料属性改善、优化和提升,是近年来的研究热点,前景广阔。


最先

组合与筛选研究兴起于1990年代中期,
在传统的研究范式下,获取材料物性数据的成本,无论是通过实验手段还是计算模拟,代价都是比较高的,虽然高通量第一原理计算自动流程和数据库解决了材料物性数据的获取问题,但是并未给出现有材料数据基础上的物性优化的方案,因此利用数据挖掘技术,实现数据驱动的材料物性筛选、预测和提升的技术路线,有着特殊重要的意义。机器学习\textrm{(Machine Learning, ML)}技术可以从大量数据中获得有价值的信息,尤其是面对高维复杂数据时,机器学习技术是确定数据间关系的有力的工具。


二十世纪七十年代以来,随着计算机科学和密度泛函理论\textrm{(Density Functional Theory, DFT)}\upcite{PRB136-864_1964,PRA140-1133_1965,Parr-Yang,CR91-651_1991}、分子动力学\textrm{(Molecular Dynamics, MD)}的发展和融合,计算材料科学\textrm{(computational materials science)}与物理、化学、工程力学以及应用数学等诸多基础和应用学科交叉日益紧密,逐渐成为一门新兴的独立学科。\upcite{Nat-Mat3-429_2004,App-CataA5-254_2003,JACS125-4306_2003,JCC5-472_2003,Mater_Sci-Tech16-1_2005,Nature392-694_1998}材料计算模拟研究的不断深入,也推动了材料模拟核心计算软件日益成熟。核心计算软件解决的是物质运动基本方程求解和基本物性计算的问题,除此之外,完整的材料模拟过程还包括“计算前处理”的问题建模、计算参数选择到“计算后处理”的数据分析,几乎每一部分都有大量具体的工作有赖人工,频繁的人机交互严重影响了模拟计算流程的顺畅与效率。进入二十一世纪,大规模高性能计算的兴起,逐渐变革了材料计算的研发模式。通过计算流程设计,衔接不同尺度的计算过程,降低人机交互频次,优化模拟流程中的任务调度,实现材料计算流程的自动化与跨尺度物性模拟,形成完善的材料数据库,已经成为材料计算软件发展的新趋势。

通过理论计算模拟不同尺度下的材料物性,不但可以节约新材料研制环节中的设计、试验和制造成本,缩短研发周期,还可能提供实验过程难以获得的信息。以燃烧催化材料的研究为例,在催化剂作用下,多相催化-氧化反应和燃烧过程的自由基反应同时发生,\upcite{ACSSym-Seri12-495_1992}这使得催化燃烧反应的研究变得非常困难,因此严重制约了燃烧催化剂研发。模拟计算可以不受燃烧实验的条件限制,对催化燃烧的微观过程给出理性的分析,为实验制备与合成催化剂研究提供理论依据。\upcite{PCC118-1999_2014,Nat-Commun8-14621_2017,CES56-2659_2001}但是催化燃烧过程是典型的跨尺度问题,对模拟计算软件和计算流程都提出了具体的需求,主要的研究难点是:




几十年来,机器学习的算法已经广泛应用于金融、导航控制、语言处理、游戏竞技、计算机可视化和生物信息学等领域。相反地,如果从非严格角度定义,任何计算机模拟人类智能的算法都可以划归为人工智能,并非一定要应用机器学习算法,也包括决策树、知识库、计算机逻辑等算法。近年来,机器学习领域的深度学习\textrm{(Deep Learning, DL)}异军突起,在很多领域都取得了很好的应用\upcite{DL_2016}。深度学习是仿照生物神经网络(意味着输入输出之间允许有多个类似神经的网络层)结构为主要代表的一种示类学习。
