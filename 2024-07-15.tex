%---------------------- TEMPLATE FOR REPORT ------------------------------------------------------------------------------------------------------%

%\thispagestyle{fancy}   % 插入页眉页脚                                        %
%%%%%%%%%%%%%%%%%%%%%%%%%%%%% 用 authblk 包 支持作者和E-mail %%%%%%%%%%%%%%%%%%%%%%%%%%%%%%%%%
%\title{More than one Author with different Affiliations}				     %
%\title{\rm{VASP}的电荷密度存储文件\rm{CHGCAR}}
%\title{面向高温合金材料设计的计算模拟软件中的几个主要问题}
\title{数据驱动视野下的材料设计}
\author[ ]{}   %
%\author[ ]{姜~骏\thanks{jiangjun@bcc.ac.cn}}   %
%\affil[ ]{北京市计算中心}    %
%\author[a]{Author A}									     %
%\author[a]{Author B}									     %
%\author[a]{Author C \thanks{Corresponding author: email@mail.com}}			     %
%%\author[a]{Author/通讯作者 C \thanks{Corresponding author: cores-email@mail.com}}     	     %
%\author[b]{Author D}									     %
%\author[b]{Author/作者 D}								     %
%\author[b]{Author E}									     %
%\affil[a]{Department of Computer Science, \LaTeX\ University}				     %
%\affil[b]{Department of Mechanical Engineering, \LaTeX\ University}			     %
%\affil[b]{作者单位-2}			    						     %
											     %
%%% 使用 \thanks 定义通讯作者								     %
											     %
\renewcommand*{\Authfont}{\small\rm} % 修改作者的字体与大小				     %
\renewcommand*{\Affilfont}{\small\it} % 修改机构名称的字体与大小			     %
\renewcommand\Authands{ and } % 去掉 and 前的逗号					     %
\renewcommand\Authands{ , } % 将 and 换成逗号					     %
\date{} % 去掉日期									     %
\date{2024-07-15}									     %

%%%%%%%%%%%%%%%%%%%%%%%%%%%%%%%%%%%%%%%%%%  不使用 authblk 包制作标题  %%%%%%%%%%%%%%%%%%%%%%%%%%%%%%%%%%%%%%%%%%%%%%
%-------------------------------The Title of The Report-----------------------------------------%
%\title{报告标题:~}   %
%-----------------------------------------------------------------------------

%----------------------The Authors and the address of The Paper--------------------------------%
%\author{
%\small
%Author1, Author2, Author3\footnote{Communication author's E-mail} \\    %Authors' Names	       %
%\small
%(The Address,City Post code)						%Address	       %
%}
%\affil[$\dagger$]{清华大学~材料加工研究所~A213}
%\affil{清华大学~材料加工研究所~A213}
%\date{}					%if necessary					       %
%----------------------------------------------------------------------------------------------%
%%%%%%%%%%%%%%%%%%%%%%%%%%%%%%%%%%%%%%%%%%%%%%%%%%%%%%%%%%%%%%%%%%%%%%%%%%%%%%%%%%%%%%%%%%%%%%%%%%%%%%%%%%%%%%%%%%%%%
%\maketitle
%\thispagestyle{fancy}   % 首页插入页眉页脚 
20世纪90年代以来,随着理论化学、计算方法的快速发展和计算机技术不断升级和更新,计算材料科学获得空前发展,它是传统材料科学与物理、化学、工程力学以及应用数学等诸多基础和应用学科交叉与融合形成的一门新兴学科,在材料研究中发挥越来越重要的作用\upcite{NatMat3-429_2004,App-CataA254-5_2003,JACS125-4306_2003,JCombChem5-472_2003,Meas_Sci-Tech16-1_2005,Nature392-694_1998}。在新材料研究中,借鉴药物组合合成与筛选\textrm{(combinatorial synthesis and screening)}思想\upcite{Science268-1738_1995}形成的高通量(\textrm{high~throughput})材料计算\upcite{Nature410-643_2001}模式,在21世纪最初十年,成为创新发展新材料、发现新现象的重要手段。材料学的主要任务之一,就是通过组分、观测尺度与材料属性的关联,发现改善和提升材料性能的途径。当前,计算材料学的研究重点主要围绕以密度泛函理论\textrm{(Density Functional Theory, DFT)}为代表的第一原理、分子动力学方法为代表的微观材料结构-属性展开。据统计,以\textrm{VASP}软件为代表的第一原理材料计算,是科学计算的执牛耳者,甚至占据了核心算力的半壁江山。\textcolor{red}{(必要性)}
\begin{figure}[h!]
\centering
\vspace*{-0.05in}
\includegraphics[width=0.6\textwidth]{Figures/Mat_Geno_Ene-1.png}
%\vskip 0.10in
%\includegraphics[height=0.85in]{Figures/Mat_Geno_Ene-3.png}
\caption{材料基因组基本理念的图示解析.}
\label{Fig:Mater_Genome}
\end{figure}

\begin{figure}[h!]
\centering
\vspace*{-0.05in}
%\includegraphics[width=0.6\textwidth]{Figures/Mat_Geno_Ene-1.png}
%\vskip 0.10in
\includegraphics[width=0.95\textwidth]{Figures/Mat_Geno_Ene-3.png}
\includegraphics[width=0.95\textwidth]{Figures/Multi_Scale-2.jpeg}
\caption{还原论:~由构成物质的最小单元出发,通过不同尺度间材料属性参数的传递,构建微观-宏观材料结构-属性的关联关系.}
\label{Fig:Mater_Genome-3}
\end{figure}
得益于高精度的多尺度计算方法和高性能并行计算技术的突破\upcite{PRL88-255506_2002,Nano-Lett3-1183_2003},计算材料可以提供大量材料数据,降低新材料的研发成本。进入21世纪以后,美国、欧洲、日本等发达国家先后启动了以变革研发模式、提升材料设计成功率,缩短材料开发周期为目的的多种类型的国家级科研专项,其中以美国的“材料基因组计划\textrm{(Materials Genome Initiative, MGI)}”\upcite{MGI_USA}最为著名。该计划的根本目的是通过增效集成各个尺度的材料模拟工具、高效实验手段和数据库,把材料研发从传统 经验式提升到科学设计,从而大大加快材料研发速度(见\textrm{Fig.}\ref{Fig:Mater_Genome})。我国也于2016年起启动了国家重点研发计划“材料基因工程关键技术与支撑平台”。随着``材料基因工程关键技术与支撑平台''项目的实施,数据在材料研发中的作用日益重要,基于对已有数据的关联分析,成为继经验探索、理性提炼和计算模拟之后,新的深化材料内在结构-性质关联关系认识的重要形式。\textcolor{red}{(重要性)}

材料的组分、结构和物性数据是新材料研发和改善材料性能提升的重要参数,在计算材料学科发展之前,人类主要通过实验手段积累材料数据,由于客观实验条件的限制,新材料的研发成本非常高。鉴于材料构成的复杂性,传统的从微观-宏观的还原论研究模式(见\textrm{Fig.}\ref{Fig:Mater_Genome-3}),在跨越材料尺度的计算中常常遭遇计算瓶颈;而着眼``多者异也''的演生论,在具体材料研究中,同样缺乏具体的执行方案,主要依赖多尺度计算在各个尺度的结构-属性关系的探索,难以形成有效的跨尺度耦合计算。\textcolor{red}{(问题挑战)}高性能计算与\textrm{DFT}相结合推动了第一原理材料模拟数据生成和结果分析的自动化,涌现了越来越多的计算材料数据库,特别在功能材料领域,高通量\textrm{DFT}计算大大推进了新材料的合成与进步。\upcite{JCED59-3232_2014, IC53-11849_2014,JPCL4-3607_2013,PCCP16-22073_2014}值得注意的是,高通量第一原理计算自动流程解决了材料物性数据的生成问题,但是自动流程无法给出现材料数据基础上的物性优化的方案。利用数据挖掘技术,实现数据驱动的材料物性筛选、预测和提升的技术路线,有着特殊重要的意义(见\textrm{Fig.}\ref{Fig:Multi_scale-modelling})。\textcolor{red}{(现状机遇)}通过大规模数据的学习(训练和验证),从大量材料数据中挖掘高维复杂数据的内在关联,变革传统材料研发模式,实现材料性能的有效预测,有着计算材料科学研究的内在迫切驱动, 成为近年来材料科学和数据科学的研究重点,前景广阔。不难预见,计算材料对高性能计算的算力需求,无论从体量还是存储来看,都极其可观。\textcolor{red}{(迫切性)}
\begin{figure}[h!]
\centering
\vspace*{-0.05in}
%\includegraphics[width=0.6\textwidth]{Figures/Mat_Geno_Ene-1.png}
%\vskip 0.10in
\includegraphics[width=0.95\textwidth]{Figures/Schematic_flow-of-multi_scale-modelling.png}
\caption{机器学习方法实现的数据驱动的材料跨尺度模拟的图示.数据驱动有望成为实现跨尺度计算的重要工具.}
\label{Fig:Multi_scale-modelling}
\end{figure}

航空发动机是飞机的心脏,单晶高温合金叶片是航空发动机的核心部件,其组分多元,工作温度高达1800$^{\circ}\mathrm{C}$,制造工艺极其复杂。其研发和生产能力已经成为评估一个国家国防装备和科学技术水平的重要发展标志。创新发展资源化第四代\ch{Ni}-基单晶高温合金是当前国防发展的重要需求,通过材料计算先期大规模筛选复杂合金材料的组合模式,有效地控制试验用合金元素的组成和比例范围,加快单晶叶片的研发速度并有效降低研发成本。以\ch{Ni}-基单晶高温合金性能优化研究为应用牵引,探索材料数据从设计、生产到流通的全过程协同的机制,成为数据驱动材料研究的新模式。\textcolor{red}{(行业范围、客户类型和应用场景)}

%面向计算材料研究领域日益增长的数据驱动的内在要求,
参赛的平台秉承材料基因工程的研究理念,变革固有研究模式对复杂材料研究的制约,实现材料数据生产流通过程所必须的模型与设计、理论方法与算法支持、算力提供和数据共享的协同,推动了以数据为核心驱动力的设计(包括模型设计和算法设计)、生产、流通全链条化。\textcolor{red}{(作品创新性)}平台适应了大数据场景下新材料研发需求,把握数据要素这个核心理念,依托市场规则完成数据在各主体间的有序流通与共享。\textcolor{red}{(作品有效性)}随着数据要素重要性的日益凸显,参赛平台的数据运营模式,对于其它行业有着显著的示范作用。\textcolor{red}{(作品可推广性)}

参赛平台依托的全国高校计算联盟(见\textrm{Fig.}\ref{Fig:Eaas-HPC}),为数据的产出提供实力雄厚的高性能计算资源;团队成员专业与分工布局合理,涵盖了材料计算软件与方法、高性能计算系统运维和合金材料性能和数据运营领域等专业,为平台的运营奠定了坚实的软硬件基础。\textcolor{red}{(技术优势和服务优势)}平台面向计算材料研究中的现实困境,通过对材料性能提升和优化过程涉及的数据要素的分析,实现研究者身份按数据要素流通阶段的角色分工,通过数据流通和分享推动新材料的研究。\textcolor{red}{(产品化优势)}。参赛平台融合了计算材料方法与软件、高性能计算运维和数据流通的运作模式,实现数据要素在不同类型的需求者间有效而安全地共享和流通,成为材料基因工程理念的重要践行者。对数据要素为中心的相关行业,平台的运营模式有着独特的示范价值。\textcolor{red}{(同类方案的竞争力)}

\begin{figure}[h!]
\centering
\vspace*{-0.05in}
%\includegraphics[width=0.6\textwidth]{Figures/Mat_Geno_Ene-1.png}
%\vskip 0.10in
\includegraphics[width=0.95\textwidth]{Figures/Eaas-HPC-Cloud.png}
\vskip 2pt
\includegraphics[width=0.95\textwidth]{Figures/Eaas-HPC-Cloud-resource.png}
\caption{全国高校计算联盟支持的计算类型(上)和为用户分配的计算资源(下)示范.}
\label{Fig:Eaas-HPC}
\end{figure}

%组合与筛选研究兴起于1990年代中期,
%在传统的研究范式下,获取材料物性数据的成本,无论是通过实验手段还是计算模拟,代价都是比较高的,虽然高通量第一原理计算自动流程和数据库解决了材料物性数据的获取问题,但是并未给出现有材料数据基础上的物性优化的方案,因此利用数据挖掘技术,实现数据驱动的材料物性筛选、预测和提升的技术路线,有着特殊重要的意义。机器学习\textrm{(Machine Learning, ML)}技术可以从大量数据中获得有价值的信息,尤其是面对高维复杂数据时,机器学习技术是确定数据间关系的有力的工具。


%二十世纪七十年代以来,随着计算机科学和密度泛函理论\textrm{(Density Functional Theory, DFT)}\upcite{PRB136-864_1964,PRA140-1133_1965,Parr-Yang,CR91-651_1991}、分子动力学\textrm{(Molecular Dynamics, MD)}的发展和融合,计算材料科学\textrm{(computational materials science)}与物理、化学、工程力学以及应用数学等诸多基础和应用学科交叉日益紧密,逐渐成为一门新兴的独立学科。\upcite{Nat-Mat3-429_2004,App-CataA5-254_2003,JACS125-4306_2003,JCC5-472_2003,Mater_Sci-Tech16-1_2005,Nature392-694_1998}材料计算模拟研究的不断深入,也推动了材料模拟核心计算软件日益成熟。核心计算软件解决的是物质运动基本方程求解和基本物性计算的问题,除此之外,完整的材料模拟过程还包括“计算前处理”的问题建模、计算参数选择到“计算后处理”的数据分析,几乎每一部分都有大量具体的工作有赖人工,频繁的人机交互严重影响了模拟计算流程的顺畅与效率。进入二十一世纪,大规模高性能计算的兴起,逐渐变革了材料计算的研发模式。通过计算流程设计,衔接不同尺度的计算过程,降低人机交互频次,优化模拟流程中的任务调度,实现材料计算流程的自动化与跨尺度物性模拟,形成完善的材料数据库,已经成为材料计算软件发展的新趋势。

%通过理论计算模拟不同尺度下的材料物性,不但可以节约新材料研制环节中的设计、试验和制造成本,缩短研发周期,还可能提供实验过程难以获得的信息。以燃烧催化材料的研究为例,在催化剂作用下,多相催化-氧化反应和燃烧过程的自由基反应同时发生,\upcite{ACSSym-Seri12-495_1992}这使得催化燃烧反应的研究变得非常困难,因此严重制约了燃烧催化剂研发。模拟计算可以不受燃烧实验的条件限制,对催化燃烧的微观过程给出理性的分析,为实验制备与合成催化剂研究提供理论依据。\upcite{PCC118-1999_2014,Nat-Commun8-14621_2017,CES56-2659_2001}但是催化燃烧过程是典型的跨尺度问题,对模拟计算软件和计算流程都提出了具体的需求,主要的研究难点是:

%几十年来,机器学习的算法已经广泛应用于金融、导航控制、语言处理、游戏竞技、计算机可视化和生物信息学等领域。相反地,如果从非严格角度定义,任何计算机模拟人类智能的算法都可以划归为人工智能,并非一定要应用机器学习算法,也包括决策树、知识库、计算机逻辑等算法。近年来,机器学习领域的深度学习\textrm{(Deep Learning, DL)}异军突起,在很多领域都取得了很好的应用\upcite{DL_2016}。深度学习是仿照生物神经网络(意味着输入输出之间允许有多个类似神经的网络层)结构为主要代表的一种示类学习。
