%---------------------- TEMPLATE FOR REPORT ------------------------------------------------------------------------------------------------------%

%\thispagestyle{fancy}   % 插入页眉页脚                                        %
%%%%%%%%%%%%%%%%%%%%%%%%%%%%% 用 authblk 包 支持作者和E-mail %%%%%%%%%%%%%%%%%%%%%%%%%%%%%%%%%
%\title{More than one Author with different Affiliations}				     %
%\title{\rm{VASP}的电荷密度存储文件\rm{CHGCAR}}
%\title{面向高温合金材料设计的计算模拟软件中的几个主要问题}
\title{关于论文的一点意见}
%\author[ ]{姜~骏}   %
%\author[ ]{姜~骏\thanks{jiangjun@bcc.ac.cn}}   %
%\affil[ ]{北京市计算中心}    %
%\author[a]{Author A}									     %
%\author[a]{Author B}									     %
%\author[a]{Author C \thanks{Corresponding author: email@mail.com}}			     %
%%\author[a]{Author/通讯作者 C \thanks{Corresponding author: cores-email@mail.com}}     	     %
%\author[b]{Author D}									     %
%\author[b]{Author/作者 D}								     %
%\author[b]{Author E}									     %
%\affil[a]{Department of Computer Science, \LaTeX\ University}				     %
%\affil[b]{Department of Mechanical Engineering, \LaTeX\ University}			     %
%\affil[b]{作者单位-2}			    						     %
											     %
%%% 使用 \thanks 定义通讯作者								     %
											     %
\renewcommand*{\Authfont}{\small\rm} % 修改作者的字体与大小				     %
\renewcommand*{\Affilfont}{\small\it} % 修改机构名称的字体与大小			     %
\renewcommand\Authands{ and } % 去掉 and 前的逗号					     %
\renewcommand\Authands{ , } % 将 and 换成逗号					     %
\date{} % 去掉日期									     %
%\date{2020-12-30}									     %

%%%%%%%%%%%%%%%%%%%%%%%%%%%%%%%%%%%%%%%%%%  不使用 authblk 包制作标题  %%%%%%%%%%%%%%%%%%%%%%%%%%%%%%%%%%%%%%%%%%%%%%
%-------------------------------The Title of The Report-----------------------------------------%
%\title{报告标题:~}   %
%-----------------------------------------------------------------------------

%----------------------The Authors and the address of The Paper--------------------------------%
%\author{
%\small
%Author1, Author2, Author3\footnote{Communication author's E-mail} \\    %Authors' Names	       %
%\small
%(The Address,City Post code)						%Address	       %
%}
%\affil[$\dagger$]{清华大学~材料加工研究所~A213}
%\affil{清华大学~材料加工研究所~A213}
%\date{}					%if necessary					       %
%----------------------------------------------------------------------------------------------%
%%%%%%%%%%%%%%%%%%%%%%%%%%%%%%%%%%%%%%%%%%%%%%%%%%%%%%%%%%%%%%%%%%%%%%%%%%%%%%%%%%%%%%%%%%%%%%%%%%%%%%%%%%%%%%%%%%%%%
\maketitle
%\thispagestyle{fancy}   % 首页插入页眉页脚 
主要的几个问题
\begin{itemize}
	\item 表述过于口语话,缺乏严格、精炼、准确的中文科学语言训练:\\
		如\textbf{摘要部分}:\\
		第二段:~``几个关键问题……独立性问题……监管的问题……不实用等问题''\\
		2.部分第一段:~``此外……最后……''\\
		\textbf{绪论部分}:\\
		第一段对第一章节列出相关部分内容基本无必要\\
		由于拼音输入``即''`既`''使用混乱\\

		总体看,相关章节的文字表述需要花大力气梳理(对第一章和结论章节的表述认为有问题部分,我用高亮标出来了)
	\item 1.2~国内外研究现状\\
		总括部分只提了国内的研究情况,对国外的研究只字未提\\
		就国内部分而言,只是罗列了研究内容,没有表达作者对相关研究的问题、难点的较为深刻的点评或对比,无法表明作者对课题研究深度有鉴别能力\\
		1.2.1-1,2.3 应该围绕``国内外研究现状'',对相关研究(比如国外研究部分)做出系统的梳理和罗列,然后指出各自的长处和不足,现在的写法成了基本原理和理论的概述——这些内容显然应该单起章节为更合适。简言之,这不是在描述``现状''
	\item 1.3~研究内容和创新点,应该是在前面综述的基础上提炼出来,因为1.2的综述写的不理想,逻辑上的1.3没有依托,而是基于1.2的原理产生出来的想法,缺乏前人研究的经验、教训总结
	\item 1.4~本文的组织结构可以附属于1.3结尾部分,没有太大必要单起
	\item 预备知识的介绍缺乏最基本的背景介绍,上来就罗列概念和定理,这样的写法非常缺乏科研论文的格式,应该将``绪论''的相关原理和这里的基本概念结合起来,凝练出论文所需要用到的主要基础知识——简言之,作为博士论文的基础知识和背景部分,应该是本专业的人在教科书基础上,结合你介绍的基础知识,能够看懂本人。所以基本概念、定理的罗列,是要有针对地服务于后面的研究,不能简单罗列
	\item 第三、四、五章的内容之间,能看出一定的关联性,但作为应用场景/对象的博物馆藏品的数字化问题的具体和特殊之处,还是应该更明确地表述出来\\
	\item 现在全文在围绕``博物馆藏品数字化资产交易''这个重要应用场景的具体性和特殊性方面的表达有比较大的加强的必要,现在全文讨论的更多是理论和原理,但相关研究在``博物馆藏品数字化资产交易''这一具体应用提到的很少,这和全论文一开始有针对的提``博物馆藏品数字化资产交易''缺乏呼应
\end{itemize}
