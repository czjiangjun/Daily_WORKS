%---------------------- TEMPLATE FOR REPORT ------------------------------------------------------------------------------------------------------%

%\thispagestyle{fancy}   % 插入页眉页脚                                        %
%%%%%%%%%%%%%%%%%%%%%%%%%%%%% 用 authblk 包 支持作者和E-mail %%%%%%%%%%%%%%%%%%%%%%%%%%%%%%%%%
%\title{More than one Author with different Affiliations}				     %
%\title{\rm{VASP}的电荷密度存储文件\rm{CHGCAR}}
%\title{面向高温合金材料设计的计算模拟软件中的几个主要问题}
\title{\rm{VASP}中的\rm{FFT}处理}
\author[ ]{姜~骏}   %
%\author[ ]{姜~骏\thanks{jiangjun@bcc.ac.cn}}   %
%\affil[ ]{北京市计算中心}    %
%\author[a]{Author A}									     %
%\author[a]{Author B}									     %
%\author[a]{Author C \thanks{Corresponding author: email@mail.com}}			     %
%%\author[a]{Author/通讯作者 C \thanks{Corresponding author: cores-email@mail.com}}     	     %
%\author[b]{Author D}									     %
%\author[b]{Author/作者 D}								     %
%\author[b]{Author E}									     %
%\affil[a]{Department of Computer Science, \LaTeX\ University}				     %
%\affil[b]{Department of Mechanical Engineering, \LaTeX\ University}			     %
%\affil[b]{作者单位-2}			    						     %
											     %
%%% 使用 \thanks 定义通讯作者								     %
											     %
\renewcommand*{\Authfont}{\small\rm} % 修改作者的字体与大小				     %
\renewcommand*{\Affilfont}{\small\it} % 修改机构名称的字体与大小			     %
\renewcommand\Authands{ and } % 去掉 and 前的逗号					     %
\renewcommand\Authands{ , } % 将 and 换成逗号					     %
\date{} % 去掉日期									     %
%\date{2020-12-30}									     %

%%%%%%%%%%%%%%%%%%%%%%%%%%%%%%%%%%%%%%%%%%  不使用 authblk 包制作标题  %%%%%%%%%%%%%%%%%%%%%%%%%%%%%%%%%%%%%%%%%%%%%%
%-------------------------------The Title of The Report-----------------------------------------%
%\title{报告标题:~}   %
%-----------------------------------------------------------------------------

%----------------------The Authors and the address of The Paper--------------------------------%
%\author{
%\small
%Author1, Author2, Author3\footnote{Communication author's E-mail} \\    %Authors' Names	       %
%\small
%(The Address,City Post code)						%Address	       %
%}
%\affil[$\dagger$]{清华大学~材料加工研究所~A213}
%\affil{清华大学~材料加工研究所~A213}
%\date{}					%if necessary					       %
%----------------------------------------------------------------------------------------------%
%%%%%%%%%%%%%%%%%%%%%%%%%%%%%%%%%%%%%%%%%%%%%%%%%%%%%%%%%%%%%%%%%%%%%%%%%%%%%%%%%%%%%%%%%%%%%%%%%%%%%%%%%%%%%%%%%%%%%
\maketitle
%\begin{figure}

\begin{lrbox}{\mysavebox}%
\tikzstyle{startstop} = [rectangle,rounded corners, minimum width=7cm,minimum height=1cm,text centered,text width =7cm, draw=black,fill=red!30]
\tikzstyle{io} = [trapezium, trapezium left angle = 70,trapezium right angle=110,minimum width=3cm,minimum height=1cm,text centered,text width =3cm,draw=black,fill=blue!30]
\tikzstyle{process} = [rectangle,minimum width=7cm,minimum height=1cm,text centered,text width =7cm,draw=black,fill=orange!30]
\tikzstyle{process-sub} = [rectangle,minimum width=5cm,minimum height=1cm,text centered,text width =5cm,draw=black,fill=blue!10]
\tikzstyle{decision} = [diamond,aspect = 3,text centered,draw=black,fill=green!30]
\tikzstyle{arrow} = [thick,->,>=stealth]
\tikzstyle{straightline} = [line width = 1pt,-]
\tikzstyle{dashedline} = [dashed, line width = 1pt,-]
\tikzstyle{point}=[coordinate]

\begin{tikzpicture}[node distance=1.8cm]
	\node (start) [startstop] {模块与变量声明、初始赋值\\打开必要的文档\\输出版本信息};
%\node (input1) [io,below of=start] {输入聚类的个数 $k$ 和最大迭代次数 $n$ };
	\node (process1) [process,below of=start,yshift=-1cm] {读入\textrm{POSCAR}文件头\\初次读入\textrm{POTCAR}\\读入\textrm{INCAR}\\准备芯层移动相关的计算\\$\vec k$~空间能量展宽相关准备};
	\node (process2) [process,below of=process1,yshift=-1cm] {再次读入\textrm{POTCAR}\\准备\textrm{LDA}+\textit{U}初始化};
	\node (process3) [process,below of=process2] {原子\textrm{PAW}计算的电荷密度差\\一致性检验};
	\node (process4) [process-sub,below of=process3] {读入\textrm{POSCAR}};
	\node (process5) [process,below of=process4] {响应函数等相关初始设置\\检索交换-相关势列表};
	\node (process6) [process,below of=process5] {晶体对称性初始化\\读入\textrm{KPOINTS}文件\\生成与$\vec k$~点有关的量};
	\node (process7) [process,below of=process6] {设置波函数有关的基本量\\生成$\vec k$~空间积分所需的网格点};
	\node (process8) [process,below of=process7] {输出各类初始化信息\\晶格参数、$\vec k$点信息};
	\node (process9) [process,below of=process8] {\textrm{allocate}各类数组\\包括电荷、势、波函数};
	\node (process10) [process,below of=process9] {原子\textrm{PAW}有关势函数计算};
	\node (process11) [process,below of=process10] {\textcolor{red}{\hei{构造初始电荷密度}}};
	\node (process12) [process,below of=process11] {\textcolor{red}{\hei{计算实空间或倒空间投影子算符}}};
	\node (process13) [process,below of=process12] {初始化能量本征值、\textrm{Fermi}能\\积分权重、缀加电荷};
	\node (process14) [decision,below of=process13,yshift=-1cm] {判断\textrm{INCAR}文件中是否提供积分权重};
	\node (process15) [io,below of=process14,xshift=-5cm, yshift=-1cm] {读入\textrm{INCAR}\\积分权重};
	\node (process16) [process,below of=process14, yshift=-3cm] {\textcolor{red}{\hei{计算投影算符作用\\于波函数的投影}}};
	\node (process17) [process,below of=process16] {\textcolor{red}{\hei{计算$\mathbf{D}_{ij}$}}};
	\node (process18) [process,below of=process17] {\textcolor{red}{\hei{波函数正交化}}};
	\node (process19) [process,below of=process18] {\textcolor{red}{\hei{计算初始电荷密度}}};
	\node (process20) [process,below of=process19] {\textcolor{red}{\hei{重新计算局域势}}};
	\node (process21) [process,below of=process20] {\textcolor{red}{\hei{一些特殊计算的准备}}};

%\node (decision1) [decision,below of=process2,yshift=-0.5cm] {是否收敛或迭代次数达到 $n$ };
%\node (stop) [startstop,below of=decision1,node distance=3cm] {输出聚类结果};
	\node(point1)[point,left of=process3 ,xshift =-1cm]{};
	\node(point2)[point,above of=process2]{};
	\node(point3)[point,left of=process1, xshift =-3cm]{};
	\node(point4)[point,left of=process6, xshift =-4cm]{};
	\node(point5)[point,left of=process9, xshift =-6cm]{};

\draw [arrow] (start) -- (process1);
%\draw [arrow] (input1) -- (process1);
\draw [arrow] (process1) -- (process2);
\draw [arrow] (process2) -- (process3);
\draw [arrow] (process3) -- (process4);
\draw [arrow] (process4) -- (process5);
\draw [arrow] (process5) -- (process6);
\draw [arrow] (process6) -- (process7);
\draw [arrow] (process7) -- (process8);
\draw [arrow] (process8) -- (process9);
\draw [arrow] (process9) -- (process10);
\draw [arrow] (process10) -- (process11);
%\draw [dashedline] (process3.west) -- (point1);
%\draw [dashedline] (point1) -- (point3);
%\draw [dashedline] (point3) -- (process1.west);
%\draw [dashedline] (process3) -- (point1);
%\draw [dashedline] (point1) |- (point2);
%\draw [dashedline] (point2) -- (process2.north);
%\draw [dashedline] (process7.west) -- (process3.east);
%\draw [dashedline] (process3.north) -- (point4);
%\draw [dashedline] (point4) -- (process6.west);
%\draw [arrow] (process4) -- (process5);
%\draw [arrow] (process2.east) -| (process7.north);
%\draw [arrow] (process6) -- (process8);
%\draw [arrow] (process6) |- (process7);
%\draw [arrow] (process7) |- (process8.east);
%\draw [arrow] (process8) -- node[anchor=south] {否} (process9);
%\draw [arrow] (process7) -- (process9);
%\draw [arrow] (decision1) -- node[anchor=east] {是} (stop);
%\draw [straightline] (decision1) -|  (point1);
%\draw [straightline] (process6) -|  (point5);
%\draw [arrow] (point1) -- node[anchor=south] {否} (input1);
%\draw [arrow] (point5) -- (process9);
\end{tikzpicture}
\end{lrbox}%
\ifdim\ht\mysavebox>\textheight
    \setlength{\myrest}{\ht\mysavebox}%
    \loop\ifdim\myrest>\textheight
        \newpage\par\noindent
        \clipbox{0 {\myrest-\textheight} 0 {\ht\mysavebox-\myrest}}{\usebox{\mysavebox}}%
        \addtolength{\myrest}{-\textheight}%
    \repeat
    \newpage\par\noindent
    \clipbox{0 0 0 {\ht\mysavebox-\myrest}}{\usebox{\mysavebox}}%
\else
    \usebox{\mysavebox}%
\fi

%\caption{\textrm{The Maiin-Flow of VASP.}}
%\label{Fig:VASP_MAIN-FLOW}
%\end{figure}
%%%%%%%%%%%%%%%%%%%%%%%%%%%%%%%%%%%%%%%%  绘制流程图  %%%%%%%%%%%%%%%%%%%%%%%%%%%%%%%%%%%%%%%%%%%%%%%%%%%%%%%%%%%%%%%%%
%\tikzstyle{startstop} = [rectangle,rounded corners, minimum width=3cm,minimum height=1cm,text centered,text width =3cm, draw=black,fill=red!30]
%\tikzstyle{io} = [trapezium, trapezium left angle = 70,trapezium right angle=110,minimum width=3cm,minimum height=1cm,text centered,text width =3cm,draw=black,fill=blue!30]
%\tikzstyle{process} = [rectangle,minimum width=3cm,minimum height=1cm,text centered,text width =3cm,draw=black,fill=orange!30]
%\tikzstyle{decision} = [diamond,aspect = 3,text centered,draw=black,fill=green!30]
%\tikzstyle{arrow} = [thick,->,>=stealth]
%\tikzstyle{straightline} = [line width = 1pt,-]
%\tikzstyle{point}=[coordinate]
%
%\begin{tikzpicture}[node distance=2cm]
%\node (start) [startstop] {开始};
%\node (input1) [io,below of=start] {输入聚类的个数 $k$ 和最大迭代次数 $n$ };
%\node (process1) [process,below of=input1] {初始化 $k$ 个聚类中心};
%\node (process2) [process,below of=process1] {分配各数据对象到距离最近的类中};
%\node (decision1) [decision,below of=process2,yshift=-0.5cm] {是否收敛或迭代次数达到 $n$ };
%\node (stop) [startstop,below of=decision1,node distance=3cm] {输出聚类结果};
%\node(point1)[point,left of=input1,node distance=5cm]{};
%
%\draw [arrow] (start) -- (input1);
%\draw [arrow] (input1) -- (process1);
%\draw [arrow] (process1) -- (process2);
%\draw [arrow] (process2) -- (decision1);
%\draw [arrow] (decision1) -- node[anchor=east] {是} (stop);
%\draw [straightline] (decision1) -|  (point1);
%\draw [arrow] (point1) -- node[anchor=south] {否} (input1);
%\end{tikzpicture}

%%%%%%%%%%%%%%%%%%%%%%%%%%%%%%%%%%%%%%%%  绘制流程图  %%%%%%%%%%%%%%%%%%%%%%%%%%%%%%%%%%%%%%%%%%%%%%%%%%%%%%%%%%%%%%%%%
%\begin{figure}
%\scriptsize
%\tikzstyle{format}=[rectangle,draw,thin,fill=white]
%%定义语句块的颜色,形状和边
%\tikzstyle{test}=[diamond,aspect=2,draw,thin]
%%定义条件块的形状,颜色
%\tikzstyle{point}=[coordinate,on grid,]
%%像素点,用于连接转移线
%\begin{tikzpicture}%[node distance=10mm,auto,>=late',thin,start chain=going below,every join/.style={norm},]
%%start chain=going below指明了流程图的默认方向,node distance=8mm则指明了默认的node距离。这些可以在定义node的时候更改,比如说
%%\node[point,right of=n3,node distance=10mm] (p0){};
%%这里声明了node p0,它在node n3 的右边,距离是10mm。
%\node[format] (start){Start};
%\node[format,below of=start,node distance=7mm] (define){Some defines};
%\node[format,below of=define,node distance=7mm] (PCFinit){PCF8563 Initialize};
%\node[format,below of=PCFinit,node distance=7mm] (DS18init){DS18 Initialize};
%\node[format,below of=DS18init,node distance=7mm] (LCDinit){LCD Initialize};
%\node[format,below of=LCDinit,node distance=7mm] (processtime){Processtime};
%\node[format,below of=processtime,node distance=7mm] (keyinit){Key Initialize};
%\node[test,below of=keyinit,node distance=15mm](setkeycheck){Check Set Key};
%\node[point,left of=setkeycheck,node distance=18mm](point3){};
%\node[format,below of=setkeycheck,node distance=15mm](readtime){Read Time};
%\node[point,right of=readtime,node distance=15mm](point4){};
%\node[format,below of=readtime](processtime1){Processtime};
%\node[format,below of=processtime1](gettemp){Get Temperature};
%\node[format,below of=gettemp](display){Display All Data};
%\node[format,right of=setkeycheck,node distance=40mm](setsetflag){Set SetFlag=1};
%\node[format,below of=setsetflag](setinit){Set Mode Initialize};
%\node[format,below of=setinit](checksetting){Checksetting()};
%\node[test,below of=checksetting,node distance=15mm](savecheck){Check Save Key};
%\node[format,below of=savecheck,node distance=15mm](clearsetflag){Clear SetFlag=0};
%\node[format,below of=clearsetflag](settime){Set Time};
%\node[point,below of=display,node distance=7mm](point1){};
%\node[point,below of=settime,node distance=7mm](point2){};
%%\node[format] (n0) at(4,4){A}; 直接指定位置
%%定义完node之后进行连线,
%%\draw[->] (n0.south) -- (n1); 带箭头实线
%%\draw[-] (n0.south) -- (n1); 不带箭头实线
%%\draw[&lt;->] (n0.south) -- (n1.north);   双箭头
%%\draw[&lt;-,dashed] (n1.south) -- (n2.north); 带箭头虚线 
%%\draw[&lt;-] (n0.south) to node{Yes} (n1.north);  带字,字在箭头方向右边
%%\draw[->] (n1.north) to node{Yes} (n0.south);  带字,字在箭头方向左边
%%\draw[->] (n1.north) to[out=60,in=300] node{Yes} (n0.south);  曲线
%%\draw[->,draw=red](n2)--(n1);  带颜色的线
%\draw[->] (start)--(define);
%\draw[->] (define)--(PCFinit);
%\draw[->](PCFinit)--(DS18init);
%\draw[->](DS18init)--(LCDinit);
%\draw[->](LCDinit)--(processtime);
%\draw[->](processtime)--(keyinit);
%\draw[->](keyinit)--(setkeycheck);
%\draw[->](setkeycheck)--node[above]{Yes}(setsetflag);
%\draw[->](setkeycheck) --node[left]{No} (readtime);
%\draw[->](readtime)--(processtime1);
%\draw[->](processtime1)--(gettemp);
%\draw[->](gettemp)--(display);
%\draw[-](display)--(point1);
%\draw[-](point1)-|(point3);
%\draw[->](point3)--(setkeycheck.west);
%\draw[->](setsetflag)--(setinit);
%\draw[->](setinit)--(checksetting);
%\draw[->](checksetting)--(savecheck);
%\draw[->](savecheck)--node[left]{Yes}(clearsetflag);
%\draw[->](savecheck.west)|-node[left]{No}(checksetting);
%\draw[->](clearsetflag)--(settime);
%\draw[-](settime)--(point2);
%\draw[-](point2)-|(point4);
%\draw[->](point4)--(readtime.east);
%\end{tikzpicture}
%\end{figure}
