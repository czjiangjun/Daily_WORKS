%---------------------- TEMPLATE FOR REPORT ------------------------------------------------------------------------------------------------------%

%\thispagestyle{fancy}   % 插入页眉页脚                                        %
%%%%%%%%%%%%%%%%%%%%%%%%%%%%% 用 authblk 包 支持作者和E-mail %%%%%%%%%%%%%%%%%%%%%%%%%%%%%%%%%
%\title{More than one Author with different Affiliations}				     %
%\title{\rm{VASP}的电荷密度存储文件\rm{CHGCAR}}
%\title{面向高温合金材料设计的计算模拟软件中的几个主要问题}
\title{\rm{VASP}中的主程序结构}
\author[ ]{姜~骏}   %
%\author[ ]{姜~骏\thanks{jiangjun@bcc.ac.cn}}   %
%\affil[ ]{北京市计算中心}    %
%\author[a]{Author A}									     %
%\author[a]{Author B}									     %
%\author[a]{Author C \thanks{Corresponding author: email@mail.com}}			     %
%%\author[a]{Author/通讯作者 C \thanks{Corresponding author: cores-email@mail.com}}     	     %
%\author[b]{Author D}									     %
%\author[b]{Author/作者 D}								     %
%\author[b]{Author E}									     %
%\affil[a]{Department of Computer Science, \LaTeX\ University}				     %
%\affil[b]{Department of Mechanical Engineering, \LaTeX\ University}			     %
%\affil[b]{作者单位-2}			    						     %
											     %
%%% 使用 \thanks 定义通讯作者								     %
											     %
\renewcommand*{\Authfont}{\small\rm} % 修改作者的字体与大小				     %
\renewcommand*{\Affilfont}{\small\it} % 修改机构名称的字体与大小			     %
\renewcommand\Authands{ and } % 去掉 and 前的逗号					     %
\renewcommand\Authands{ , } % 将 and 换成逗号					     %
\date{} % 去掉日期									     %
%\date{2020-12-30}									     %

%%%%%%%%%%%%%%%%%%%%%%%%%%%%%%%%%%%%%%%%%%  不使用 authblk 包制作标题  %%%%%%%%%%%%%%%%%%%%%%%%%%%%%%%%%%%%%%%%%%%%%%
%-------------------------------The Title of The Report-----------------------------------------%
%\title{报告标题:~}   %
%-----------------------------------------------------------------------------

%----------------------The Authors and the address of The Paper--------------------------------%
%\author{
%\small
%Author1, Author2, Author3\footnote{Communication author's E-mail} \\    %Authors' Names	       %
%\small
%(The Address,City Post code)						%Address	       %
%}
%\affil[$\dagger$]{清华大学~材料加工研究所~A213}
%\affil{清华大学~材料加工研究所~A213}
%\date{}					%if necessary					       %
%----------------------------------------------------------------------------------------------%
%%%%%%%%%%%%%%%%%%%%%%%%%%%%%%%%%%%%%%%%%%%%%%%%%%%%%%%%%%%%%%%%%%%%%%%%%%%%%%%%%%%%%%%%%%%%%%%%%%%%%%%%%%%%%%%%%%%%%
\maketitle
%\begin{figure}

\begin{tikzpicture}[
    box/.style={rectangle,draw,fill=DarkGray!20,node distance=1cm,text width=15em,text centered,rounded corners,minimum height=2em,thick},
    box1/.style={rectangle,draw,fill=green!50,node distance=1cm,text width=12em,text centered,rounded corners,minimum height=2em,thick},
    box2/.style={rectangle,draw,fill=magenta!75,node distance=1cm,text width=12em,text centered,rounded corners,minimum height=2em,thick},
    box3/.style={rectangle,draw,fill=blue!30,node distance=1cm,text width=12em,text centered,rounded corners,minimum height=2em,thick},
    box4/.style={rectangle,draw,fill=violet!45,node distance=2cm,text width=13em,text centered,rounded corners,minimum height=2em,thick},
%    arrow/.style={draw,-latex', thick},
    arrow/.style={draw,-latex', red, line width=2pt},
  ]
%  \node [box] (potential) {$v_{\text{ext},s}(\vec r)=v_\text{H}(\vec r) + v_\text{xc}(\vec r) + v_\text{ext}(\vec r)$};
  \node [box1,text width=8em] (Softwares) {实际程序\\实现和应用};
%  \node [box,below=0.5 of potential] (hamiltonian) {$\hat{H}_{KS}=-\frac{\hbar^2}{2m}\vec{\nabla}^2 + v_{\text{ext},s}(\vec r)$};
  \node [box2,below=0.5 of Softwares] (Method) {基组和势函数处理方法\\$\vec k$空间积分方案};
%  \node [box,below=0.5 of hamiltonian] (se) {$\hat{H}_{KS} \phi_i(\vec r)= E_i \phi_i(\vec r)$};
  \node [box3,below=0.7 of Method, text width=13em] (Theory) {基本理论\\\textrm{DFT}和\textrm{Band Structure}};
%  \node [box,below=0.5 of se] (density) {$\rho(\vec r)=\sum_{i=1}^n f_i\,|\phi_i(\vec r_i)|^2$};
  \node [box4,below=0.8 of Theory, text width=15em] (Basic-Equ) {(学科)基本方程\\量子力学:~\textrm{Schr\"odinger}方程\\$\hat{\mathbf H}\Psi=E\Psi$};
%  \node [box,below=0.5 of density] (criterion) {Convergence criterion satisfied?};

%  \node [box,above=1.5 of potential, fill=orange!30, text width=20em] (initial) {Supply initial density guess $\rho_\text{ini}(\vec r)$ to Kohn Sham equations};
  \node [box4,left=0.5 of Basic-Equ, fill=violet!30, text width=5em, minimum height=5em] (PDE) {经典数值\\分析算法};
%  \node [box,below=1.5 of criterion, fill=blue!30, text width=20em] (energy) {Use $\rho_\text{fin}(\vec r)$ to minimize total energy functional $E_{V_\text{ext}}[\rho]=T_{e,s}[\phi_i\{\rho\}] + V_{ee,H}[\rho] + E_{xc}[\rho] + V_{eI}[\rho]$};
  \node [box4,left=1.0 of Method, fill=magenta!50, text width=5em] (Algorithm) {具体\\算法\\实现};

  \node [box4,right=0.5 of Basic-Equ, fill=violet!30, text width=5em] (OS) {操作\\系统\\与\\编程\\技术};
  \node [box4,right=1.0 of Method, fill=magenta!50, text width=5em] (coding) {程序语言};
%  \node [box4,right=1.0 of Method, fill=magenta!50, text width=5em, draw=red, thick] at (-5,0) (Figure) {\includegraphics[width=0.5in, height=0.4in]{Figures/Multi-Scale-6.png}};
%  \path [arrow] (initial) -- (potential);
  \path [arrow] (Method) -- (Softwares);
  \path [arrow] (Theory) -- (Method);
  \path [arrow] (Basic-Equ) -- (Theory);
%  \path [arrow] (density) -- (criterion);

  \path [arrow] (PDE) -- (Algorithm);
  \path [arrow] (OS) -- (coding);
 % \path [arrow] (OS) -- (Figure);
  \path [arrow] (coding) |- (Softwares.east);
  \path [arrow] (Algorithm) |- (Softwares.west);
  \path [arrow] (Method.east) -- (coding.west);
  \path [arrow] (Method.west) -- (Algorithm.east);

%%%%%% ------------ 定义 虚线方框转角 -----------------  %%%%%%
%  \path
%  (potential.north west) ++(-1em,1em) coordinate (potential fit)
%  (criterion.south east) ++(1em,-1em) coordinate (criterion fit);

%%%%%% ------------ 绘制 虚线方框 -----------------  %%%%%%
%  \node [rectangle,draw,dashed,inner sep=1em,fit=(potential fit) (criterion fit)] (enclosure) {};
%  \node [above=-0.8em of enclosure,anchor=south,draw,outer sep=0pt,fill=white] (enclosure label) {\Large\textbf{Kohn-Sham method}};

%  \path [arrow] (criterion) -- (energy) node [midway,left=0.1,draw,outer sep=0pt,fill=white] (TextNode) {Yes};
%  \path [draw,thick] (criterion.south) ++(0em,-1em) -- (criterion fit) node [midway,below=0.1,sloped,draw,outer sep=0pt,fill=white] (TextNode) {No};
%  \draw [arrow] (criterion fit) |- (potential.east);
\end{tikzpicture}

\begin{lrbox}{\mysavebox}%
\tikzstyle{startstop} = [rectangle,rounded corners, minimum width=7cm,minimum height=1cm,text centered,text width =7cm, draw=black,fill=red!30]
\tikzstyle{io} = [trapezium, trapezium left angle = 70,trapezium right angle=110,minimum width=3cm,minimum height=1cm,text centered,text width =3cm,draw=black,fill=blue!30]
\tikzstyle{process} = [rectangle,minimum width=7cm,minimum height=1cm,text centered,text width =7cm,draw=black,fill=orange!30]
\tikzstyle{process-sub} = [rectangle,minimum width=5cm,minimum height=1cm,text centered,text width =5cm,draw=black,fill=blue!10]
\tikzstyle{decision} = [diamond,aspect = 3,text centered,draw=black,fill=green!30]
\tikzstyle{arrow} = [thick,->,>=stealth]
\tikzstyle{straightline} = [line width = 1pt,-]
\tikzstyle{dashedline} = [dashed, line width = 1pt,-]
\tikzstyle{point}=[coordinate]

\begin{tikzpicture}[node distance=1.8cm]
	\node (start) [startstop] {程序启动\\模块与变量声明、初始赋值\\打开必要的文档};
	\node (process0) [io,below of=start] {输出版本信息};
%\node (input1) [io,below of=start] {输入聚类的个数 $k$ 和最大迭代次数 $n$ };
	\node (processin1-1) [io,below of=process0] {读入\textrm{POSCAR}文件头};
	\node (processin1-2) [io,below of=processin1-1] {初次读入\\\textrm{POTCAR}文件};
	\node (processin1-3) [io,below of=processin1-2] {读入\textrm{INCAR}文件};
        \node (process1) [process,below of=processin1-3] {准备芯层移动相关的计算\\$\vec k$~空间能量展宽相关准备};
	\node (process2) [io,below of=process1] {再次读入\\\textrm{POTCAR}文件};
	\node (processin2) [process,below of=process2] {准备\textrm{LDA}+\textit{U}初始化};
	\node (process3) [process,below of=processin2] {原子\textrm{PAW}计算的电荷密度差\\一致性检验};
	\node (process4) [io,below of=process3] {读入完整的\textrm{POSCAR}文件};
	\node (process5) [process,below of=process4] {响应函数等相关初始设置\\检索交换-相关势列表};
	\node (process6) [process,below of=process5] {晶体对称性初始化};
	\node (processin6) [io,below of=process6] {读入\textrm{KPOINTS}文件};
	\node (process6-1) [process,below of=processin6] {生成与$\vec k$~点有关的量};
	\node (process7) [process,below of=process6-1] {设置波函数有关的基本量\\生成$\vec k$~空间积分所需的网格点};
	\node (process8) [io,below of=process7] {输出各类初始化信息};
	\node (processout8-1) [io,below of=process8] {输出晶格参数信息};
	\node (processout8-2) [io,below of=processout8-1] {输出$\vec k$点信息};
	\node (process9) [process,below of=processout8-2] {\textrm{allocate}各类数组\\包括电荷、势、波函数};
	\node (process10) [process,below of=process9] {原子\textrm{PAW}有关势函数计算};
	\node (process11) [process,below of=process10] {\textcolor{red}{\hei{构造初始电荷密度}}};
	\node (process12) [process,below of=process11] {\textcolor{red}{\hei{计算实空间或倒空间投影子算符}}};
	\node (process13) [process,below of=process12] {初始化能量本征值、\textrm{Fermi}能\\积分权重、缀加电荷};
	\node (process14) [decision,below of=process13,yshift=-1cm] {判断\textrm{INCAR}文件中是否提供积分权重};
	\node (process15) [io,below of=process14,xshift=-5cm, yshift=-1cm] {读入\textrm{INCAR}文件\\中的积分权重};
	\node (process16) [process,below of=process14, yshift=-3cm] {\textcolor{red}{\hei{计算投影算符作用\\于波函数的投影}}};
	\node (process17) [process,below of=process16] {\textcolor{red}{\hei{计算$\mathbf{D}_{ij}$}}};
	\node (process18) [process,below of=process17] {\textcolor{red}{\hei{波函数正交化}}};
	\node (process19) [process,below of=process18] {\textcolor{red}{\hei{计算初始电荷密度}}};
	\node (process20) [process,below of=process19] {\textcolor{red}{\hei{重新计算局域势}}};
	\node (process21) [process,below of=process20] {\textcolor{red}{\hei{一些特殊计算的准备}}};
	\node (process22) [process,below of=process21] {\textcolor{red}{\hei{电子步计算\\(总能量最小化)}}};
	\node (process23) [process,below of=process22] {能带占据数检查};
	\node (process24) [process,below of=process23] {\textcolor{red}{\hei{静态受力和应力计算}}};
	\node (process25) [process,below of=process24] {检查受力和总能的一致性};
	\node (process26) [process,below of=process25] {重新计算倒空间势函数};
	\node (process27) [decision,below of=process26] {离子运动判据};
	\node (process28) [process-sub,below of=process27,xshift=-3cm] {\textcolor{red}{\hei{分子动力\\学计算}}};
	\node (process29) [process-sub,below of=process27,xshift=3cm] {\textcolor{red}{\hei{晶格弛豫\\计算}}};
	\node (process30) [process,below of=process27,yshift=-3.6cm] {\textcolor{red}{\hei{离子位置变换后\\重新计算投影函数、\\局域势}}};
	\node (process31) [decision,below of=process30] {收敛判据};
	\node (process32) [io,below of=process31] {输出电荷密度};
	\node (process33) [io,below of=process32] {输出能量本征值};
	\node (process34) [io,below of=process33] {输出当前电荷密度};
	\node (process35) [process,below of=process34] {计算光学性质};
	\node (process36) [process,below of=process35] {计算双电子\\多中心积分};
	\node (process37) [process,below of=process36] {计算总的\textrm{DOS}、\\分波占据和\\分波\textrm{DOS}};
	\node (process38) [io,below of=process37] {输出\textrm{DOS}和\\分波\textrm{DOS}};
	\node (stop) [startstop,below of=process38] {程序结束};

%\node (decision1) [decision,below of=process2,yshift=-0.5cm] {是否收敛或迭代次数达到 $n$ };
%\node (stop) [startstop,below of=decision1,node distance=3cm] {输出聚类结果};
	\node(point1)[point,left of=process14 ,xshift =-3.2cm]{};
	\node(point2)[point,left of=process16, xshift =-3.2cm]{};
	\node(point3)[point,left of=process27, xshift =-1.2cm]{};
	\node(point4)[point,left of=process27, xshift =4.8cm]{};
	\node(point5)[point,below of=process28]{};
	\node(point6)[point,below of=process29]{};
	\node(point7)[point,above of=process30]{};
	\node(point8)[point,left of=process22,xshift =-5cm,yshift= 1cm]{};
	\node(point10)[point,above of=process22,yshift=-0.8cm]{};
	\node(point9)[point,left of=process31,xshift = -5cm]{};

\draw [arrow] (start) -- (process0);
\draw [arrow] (process0) -- (processin1-1);
\draw [arrow] (processin1-1) -- (processin1-2);
\draw [arrow] (processin1-2) -- (processin1-3);
%\draw [arrow] (input1) -- (process1);
\draw [arrow] (processin1-3) -- (process1);
\draw [arrow] (process1) -- (process2);
\draw [arrow] (process2) -- (processin2);
\draw [arrow] (processin2) -- (process3);
\draw [arrow] (process3) -- (process4);
\draw [arrow] (process4) -- (process5);
\draw [arrow] (process5) -- (process6);
\draw [arrow] (process6) -- (processin6);
\draw [arrow] (processin6) -- (process6-1);
\draw [arrow] (process6-1) -- (process7);
\draw [arrow] (process7) -- (process8);
\draw [arrow] (process8) -- (processout8-1);
\draw [arrow] (processout8-1) -- (processout8-2);
\draw [arrow] (processout8-2) -- (process9);
\draw [arrow] (process9) -- (process10);
\draw [arrow] (process10) -- (process11);
\draw [arrow] (process11) -- (process12);
\draw [arrow] (process12) -- (process13);
\draw [arrow] (process13) -- (process14);
\draw [straightline] (process14.west) -- node[anchor=south]{是}(point1);
\draw [arrow] (point1) -- (process15);
\draw [straightline] (process15) -- (point2);
\draw [arrow] (point2) -- (process16.west);
\draw [arrow] (process14) -- node[anchor=east]{否} (process16);
\draw [arrow] (process16) -- (process17);
\draw [arrow] (process17) -- (process18);
\draw [arrow] (process18) -- (process19);
\draw [arrow] (process19) -- (process20);
\draw [arrow] (process20) -- (process21);
\draw [arrow] (process21) -- (process22);
\draw [straightline] (point8) -- (point10);
\draw [arrow] (point10) -- (process22);
\draw [arrow] (process22) -- (process23);
\draw [arrow] (process23) -- (process24);
\draw [arrow] (process24) -- (process25);
\draw [arrow] (process25) -- (process26);
\draw [arrow] (process26) -- (process27);
\draw [straightline] (process27.west) -- node[anchor=south]{分子动力学}(point3);
\draw [straightline] (process27.east) -- node[anchor=south]{结构弛豫}(point4);
\draw [arrow] (point3) -- (process28);
\draw [arrow] (point4) -- (process29);
\draw [straightline] (process28) -- (point5);
\draw [straightline] (process29) -- (point6);
\draw [straightline] (point5) -- (point7);
\draw [straightline] (point6) -- (point7);
\draw [arrow] (point7) -- (process30);
\draw [arrow] (process30) -- (process31);
\draw [straightline] (process31.west) -- node[anchor=south]{否}(point9);
\draw [straightline] (point9) -- (point8);
\draw [arrow] (process31) -- node[anchor=west]{是}(process32);
\draw [arrow] (process32) -- (process33);
\draw [arrow] (process33) -- (process34);
\draw [arrow] (process34) -- (process35);
\draw [arrow] (process35) -- (process36);
\draw [arrow] (process36) -- (process37);
\draw [arrow] (process37) -- (process38);
\draw [arrow] (process38) -- (stop);
%\draw [dashedline] (process3.west) -- (point1);
%\draw [dashedline] (point1) -- (point3);
%\draw [dashedline] (point3) -- (process1.west);
%\draw [dashedline] (process3) -- (point1);
%\draw [dashedline] (point1) |- (point2);
%\draw [dashedline] (point2) -- (process2.north);
%\draw [dashedline] (process7.west) -- (process3.east);
%\draw [dashedline] (process3.north) -- (point4);
%\draw [dashedline] (point4) -- (process6.west);
%\draw [arrow] (process4) -- (process5);
%\draw [arrow] (process2.east) -| (process7.north);
%\draw [arrow] (process6) -- (process8);
%\draw [arrow] (process6) |- (process7);
%\draw [arrow] (process7) |- (process8.east);
%\draw [arrow] (process8) -- node[anchor=south] {否} (process9);
%\draw [arrow] (process7) -- (process9);
%\draw [arrow] (decision1) -- node[anchor=east] {是} (stop);
%\draw [straightline] (decision1) -|  (point1);
%\draw [straightline] (process6) -|  (point5);
%\draw [arrow] (point1) -- node[anchor=south] {否} (input1);
%\draw [arrow] (point5) -- (process9);
\end{tikzpicture}
\end{lrbox}%
\ifdim\ht\mysavebox>\textheight
    \setlength{\myrest}{\ht\mysavebox}%
    \loop\ifdim\myrest>\textheight
        \newpage\par\noindent
        \clipbox{0 {\myrest-\textheight} 0 {\ht\mysavebox-\myrest}}{\usebox{\mysavebox}}%
        \addtolength{\myrest}{-\textheight}%
    \repeat
    \newpage\par\noindent
    \clipbox{0 0 0 {\ht\mysavebox-\myrest}}{\usebox{\mysavebox}}%
\else
    \usebox{\mysavebox}%
\fi

%\caption{\textrm{The Maiin-Flow of VASP.}}
%\label{Fig:VASP_MAIN-FLOW}
%\end{figure}


\begin{tikzpicture}[
    box/.style={rectangle,draw,fill=darkgray!20,node distance=1cm,text width=15em,text centered,rounded corners,minimum height=2em,thick},
    box1/.style={diamond, aspect=7, inner sep=2pt, draw,fill=green!50,node distance=1cm,text width=12em,text centered,rounded corners,minimum height=2em,thick},
    box2/.style={rectangle,draw,fill=magenta!75,node distance=1cm,text width=12em,text centered,rounded corners,minimum height=2em,thick},
    box3/.style={rectangle,draw,fill=blue!30,node distance=1cm,text width=12em,text centered,rounded corners,minimum height=2em,thick},
    box4/.style={rectangle,draw,fill=violet!45,node distance=2cm,text width=13em,text centered,rounded corners,minimum height=2em,thick},
%    arrow/.style={draw,-latex', thick},
    arrow/.style={draw,-latex', red},
  ]
%  \node [box] (potential) {$v_{\text{ext},s}(\vec r)=v_\text{H}(\vec r) + v_\text{xc}(\vec r) + v_\text{ext}(\vec r)$};
  \node [box, fill=brown!80, text width=2.5em, minimum height=0.5em,xshift=-30.0em, yshift=-7.5em] at (current page.west)(Symmetry-OP) {\fontsize{5.0pt}{2.5pt}\selectfont{对称操作}};
  \node [box, fill=blue!30, text width=7.0em, minimum height=0.5em] at(-7.5,12.45) (Init-Struct) {\fontsize{5.0pt}{2.5pt}\selectfont{初始结构~(经验或猜测)}};
  \node [box, fill=blue!30, text width=5.0em, minimum height=0.5em] at(-7.5,11.35) (Init-Charge) {\fontsize{5.0pt}{2.5pt}\selectfont{初始电荷密度$\rho_0$}};
  \node [box, draw=blue, fill=white, text width=8.0em, minimum height=0.5em] at(-4.0,11.6) (DFT-SCF) {\fontsize{16.0pt}{4.5pt}\selectfont{\textcolor{blue}{\bf{DFT-$SCF$}}}};
%  \node [dot,  below=0.1 of Init-Charge] (Corner1) {};
  \node [box, below=0.6 of Symmetry-OP, fill=blue!120, text width=5.0em, minimum height=1em] (Potential) {\fontsize{5.0pt}{2.5pt}\selectfont{\textcolor{white}{全势\textrm{(FP)}方法\vskip -1.3em 赝势\textrm{(PP)}方法}}};
  \node [box, right=0.3 of Potential, fill=blue!60, text width=6.0em, minimum height=0.5em] (Hartree) {\fontsize{5.0pt}{2.5pt}\selectfont{\textrm{Hartree}势$V_\mathrm{H}(\vec r)$}};
  \node [box, right=0.8 of Hartree, fill=blue!60, text width=7.0em, minimum height=0.5em] (Exch-Corr) {\fontsize{5.0pt}{2.5pt}\selectfont{交换-相关势$V_{\mathrm{XC}}[\rho(\vec r)]$}};
  \node [box, below=0.2 of Potential, fill=orange!80, text width=4.0em, minimum height=0.5em] (Basis-set) {\fontsize{5.0pt}{2.5pt}\selectfont{基函数}};
  \node [box, right=0.8 of Basis-set, fill=red!80, text width=7.0em, minimum height=0.5em] (Fock-matrix) {\fontsize{5.0pt}{2.5pt}\selectfont{构造\textrm{Fock}矩阵}};
  \node [box, below=0.2 of Fock-matrix, fill=green!80, text width=7.0em, minimum height=0.5em] (Kohn-Sham) {\fontsize{5.0pt}{2.5pt}\selectfont{求解\textrm{Kohn-Sham}方程\vskip -1.3em (\textrm{Fock}矩阵对角化)}};
  \node [box, below=0.2 of Kohn-Sham, fill=blue!80, text width=7.0em, minimum height=0.5em] (Wave-Eigen) {\fontsize{5.0pt}{2.5pt}\selectfont{\textcolor{white}{本征态波函数$\Psi_i(\vec r)$,\vskip -1.3em 能量本征值$\varepsilon_i$}}};
%  \node [box, draw=blue, fill=white, text width=5.0em, minimum height=1em] at(-3.0,-2.7) (k-space) {\fontsize{5.0pt}{2.5pt}\selectfont{$\vec k$-空间积分方案}};
  \node [box, below=0.2 of Wave-Eigen, fill=blue!80, text width=7.0em, minimum height=0.5em] (Charge) {\fontsize{5.0pt}{2.5pt}\selectfont{\textcolor{white}{计算电荷密度$\rho$}}};
  \node [box, draw=blue, below=2.60 of Exch-Corr, fill=purple!30, text width=5.0em, minimum height=0.5em] (k-space) {\fontsize{5.0pt}{2.5pt}\selectfont{$\vec k$-空间积分方案}};
  \node [box, below=0.2 of Charge, fill=blue!80, text width=5.5em, minimum height=0.5em] (ele-ele) {\fontsize{4.0pt}{2.5pt}\selectfont{\textcolor{white}{电子-电子静电作用$E_{e-e}$}}};
  \node [box, left=0.05 of ele-ele, fill=blue!80, text width=5.5em, minimum height=0.5em] (Ion-ele) {\fontsize{4.0pt}{2.5pt}\selectfont{\textcolor{white}{离子-电子静电作用$E_{\mathrm{N-}e}$}}};
%  \node [box, left=2.4 of Basis-set, fill=blue!80, text width=4.6em, minimum height=0.5em] at(-11,0.0) (Ion-Static) {\fontsize{4.0pt}{2.5pt}\selectfont{离子静电作用$E_{\mathrm{N-N}}$}};
  \node [box, left=0.05 of Ion-ele, fill=blue!80, text width=4.6em, minimum height=0.5em] (Ion-Static) {\fontsize{4.0pt}{2.5pt}\selectfont{\textcolor{white}{离子静电作用$E_{\mathrm{N-N}}$}}};
  \node [box, right=0.05 of ele-ele, fill=blue!80, text width=5.0em, minimum height=0.5em] (Kinetic-Ene) {\fontsize{4.0pt}{2.5pt}\selectfont{\textcolor{white}{电子动能$T=\sum T_i$}}};
  \node [box, right=0.05 of Kinetic-Ene, fill=blue!80, text width=5.0em, minimum height=0.5em] (Exch-Corr-Ene) {\fontsize{4.0pt}{2.5pt}\selectfont{\textcolor{white}{交换-相关能$E[\rho(\vec r)]$}}};
%%%%%% ------------ 定义 虚线方框转角 -----------------  %%%%%%
\path
  (Init-Charge.south) ++(0.0 em, -0.3em) coordinate (Potential-fit1)
  (Symmetry-OP.east) ++(1.5em, 0.0em) coordinate (Symmetry-OP-fit)
  (Hartree.south) ++(0.0 em, -0.3em) coordinate (Hartree-fit)
  (Exch-Corr.south) ++(0.0em, -0.3em) coordinate (Exch-Corr-fit)
  (Potential.north west) ++(-4.0em,0.15em) coordinate (Potential-fit)
  (Exch-Corr-Ene.south east) ++(0.1em,-0.15em) coordinate (Exch-Corr-Ene-fit1)
  (Fock-matrix.west) ++(-1.0 em, 0.75em) coordinate (Fock-matrix-fit)
  (Kohn-Sham.east) ++(1.0em, -1.0em) coordinate (Kohn-Sham-fit)
  (Ion-ele.north west) ++(-5.5em,0.03em) coordinate (Ion-ele-fit)
  (Exch-Corr-Ene.south east) ++(0.04em,-0.03em) coordinate (Exch-Corr-Ene-fit);

  \node [rectangle,draw=red, inner sep=0.1em,fit=(Ion-ele-fit) (Exch-Corr-Ene-fit),thick] (enclosure2) {};
  \node [box1, below=0.05 of enclosure2, fill=purple!80, text width=7.0em, minimum height=0.5em] (Criterion) {\fontsize{5.0pt}{2.5pt}\selectfont{收敛判据}};
  \node [box, below=0.2 of Criterion, fill=purple!80, text width=7.0em, minimum height=0.5em] (SCF-output) {\fontsize{4.2pt}{1.5pt}\selectfont{输出电荷密度$\rho(\vec r)$\vskip 0.2em 本征态波函数$\Psi_i(\vec r)$ \vskip -1.9em 能量本征值$\varepsilon_i$}};
%  \node [box,below=0.5 of potential] (hamiltonian) {$\hat{H}_{KS}=-\frac{\hbar^2}{2m}\vec{\nabla}^2 + v_{\text{ext},s}(\vec r)$};
%  \node [box,below=0.5 of hamiltonian] (se) {$\hat{H}_{KS} \phi_i(\vec r)= E_i \phi_i(\vec r)$};
%  \node [box,below=0.5 of se] (density) {$\rho(\vec r)=\sum_{i=1}^n f_i\,|\phi_i(\vec r_i)|^2$};
%  \node [box,below=0.5 of density] (criterion) {Convergence criterion satisfied?};

%  \node [box,above=1.5 of potential, fill=orange!30, text width=20em] (initial) {Supply initial density guess $\rho_\text{ini}(\vec r)$ to Kohn Sham equations};
%  \node [box,below=1.5 of criterion, fill=blue!30, text width=20em] (energy) {Use $\rho_\text{fin}(\vec r)$ to minimize total energy functional $E_{V_\text{ext}}[\rho]=T_{e,s}[\phi_i\{\rho\}] + V_{ee,H}[\rho] + E_{xc}[\rho] + V_{eI}[\rho]$};

\path
  (Basis-set.south) ++(0.0em,-6.55em) coordinate (Basis-set-fit)
  (Charge.west) ++(-1.25em,0.0em) coordinate (Charge-fit)
  (Wave-Eigen.west) ++(-1.25em,0.0em) coordinate (Wave-Eigen-fit)
  (Potential.north west) ++(-3.9em,0.20em) coordinate (Potential-fit2)
  (Criterion.west) ++(-8.9em,0.0em) coordinate (Criterion-fit)
  (SCF-output.west) ++(-11.40em,-1.5em) coordinate (SCF-output-fit)
  (Symmetry-OP.west) ++(-5.5em, 0.0em) coordinate (Symmetry-OP-fit2)
  (k-space.east) ++(4.35em, 0.0em) coordinate (k-space-fit) ;

%  \path [arrow] (initial) -- (potential);
  \path [arrow, gray] (Init-Struct) -- (Init-Charge);
  \path [arrow] (Symmetry-OP.east) -| (Init-Charge.north);
  \path [arrow, blue!90] (Potential) -- (Hartree);

  \path [draw, magenta](Init-Charge.south) ++(0em, -0.0em) --(Potential-fit1);
  \path [arrow, draw=magenta] (Potential-fit1) -| (Hartree.north);
  \path [draw, magenta](Init-Charge.south) ++(0em, -0.0em) --(Potential-fit1);
  \path [arrow, draw=magenta] (Potential-fit1) -| (Exch-Corr.north);
  \path [draw, magenta](Hartree.south) -- (Hartree-fit) ++ (0em, 0.0em);
  \path [arrow, draw=magenta] (Hartree-fit) -| (Fock-matrix.north);
  \path [draw, magenta](Exch-Corr.south) --(Exch-Corr-fit) ++(0em, 0.0em);
  \path [arrow, draw=magenta] (Exch-Corr-fit) -| (Fock-matrix.north);

  \path [arrow, draw=orange] (Basis-set) -- (Fock-matrix);
  \path [arrow, draw=orange] (Basis-set.north) -- (Potential.south);
  \path [draw, dotted, brown] (Symmetry-OP.east) ++(0em, 0.0em)-- (Symmetry-OP-fit);
  \path [arrow, dotted, draw=brown] (Symmetry-OP-fit) |- (Fock-matrix.west);

%%%%%% ------------ 绘制 虚线方框 -----------------  %%%%%%
  \node [rectangle,draw,dashdotted,inner sep=0.1em,fit=(Potential-fit) (Exch-Corr-Ene-fit1)] (enclosure) {};


  \path [arrow, draw=black] (Fock-matrix) -- (Kohn-Sham);
  \path [arrow, draw=black] (Kohn-Sham) -- (Wave-Eigen);
  \path [arrow, draw=black] (Wave-Eigen) -- (Charge);
%  \node [rectangle,draw=red, fill=gray!30, inner sep=0.1em,fit=(Fock-matrix-fit) (Kohn-Sham-fit), thick] (enclosure1) {};
  \node [rectangle,draw=red, dotted, inner sep=0.1em,fit=(Fock-matrix-fit) (Kohn-Sham-fit), thick] (enclosure1) {};
 % \path [arrow] (OS) -- (Figure);
  \path [arrow, draw=cyan] (k-space.west) -| (Charge.north);

  \node [above=0.2em of Ion-Static, anchor=south,outer sep=0pt] (Etot-label){\fontsize{5.0pt}{2.5pt}\selectfont{\textcolor{red}{$\mathbf{E}_{TOT}$}}};
  \path [arrow] (Charge) --(enclosure2);
  \path [draw, magenta] (Charge) --(Charge-fit);
  \path [draw, magenta] (Charge-fit) |-(Potential-fit1);
  \path [draw, magenta] (Wave-Eigen) --(Wave-Eigen-fit);
  \path [draw, magenta] (Wave-Eigen-fit) |-(Potential-fit1);
  \path [draw, orange] (Basis-set) --(Basis-set-fit);
  \path [draw, orange] (Basis-set-fit) -|(enclosure2);

  \path [arrow] (enclosure2)-- (Criterion);


  \path [arrow] (Criterion)--(SCF-output) node[midway,right=0.1,outer sep=0pt](TextNode1){\fontsize{4.0pt}{2.5pt}\selectfont{\textrm{\textcolor{black}{Yes}}}};
  \path [draw,magenta] (Criterion.west)++(0.25em, 0.0em)--(Criterion-fit) node[midway,above=0.05,outer sep=0pt](TextNode1){\fontsize{4.0pt}{2.5pt}\selectfont{\textrm{\textcolor{black}{NO}}}};
  \path [draw,magenta] (Criterion-fit) |-(Potential-fit1);
  \node [box, right=0.3 of SCF-output, fill=green!50, text width=7.0em, minimum height=0.5em] (SCF-properties) {\fontsize{5.0pt}{2.5pt}\selectfont{各类物理数据与性质计算}};
  \node [box, fill=magenta!50, text width=8.5em, draw=pink, minimum height=0.5em, thick] at (-3.8,9.15) (Figure) {\includegraphics[width=1.05in, height=0.7in,viewport=0 0 1172 863,clip]{DFT.png}};

  \path [arrow, draw=purple] (SCF-output.east) --(SCF-properties.west);
  \path [draw, dotted, brown] (Symmetry-OP.west) --(Symmetry-OP-fit2);
  \path [draw, dotted, brown] (Symmetry-OP-fit2) --(SCF-output-fit);
  \path [arrow, dotted, draw=brown](SCF-output-fit)-|(SCF-properties.south);
  \path [draw, cyan] (k-space) --(k-space-fit);
  \path [arrow, cyan](k-space-fit)|-(SCF-properties.east);
  \path [arrow, draw, fill=red, line width=2.5pt] (enclosure1.east) -- (Figure.west);

%  \node [above=-0.8em of enclosure,anchor=south,draw,outer sep=0pt,fill=white] (enclosure label) {\Large\textbf{Kohn-Sham method}};

%  \path [arrow] (criterion) -- (energy) node [midway,left=0.1,draw,outer sep=0pt,fill=white] (TextNode) {Yes};
%  \path [draw,thick] (criterion.south) ++(0em,-1em) -- (criterion fit) node [midway,below=0.1,sloped,draw,outer sep=0pt,fill=white] (TextNode) {No};
%  \draw [arrow] (criterion fit) |- (potential.east);
\end{tikzpicture}

\begin{tikzpicture}[
    box-rect/.style={rectangle,draw,fill=darkgray!20,node distance=1cm,text width=15em,text centered,rounded corners,minimum height=2em,thick},
    box-diam/.style={diamond, aspect=7, inner sep=2pt, draw,fill=green!50,node distance=1cm,text width=12em,text centered,rounded corners,minimum height=2em,thick},
    arrow/.style={draw,-latex', red},
  ]

  \node [box-rect, fill=magenta!50, text width=8.5em, draw=pink, minimum height=0.5em, thick] at (current page.west)(Figure) {\includegraphics[width=1.25in, height=0.7in,viewport=0 0 103 55,clip]{The-Periodic-Table-of-perovskene-materials.jpg}};
  \node [box-rect, right=0.9 of Figure, fill=gray!30, text width=12.5em, minimum height=0.5em] (Highthrough-Put) {\fontsize{7.5pt}{5.5pt}\selectfont{高通量材料计算\textrm{Python/Web}服务}};
  \node [box-rect, right=1.0 of Highthrough-Put, fill=white, text width=0.5em, minimum height=1.5em, yshift=-3.5em, very thin] (Modeling-Initial) {\fontsize{7.5pt}{2.5pt}\selectfont{结构建模与初始化}};
  \node [box-rect, below=0.5 of Figure, fill=red!30, text width=7.5em, minimum height=0.5em] (Modeling-mode) {\fontsize{7.5pt}{5.5pt}\selectfont{计算结构建模\\\textrm{+ASE\_Atoms}模块}};
  \node [box-rect, right=1.0 of Modeling-mode, fill=red!30, text width=4.5em, minimum height=0.5em] (User-defineded) {\fontsize{7.5pt}{5.5pt}\selectfont{用户需求自定义}};
  \node [box-rect, right=1.0 of User-defineded, fill=brown!30, text width=4.5em, minimum height=0.5em] (Process-Design) {\fontsize{7.5pt}{5.5pt}\selectfont{计算流程设计}};
  \node [box-rect, below=0.5 of Modeling-mode, fill=green!30, text width=10.5em, minimum height=0.5em] (Modeling-Analysis) {\fontsize{7.5pt}{5.5pt}\selectfont{组分、模型与结构对称性分析}};

  \node [box-rect, below=1.5 of Modeling-Analysis, fill=blue!50, text width=6.5em, minimum height=0.5em, xshift=3.5em] (Process-Manage) {\fontsize{7.5pt}{2.5pt}\selectfont{计算流程管理\\\textrm{MP\_FireWorks}模块}};
  \node [box-rect, right=1.95 of Process-Manage, fill=red!50, text width=9.0em, minimum height=6.0em] (Multi-Scale) {\fontsize{7.5pt}{5.5pt}\selectfont{\vskip -25pt高性能计算任务\\\textrm{DFT-MD-$\cdots$}\\\textrm{(跨尺度计算)}\\\textrm{+}集成核心计算软件接口}};
  \node [box-rect, right=2.1 of Process-Manage, fill=orange!50, text width=8.2em, minimum height=0.5em, yshift=-1.8em] (Custodian) {\fontsize{7.5pt}{5.5pt}\selectfont{容错机制\textrm{Cudodian}模块}};
  \node [box-rect, below=0.05 of Modeling-Initial, fill=white, text width=0.5em, minimum height=1.5em, very thin] (Multi-Scale-Simulation) {\fontsize{7.5pt}{2.5pt}\selectfont{多尺度计算与模拟}};
  \node [below=1.0 of Process-Manage, text width=6.5em, minimum height=0.5em] (DataBase) {\fontsize{7.5pt}{5.5pt}\selectfont{\textcolor{violet}{\textrm{MongoDB}数据库}}};

  \node [box-rect, below=0.8 of DataBase, fill=olive!60, text width=12.5em, minimum height=2.5em] (Data-Manage) {\fontsize{7.5pt}{5.5pt}\selectfont{计算结果数据与管理}};
  \node [box-rect, below=0.8 of Data-Manage, fill=lime!70, text width=6.5em, minimum height=1.5em, xshift=-3.5em] (Result-Analysis) {\fontsize{7.5pt}{5.5pt}\selectfont{计算结果可视化分析}};
  \node [box-rect, right=0.5 of Result-Analysis, fill=teal!70, text width=15.5em, minimum height=1.5em] (Machine-Learning) {\fontsize{7.5pt}{5.5pt}\selectfont{\textrm{Python}集成的机器学习算法\textrm{GPR/BO/ANN}支持}};
  \node [box-rect, below=2.5 of Data-Manage, fill=darkgray!70, text width=15.5em, minimum height=1.5em, xshift=5.0em] (Application) {\fontsize{7.5pt}{5.5pt}\selectfont{\textcolor{white}{特色应用场景:~原子间多体相互作用}}};
  \node [box-rect, below=0.05 of Multi-Scale-Simulation, fill=white, text width=0.5em, minimum height=1.5em, very thin] (Result-Machine-Learning) {\fontsize{7.5pt}{2.5pt}\selectfont{计算结果分析与机器学习}};

%%%%%% ------------ 定义 虚线方框转角 -----------------  %%%%%%
\path
  (Figure.north) ++(-5.0 em, -0.6em) coordinate (Model-fit)
  (Modeling-Analysis.south) ++(5.0em, 0.6em) coordinate (Symmetry-fit)
  (Modeling-Analysis.south) ++(-6.0em, -0.2em) coordinate (Symmetry-seperate)
  (Highthrough-Put.south) ++(-4.2em, 0.0em) coordinate (Highthrough-1)
  (Highthrough-Put.south) ++(3.45em, -0.0em) coordinate (Highthrough-2)
  (Highthrough-Put.west) ++(-1.50em, 0.0em) coordinate (Highthrough-3)
  (Process-Manage.north) ++(-8.6em, 0.8em) coordinate (DataBase-1)
  (Data-Manage.south) ++ (7.0em, -0.1em) coordinate (DataBase-2)
  (Process-Design.south)++(-2.0em, -1.50em) coordinate (Process-Design-1)
  (Process-Design.south)++(-10.5em, -1.50em) coordinate (Process-Design-2)
  (Process-Manage.north)++(2.7em, 0.0em) coordinate (Process-Manage-1)
  (Modeling-Analysis.south)++(-2.0em, 0.0em) coordinate (Modeling-Analysis-1)
  (Data-Manage.north) ++ (-5.5em, 0.0em) coordinate (DataBase-3)
  (Application.north)++(-6.51em, 0.0em) coordinate(Application-1)
  (Application.north)++(-4.35em, 0.0em) coordinate(Application-2)
  (Application.north)++(4.43em, 0.0em) coordinate(Application-3)
  (Machine-Learning.north) ++(-7.5em, -0.5em) coordinate (Machine-Learning-1)
  (Machine-Learning.south) ++(7.5em, 0.5em) coordinate (Machine-Learning-2);

\path [arrow, gray,thick] (Figure.south) -- (Modeling-mode);
\path [arrow, gray,thick] (Highthrough-Put.west) -- (Highthrough-3) |- (Modeling-mode.east);
\path [arrow, gray,thick] (Modeling-mode.south) -- (Modeling-Analysis);
\path [arrow, gray,thick] (User-defineded.west) -- (Modeling-mode.east);
\path [arrow, gray,thick] (User-defineded.east) -- (Process-Design.west);
\path [arrow, gray,thick] (Highthrough-1) -- (User-defineded.north);
\path [arrow, gray,thick] (Highthrough-2) -- (Process-Design.north);

\path [arrow, draw=blue!30, line width=4pt] (Process-Design.south) -- (Multi-Scale.north);
\path [arrow, draw=black, line width=3pt] (Process-Design.south)++(-2.0em, 0.0em) |- (Process-Design-2) -- (Process-Manage-1);
%\path [draw=black, line width=3pt] (Process-Design-1) -- (Process-Design-2);
%\path [arrow, draw=black, line width=3pt] (Process-Design-2) -- (Process-Manage-1);
\path [arrow, draw=green!70!red, opacity=0.50, line width=3pt] (Modeling-Analysis-1) -- (DataBase-3);
\path [arrow, draw=red!80, line width=3pt] (Process-Manage.east) -- (Multi-Scale.west);
\path [arrow, draw=black,  line width=3pt] (Multi-Scale.south) |- (Data-Manage.east);

\path [arrow, draw=black, thick](Data-Manage.south)++(-3.45em, 0.0em) -- (Result-Analysis.north);
\path [arrow, draw=black, thick](Result-Analysis.south)++(2.0em, 0.0em) -- (Application-1);
\path [arrow, draw=black, thick](Data-Manage.south)++(0.65em, 0.0em) -- (Application-2);
\path [arrow, draw=black, thick](Machine-Learning.north) |- (Data-Manage.east);
\path [arrow, draw=black, thick](Machine-Learning.south) -- (Application-3);

%%%%%% ------------ 绘制 虚线 -----------------  %%%%%%
%\draw [gray, dotted, line width=2pt] (Symmetry-seperate) -- (Modeling-Initial.south);
\draw [gray, dashed, ultra thick] (Symmetry-seperate) -- (Modeling-Initial.south);
%\draw [gray, densely dotted, ultra thick] (Symmetry-seperate) -- (Modeling-Initial.south);
\draw [gray, dotted, line width=2pt] (DataBase.south)++(-9.5em,-1.58em) -- (Multi-Scale-Simulation.south);
%%%%%% ------------ 绘制 虚线方框 -----------------  %%%%%%
\node [rectangle,draw=blue!30, dashdotted, inner sep=0.8em,fit=(Model-fit) (Symmetry-fit), very thick] (enclosure) {};
%\node [rectangle,draw=blue!30, dash dot, inner sep=0.8em,fit=(Model-fit) (Symmetry-fit), very thick] (enclosure) {};
%\node [rectangle,draw=blue!30, dash dot dot, inner sep=0.8em,fit=(Model-fit) (Symmetry-fit), very thick] (enclosure) {};
%\node [rectangle,draw=violet, dash, inner sep=0.8em,fit=(DataBase-1) (DataBase-2), thick] (enclosure) {};
\node [rectangle,draw=violet, dashed, inner sep=0.8em,fit=(DataBase-1) (DataBase-2), thick] (enclosure) {};

\node [rectangle,draw=red, inner sep=0.8em,fit=(Machine-Learning-1) (Machine-Learning-2), line width=3.0pt] (enclosure) {};
\end{tikzpicture}

\begin{tikzpicture}[
    box-rect/.style={rectangle,draw,fill=darkgray!20,node distance=1cm,text width=15em,text centered,rounded corners,minimum height=2em,thick},
    box-diam/.style={diamond, aspect=7, inner sep=2pt, draw,fill=green!50,node distance=1cm,text width=12em,text centered,rounded corners,minimum height=2em,thick},
    arrow/.style={draw,-latex', red},
  ]
	
  \node [box-rect, right=1.0 of Highthrough-Put, fill=white, text width=0.5em, minimum height=1.5em, yshift=-3.5em, very thin] (Modeling-Initial) {\fontsize{7.5pt}{2.5pt}\selectfont{建模阶段}};
  \node [box-rect, right=1.0 of Highthrough-Put, fill=white, text width=0.5em, minimum height=1.5em, yshift=-3.5em, very thin] (Modeling-Screen) {\fontsize{7.5pt}{2.5pt}\selectfont{筛选优化}};
\end{tikzpicture}
%%%%%%%%%%%%%%%%%%%%%%%%%%%%%%%%%%%%%%%%  绘制流程图  %%%%%%%%%%%%%%%%%%%%%%%%%%%%%%%%%%%%%%%%%%%%%%%%%%%%%%%%%%%%%%%%%
%\tikzstyle{startstop} = [rectangle,rounded corners, minimum width=3cm,minimum height=1cm,text centered,text width =3cm, draw=black,fill=red!30]
%\tikzstyle{io} = [trapezium, trapezium left angle = 70,trapezium right angle=110,minimum width=3cm,minimum height=1cm,text centered,text width =3cm,draw=black,fill=blue!30]
%\tikzstyle{process} = [rectangle,minimum width=3cm,minimum height=1cm,text centered,text width =3cm,draw=black,fill=orange!30]
%\tikzstyle{decision} = [diamond,aspect = 3,text centered,draw=black,fill=green!30]
%\tikzstyle{arrow} = [thick,->,>=stealth]
%\tikzstyle{straightline} = [line width = 1pt,-]
%\tikzstyle{point}=[coordinate]
%
%\begin{tikzpicture}[node distance=2cm]
%\node (start) [startstop] {开始};
%\node (input1) [io,below of=start] {输入聚类的个数 $k$ 和最大迭代次数 $n$ };
%\node (process1) [process,below of=input1] {初始化 $k$ 个聚类中心};
%\node (process2) [process,below of=process1] {分配各数据对象到距离最近的类中};
%\node (decision1) [decision,below of=process2,yshift=-0.5cm] {是否收敛或迭代次数达到 $n$ };
%\node (stop) [startstop,below of=decision1,node distance=3cm] {输出聚类结果};
%\node(point1)[point,left of=input1,node distance=5cm]{};
%
%\draw [arrow] (start) -- (input1);
%\draw [arrow] (input1) -- (process1);
%\draw [arrow] (process1) -- (process2);
%\draw [arrow] (process2) -- (decision1);
%\draw [arrow] (decision1) -- node[anchor=east] {是} (stop);
%\draw [straightline] (decision1) -|  (point1);
%\draw [arrow] (point1) -- node[anchor=south] {否} (input1);
%\end{tikzpicture}

%%%%%%%%%%%%%%%%%%%%%%%%%%%%%%%%%%%%%%%%  绘制流程图  %%%%%%%%%%%%%%%%%%%%%%%%%%%%%%%%%%%%%%%%%%%%%%%%%%%%%%%%%%%%%%%%%
%\begin{figure}
%\scriptsize
%\tikzstyle{format}=[rectangle,draw,thin,fill=white]
%%定义语句块的颜色,形状和边
%\tikzstyle{test}=[diamond,aspect=2,draw,thin]
%%定义条件块的形状,颜色
%\tikzstyle{point}=[coordinate,on grid,]
%%像素点,用于连接转移线
%\begin{tikzpicture}%[node distance=10mm,auto,>=late',thin,start chain=going below,every join/.style={norm},]
%%start chain=going below指明了流程图的默认方向,node distance=8mm则指明了默认的node距离。这些可以在定义node的时候更改,比如说
%%\node[point,right of=n3,node distance=10mm] (p0){};
%%这里声明了node p0,它在node n3 的右边,距离是10mm。
%\node[format] (start){Start};
%\node[format,below of=start,node distance=7mm] (define){Some defines};
%\node[format,below of=define,node distance=7mm] (PCFinit){PCF8563 Initialize};
%\node[format,below of=PCFinit,node distance=7mm] (DS18init){DS18 Initialize};
%\node[format,below of=DS18init,node distance=7mm] (LCDinit){LCD Initialize};
%\node[format,below of=LCDinit,node distance=7mm] (processtime){Processtime};
%\node[format,below of=processtime,node distance=7mm] (keyinit){Key Initialize};
%\node[test,below of=keyinit,node distance=15mm](setkeycheck){Check Set Key};
%\node[point,left of=setkeycheck,node distance=18mm](point3){};
%\node[format,below of=setkeycheck,node distance=15mm](readtime){Read Time};
%\node[point,right of=readtime,node distance=15mm](point4){};
%\node[format,below of=readtime](processtime1){Processtime};
%\node[format,below of=processtime1](gettemp){Get Temperature};
%\node[format,below of=gettemp](display){Display All Data};
%\node[format,right of=setkeycheck,node distance=40mm](setsetflag){Set SetFlag=1};
%\node[format,below of=setsetflag](setinit){Set Mode Initialize};
%\node[format,below of=setinit](checksetting){Checksetting()};
%\node[test,below of=checksetting,node distance=15mm](savecheck){Check Save Key};
%\node[format,below of=savecheck,node distance=15mm](clearsetflag){Clear SetFlag=0};
%\node[format,below of=clearsetflag](settime){Set Time};
%\node[point,below of=display,node distance=7mm](point1){};
%\node[point,below of=settime,node distance=7mm](point2){};
%%\node[format] (n0) at(4,4){A}; 直接指定位置
%%定义完node之后进行连线,
%%\draw[->] (n0.south) -- (n1); 带箭头实线
%%\draw[-] (n0.south) -- (n1); 不带箭头实线
%%\draw[&lt;->] (n0.south) -- (n1.north);   双箭头
%%\draw[&lt;-,dashed] (n1.south) -- (n2.north); 带箭头虚线 
%%\draw[&lt;-] (n0.south) to node{Yes} (n1.north);  带字,字在箭头方向右边
%%\draw[->] (n1.north) to node{Yes} (n0.south);  带字,字在箭头方向左边
%%\draw[->] (n1.north) to[out=60,in=300] node{Yes} (n0.south);  曲线
%%\draw[->,draw=red](n2)--(n1);  带颜色的线
%\draw[->] (start)--(define);
%\draw[->] (define)--(PCFinit);
%\draw[->](PCFinit)--(DS18init);
%\draw[->](DS18init)--(LCDinit);
%\draw[->](LCDinit)--(processtime);
%\draw[->](processtime)--(keyinit);
%\draw[->](keyinit)--(setkeycheck);
%\draw[->](setkeycheck)--node[above]{Yes}(setsetflag);
%\draw[->](setkeycheck) --node[left]{No} (readtime);
%\draw[->](readtime)--(processtime1);
%\draw[->](processtime1)--(gettemp);
%\draw[->](gettemp)--(display);
%\draw[-](display)--(point1);
%\draw[-](point1)-|(point3);
%\draw[->](point3)--(setkeycheck.west);
%\draw[->](setsetflag)--(setinit);
%\draw[->](setinit)--(checksetting);
%\draw[->](checksetting)--(savecheck);
%\draw[->](savecheck)--node[left]{Yes}(clearsetflag);
%\draw[->](savecheck.west)|-node[left]{No}(checksetting);
%\draw[->](clearsetflag)--(settime);
%\draw[-](settime)--(point2);
%\draw[-](point2)-|(point4);
%\draw[->](point4)--(readtime.east);
%\end{tikzpicture}
%\end{figure}
