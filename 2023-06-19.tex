%---------------------- TEMPLATE FOR REPORT ------------------------------------------------------------------------------------------------------%

%\thispagestyle{fancy}   % 插入页眉页脚                                        %
%%%%%%%%%%%%%%%%%%%%%%%%%%%%% 用 authblk 包 支持作者和E-mail %%%%%%%%%%%%%%%%%%%%%%%%%%%%%%%%%
%\title{More than one Author with different Affiliations}				     %
%\title{\rm{VASP}的电荷密度存储文件\rm{CHGCAR}}
%\title{面向高温合金材料设计的计算模拟软件中的几个主要问题}
\title{}
\author[ ]{}   %
%\author[ ]{姜~骏\thanks{jiangjun@bcc.ac.cn}}   %
%\affil[ ]{北京市计算中心}    %
%\author[a]{Author A}									     %
%\author[a]{Author B}									     %
%\author[a]{Author C \thanks{Corresponding author: email@mail.com}}			     %
%%\author[a]{Author/通讯作者 C \thanks{Corresponding author: cores-email@mail.com}}     	     %
%\author[b]{Author D}									     %
%\author[b]{Author/作者 D}								     %
%\author[b]{Author E}									     %
%\affil[a]{Department of Computer Science, \LaTeX\ University}				     %
%\affil[b]{Department of Mechanical Engineering, \LaTeX\ University}			     %
%\affil[b]{作者单位-2}			    						     %
											     %
%%% 使用 \thanks 定义通讯作者								     %
											     %
\renewcommand*{\Authfont}{\small\rm} % 修改作者的字体与大小				     %
\renewcommand*{\Affilfont}{\small\it} % 修改机构名称的字体与大小			     %
\renewcommand\Authands{ and } % 去掉 and 前的逗号					     %
\renewcommand\Authands{ , } % 将 and 换成逗号					     %
\date{} % 去掉日期									     %
%\date{2020-12-30}									     %

%%%%%%%%%%%%%%%%%%%%%%%%%%%%%%%%%%%%%%%%%%  不使用 authblk 包制作标题  %%%%%%%%%%%%%%%%%%%%%%%%%%%%%%%%%%%%%%%%%%%%%%
%-------------------------------The Title of The Report-----------------------------------------%
%\title{报告标题:~}   %
%-----------------------------------------------------------------------------

%----------------------The Authors and the address of The Paper--------------------------------%
%\author{
%\small
%Author1, Author2, Author3\footnote{Communication author's E-mail} \\    %Authors' Names	       %
%\small
%(The Address,City Post code)						%Address	       %
%}
%\affil[$\dagger$]{清华大学~材料加工研究所~A213}
%\affil{清华大学~材料加工研究所~A213}
%\date{}					%if necessary					       %
%----------------------------------------------------------------------------------------------%
%%%%%%%%%%%%%%%%%%%%%%%%%%%%%%%%%%%%%%%%%%%%%%%%%%%%%%%%%%%%%%%%%%%%%%%%%%%%%%%%%%%%%%%%%%%%%%%%%%%%%%%%%%%%%%%%%%%%%
\maketitle
%\thispagestyle{fancy}   % 首页插入页眉页脚 
\section{新型能源系统储能}
随着清洁能源技术的不断发展和能源转型的持续推进,对高效可靠、可持续的能源储存和利用方案的需求日益迫切。作为能量储存和释放物理基础的材料创新,更是重中之重。新型能源系统的储能材料在推动清洁能源转型和实现可持续发展方面起着关键作用。随着近年来新型储能材料的不断涌现,为实现清洁能源的高效利用和能源转型提供重要保证。 

当前新型的能源系统储能材料主要有%以下是关于新型能源系统储能材料的综述。

\begin{itemize}
	\item 锂离子电池材料\\
锂离子电池是目前应用最广泛的储能技术之一,其核心是正负极材料。随着对于电动汽车和可再生能源储能需求的增加,研究人员不断改进和发展新型锂离子电池材料,以提高能量密度、循环寿命和安全性能。例如,钴酸锂、磷酸铁锂、钛酸锂等正极材料和石墨、硅基材料等负极材料的研究和优化。

\item 超级电容器材料\\
超级电容器(超级电容)是一种具有高能量密度和高功率密度的储能装置。其核心是电极材料,常用的材料包括活性炭、金属氧化物、导电聚合物等。研究人员致力于开发新型超级电容器材料,以提高储能容量、循环寿命和快速充放电性能。

\item 燃料电池储氢材料\\
燃料电池是一种将氢气和氧气直接转化为电能的装置,其核心是催化剂和电解质材料。研究人员不断改进和创新催化剂材料,如贵金属合金、非贵金属催化剂等,以提高燃料电池的催化活性和稳定性。同时,也在探索新型电解质材料,如聚合物电解质和固体氧化物燃料电池的陶瓷电解质等。

\item 纳米材料和多孔材料\\
纳米材料和多孔材料具有较大的比表面积和储能容量,对于储能领域具有重要意义。例如,纳米材料可以用于改善锂离子电池和超级电容器的电极性能,提高能量密度和循环寿命。多孔材料可以用于物理吸附储氢技术,提供高容量和快速氢气吸附释放的特性。

\item 其他新型储能材料\\
还有其他一些新型储能材料正在被研究和开发,如钠离子电池材料、锌空气电池材料、流电池材料等。这些材料具有不同的储能机制和特性,为能源系统提供了更多的选择和可能性。
\end{itemize}


清洁能源发展是指利用可再生能源或低碳能源来替代传统的化石燃料能源,以减少对环境的污染和气候变化的影响。随着全球能源需求的增长和对环境问题的日益关注,清洁能源已成为国际社会关注的焦点之一。

清洁能源的主要来源包括太阳能、风能、水能、地热能等可再生能源,以及核能等低碳能源。这些能源具有广泛的应用潜力,可以用于发电、供热、交通等领域。与传统的化石燃料相比,清洁能源具有许多优势,例如:可再生性、减少温室气体排放、降低能源成本、促进经济发展等。

全球范围内,越来越多的国家和地区开始加大对清洁能源的发展和利用力度。许多国家制定了政策措施,鼓励清洁能源的投资和研发,推动可再生能源的普及和应用。同时,清洁能源技术也在不断创新和进步,太阳能电池、风力发电机、生物质能利用等技术不断发展,使清洁能源的成本逐渐降低,效率逐步提高。

清洁能源的发展对于环境保护和气候变化具有重要意义。传统的化石燃料能源产生大量的二氧化碳等温室气体排放,对全球气候变化产生严重影响。而清洁能源则可以大幅度减少温室气体的排放,有效遏制气候变化的进程。此外,清洁能源的利用还可以减少空气污染和水污染,改善环境质量,保护生态系统的健康。

然而,清洁能源发展面临一些挑战。首先,清洁能源技术的成本仍然较高,需要进一步降低成本才能更广泛地推广应用。其次,清洁能源的可持续性和稳定性也是一个问题,如太阳能和风能的波动性,需要解决能源储存和输送的技术难题。此外,清洁能源的发展还需要解决政策、法律、市场等方面的问题,以促进其良性循环和可持续发展。

总之,清洁能源的发展是实现可持续发展和应对气候变化的重要途径之一。通过加大对清洁能源的投资和研发,制定相关政策和法规,促进技术创新和应用推广,我们可以实现清洁能源的可持续利用,为人类创造一个更加清洁、健康和可持续的未来。

储能技术是指将能量在其可用状态下存储起来,以便在需要时进行释放和利用。在清洁能源的发展中,储能技术起到了关键的作用,帮助解决可再生能源的波动性和间歇性问题,提高能源利用效率。

储能材料是实现储能技术的核心组成部分之一。在其中,储氢材料作为一种重要的储能材料之一,具有很高的潜力。储氢材料能够吸收和存储氢气,在需要时释放出氢气供能源系统使用。以下是关于储能和储氢材料的综述。

储能材料的分类:
储能材料可以分为化学储能材料、物理储能材料和电化学储能材料三类。

化学储能材料:例如,电池和燃料电池利用化学反应将能量储存起来,通过反应产生电能供应使用。
物理储能材料:例如,压缩空气储能(CAES)和储热技术将能量以物理形式存储,如通过压缩空气或热储存介质。
电化学储能材料:例如,锂离子电池、超级电容器和燃料电池等,利用电化学反应在电极之间储存和释放能量。
储氢材料的类型:
储氢材料可分为物理吸附储氢和化学储氢两类。

物理吸附储氢:物理吸附储氢是指氢气以物理方式吸附在材料表面,如多孔材料、碳纳米管等。物理吸附储氢具有高氢气吸附容量和较低的工作温度,但氢气的释放和吸附速率相对较慢。
化学储氢:化学储氢是指氢气以化学键形式储存在材料中,如金属氢化物、化合物氢化物等。化学储氢具有高密度储氢和较快的释放速率,但对于反应条件和储氢容量方面有一定的限制。
储能和储氢材料的发展趋势:
近年来,研究人员一直在探索和开发更高效、可持续的储能和储氢材料。一些关键的发展趋势包括:

提高能量密度:研究人员努力开发新型材料,以提高储能和储氢系统的能量密度,实现更高效的能源储存和利用。
改善循环稳定性:对于储能和储氢材料来说,循环稳定性是一个重要指标。研究人员致力于开发具有良好循环稳定性的材料,以确保储能系统的长寿命和高性能。
节约材料资源:可持续性是储能技术发展的重要考虑因素之一。研究人员正在寻求减少材料使用量、提高资源利用效率和开发可再生材料的方法。
结合多种储能技术:为了克服单一储能技术的局限性,研究人员也在探索将不同的储能技术结合起来,形成混合储能系统,以实现更高效的能源转换和利用。
综上所述,储能和储氢材料在清洁能源领域发挥着重要作用。随着技术的不断创新和进步,储能和储氢材料将继续发展,为可持续能源的应用和推广提供更好的支持。



宁波市在新型能源存储发展方面有以下几个思路:

储能技术创新:宁波市鼓励科技创新和研发,致力于发展新型能源储存技术。通过引进和培育高新技术企业,加强与科研机构和大学的合作,推动能源储存技术的突破和创新。重点发展电池储能、超级电容器、氢能储存等新型能源存储技术,提高能源存储效率和可靠性。

储能设施建设:宁波市积极推动新型能源储存设施的建设。加大投资力度,建设大规模的储能电站和分布式能源存储系统,提高能源的可调度性和供应稳定性。同时,结合城市建设规划,合理规划和布局储能设施,使其与现有能源设施相互衔接,实现能源的高效利用和优化配置。

智能能源管理系统:宁波市提倡建设智能能源管理系统,将储能技术与智能电网、能源互联网相结合。通过建立数据监测、分析和控制系统,实现对能源的实时监测、预测和调度,优化能源供需平衡,提高能源利用效率和系统安全性。智能能源管理系统将有助于提升新型能源存储系统的整体性能和运营效益。

政策支持和产业培育:宁波市出台一系列政策支持和激励措施,鼓励企业参与新型能源存储产业的发展。提供资金支持、税收优惠和技术支持等政策,吸引和培育储能技术企业和产业链上下游企业的发展。通过政策引导和市场培育,形成新型能源存储产业集聚效应,推动宁波市成为新能源存储技术和产品的研发和生产基地。

通过以上思路,宁波市将促进新型能源存储技术的发展和应用,提高清洁能源的可靠性和可持续性,推动能源结构转型和能源消费方式的改变,为宁波市的可持续发展做出贡献。


宁波市在清洁能源技术发展方面采取了以下战略:

提升清洁能源比重:宁波市致力于提高清洁能源在能源结构中的比重。通过加大可再生能源开发和利用的力度,特别是风能、太阳能和水能等方面的开发,推动清洁能源占比的增加。同时,逐步减少对传统化石能源的依赖,降低碳排放,实现绿色低碳发展。

积极推动新能源技术创新:宁波市鼓励新能源技术的创新和应用。加强与高校、科研院所和企业的合作,加大科研投入,推动新能源技术的研发和成果转化。重点关注新能源发电、能源存储、能源管理和智能电网等领域的技术创新,提高清洁能源的效率、稳定性和可持续性。

建设清洁能源示范项目:宁波市积极打造清洁能源示范项目,以示范带动整个行业的发展。通过建设风电场、太阳能光伏电站、生物质能源项目等,展示清洁能源的可行性和优势。这些示范项目不仅提供了可靠的清洁能源供应,还为其他地区和企业提供了学习和借鉴的样本。

推广清洁能源利用:宁波市大力推广清洁能源的利用。鼓励居民和企业采用清洁能源供暖、清洁能源交通和清洁能源照明等方式,减少对传统能源的消耗。同时,提供相应的政策支持和激励措施,鼓励广大群众参与到清洁能源利用中来,共同推动可持续能源的普及和应用。

加强政策引导和支持:宁波市制定了一系列的政策文件,为清洁能源技术发展提供支持和引导。包括财政奖励、税收优惠、技术支持等方面的政策,吸引和扶持清洁能源企业和项目的发展。同时,建立健全的管理和监管机制,加强政府与企业、社会的合作,形成共同推动清洁能源发展的良好局面。

通过上述战略的实施,宁波市将不断推动清洁能源技术的发展和应用,加快能源结构转型,促进经济可持续发展,实现绿色低碳的宁波愿景。

宁波市在新型能源系统储能发展方面采取了以下战略:

投资建设储能设施:宁波市鼓励投资建设不同类型的储能设施,包括电池储能、抽水蓄能、氢能储能等。通过增加储能容量和提升储能效率,实现对可再生能源的平稳接纳和调峰调频能力的提升。特别是注重与清洁能源发电项目的结合,构建以储能系统为支撑的可持续能源系统。

优化能源系统管理:宁波市重视能源系统的优化管理,通过智能化技术和数据分析,实现储能设施与能源系统的智能协调运行。通过电网调度、能量管理和储能优化等手段,提高储能系统的运行效率和经济性,最大限度地发挥储能的作用。

推动储能技术创新:宁波市积极推动储能技术的创新研发。鼓励企业、科研机构和高校加大在储能材料、储能装置和储能系统控制等方面的研究力度。特别是注重新型储能技术的研发,如钠离子电池、固态电池、氢储能等,推动储能技术的突破和应用。

加强政策引导和支持:宁波市制定相关政策文件,为储能发展提供支持和引导。包括财政奖励、税收优惠、技术支持等方面的政策,吸引和扶持储能企业和项目的发展。同时,建立健全的管理和监管机制,加强政府与企业、社会的合作,形成共同推动储能发展的良好局面。

促进产学研用合作:宁波市鼓励产学研用的深度合作,加强储能领域的技术交流和合作。通过建立产学研用联盟、共享研发资源和推动技术转移,促进储能技术的快速转化和产业化。同时,加强国际合作与交流,借鉴国际先进经验和技术,提升宁波市在储能领域的创新能力和国际竞争力。

通过以上战略的实施,宁波市将推动新型能源系统储能技术的发展和应用,提高能源的可持续利用率和清洁能源的消纳能力,推动宁波市能源转型和可持续发展。
\section{功能材料制造}
高熵合金是一种新兴的材料类别,具有独特的结构和性能特点。本文将为您提供高熵合金的综述。

高熵合金的定义:高熵合金是由五个或更多种等摩尔比元素组成的合金。与传统合金不同,高熵合金中各元素的摩尔比接近相等,形成均匀的原子混合。

结构特点:高熵合金的最显著特点之一是其高熵效应。高熵效应意味着合金中存在大量的局域无序,使得合金具有非常复杂的原子结构。这种无序结构有助于提高合金的抗变形性能和耐腐蚀性能。

性能特点:高熵合金具有一系列优异的性能。首先,高熵合金具有出色的机械性能,包括高硬度、高强度和良好的韧性。其次,高熵合金表现出良好的耐热性和高温稳定性,使其在高温环境下具有优异的性能。此外,高熵合金还表现出优异的耐腐蚀性和抗氧化性能。

制备方法:高熵合金的制备方法多种多样,包括机械合金化、熔融混合、物理气相沉积等。这些方法可以实现原子级的均匀混合,从而形成高熵合金的特殊结构。

应用领域:高熵合金在多个领域具有广泛的应用潜力。例如,在航空航天、能源、汽车和电子等领域,高熵合金可用于制造高温结构件、耐蚀材料、导电材料等。其出色的性能使得高熵合金成为一种具有重要应用前景的新型材料。

需要注意的是,高熵合金领域仍处于不断发展和探索的阶段,许多方面仍需进一步研究和理解。然而,高熵合金的独特性质和潜力已经引起了广泛的科学界和工业界的关注,将为材料科学和工程领域带来新的发展机遇。


新功能材料制造是一个广泛的领域,涉及到各种新材料的设计、合成和制备方法。下面是一些在新功能材料制造中常见的方法和技术:

纳米材料制造:纳米材料具有独特的尺寸效应和界面效应,常常展现出与宏观材料不同的性能。纳米材料的制造方法包括溶胶凝胶法、熔融法、气相沉积等,可以通过控制制备条件来实现纳米尺寸的材料。

3D打印:3D打印技术是一种逐层制造的方法,可以根据设计模型直接将材料打印成所需形状。它具有快速、灵活和可定制化的优势,被广泛应用于制造复杂形状和结构的新功能材料。

激光制造:激光制造技术包括激光熔化、激光烧结、激光沉积等,通过激光能量对材料进行加热和处理,实现高精度和高效率的制造。激光制造可以用于制备金属、陶瓷、聚合物等各种材料。

薄膜制备:薄膜制备技术用于制备具有特殊功能和性能的薄膜材料,如光学薄膜、防腐蚀薄膜、导电薄膜等。常见的薄膜制备方法包括物理气相沉积、化学气相沉积、溅射法等。

生物制造:生物制造技术利用生物体内的细胞、组织和生物大分子来合成和组装材料。例如,利用生物体外的微生物发酵产生特定化合物或纳米颗粒,利用生物材料如细胞外基质或生物胶体构建复杂的结构等。

材料设计和模拟:材料设计和模拟技术利用计算机模拟和模型预测的方法,通过理论计算和仿真来指导新功能材料的设计和优化。这种方法可以节省时间和成本,提高材料研发的效率。

多功能复合材料:多功能复合材料将不同类型的材料组合在一起,以获得多种功能和性能。这些复合材料可以通过层叠、混合、交联等方法制备,应用于各种领域,如结构材料、传感器、电子器件等。

以上只是新功能材料制造领域中的一些常见方法和技术,随着科学技术的不断进步,新的制造方法和技术也在不断涌现,为新功能材料的制备提供更多可能性。



功能材料制造在宁波的前景非常广阔。宁波是中国东部沿海地区的重要制造业基地,拥有良好的产业基础、完善的供应链和丰富的科技创新资源,为功能材料制造提供了良好的发展环境。

以下是宁波功能材料制造的一些前景:

新能源材料:宁波在新能源领域具有一定的优势,包括光伏、储能、电动汽车等。功能材料在新能源领域的应用非常广泛,如高性能电池材料、光伏材料、储能材料等。宁波的制造业基础和创新能力为新能源功能材料的研发和生产提供了良好的支持。

先进制造材料:宁波在机械制造、汽车制造、船舶制造等领域具有雄厚的实力。先进制造材料如高强度钢材、高温合金、先进陶瓷等在这些领域中起到重要的作用。宁波的制造业发展为先进制造材料的生产和应用提供了广阔的市场需求。

新型功能材料:随着科学技术的不断进步,新型功能材料的研发和制造成为未来的重要发展方向。宁波拥有多所高水平的大学和研究机构,为新型功能材料的研究提供了强大的科研支持。例如,高熵合金、二维材料、生物材料等都是新兴的功能材料领域,宁波可以通过加强科研合作和技术创新,在这些领域中取得突破。

绿色环保材料:宁波高度重视绿色环保发展,功能材料制造中的绿色环保材料需求将得到进一步提升。例如,可降解材料、环保涂料、新型过滤材料等在环保领域有着广泛的应用前景。宁波的制造业转型升级将对绿色环保材料的研发和制造提供更多机遇。

总的来说,宁波作为制造业基地和科技创新中心,具备了发展功能材料制造的良好条件。政府的政策支持、产业链的完善、科研机构的支持以及企业的创新能力将共同推动宁波功能材料制造的蓬勃发展。

宁波市在新功能材料发展方面采取了以下战略:

积极引导技术创新:宁波市鼓励企业、科研机构和高等院校加大对新功能材料技术的研发投入。通过建立科研项目资助、技术转移和知识产权保护等支持机制,激发创新活力,推动新功能材料技术的突破和应用。重点关注具有高性能、高附加值和环保特性的新材料,如先进金属材料、高性能复合材料、新型能源材料等。

建设创新平台和示范基地:宁波市致力于建设新功能材料的创新平台和示范基地。通过构建科研院所、企业和高校合作的创新联盟,加强技术交流和合作,推动新功能材料技术的集聚和创新。同时,建设新功能材料产业园区和示范基地,提供完善的配套服务和优惠政策,吸引和扶持新功能材料相关企业的发展。

产业链协同发展:宁波市鼓励新功能材料产业链的协同发展。加强与上下游产业环节的合作,推动新功能材料的研发、生产、应用和推广。通过促进产学研用的深度融合,提高新功能材料的产业化水平,形成完整的产业链和价值链,推动宁波市新功能材料产业的快速发展。

培育龙头企业和品牌:宁波市鼓励培育新功能材料领域的龙头企业和品牌。通过提供政策扶持、资金支持和市场拓展等措施,支持优秀企业在新功能材料领域的创新和发展。同时,加强品牌建设和知名度推广,提高宁波市新功能材料的市场竞争力和影响力。

加强国际合作与交流:宁波市积极开展国际合作与交流,借鉴和吸收国际先进的新功能材料技术和管理经验。通过组织国际学术会议、技术交流和合作项目,加强与国外企业、科研机构和高校的合作,提升宁波市在新功能材料领域的国际影响力和竞争力。

通过以上战略的实施,宁波市将不断推动新功能材料的研发和应用,促进新功能材料产业的快速发展,为宁波市经济转型升级和可持续发展提供重要支撑。


宁波市在高熵合金发展方面采取了以下战略:

科研创新与技术突破:宁波市鼓励企业、科研机构和高等院校加大在高熵合金领域的科研投入,推动关键技术的突破和创新。通过引进高水平人才、加强科研项目资助和建设研发平台,提升高熵合金材料的研发水平和技术竞争力。重点关注高熵合金的组织结构设计、合金配方优化、加工工艺控制等关键技术,提高高熵合金的性能和应用范围。

建立产学研用合作平台:宁波市鼓励企业与高校、科研院所建立产学研用合作平台,推动高熵合金的产学研用深度融合。通过共同开展科研项目、共享实验设施和技术资源,实现高熵合金技术的快速转化和产业化。同时,加强与行业协会和标准化组织的合作,推动高熵合金标准的制定和推广应用,提升高熵合金产业的整体水平和规范化发展。

建设示范项目和示范应用:宁波市鼓励建设高熵合金的示范项目和示范应用,展示其在各个领域的应用潜力和经济效益。通过支持高熵合金在航空航天、能源、汽车、船舶等关键领域的应用,推动高熵合金的市场拓展和技术推广。同时,注重技术示范和技术培训,提升相关产业链的技术水平和市场竞争力。

加强人才培养和引进:宁波市注重高熵合金领域的人才培养和引进。通过建立人才培养机制、提供奖励和激励政策,吸引和培养一批高水平的高熵合金人才。同时,加大引进海外优秀人才的力度,引领和推动高熵合金领域的创新和发展。通过人才培养和引进,提升宁波市在高熵合金领域的人才优势和技术实力。

加强国际合作与交流:宁波市积极开展国际合作与交流,加强与国际高熵合金领域的企业、科研机构和高校的合作。通过组织国际学术会议、技术交流和合作项目,引进先进技术和经验,促进国内外高熵合金领域的合作与交流,提升宁波市在全球高熵合金产业链中的地位和影响力。

通过以上战略的实施,宁波市将加快推动高熵合金技术的创新与应用,培育高熵合金产业集群,提升宁波在高熵合金领域的核心竞争力,促进高熵合金产业的健康发展。
