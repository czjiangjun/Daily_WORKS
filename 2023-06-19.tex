%---------------------- TEMPLATE FOR REPORT ------------------------------------------------------------------------------------------------------%

%\thispagestyle{fancy}   % 插入页眉页脚                                        %
%%%%%%%%%%%%%%%%%%%%%%%%%%%%% 用 authblk 包 支持作者和E-mail %%%%%%%%%%%%%%%%%%%%%%%%%%%%%%%%%
%\title{More than one Author with different Affiliations}				     %
%\title{\rm{VASP}的电荷密度存储文件\rm{CHGCAR}}
%\title{面向高温合金材料设计的计算模拟软件中的几个主要问题}
\title{}
\author[ ]{}   %
%\author[ ]{姜~骏\thanks{jiangjun@bcc.ac.cn}}   %
%\affil[ ]{北京市计算中心}    %
%\author[a]{Author A}									     %
%\author[a]{Author B}									     %
%\author[a]{Author C \thanks{Corresponding author: email@mail.com}}			     %
%%\author[a]{Author/通讯作者 C \thanks{Corresponding author: cores-email@mail.com}}     	     %
%\author[b]{Author D}									     %
%\author[b]{Author/作者 D}								     %
%\author[b]{Author E}									     %
%\affil[a]{Department of Computer Science, \LaTeX\ University}				     %
%\affil[b]{Department of Mechanical Engineering, \LaTeX\ University}			     %
%\affil[b]{作者单位-2}			    						     %
											     %
%%% 使用 \thanks 定义通讯作者								     %
											     %
\renewcommand*{\Authfont}{\small\rm} % 修改作者的字体与大小				     %
\renewcommand*{\Affilfont}{\small\it} % 修改机构名称的字体与大小			     %
\renewcommand\Authands{ and } % 去掉 and 前的逗号					     %
\renewcommand\Authands{ , } % 将 and 换成逗号					     %
\date{} % 去掉日期									     %
%\date{2020-12-30}									     %

%%%%%%%%%%%%%%%%%%%%%%%%%%%%%%%%%%%%%%%%%%  不使用 authblk 包制作标题  %%%%%%%%%%%%%%%%%%%%%%%%%%%%%%%%%%%%%%%%%%%%%%
%-------------------------------The Title of The Report-----------------------------------------%
%\title{报告标题:~}   %
%-----------------------------------------------------------------------------

%----------------------The Authors and the address of The Paper--------------------------------%
%\author{
%\small
%Author1, Author2, Author3\footnote{Communication author's E-mail} \\    %Authors' Names	       %
%\small
%(The Address,City Post code)						%Address	       %
%}
%\affil[$\dagger$]{清华大学~材料加工研究所~A213}
%\affil{清华大学~材料加工研究所~A213}
%\date{}					%if necessary					       %
%----------------------------------------------------------------------------------------------%
%%%%%%%%%%%%%%%%%%%%%%%%%%%%%%%%%%%%%%%%%%%%%%%%%%%%%%%%%%%%%%%%%%%%%%%%%%%%%%%%%%%%%%%%%%%%%%%%%%%%%%%%%%%%%%%%%%%%%
\maketitle
%\thispagestyle{fancy}   % 首页插入页眉页脚 
\section{新型能源系统储能}
随着清洁能源技术的不断发展和能源转型的持续推进,对高效可靠、可持续的能源储存和利用方案的需求日益迫切。作为能量储存和释放物理基础的材料创新,更是重中之重。新型能源系统的储能材料在推动清洁能源转型和实现可持续发展方面起着关键作用。随着近年来新型储能材料的不断涌现,为实现清洁能源的高效利用和能源转型提供重要保证。 

当前新型的能源系统储能材料主要有%以下是关于新型能源系统储能材料的综述。

\begin{itemize}
	\item 锂离子电池材料\\
锂离子电池是目前应用最广泛的储能技术之一,其核心是正负极材料。随着对于电动汽车和可再生能源储能需求的增加,研究人员不断改进和发展新型锂离子电池材料,以提高能量密度、循环寿命和安全性能。例如,钴酸锂、磷酸铁锂、钛酸锂等正极材料和石墨、硅基材料等负极材料的研究和优化。

\item 超级电容器材料\\
超级电容器(超级电容)是一种具有高能量密度和高功率密度的储能装置。其核心是电极材料,常用的材料包括活性炭、金属氧化物、导电聚合物等。研究人员致力于开发新型超级电容器材料,以提高储能容量、循环寿命和快速充放电性能。

\item 燃料电池储氢材料\\
燃料电池是一种将氢气和氧气直接转化为电能的装置,其核心是催化剂和电解质材料。研究人员不断改进和创新催化剂材料,如贵金属合金、非贵金属催化剂等,以提高燃料电池的催化活性和稳定性。同时,也在探索新型电解质材料,如聚合物电解质和固体氧化物燃料电池的陶瓷电解质等。

\item 纳米材料和多孔材料\\
纳米材料和多孔材料具有较大的比表面积和储能容量,对于储能领域具有重要意义。例如,纳米材料可以用于改善锂离子电池和超级电容器的电极性能,提高能量密度和循环寿命。多孔材料可以用于物理吸附储氢技术,提供高容量和快速氢气吸附释放的特性。

\item 其他新型储能材料\\
还有其他一些新型储能材料正在被研究和开发,如钠离子电池材料、锌空气电池材料、流电池材料等。这些材料具有不同的储能机制和特性,为能源系统提供了更多的选择和可能性。
\end{itemize}


清洁能源发展是指利用可再生能源或低碳能源来替代传统的化石燃料能源,以减少对环境的污染和气候变化的影响。随着全球能源需求的增长和对环境问题的日益关注,清洁能源已成为国际社会关注的焦点之一。

清洁能源的主要来源包括太阳能、风能、水能、地热能等可再生能源,以及核能等低碳能源。这些能源具有广泛的应用潜力,可以用于发电、供热、交通等领域。与传统的化石燃料相比,清洁能源具有许多优势,例如:可再生性、减少温室气体排放、降低能源成本、促进经济发展等。

全球范围内,越来越多的国家和地区开始加大对清洁能源的发展和利用力度。许多国家制定了政策措施,鼓励清洁能源的投资和研发,推动可再生能源的普及和应用。同时,清洁能源技术也在不断创新和进步,太阳能电池、风力发电机、生物质能利用等技术不断发展,使清洁能源的成本逐渐降低,效率逐步提高。

清洁能源的发展对于环境保护和气候变化具有重要意义。传统的化石燃料能源产生大量的二氧化碳等温室气体排放,对全球气候变化产生严重影响。而清洁能源则可以大幅度减少温室气体的排放,有效遏制气候变化的进程。此外,清洁能源的利用还可以减少空气污染和水污染,改善环境质量,保护生态系统的健康。

然而,清洁能源发展面临一些挑战。首先,清洁能源技术的成本仍然较高,需要进一步降低成本才能更广泛地推广应用。其次,清洁能源的可持续性和稳定性也是一个问题,如太阳能和风能的波动性,需要解决能源储存和输送的技术难题。此外,清洁能源的发展还需要解决政策、法律、市场等方面的问题,以促进其良性循环和可持续发展。

总之,清洁能源的发展是实现可持续发展和应对气候变化的重要途径之一。通过加大对清洁能源的投资和研发,制定相关政策和法规,促进技术创新和应用推广,我们可以实现清洁能源的可持续利用,为人类创造一个更加清洁、健康和可持续的未来。

储能技术是指将能量在其可用状态下存储起来,以便在需要时进行释放和利用。在清洁能源的发展中,储能技术起到了关键的作用,帮助解决可再生能源的波动性和间歇性问题,提高能源利用效率。

储能材料是实现储能技术的核心组成部分之一。在其中,储氢材料作为一种重要的储能材料之一,具有很高的潜力。储氢材料能够吸收和存储氢气,在需要时释放出氢气供能源系统使用。以下是关于储能和储氢材料的综述。

储能材料的分类:
储能材料可以分为化学储能材料、物理储能材料和电化学储能材料三类。

化学储能材料:例如,电池和燃料电池利用化学反应将能量储存起来,通过反应产生电能供应使用。
物理储能材料:例如,压缩空气储能(CAES)和储热技术将能量以物理形式存储,如通过压缩空气或热储存介质。
电化学储能材料:例如,锂离子电池、超级电容器和燃料电池等,利用电化学反应在电极之间储存和释放能量。
储氢材料的类型:
储氢材料可分为物理吸附储氢和化学储氢两类。

物理吸附储氢:物理吸附储氢是指氢气以物理方式吸附在材料表面,如多孔材料、碳纳米管等。物理吸附储氢具有高氢气吸附容量和较低的工作温度,但氢气的释放和吸附速率相对较慢。
化学储氢:化学储氢是指氢气以化学键形式储存在材料中,如金属氢化物、化合物氢化物等。化学储氢具有高密度储氢和较快的释放速率,但对于反应条件和储氢容量方面有一定的限制。
储能和储氢材料的发展趋势:
近年来,研究人员一直在探索和开发更高效、可持续的储能和储氢材料。一些关键的发展趋势包括:

提高能量密度:研究人员努力开发新型材料,以提高储能和储氢系统的能量密度,实现更高效的能源储存和利用。
改善循环稳定性:对于储能和储氢材料来说,循环稳定性是一个重要指标。研究人员致力于开发具有良好循环稳定性的材料,以确保储能系统的长寿命和高性能。
节约材料资源:可持续性是储能技术发展的重要考虑因素之一。研究人员正在寻求减少材料使用量、提高资源利用效率和开发可再生材料的方法。
结合多种储能技术:为了克服单一储能技术的局限性,研究人员也在探索将不同的储能技术结合起来,形成混合储能系统,以实现更高效的能源转换和利用。
综上所述,储能和储氢材料在清洁能源领域发挥着重要作用。随着技术的不断创新和进步,储能和储氢材料将继续发展,为可持续能源的应用和推广提供更好的支持。


\section{功能材料制造}
