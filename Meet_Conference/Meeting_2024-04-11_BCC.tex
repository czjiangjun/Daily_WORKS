{\CJKfamily{hei}姜骏}:~确认清明节前赵琉涛主任提供贵部门的\textrm{word}版数据文档说明(不含数据表头)是否可用

{\CJKfamily{hei}任鹏举}:~相关文档已收悉,具体数据准备还需要跟团队成员沟通

{\CJKfamily{hei}姜骏}:~孟老师处已经表示,贵方需要的大模型可以做。另外需要向您这边确认,贵方提供的网站,上面的论文都是英文的,数据是否有意义,是否需要转换成中文

{\CJKfamily{hei}任鹏举}:~数据是有意义的,温晓东老师关心的不是数据形式,更关心的是数据和模型能否真的有用,所以必须把数据整理好。当前的数据整理要求比较高,我们可以提供一部分(数据)

{\CJKfamily{hei}孟凡银}:~此前赵主任提供提供的数据文档说明是按知识图谱模式设计的。网上提供的\textrm{pdf}格式的英文文献,我们可以做数据预处理,直接解析

{\CJKfamily{hei}任鹏举}:~英文文献的数据不需要太多预处理,只做简单分类和介绍。目前的学术论文肯定以英文为主,法规、行业报告等中文文献有面,但需要报请公司后才能指导可否公开。对模型的一个挑战:~按照目前文档建议的数据模式,问答中提问问题是中文形式,支持回答的大部分是英文文献知识,英文-中文交互的具体实现需要您这边考虑合理可行的方案。

{\CJKfamily{hei}孟凡银}:~中-英文交互效果不好,相关方案我们可以做。现在需要有价值的数据,关于数据准备我们有通用的解决方案。

{\CJKfamily{hei}任鹏举}:~通用的文本识别和现成的分词和挖掘,用自然语言模型处理,对专业的文本,分词效果未必好。按提供的文档模式整理文本,有些是格式化的,整理完的数据会很有价值。如果我们去整理,质量会高很多。如化学反应方程式等,整理数据比挖掘数据质量高。实验条件和结果,中文文献质量低,我们将中英文文献整理后再筛查。

{\CJKfamily{hei}任鹏举}:~确认数据类型和数据描述,是否可以提供定量的结果。(如“甲醇产率50\%”“油品产率60\%”等这些结果,孟表示认可)

{\CJKfamily{hei}孟凡银}:~数据只要按照文档说明的格式即可

{\CJKfamily{hei}任鹏举}:~过于基础的东西(如相关名词的定义),我们不关心,不要做

{\CJKfamily{hei}孟凡银}:~先做核心点,没用的可以先不做

{\CJKfamily{hei}任鹏举}:~实体数据的“关系”的定义是柔性的还是预先定义好的

{\CJKfamily{hei}孟凡银}:~能定义好数据关系最好

{\CJKfamily{hei}任鹏举}:~提供的数据是字典形式描述的:~(1)实体、(2)实体-实体间的关系(实体的描述)$\cdots$先按照数据要求整理出一部分(未必能完全遵照要求),提供后看是否需要调整

{\CJKfamily{hei}任鹏举}:~(问答模型中)英文-中文交互的问题需要您那边先考虑一个方案,看看是否可行,这是“刚需”

{\CJKfamily{hei}姜骏}:~根据孟老师的建议,鹏举老师这边先把能整理的数据先整理出来,提交到模型看一下效果

双方继续保持沟通,推进项目前进
