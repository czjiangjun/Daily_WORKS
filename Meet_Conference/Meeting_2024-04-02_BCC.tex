(1)双方成员介绍

(2)交流环节:\\
{\CJKfamily{hei}鲁保旺~博士}:~介绍有关研究背景和研究内容
\begin{itemize}
	\item \ch{CO2}催化还原经过\ch{CO},再加\ch{H}还原形成\ch{CH4}
	\item 金属\ch{Ni}在\ch{SiO2}基底上,不同担载(\ch{CeO2}或\ch{CaO})条件下的反应活性
	\item 实验结果需要理论与计算的结果支持
\end{itemize}
鲁博士报告后,双方就理论计算的模型搭建与实验细节作了详细讨论,历时约1.5小时

(3)双方达成初步研究合作意向,并口头约定:
\begin{itemize}
	\item 由北京市计算中心材料计算团队承担本次\ch{CO2}催化还原的建模、计算和理论模拟任务
	\item 模拟与计算过程中及时沟通与讨论
	\item 计算服务时长初步限定为2个月(60天)
	\item 本次学术合作主要目标成果:~发表论文(共同一作)
	\item 期待以本次学术合作为起点,为今后的合作奠定基础
	\item 学术交流与讨论过程中,将适当专家咨询费形式,感谢参与工作的团队成员的劳动成果
\end{itemize}
