{\CJKfamily{hei}姜骏}:~前期工作的情况
\begin{itemize}
	\item 感谢中科合成油提供机会,通过化学-化工知识图谱项目推动双方合作
	\item 2023年11月双方领导碰面后,煤化工、石油化工、天然气、生物质为起点的数据及产业链没有及时落实
	\item 期待这次一线工作人员对接,能够尽快推进相关工作
\end{itemize}

{\CJKfamily{hei}任鹏举}:~对知识图谱工作的设想和期待
\begin{itemize}
	\item 前期工作的基础,再次明确要求:~不是单纯的知识图谱展示,而是期待能够成为长期合作的起点
	\item 面对人工智能全方位的转型
		\begin{itemize}
			\item 智能实验室
			\item 定位 \ch{C}基材料(能源~材料)
			\item 充分发挥人工智能的作用
		\end{itemize}
	\item 机遇:~知识图谱项目作为切入点
	\item 与科学院文献情报中心合作:~文献信息梳理~(项目已结题)
	\item 关于“智能科学家”项目主要挑战
		\begin{itemize}
			\item 传统的智能:~机械臂的自动化,动作可分解,可自动控制
			\item 科研活动的重复动作自动化:~不统一、非确定性;仪器表征方法多:~操作、接口的多样性;软件的缺乏
			\item 科研数据的优化
		\end{itemize}
	\item 后续工作的设想:~初步的原型机
		\begin{itemize}
			\item 依托软件,完成数据的梳理
			\item 形成知识
			\item 专业知识网络(刚性)-大模型(柔性)相结合:~\textcolor{red}{期待专业领域新知识的发现}
		\end{itemize}
	\item 将当前的具体需求,及时落地
\end{itemize}

{\CJKfamily{hei}孟凡银}:~通用知识图谱与行业学科的结合
\begin{itemize}
	\item 大模型的预测能力
	\item 大模型的预测路径
	\item 基于图像识别的学习推理:~卷积神经网络方法
	\item 知识图谱抽取的普遍关系:\\
		实体-关系-属性
	\item 自然语言的模糊性:\\
		知识图谱-语义关系-知识点网络:~大模型
\end{itemize}

{\CJKfamily{hei}任鹏举}:~关于专业知识(图谱)与大模型的思考与阐述
\begin{itemize}
	\item 基于专业知识(刚性)-\textrm{ChatGPT}模型(柔性)融合思想的详细阐述~姜骏参与了讨论
\end{itemize}

{\CJKfamily{hei}孟凡银}:~说明计算机技术上相关问题的实现策略并展示案例
\begin{itemize}
	\item 明确技术思路上可行,需要数据支持
\end{itemize}

		
{\CJKfamily{hei}任鹏举}:~期待尝试后有成果(不在大,在希望)
\begin{itemize}
	\item 要求提供此前生成知识图谱的文件示例(\textrm{excel}或任何格式)
	\item 参考文件示例提供专业知识数据(文件数据是否有要求)~孟凡银明确:~无特殊要求
	\item \textcolor{blue}{期待尝试知识图谱(刚性)-\textrm{ChatGPT}模型(柔性)的结合}
	\item 能体现人工智能的效果,为领导提供信心:~\textcolor{red}{有助于推进下一阶段的合作,争取主动}
	\item 远期目标:~行业领域内乃至市场需求
\end{itemize}

{\CJKfamily{hei}双方共同约定}
\begin{itemize}
	\item \textcolor{red}{相关技术文档的准备}
	\item \textcolor{blue}{定期讨论机制:~每两周一次}:~周四上午\textrm{10:00}
	\item 有问题随时沟通
	\item 关于知识产权归属的约定:~友好协商,争取共赢
\end{itemize}
