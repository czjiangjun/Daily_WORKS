%---------------------- TEMPLATE FOR REPORT ------------------------------------------------------------------------------------------------------%

\thispagestyle{fancy}   % 插入页眉页脚                                        %
%%%%%%%%%%%%%%%%%%%%%%%%%%%%% 用 authblk 包 支持作者和E-mail %%%%%%%%%%%%%%%%%%%%%%%%%%%%%%%%%
%\title{More than one Author with different Affiliations}				     %
\title{\hei \textrm{2024}年度项目资金动态监管系统培训}
%\title{\hei 清华大学讨论会议纪要}
%\title{\hei 与中科合成油讨论纪要 (IV)}
%\title{\hei 与北京低碳清洁能源研究院讨论纪要}
\author[ ]{记录员:~姜骏}   %
%\author[ ]{记录员:~高朋林}   %
%\author[ ]{姜~骏\thanks{jiangjun@bcc.ac.cn}}   %
%\author[a]{Author A}									     %
%\author[a]{Author B}									     %
%\author[a]{Author C \thanks{Corresponding author: email@mail.com}}			     %
%\author[a]{Author/通讯作者 C \thanks{Corresponding author: cores-email@mail.com}}     	     %
%\author[b]{Author D}									     %
%\author[b]{Author/作者 D}								     %
%\author[b]{Author E}									     %
%\affil[ ]{会议地点:~双清大厦4号楼16层会议室}    %
%\affil[ ]{会议地点:~计算中心\textrm{317}会议室}    %
%\affil[ ]{讨论地点:~北京低碳清洁能源研究院}    %
\affil[ ]{在线讨论}    %
%\affil[a]{Department of Computer Science, \LaTeX\ University}				     %
%\affil[b]{Department of Mechanical Engineering, \LaTeX\ University}			     %
%\affil[b]{作者单位-2}			    						     %
											     %
%%% 使用 \thanks 定义通讯作者								     %
											     %
\renewcommand*{\Authfont}{\small\rm} % 修改作者的字体与大小				     %
\renewcommand*{\Affilfont}{\small\it} % 修改机构名称的字体与大小			     %
\renewcommand\Authands{ and } % 去掉 and 前的逗号					     %
\renewcommand\Authands{ , } % 将 and 换成逗号						     %
%\date{} % 不显示日期									     %
%\date{2020-12-30}									     %
\date{\footnotesize\rm\today}							     %

%%%%%%%%%%%%%%%%%%%%%%%%%%%%%%%%%%%%%%%%%%  不使用 authblk 包制作标题  %%%%%%%%%%%%%%%%%%%%%%%%%%%%%%%%%%%%%%%%%%%%%%
%-------------------------------The Title of The Report-----------------------------------------%
%\title{报告标题:~}   %
%-----------------------------------------------------------------------------

%----------------------The Authors and the address of The Paper--------------------------------%
%\author{
%\small
%Author1, Author2, Author3\footnote{Communication author's E-mail} \\    %Authors' Names	       %
%\small
%(The Address,City Post code)						%Address	       %
%}
%\affil[$\dagger$]{清华大学~材料加工研究所~A213}
%\affil{清华大学~材料加工研究所~A213}
%\date{}					%if necessary					       %
%----------------------------------------------------------------------------------------------%
%%%%%%%%%%%%%%%%%%%%%%%%%%%%%%%%%%%%%%%%%%%%%%%%%%%%%%%%%%%%%%%%%%%%%%%%%%%%%%%%%%%%%%%%%%%%%%%%%%%%%%%%%%%%%%%%%%%%%
\maketitle
%\thispagestyle{fancy}   % 首页插入页眉页脚 

%-------%%%%%%%% PREPARE FOR THE ABSTRACT BIBLIOGRAPH FIG. AND TAB. %%%%%%%%-----------%
%
%\begin{CJK}{UTF8}{gbsn} %针对文字编码为unix %CJK自带的utf-8简体字体有gbsn(宋体)和gkai(楷体)
%\begin{CJK}{GBK}{hei}	%针对文字编码为doc
%\begin{CJK}{GBK}{hei}	 %针对文字编码为doc
%\CJKindent     %在CJK环境中,中文段落起始缩进2个中文字符
%\indent
%
\renewcommand{\abstractname}{\small{\CJKfamily{hei} 到\quad 会\quad 人\quad 员}} %\CJKfamily{hei} 设置中文字体,字号用\large \small来设
%\renewcommand{\contentsname}{\centering\CJKfamily{hei} 目~~~录}
\renewcommand{\refname}{\centering\CJKfamily{hei} 参考文献}
%\renewcommand{\figurename}{\CJKfamily{hei} 图.}
\renewcommand{\figurename}{{\bf Fig}.}
%\renewcommand{\tablename}{\CJKfamily{hei} 表.}
\renewcommand{\tablename}{{\bf Tab}.}
%\renewcommand{\thesubfigure}{\roman{subfigure}}  \makeatletter %子图标记罗马字母
%\renewcommand{\thesubfigure}{\tiny(\alph{subfigure})}  \makeatletter %子图标记英文字母
%\renewcommand{\thesubfigure}{}  \makeatletter %子图无标记

%将图表的Caption写成 图(表) Num. 格式
\makeatletter
\long\def\@makecaption#1#2{%
  \vskip\abovecaptionskip
  \sbox\@tempboxa{#1. #2}%
  \ifdim \wd\@tempboxa >\hsize
    #1. #2\par
  \else
    \global \@minipagefalse
    \hb@xt@\hsize{\hfil\box\@tempboxa\hfil}%
  \fi
  \vskip\belowcaptionskip}
\makeatother

\newcommand{\keywords}[1]{{\hspace{0pt}\small{\CJKfamily{hei} 关键词:}{\hspace{2ex}{#1}}\bigskip}}
\newcommand{\peopinfo}[3]{\small\CJKfamily{hei} #1:~#2,~~#3.~~~}
\newcommand{\peopsimp}[2]{\small\CJKfamily{hei} #1~~#2.}
%
%----------------------------------------------------------------------------------------------------------------------------------------------------%
%
\thispagestyle{fancy}   % 插入页眉页脚                                        %
%------%%%%%%%%%%%%%%------      到会人员信息     -------%%%%%%%%%%%%%--------%
\begin{abstract}
%		%		\item \peopinfo{姜骏}{北京市计算中心}{副研究员}   %到会人员信息(姓名、单位、职称)
%		\peopinfo{王崇愚}{清华大学}{科学院院士}\peopinfo{毛勇}{云南大学}{教授}\peopinfo{毛勇-博士生}{云南大学}{}\\
%		\peopinfo{吴健}{清华大学}{教授}\peopinfo{于涛}{钢铁研究总院}{教授}\peopinfo{姜骏}{北京市计算中心}{副研究员} 
%\peopsimp{聂淼}{}\peopsimp{陶应龙}{}\peopsimp{王彩群}{}\peopsimp{高朋林}{}\peopsimp{姜骏}{}
\peopsimp{姚洁}{}\peopsimp{姜骏}{}
    %%  会议纪录目录:~到会人员信息
%	\begin{itemize}
%		\item \peopinfo{姜骏}{北京市计算中心}{副研究员}   %到会人员信息(姓名、单位、职称)
%	\end{itemize}
	\noindent
%		\peopinfo{王崇愚}{清华大学}{科学院院士}\peopinfo{毛勇}{云南大学}{教授}\peopinfo{毛勇-博士生}{云南大学}{}\\
%		\peopinfo{吴健}{清华大学}{教授}\peopinfo{于涛}{钢铁研究总院}{教授}\peopinfo{姜骏}{北京市计算中心}{副研究员} 
%\peopsimp{聂淼}{}\peopsimp{陶应龙}{}\peopsimp{王彩群}{}\peopsimp{高朋林}{}\peopsimp{姜骏}{}
\peopsimp{姚洁}{}\peopsimp{姜骏}{}
\end{abstract}

%\keywords{Keyword1; Keyword2; Keyword3}

%%%%%%%%%%%%%%%%%%%%%%%%%%%%%%%%%%%%%%%%%%%%%%%%%%%%%%%%%%%%%%%%%%%%%%%%%%%%%%%%%%%%%%%%%%%%%%%%%%
%
%---------------The Content of The Paper----------------------%
%\newpage
%\pagestyle{plain}   % 删除页眉                               %
%\addcontentsline{toc}{subsection}{\CJKfamily{hei} 目~录}
%\tableofcontents %% 制作目录(目录是根据标题自动生成的)
%--------%%%%%%%%%%%%%%%%%%%%%%%%%%%%%%%%%%%%%%%%%%%%%--------%
%
%----------------------------------------------------------------------------------------------------------------------------------------------------%
%

\section*{会议内容}
{\CJKfamily{hei}姜骏}:~前期工作的情况
\begin{itemize}
	\item 感谢中科合成油提供机会,通过化学-化工知识图谱项目推动双方合作
	\item 2023年11月双方领导碰面后,煤化工、石油化工、天然气、生物质为起点的数据及产业链没有及时落实
	\item 期待这次一线工作人员对接,能够尽快推进相关工作
\end{itemize}

{\CJKfamily{hei}任鹏举}:~对知识图谱工作的设想和期待
\begin{itemize}
	\item 前期工作的基础,再次明确要求:~不是单纯的知识图谱展示,而是期待能够成为长期合作的起点
	\item 面对人工智能全方位的转型
		\begin{itemize}
			\item 智能实验室
			\item 定位 \ch{C}基材料(能源~材料)
			\item 充分发挥人工智能的作用
		\end{itemize}
	\item 机遇:~知识图谱项目作为切入点
	\item 与科学院文献情报中心合作:~文献信息梳理~(项目已结题)
	\item 关于“智能科学家”项目主要挑战
		\begin{itemize}
			\item 传统的智能:~机械臂的自动化,动作可分解,可自动控制
			\item 科研活动的重复动作自动化:~不统一、非确定性;仪器表征方法多:~操作、接口的多样性;软件的缺乏
			\item 科研数据的优化
		\end{itemize}
	\item 后续工作的设想:~初步的原型机
		\begin{itemize}
			\item 依托软件,完成数据的梳理
			\item 形成知识
			\item 专业知识网络(刚性)-大模型(柔性)相结合:~\textcolor{red}{期待专业领域新知识的发现}
		\end{itemize}
	\item 将当前的具体需求,及时落地
\end{itemize}

{\CJKfamily{hei}孟凡银}:~通用知识图谱与行业学科的结合
\begin{itemize}
	\item 大模型的预测能力
	\item 大模型的预测路径
	\item 基于图像识别的学习推理:~卷积神经网络方法
	\item 知识图谱抽取的普遍关系:\\
		实体-关系-属性
	\item 自然语言的模糊性:\\
		知识图谱-语义关系-知识点网络:~大模型
\end{itemize}

{\CJKfamily{hei}任鹏举}:~关于专业知识(图谱)与大模型的思考与阐述
\begin{itemize}
	\item 基于专业知识(刚性)-\textrm{ChatGPT}模型(柔性)融合思想的详细阐述~姜骏参与了讨论
\end{itemize}

{\CJKfamily{hei}孟凡银}:~说明计算机技术上相关问题的实现策略并展示案例
\begin{itemize}
	\item 明确技术思路上可行,需要数据支持
\end{itemize}

		
{\CJKfamily{hei}任鹏举}:~期待尝试后有成果(不在大,在希望)
\begin{itemize}
	\item 要求提供此前生成知识图谱的文件示例(\textrm{excel}或任何格式)
	\item 参考文件示例提供专业知识数据(文件数据是否有要求)~孟凡银明确:~无特殊要求
	\item \textcolor{blue}{期待尝试知识图谱(刚性)-\textrm{ChatGPT}模型(柔性)的结合}
	\item 能体现人工智能的效果,为领导提供信心:~\textcolor{red}{有助于推进下一阶段的合作,争取主动}
	\item 远期目标:~行业领域内乃至市场需求
\end{itemize}

{\CJKfamily{hei}双方共同约定}
\begin{itemize}
	\item \textcolor{red}{相关技术文档的准备}
	\item \textcolor{blue}{定期讨论机制:~每两周一次}:~周四上午\textrm{10:00}
	\item 有问题随时沟通
	\item 关于知识产权归属的约定:~友好协商,争取共赢
\end{itemize}
