\documentclass[10pt,a4paper]{article}

%%%%%%%%%%%%%%%%% CJK 中文版面控制  %%%%%%%%%%%%%%%%%%%%%%%%%%%%%%
\usepackage{xeCJK} % CTEX-CJK 中文支持                            %
\usepackage{CJKutf8} % Texlive 中文支持                          %
\usepackage{CJKnumb} %中文序号                                   %
\usepackage{indentfirst} % 中文段落首行缩进                      %
%\setlength\parindent{22pt}       % 段落起始缩进量               %
\renewcommand{\baselinestretch}{1.2} % 中文行间距调整            %
\setlength{\textwidth}{16cm}                                     %
\setlength{\textheight}{24cm}                                    %
\setlength{\topmargin}{-1cm}                                     %
\setlength{\oddsidemargin}{0.1cm}                                %
\setlength{\evensidemargin}{\oddsidemargin}                      %
%%%%%%%%%%%%%%%%%%%%%%%%%%%%%%%%%%%%%%%%%%%%%%%%%%%%%%%%%%%%%%%%%%

\usepackage{authblk}					 %作者地址和E-mail
\usepackage{amsmath,amsthm,amsfonts,amssymb,bm}          %数学公式
\usepackage{mathrsfs}                                    %英文花体

\usepackage{longtable}                                   %使用长表格

%%%%%%%%%%%%%%%%%%%%%%%%%  参考文献引用 %%%%%%%%%%%%%%%%%%%%%%%%%%%
%%尽量使用 BibTeX(含有超链接,数据库的条目URL即可)                %
%%%%%%%%%%%%%%%%%%%%%%%%%%%%%%%%%%%%%%%%%%%%%%%%%%%%%%%%%%%%%%%%%%%

\usepackage[numbers,sort&compress]{natbib} %紧密排列             %
\usepackage[sectionbib]{chapterbib}        %每章节单独参考文献   %
\usepackage{hypernat}                                                                         %
\usepackage[dvipdfm,bookmarksopen=true,pdfstartview=FitH,CJKbookmarks]{hyperref}              %
\hypersetup{bookmarksnumbered,colorlinks,linkcolor=green,citecolor=blue,urlcolor=red}         %
%参考文献含有超链接引用时需要下列宏包,注意与natbib有冲突        %
%\usepackage[dvipdfm]{hyperref}                                  %
%\usepackage{hypernat}                                           %
\newcommand{\upcite}[1]{\hspace{0ex}\textsuperscript{\cite{#1}}} %

%%%%%%%%%%%%%%%%%%%%%%%%%%%%%%%%%%%%%%%%%%%%%%%%%%%%%%%%%%%%%%%%%%%%%%%%%%%%%%%%%%%%%%%%%%%%%%%
%\AtBeginDvi{\special{pdf:tounicode GBK-EUC-UCS2}} %CTEX用dvipdfmx的话,用该命令可以解决      %
%						   %pdf书签的中文乱码问题		      %
%%%%%%%%%%%%%%%%%%%%%%%%%%%%%%%%%%%%%%%%%%%%%%%%%%%%%%%%%%%%%%%%%%%%%%%%%%%%%%%%%%%%%%%%%%%%%%%

%%%%%%%%%%%%%%%%%%%%%  % EPS 图片支持  %%%%%%%%%%%%%%%%%%%%%%%%%%%
\usepackage{graphicx}                                            %
%%%%%%%%%%%%%%%%%%%%%%%%%%%%%%%%%%%%%%%%%%%%%%%%%%%%%%%%%%%%%%%%%%


\begin{document}
%\begin{CJK}{UTF8}{}	%针对文字编码为unix
%\begin{CJK}{GBK}{hei}	%针对文字编码为doc
%\CJKindent     %在CJK环境中,中文段落起始缩进2个中文字符
\graphicspath{{Figures}}
%
\renewcommand{\abstractname}{\small{\CJKfamily{hei} 摘\quad 要}} %\CJKfamily{hei} 设置中文字体,字号用\big \small来设
\renewcommand{\refname}{\centering\CJKfamily{hei} 参考文献}
%\renewcommand{\figurename}{\CJKfamily{hei} 图.}
\renewcommand{\figurename}{{\bf Fig}.}
%\renewcommand{\tablename}{\CJKfamily{hei} 表.}
\renewcommand{\tablename}{{\bf Tab}.}

%将图表的Caption写成 图(表) Num. 格式
\makeatletter
\long\def\@makecaption#1#2{%
  \vskip\abovecaptionskip
  \sbox\@tempboxa{#1. #2}%
  \ifdim \wd\@tempboxa >\hsize
    #1. #2\par
  \else
    \global \@minipagefalse
    \hb@xt@\hsize{\hfil\box\@tempboxa\hfil}%
  \fi
  \vskip\belowcaptionskip}
\makeatother

\newcommand{\keywords}[1]{{\hspace{0\ccwd}\small{\CJKfamily{hei} 关键词:}{\hspace{2ex}{#1}}\bigskip}}

%%%%%%%%%%%%%%%%%%中文字体设置%%%%%%%%%%%%%%%%%%%%%%%%%%%
%默认字体 defalut fonts \TeX 是一种排版工具 \\		%
%{\bfseries 粗体 bold \TeX 是一种排版工具} \\		%
%{\CJKfamily{song}宋体 songti \TeX 是一种排版工具} \\	%
%{\CJKfamily{hei} 黑体 heiti \TeX 是一种排版工具} \\	%
%{\CJKfamily{kai} 楷书 kaishu \TeX 是一种排版工具} \\	%
%{\CJKfamily{fs} 仿宋 fangsong \TeX 是一种排版工具} \\	%
%%%%%%%%%%%%%%%%%%%%%%%%%%%%%%%%%%%%%%%%%%%%%%%%%%%%%%%%%

%\addcontentsline{toc}{section}{Bibliography}

%-------------------------------The Title of The Paper-----------------------------------------%
\title{2018.09.29-30~会议记录}
%----------------------------------------------------------------------------------------------%

%----------------------The Authors and the address of The Paper--------------------------------%
\author{
\small
%Author1, Author2, Author3\footnote{Communication author's E-mail} \\    %Authors' Names	       %
\small
%(The Address,City Post code)						%Address	       %
}
%\affil[$\dagger$]{清华大学~材料加工研究所~A213}
\affil{清华大学~材料加工研究所~A213}
\date{}					%if necessary					       %
%----------------------------------------------------------------------------------------------%
\maketitle

%-------------------------------------------------------------------------------The Abstract and the keywords of The Paper----------------------------------------------------------------------------%
%\begin{abstract}
%The content of the abstract
%\end{abstract}

%\keywords {Keyword1; Keyword2; Keyword3}
%-----------------------------------------------------------------------------------------------------------------------------------------------------------------------------------------------------%

%----------------------------------------------------------------------------------------The Body Of The Paper----------------------------------------------------------------------------------------%
%Introduction
\setcounter{section}{-1}
\section{Introductioo}
衣丰涛\;处长
\begin{itemize}
	\item 监督:~纳税人的钱
	\item 中期考核
	\item 特殊性:~基础研究
	\item 关注
\end{itemize}
研究中要注重
\begin{enumerate}
	\item 突出材料基因的理念和方法
	\item 采用和未采用这类方法的效果对比
\end{enumerate}
组织实施要求高,建议
\begin{itemize}
	\item 听从安排,发挥专家组的作用:~尽量邀请专家组专家出席
	\item 希望项目任务实施好、资金使用好
\end{itemize}

朱付元
\begin{enumerate}
	\item 感谢领导支持
	\item 合作单位多,偏基础;配合需加强,更好地保证项目执行;做好项目管理工作
\end{enumerate}

朱宏伟
\begin{itemize}
	\item 学院保证做好支撑工作
\end{itemize}

\textbf{基体合影}

\section{项目年度报告-1}
许庆彦老师:
\begin{itemize}
	\item 年度报告\\
		\begin{itemize}
			\item 背景和意义:~大于2000K的高温的单晶高温合金
			\item 项目目标:开发四代单晶高温合金
			\item 考核指标: >5000/四代单晶高温合金
			\item 研究内容
			\item 技术路线和研究方法
			\item 任务分解:~中心任务:~全链条多尺度集成高通量计算算法及相关软件\\
				课题1(软件)/课题2(物理)/课题3(实验与验证)
			\item 项目进度安排
			\item 里程碑计划(项目) :什么是“材料基因”
			\item 研究成果 
				\begin{itemize}
					\item 课题1-\textrm{MIP}高通量材料计算平台整体框架
					\item 课题2-五元系合金物理(相图)
					\item 课题3-热处理对合金组织演变影响试验
				\end{itemize}
			\item 项目组织实施机制 项目管理委员会、专家组
			\item 项目经费情况:~分直接经费/间接经费
		\end{itemize}

		\subsection{课题1:~上海大学:~张武}
		\begin{itemize}
			\item 任务分解\\
				任务书(整体任务)
				\begin{itemize}
					\item 自主知识产权、多尺度计算算法与软件
					\item 部署机器
				\end{itemize}
				额外考虑:~语言设计
			\item 高通量软件的实现
				\textrm{Materail Informaton Platform(MIP)}系统
				\begin{itemize}
					\item 简化平台对资源的配置
					\item \textrm{MIP}优势
					\item 软件架构模型:~可视化/模块化接口/自动化流程管理/计算资源及队列兼容性/计算过程容错/数据存储优化/标准化语言描述/针对高温合金材料(批量作业生成)
					\item 基于\textrm{MP}软件包的高通量计算
					\item 课题展望
				\end{itemize}
				
		\end{itemize}
		问题:
		\begin{itemize}
			\item 计算资源的利用率
			\item 知识产权和关系
			\item 计算成本考虑\\

		\end{itemize}
		\subsection{课题2:~九所:~张平}
		\subsection{课题3:~621所:~李嘉荣}
	
王崇愚先生报告(见照片):

\section{项目年度会议-2}
\subsection{课题1:~上海大学~倪剑樾:~高通量材料计算平台模型中间件简介}
软件平台
\begin{itemize}
	\item 平台模型建立与实现:~\textrm{ADDS~}模型\\
在模型上建立平台,特别是用户平台/通过标准化模型建立任务的配置和分发
	\item 平台中间件通用接口
	\item 平台数据中间件接口\\
		异构文件信息收取/机器学习
	\item \textrm{MIP~}的标准化描述语言设计:~标准与规则
	\item 工作展望
\end{itemize}

\subsection{课题1:~上海交通大学~朱虹、孔令体}
\begin{itemize}
	\item 进展二:~\textrm{Ni/Ni$_3$Al~}界面高通量计算流程设计
\end{itemize}
\subsection{课题1:~清华大学~都志辉}
\begin{itemize}
	\item 高通量计算:~特点与建模
		\begin{itemize}
			\item 特点:~相互独立、计算密集/可类聚成不同的作业群
			\item 建模:~多个作业群集合/多个作业集合/优化目标\textrm{min}
		\end{itemize}
	\item 基本假设
		\begin{itemize}
			\item 以相近作业充分弛豫后的结果为基础,构建新作业的的初始态,大幅度缩短新作业的弛豫时间
			\item 考虑不同初始态的影响
			\item 可行性验证
			\item 方法有效性验证:~启发式调度算法/最优化算法
		\end{itemize}
	\item 未解决的问题
		\begin{itemize}
			\item 启发式算法:~更具体的启发式知识/物理与计算
			\item 最优算法:~采用不同作业结果构建初始态的效果
		\end{itemize}
	\item 未来工作建议
\end{itemize}

\subsection{课题1:~北京市计算中心~姜骏}

\subsection{课题1:~中科院数学所~黄记祖}
\textrm{VASP~}弛豫问题\\
弛豫方法的改进:~局域极小值,可能的改进方法\\
提供计算资源

\subsection{课题2:~上海大学~刘悦}
单晶高温合金中的数据关联
\begin{itemize}
	\item 数据采集:~数据存储:~文献-数据-方法的\textrm{Ni}-基单晶高温合金材料数据
	\item 算法研究:~集成学习:~数据驱动的机器学习算法库
		\begin{itemize}
			\item 非自动化:~专家模式
			\item 半自动化(模板)
			\item 全自动化
		\end{itemize}
	\item 实际应用
		\begin{itemize}
			\item 面向蠕变寿命的预测:~基于聚类的最优回归集成学习方法
				\begin{itemize}
					\item 学习样本构造
					\item 数据清洗
					\item 数据变换
				\end{itemize}
			\item 特征选择:~新属性的构造
			\item 基于主动学习的多层级交互式特征分析方法
		\end{itemize}
	\item 应用平台
\end{itemize}

\subsection{课题2:~上海大学~刘轶}
高温合金的相互作用势函数
\begin{itemize}
	\item \textrm{ReaxFF~}反应力场
		\begin{itemize}
			\item \textrm{EOS~}:~\textrm{Brich-Murnaghan~}方程拟合
		\end{itemize}
\end{itemize}
$\ast$王崇愚先生评论提到,余瑞璜是\textrm{Junior Bragg}的学生

\subsection{课题2:~上海大学~鲁晓刚}
与\textrm{2018.7}的报告内容相似

\subsection{课题2:~清华大学~林皎}
见照片

\subsection{课题2:~中科院力学所~王云江}
原子有限元方案与分子动力学相耦合算法发展
\begin{itemize}
	\item 背景:~蠕变\textrm{(Creep)}基础
	\item 多尺度结构演化与宏观物性
	\item 多尺度计算:原子-连续介质力学多尺度蠕变模拟框架
\end{itemize}

\subsection{课题2:~北京理工大学~郭伟}
\textrm{Ta,~W,~Re~-~Doped Ni$_3$Al and Ni}\\
\textrm{How spin-polarization spin-orbit coupling effects make differences}\\
弛豫:~晶胞形状固定\\
\begin{itemize}
	\item \textrm{Structure:~spin effect on Lattice of Doped Ni$_3$Al}
	\item \textrm{spin effect on Bulk modulus of Doped Ni$_3$Al}
	\item \textrm{spin effect on Formation Energy of Doped Ni$_3$Al}
\end{itemize}

\subsection{课题2:~北京交通大学~郭雅芳组}
时间多尺度模拟方法
\begin{itemize}
	\item \textrm{Autonomous Basin Climbing (ABC)-E/Activation-Relaxation Technique (ART)/Metadynamics/Temperature Accelerated Dynamics(TAD)}\\
		结合\textrm{DFT+NEB}\\
		实现\textrm{Atomistic Kinetic Monte Carlo (AKMC)}\\
		自扩散/合金元素扩散+疲劳载荷
\end{itemize}

\subsection{课题2:~九所~李孜}
\begin{itemize}
	\item 原子扩散/偏聚:~\textrm{Ab initio-NEB-Kinetic-Monte Carlo (KMC)}
		\begin{itemize}
			\item 界面元素掺杂计算测试
			\item 电子结构
		\end{itemize}
	\item 能带-声子谱反折叠
		\begin{itemize}
			\item 能带反折叠
			\item 磁结构遗传算法搜索结构(\textrm{MagGen})
		\end{itemize}
	\item 数据管理与分析
		\begin{itemize}
			\item 数据归档
			\item 数据归档平台
		\end{itemize}
\end{itemize}


\subsection{课题3:~钢铁研究总院~于涛}
\begin{itemize}
	\item 实验进展:~背景/实验考虑
	\item 冶炼合金工艺
	\item \textrm{JMatPro~}软件应用:~固溶温度、错配度、相平衡、拉伸强度、热物性
	\item 几点思考:~一般模式(文献)/探索模式:~资源轻质化四代合金
\end{itemize}


%\begin{equation}
%\end{equation}
%-------------------The Figure Of The Paper------------------
%\begin{figure}[h!]
%\centering
%\includegraphics[height=3.35in,width=2.85in,viewport=0 0 400 475,clip]{PbTe_Band_SO.eps}
%\hspace{0.5in}
%\includegraphics[height=3.35in,width=2.85in,viewport=0 0 400 475,clip]{EuTe_Band_SO.eps}
%\caption{\small Band Structure of PbTe (a) and EuTe (b).}%(与文献\cite{EPJB33-47_2003}图1对比)
%\label{Pb:EuTe-Band_struct}
%\end{figure}

%-------------------The Equation Of The Paper-----------------
%\begin{equation}
%\varepsilon_1(\omega)=1+\frac2{\pi}\mathscr P\int_0^{+\infty}\frac{\omega'\varepsilon_2(\omega')}{\omega'^2-\omega^2}d\omega'
%\label{eq:magno-1}
%\end{equation}

%\begin{equation} 
%\begin{split}
%\varepsilon_2(\omega)&=\frac{e^2}{2\pi m^2\omega^2}\sum_{c,v}\int_{BZ}d{\vec k}\left|\vec e\cdot\vec M_{cv}(\vec k)\right|^2\delta [E_{cv}(\vec k)-\hbar\omega] \\
% &= \frac{e^2}{2\pi m^2\omega^2}\sum_{c,v}\int_{E_{cv}(\vec k=\hbar\omega)}\left|\vec e\cdot\vec M_{cv}(\vec k)\right|^2\dfrac{dS}{\nabla_{\vec k}E_{cv}(\vec k)}
% \end{split}
%\label{eq:magno-2}
%\end{equation}

%-------------------The Table Of The Paper----------------------
%\begin{table}[!h]
%\tabcolsep 0pt \vspace*{-12pt}
%\caption{The representative $\vec k$ points contributing to $\sigma_2^{xy}$ of interband transition in EuTe around 2.5 eV.}
%\label{Table-EuTe_Sigma}
%\begin{minipage}{\textwidth}
%%\begin{center}
%\centering
%\def\temptablewidth{1.01\textwidth}
%\rule{\temptablewidth}{1pt}
%\begin{tabular*} {\temptablewidth}{@{\extracolsep{\fill}}cccccc}

%-------------------------------------------------------------------------------------------------------------------------
%&Peak (eV)  & {$\vec k$}-point            &Band{$_v$} to Band{$_c$}  &Transition Orbital
%Components\footnote{波函数主要成分后的括号中,$5s$、$5p$和$5p$、$4f$、$5d$分别指碲和铕的原子轨道。} &Gap (eV)   \\ \hline
%-------------------------------------------------------------------------------------------------------------------------
%&2.35       &(0,0,0)         &33$\rightarrow$34    &$4f$(31.58)$5p$(38.69)$\rightarrow$$5p$      &2.142   \\% \cline{3-7}
%&       &(0,0,0)         &33$\rightarrow$34    &$4f$(31.58)$5p$(38.69)$\rightarrow$$5p$      &2.142   \\% \cline{3-7}
%-------------------------------------------------------------------------------------------------------------------------

%\end{tabular*}
%\rule{\temptablewidth}{1pt}\\
%%\end{center}
%\end{minipage}
%\end{table}

%-------------------The Long Table Of The Paper--------------------
%\begin{small}
%%\begin{minipage}{\textwidth}
%%\begin{longtable}[l]{|c|c|cc|c|c|} %[c]指定长表格对齐方式
%\begin{longtable}[c]{|c|c|p{1.9cm}p{4.6cm}|c|c|}
%\caption{Assignment for the peaks of EuB$_6$}
%\label{tab:EuB6-1}\\ %\\长表格的caption中换行不可少
%\hline
%%
%--------------------------------------------------------------------------------------------------------------------------------
%\multicolumn{2}{|c|}{\bfseries$\sigma_1(\omega)$谱峰}&\multicolumn{4}{c|}{\bfseries部分重要能带间电子跃迁\footnotemark}\\ \hline
%\endfirsthead
%--------------------------------------------------------------------------------------------------------------------------------
%%
%\multicolumn{6}{r}{\it 续表}\\
%\hline
%--------------------------------------------------------------------------------------------------------------------------------
%标记 &峰位(eV) &\multicolumn{2}{c|}{有关电子跃迁} &gap(eV)  &\multicolumn{1}{c|}{经验指认} \\ \hline
%\endhead
%--------------------------------------------------------------------------------------------------------------------------------
%%
%\multicolumn{6}{r}{\it 续下页}\\
%\endfoot
%\hline
%--------------------------------------------------------------------------------------------------------------------------------
%%
%%\hlinewd{0.5$p$t}
%\endlastfoot
%--------------------------------------------------------------------------------------------------------------------------------
%%
%% Stuff from here to \endlastfoot goes at bottom of last page.
%%
%--------------------------------------------------------------------------------------------------------------------------------
%标记 &峰位(eV)\footnotetext{见正文说明。} &\multicolumn{2}{c|}{有关电子跃迁\footnotemark} &gap(eV) &\multicolumn{1}{c|}{经验指认\upcite{PRB46-12196_1992}}\\ \hline
%--------------------------------------------------------------------------------------------------------------------------------
%
%     &0.07 &\multicolumn{2}{c|}{电子群体激发$\uparrow$} &--- &电子群\\ \cline{2-5}
%\raisebox{2.3ex}[0pt]{$\omega_f$} &0.1 &\multicolumn{2}{c|}{电子群体激发$\downarrow$} &--- &体激发\\ \hline
%--------------------------------------------------------------------------------------------------------------------------------
%
%     &1.50 &\raisebox{-2ex}[0pt][0pt]{20-22(0,1,4)} &2$p$(10.4)4$f$(74.9)$\rightarrow$ &\raisebox{-2ex}[0pt][0pt]{1.47} &\\%\cline{3-5}
%     &1.50$^\ast$ & &2$p$(17.5)5$d_{\mathrm E}$(14.0)$\uparrow$ & &4$f$$\rightarrow$5$d_{\mathrm E}$\\ \cline{3-5}
%     \raisebox{2.3ex}[0pt][0pt]{$a$} &(1.0$^\dagger$) &\raisebox{-2ex}[0pt][0pt]{20-22(1,2,6)} &\raisebox{-2ex}[0pt][0pt]{4$f$(89.9)$\rightarrow$2$p$(18.7)5$d_{\mathrm E}$(13.9)$\uparrow$}\footnotetext{波函数主要成分后的括号中,2$s$、2$p$和5$p$、4$f$、5$d$、6$s$分别指硼和铕的原子轨道;5$d_{\mathrm E}$、5$d_{\mathrm T}$分别指铕的(5$d_{z^2}$,5$d_{x^2-y^2}$和5$d_{xy}$,5$d_{xz}$,5$d_{yz}$)轨道,5$d_{\mathrm{ET}}$(或5$d_{\mathrm{TE}}$)则指5个5$d$轨道成分都有,成分大的用脚标的第一个字母标示;2$ps$(或2$sp$)表示同时含有硼2$s$、2$p$轨道成分,成分大的用第一个字母标示。$\uparrow$和$\downarrow$分别标示$\alpha$和$\beta$自旋电子跃迁。} &\raisebox{-2ex}[0pt][0pt]{1.56} &激子跃迁。 \\%\cline{3-5}
%     &(1.3$^\dagger$) & & & &\\ \hline
%--------------------------------------------------------------------------------------------------------------------------------

%     & &\raisebox{-2ex}[0pt][0pt]{19-22(0,0,1)} &2$p$(37.6)5$d_{\mathrm T}$(4.5)4$f$(6.7)$\rightarrow$ & & \\\nopagebreak %\cline{3-5}
%     & & &2$p$(24.2)5$d_{\mathrm E}$(10.8)4$f$(5.1)$\uparrow$ &\raisebox{2ex}[0pt][0pt]{2.78} &a、b、c峰可能 \\ \cline{3-5}
%     & &\raisebox{-2ex}[0pt][0pt]{20-29(0,1,1)} &2$p$(35.7)5$d_{\mathrm T}$(4.8)4$f$(10.0)$\rightarrow$ & &包含有复杂的\\ \nopagebreak%\cline{3-5}
%     &2.90 & &2$p$(23.2)5$d_{\mathrm E}$(13.2)4$f$(3.8)$\uparrow$ &\raisebox{2ex}[0pt][0pt]{2.92} &强激子峰。$^{\ast\ast}$\\ \cline{3-5}
%$b$  &2.90$^\ast$ &\raisebox{-2ex}[0pt][0pt]{19-22(0,1,1)} &2$p$(33.9)4$f$(15.5)$\rightarrow$ & &B2$s$-2$p$的价带 \\ \nopagebreak%\cline{3-5}
%     &3.0 & &2$p$(23.2)5$d_{\mathrm E}$(13.2)4$f$(4.8)$\uparrow$ &\raisebox{2ex}[0pt][0pt]{2.94} &顶$\rightarrow$B2$s$-2$p$导\\ \cline{3-5}
%     & &12-15(0,1,2) &2$p$(39.3)$\rightarrow$2$p$(25.2)5$d_{\mathrm E}$(8.6)$\downarrow$ &2.83 &带底跃迁。\\ \cline{3-5}
%     & &14-15(1,1,1) &2$p$(42.5)$\rightarrow$2$p$(29.1)5$d_{\mathrm E}$(7.0)$\downarrow$ &2.96 & \\\cline{3-5}
%     & &13-15(0,1,1) &2$p$(40.4)$\rightarrow$2$p$(28.9)5$d_{\mathrm E}$(6.6)$\downarrow$ &2.98 & \\ \hline
%--------------------------------------------------------------------------------------------------------------------------------
%%\hline
%%\hlinewd{0.5$p$t}
%\end{longtable}
%%\end{minipage}{\textwidth}
%%\setlength{\unitlength}{1cm}
%%\begin{picture}(0.5,2.0)
%%  \put(-0.02,1.93){$^{1)}$}
%%  \put(-0.02,1.43){$^{2)}$}
%%\put(0.25,1.0){\parbox[h]{14.2cm}{\small{\\}}
%%\put(-0.25,2.3){\line(1,0){15}}
%%\end{picture}
%\end{small}

%-----------------------------------------------------------------------------------------------------------------------------------------------------------------------------------------------------%


%--------------------------------------------------------------------------The Biblography of The Paper-----------------------------------------------------------------%
%\newpage																				%
%-----------------------------------------------------------------------------------------------------------------------------------------------------------------------%
%\begin{thebibliography}{99}																		%
%%\bibitem{PRL58-65_1987}H.Feil, C. Haas, {\it Phys. Rev. Lett.} {\bf 58}, 65 (1987).											%
%\end{thebibliography}																			%
%-----------------------------------------------------------------------------------------------------------------------------------------------------------------------%
%																					%
\phantomsection\addcontentsline{toc}{section}{Bibliography}	 %直接调用\addcontentsline命令可能导致超链指向不准确,一般需要在之前调用一次\phantomsection命令加以修正	%
\bibliography{ref/Myref}																			%
\bibliographystyle{ref/mybib}																		%
%  \nocite{*}																				%
%-----------------------------------------------------------------------------------------------------------------------------------------------------------------------%

\clearpage     %\end{CJK} 前加上\clearpage是CJK的要求

%\end{CJK*}
\end{document}
