{\CJKfamily{hei}姜骏}:~赵琉涛主任建议,关于贵单位提供的数据格式,最理想的是标准\textrm{json}格式或\textrm{macdown}格式;~论文最好能有\textrm{Latex}格式文档

{\CJKfamily{hei}任鹏举}:~大部分文献相对较老,有些还是扫描的图片,因此数据格式和论文格式的要求\textcolor{blue}{可能很难完全满足}
\vskip 3pt
数据提供,与中科院信息中心合作
\begin{itemize}
	\item 煤化工与碳一\textrm{(\ch{C1})}化学专题方向
	\item 提供了数据接口,\textcolor{red}{文献公开部分:~摘要}
\end{itemize}
\textrm{(演示并登录到接口,展示论文摘要)}

{\CJKfamily{hei}王普鑫}:~演示案例知识库
\begin{itemize}
	\item 文献检索
	\item 包含大量的图片
	\item 包含大量的表格
\end{itemize}

{\CJKfamily{hei}任鹏举}:~现有大量的古老文献,大多是扫描件

提议先拿一部分数据文献尝试,合成油方面关心:
\begin{itemize}
	\item \underline{\CJKfamily{hei}英文文献}的``投喂-学习''效果(是否有问题)
	\item 技术是否是开源的
\end{itemize}

{\CJKfamily{hei}任鹏举}:~上次讨论提到的想法:~\textcolor{red}{\CJKfamily{hei}两个网络的融合}(专业知识网络[刚性]-大模型[柔性网络]),探索煤化学反应的网络知识数据
		
{\CJKfamily{hei}孟凡银}:~将为合成油方面提供表头,希望合成油方面提供数据

{\CJKfamily{hei}任鹏举}:~根据提供的表头,有针对性地整理数据:~计划用数据来测试当前大模型的学习能力


双方共同约定:~保持沟通,推进项目前进
