王先生的建议
\begin{itemize}
	\item 选出两个材料~有难度
	\item 讲课内容要选得好
	\item 考虑防疫的影响
	\item 多学科材料设计的基础
	\item 算法、理论的研究
	\item 美国``材料基因组''战略计划报告
	\item 于涛老师:~单晶高温合金的制备
	\item 极端环境下的超高温材料:~3000$^{\circ}\mathrm{C}$,\ch{Pt}、\ch{Rh}、\ch{Ir}
	\item 材料创新基础来自材料基因组
	\item 高熵陶瓷研究的\textrm{Review}发表在\textrm{Nature}
	\item 高熵陶瓷材料:~\textrm{Compound~vs~Composite}
	\item 美国隐形功能的\textrm{F35-2/F35-3}战斗机
	\item 林新华老师:~超级计算和并行计算
\end{itemize}

毛勇教授(云南大学):
\begin{itemize}
	\item 2018年成立
	\item 云南省对云南省院士工作站的期望:~
		\begin{itemize}
			\item 推动学科的建设和发展(围绕材料基因组)
			\item 团队和人才的培养
			\item 结合云南省特色贵金属产业,做一些科研的指导工作(没有经费支持)
		\end{itemize}
	\item 课程建议以录播方式
\end{itemize}

于涛教授(钢铁研究总院):
\begin{itemize}
	\item 确定讲课对象
	\item 
\end{itemize}

刘立平教授(北京理工大学~材料学院):
\begin{itemize}
	\item 以应用需求为导向的课程
	\item 毁伤与防护国家重点实验室
	\item 金属材料研究
	\item 中央军委新材料研究中心设置在北京理工大学:~主要做高熵合金
	\item 北京理工大学在新材料:~包括材料数据库等课题
	\item 针对新材料的高超:~航天三院:~9马赫数,金属-陶瓷合成材料,发动机喷管
	\item 应用材料学院~车辆学院(坦克)~宇航学院(导弹发射)
		作为现在国家最迫切的需要:~耐高温、耐摩擦,自断破片(材料设计)、耐超高温材料:~耐氢脆、烧蚀,
		材料的成型加工和应用仍然是难题
	\item 很多实验在云南水兵器研究,发动机~011/061基地都在贵州(自成体系)
\end{itemize}
