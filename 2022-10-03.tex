%---------------------- TEMPLATE FOR REPORT ------------------------------------------------------------------------------------------------------%

%\thispagestyle{fancy}   % 插入页眉页脚                                        %
%%%%%%%%%%%%%%%%%%%%%%%%%%%%% 用 authblk 包 支持作者和E-mail %%%%%%%%%%%%%%%%%%%%%%%%%%%%%%%%%
%\title{More than one Author with different Affiliations}				     %
%\title{\rm{VASP}的电荷密度存储文件\rm{CHGCAR}}
\title{第四代高温合金材料设计与高通量模拟计算中的算法考虑}
\author[ ]{}   %
%\author[ ]{姜~骏\thanks{jiangjun@bcc.ac.cn}}   %
%\affil[ ]{北京市计算中心}    %
%\author[a]{Author A}									     %
%\author[a]{Author B}									     %
%\author[a]{Author C \thanks{Corresponding author: email@mail.com}}			     %
%%\author[a]{Author/通讯作者 C \thanks{Corresponding author: cores-email@mail.com}}     	     %
%\author[b]{Author D}									     %
%\author[b]{Author/作者 D}								     %
%\author[b]{Author E}									     %
%\affil[a]{Department of Computer Science, \LaTeX\ University}				     %
%\affil[b]{Department of Mechanical Engineering, \LaTeX\ University}			     %
%\affil[b]{作者单位-2}			    						     %
											     %
%%% 使用 \thanks 定义通讯作者								     %
											     %
\renewcommand*{\Authfont}{\small\rm} % 修改作者的字体与大小				     %
\renewcommand*{\Affilfont}{\small\it} % 修改机构名称的字体与大小			     %
\renewcommand\Authands{ and } % 去掉 and 前的逗号					     %
\renewcommand\Authands{ , } % 将 and 换成逗号					     %
\date{} % 去掉日期									     %
%\date{2020-12-30}									     %

%%%%%%%%%%%%%%%%%%%%%%%%%%%%%%%%%%%%%%%%%%  不使用 authblk 包制作标题  %%%%%%%%%%%%%%%%%%%%%%%%%%%%%%%%%%%%%%%%%%%%%%
%-------------------------------The Title of The Report-----------------------------------------%
%\title{报告标题:~}   %
%-----------------------------------------------------------------------------

%----------------------The Authors and the address of The Paper--------------------------------%
%\author{
%\small
%Author1, Author2, Author3\footnote{Communication author's E-mail} \\    %Authors' Names	       %
%\small
%(The Address,City Post code)						%Address	       %
%}
%\affil[$\dagger$]{清华大学~材料加工研究所~A213}
%\affil{清华大学~材料加工研究所~A213}
%\date{}					%if necessary					       %
%----------------------------------------------------------------------------------------------%
%%%%%%%%%%%%%%%%%%%%%%%%%%%%%%%%%%%%%%%%%%%%%%%%%%%%%%%%%%%%%%%%%%%%%%%%%%%%%%%%%%%%%%%%%%%%%%%%%%%%%%%%%%%%%%%%%%%%%
\maketitle
%\thispagestyle{fancy}   % 首页插入页眉页脚 

\section{引言}
航空发动机是飞机的心脏,也是一个国家国防装备和科学技术水平的重要发展标志。单晶高温合金叶片是航空发动机的核心部件,其工作温度高达1800$^{\circ}\mathrm{C}$。创新发展资源化第四代\ch{Ni}-基单晶高温合金是当前国防发展的重要需求。通过材料计算先期大规模筛选复杂合金材料的组合模式,可以更有效地确定试验用合金元素的组成和比例,加快单晶叶片的研发速度和降低成本。

进入20世纪90年代,伴随理论化学、计算物理方法的快速发展以及计算机软硬件技术不断升级和更新,计算材料科学获得了空前发展,它与物理、化学、工程力学以及应用数学等诸多基础和应用学科日益交叉并融合,逐渐成为一门新兴学科,在材料研究中发挥越来越重要的作用\upcite{NatMat3-429_2004,App-CataA254-5_2003,JACS125-4306_2003,JCombChem5-472_2003,Meas_Sci-Tech16-1_2005,Nature392-694_1998}。尤其值得注意的是,近年来,得益于高精度的多尺度计算方法和高性能并行计算技术的突破\upcite{PRL88-255506_2002,Nano-Lett3-1183_2003},高通量材料计算\footnote{高通量材料研究\upcite{Nature410-643_2001}最先借鉴药物合成中的组合合成与筛选\textrm{(combinatorial synthesis and screening)}的思想而出现的,兴起于1990年代中期\upcite{Science268-1738_1995},在21世纪最初十年,逐渐扩展到计算材料研究领域,形成“高通量材料计算”的理念。}在创新发展新材料、发现新现象方面显现出强大的能力,借助机器学习技术进行材料性能预测,用以提升和改善材料性能,发现出更新代次材料,逐渐成为了极具前景的研究热点。

基于高性能计算集群,材料物性设计与模拟计算过程预测目标材料的组成和物性, 已在能源材料预测\upcite{PRL108-068701_2012}、拓扑绝缘体发现\upcite{RMP82-3045_2010}、热电材料\upcite{JACS128-12140_2006}、催化材料\upcite{ACIE46-6016_2007}、轻质镁合金研究\upcite{PRB84-084101_2011}、超导材料\upcite{PRL105-217003_2010}、磁性材料\upcite{NatMat10-158_2011,JPD40-R337_2007},复杂多组元化合物表面设计\upcite{Science316-732_2007,ACSNano5-247_2011},二元或三元化合物结构稳定性判断\upcite{PRB85-144116_2012},以及高强高温合金等体系中有广泛的成功应用和尝试。因为不同尺度下的材料组成基本单元服从不同的物理规律,使用的模拟与计算软件也千差万别,因此从应用软件组织的角度说,高通量(\textrm{high~throughput})自动流程主要面向材料模拟和计算过程的文件组织、软件提交和数据分析过程的自动实现,重点围绕以下问题:~
\begin{itemize}
	\item 材料计算和模拟软件\textrm{I/O}数据生成、传递的自动化
	\item 计算和模拟作业提交、控制及执行的自动化
	\item 计算中间结果和最终结果的解析和可视化展示的自动化
\end{itemize}
伴随着高通量的新材料计算、设计和优化实现过程,一大批开放型材料物性数据库也应运而生。随着材料数据的不断积累、丰富和充实,应用数据挖掘(\textrm{data mining})技术,有望大大加快材料设计、优选的进程。

\section{高通量材料计算的主要模式}
当前高通量材料计算的自动流程主要以支持原子、分子尺度等微观模拟计算为主,目标是实现材料电子结构、分子动力学和热力学物理量计算过程的自动化,建立相应的数据库。自动流程的示意简图如\ref{Fig:MP_data_flow}所示。
\begin{figure}[h!]
\centering
\includegraphics[height=3.1in,width=4.3in,viewport=0 0 800 650,clip]{MP_data_flow.png}%}
\caption{一般材料计算自动流程中的数据流示意(引自文献\cite{CMS49-299_2010}).}%
\label{Fig:MP_data_flow}
\end{figure}
%\section{材料基因组的基本思想}

国际上知名的高效材料计算自动流程软件主要有:~\textrm{Automatic Flow(AFLOW)}\upcite{CMS49-299_2010},\textrm{Materials Project(MP)}\upcite{CMS50-2295_2011},\textrm{Quantum Materials Informatics Project (QMIP)}\upcite{QMIP_URL},\textrm{Clean Energy Project (CEP)}\upcite{JPCL2-2241_2011},\textrm{Atomic Simulation Environment (ASE)}\upcite{JPCM29-273002_2017}以及我国中科院计算机网络信息中心杨小渝等开发的\textrm{Matcloud}\upcite{CMS146-319_2018}等。这些软件主要通过集成密度泛函理论\textrm{(DFT)}为基础的第一原理和分子动力学程序,如\textrm{VASP}\upcite{VASP_manual}、\textrm{Quantum ESPRESSO (QE)}\upcite{JPCM21-395502_2009}、\textrm{ABINIT}\upcite{CPC180-2582_2009}、\textrm{Gaussian}\upcite{Gaussian-UG_2004}、\textrm{LAMMPS}\upcite{JCP117-1_1995}等,完成材料的微观电子结构和相关物性计算。与传统的材料计算不同,高通量自动流程的最终目标之一是建立系统的材料数据库,高通量计算的结果在自动流程的终端将存储到对应的材料数据库中。

%\section{主要高通量材料计算自动流程软件与实现}
高通量材料计算的自动流程是对传统材料计算与模拟过程的计算机自动化与程序化,高通量主要是面向具有复杂组成的材料的模拟计算需求。目前的材料计算以支持第一原理\textrm{DFT}计算为主,少数可支持跨尺度/多尺度\textrm{(DFT-MD)}计算。考虑到材料计算过程的一般特点不难发现,高通量材料计算自动流程主体结构主要包括: 
\begin{enumerate}
	\item 批量化的基本结构建模、计算流程参数的选择与控制(高通量计算流程的逻辑起点)
	\item 高并发度的计算任务的生成与提交,实时计算进程监控(高通量计算流程软件的核心功能)
	\item 高效能的结果数据分析、可视化与数据库管理 
\end{enumerate}
现有的材料计算流程软件围绕上述方面做出了不同的探索,并给出了各自的解决方案,由于各软件的实现采用的编程语言和支持框架不同,其主体结构之间的耦合程度不同,软件应用于具体材料的模拟计算时,效果也有较大差别。各种高通量材料计算自动流程软件功能的对比列于表\ref{Table-Cost}。
\begin{table}[!h]
\tabcolsep 0pt \vspace*{-5pt}
\begin{minipage}{\0.95\textwidth}
%\begin{center}
\centering
\caption{各种高通量材料计算自动流程软件概况}\label{Table-Cost}
\def\temptablewidth{0.92\textwidth}
\renewcommand\arraystretch{0.8} %表格宽度控制(普通表格宽度的两倍)
\rule{\temptablewidth}{1pt}
\begin{tabular*} {\temptablewidth}{@{\extracolsep{\fill}}c@{\extracolsep{\fill}}c@{\extracolsep{\fill}}c@{\extracolsep{\fill}}c@{\extracolsep{\fill}}c@{\extracolsep{\fill}}c@{\extracolsep{\fill}}c}
%-------------------------------------------------------------------------------------------------------------------------
	&\multirow{2}{*}{\fontsize{9.2pt}{7.2pt}\selectfont{编程语言}}	&\fontsize{9.2pt}{7.2pt}\selectfont{建模} &\multicolumn{2}{|c|}{\fontsize{9.2pt}{7.2pt}\selectfont{任务提交与管理}} &\multirow{2}{*}{\fontsize{9.2pt}{7.2pt}\selectfont{后处理}} &\multirow{2}{*}{\fontsize{9.2pt}{7.2pt}\selectfont{数据组织管理}} \\\cline{4-5}
	&	&\fontsize{9.2pt}{7.2pt}\selectfont{功能} &\multicolumn{1}{|c|}{\fontsize{9.2pt}{7.2pt}\selectfont{~~软件接口~~}} &\multicolumn{1}{c|}{\fontsize{9.2pt}{7.2pt}\selectfont{运行容错~~~}} & & \\\hline
	\fontsize{9.2pt}{7.2pt}\selectfont{{\textrm{AFLOW}}} &\fontsize{9.2pt}{7.2pt}\selectfont{\textrm{C++}} &\checkmark &$\triangle$ &\text{\ding{73}} &\text{\ding{73}} &\fontsize{9.2pt}{7.2pt}\selectfont{{\textrm{Django}}} \\
	\fontsize{9.2pt}{7.2pt}\selectfont{{\textrm{MP}}} &\fontsize{9.2pt}{7.2pt}\selectfont{\textrm{Python}} &\checkmark &\checkmark &\text{\ding{73}} &\text{\ding{73}} &\fontsize{9.2pt}{7.2pt}\selectfont{{\textrm{MongoDB}}} \\
	\multirow{2}{*}{\fontsize{9.2pt}{7.2pt}\selectfont{{\textrm{QMIP}}}} &\fontsize{9.2pt}{7.2pt}\selectfont{\textrm{JavaScript/SVG}} &\multirow{2}{*}{\checkmark} &\multirow{2}{*}{\checkmark} &\multirow{2}{*}{--} &\multirow{2}{*}{\checkmark} &\multirow{2}{*}{--} \\
	&\fontsize{9.2pt}{7.2pt}\selectfont{\textrm{+html/Python}} & & & & & \\
	\fontsize{9.2pt}{7.2pt}\selectfont{{\textrm{CEP}}} &\fontsize{9.2pt}{7.2pt}\selectfont{\textrm{Python}} &\checkmark &\checkmark &-- &\checkmark &\fontsize{9.2pt}{7.2pt}\selectfont{{\textrm{Django/MySQL}}} \\
	\fontsize{9.2pt}{7.2pt}\selectfont{{\textrm{ASE}}} &\fontsize{9.2pt}{7.2pt}\selectfont{\textrm{Python}} &\text{\ding{73}} &\text{\ding{73}} &-- &$\triangle$ &-- \\
	\multirow{2}{*}{\fontsize{9.2pt}{7.2pt}\selectfont{{\textrm{MatCloud}}}} &\fontsize{9.2pt}{7.2pt}\selectfont{\textrm{JavaScript}} &\multirow{2}{*}{\checkmark} &\multirow{2}{*}{$\triangle$} &\multirow{2}{*}{\checkmark} &\multirow{2}{*}{\checkmark} &\multirow{2}{*}{\fontsize{9.2pt}{7.2pt}\selectfont{{\textrm{MongoDB}}}} \\
	&\fontsize{9.2pt}{7.2pt}\selectfont{\textrm{+.NETCore}} & & & & &
\end{tabular*}
\rule{\temptablewidth}{1pt}
\fontsize{8.2pt}{5.2pt}\selectfont{
%\begin{description}
%	\item[\text{\ding{73}}]~表示该功能较突出
%	\item[\checkmark]~表示该功能基本满足需求
%	\item[$\triangle$]~表示该功能存在不足
%\end{description}
	\begin{itemize}
		\item \text{\ding{73}}~表示该功能较突出;~\checkmark~表示该功能基本满足需求;~$\triangle$~表示该功能存在不足
	\end{itemize}}
\end{minipage}
%\end{center}
\end{table}

不难看出,当前高通量材料计算自动流程软件只是初步具备了“作业提交-模型计算-数据入库”的框架,面向\ch{Ni}-基单晶高温合金材料这种高复杂度\footnote{设计的高温合金材料计算模型包含结构复杂度和计算规模复杂度两个方面,结构复杂度主要指计算模型,需要包含9种过渡金属元素,各元素及其组分在模型中的分布有较大变化;计算规模复杂度是每个模型的模拟计算都需要大量的计算资源和计算能力支持。}的计算需求,还需要从软件开发和算法改进等多方面提升软件的性能,才有可能解决当前单晶高温合金材料计算模拟的主要技术瓶颈。

\section{高通量材料计算中的算法改进与实现}
一般地,通用的计算机集群都配备管理调度系统,负责对硬件、用户和计算资源的有效分配、组织和管理,其中作业管理系统支持用户队列产生的作业提交、调度和监控。材料模拟计算流程主要针对材料模拟所需的模型创建、参数选择和确定、以及物理信息的提取,主要对计算作业的提交、计算软件的组织和运行产生影响。一旦作业被提交,计算进程就由作业管理系统接管,自动流程一般不会对计算进程予以干预,主要负责对计算过程的产生的文件和输出的监控,直到核心计算软件进程结束,自动流程再从作业管理系统处接管作业的控制权,通过必要判断决定是否继续启动后续作业进程。材料计算核心软件一般都可以并行执行,如流行的第一原理计算软件\textrm{VASP}、\textrm{ABINIT}、\textrm{Siesta}以及量子化学软件\textrm{Gaussian}、\textrm{ADF}等,都有较好的\textrm{MPI}并行能力;分子动力学软件如\textrm{LAMMPS}、\textrm{Gromacs}等不仅可以具备\textrm{MPI}并行能力,近年来利用\textrm{GPU}加速,计算能力得到进一步的提升。

高通量材料计算的软件与算法改进,将主要从三个层次予以考虑:
\begin{itemize}
	\item 提升核心计算软件的并行规模和能力
	\item 提高计算流程软件的作业并发和负载均衡能力
	\item 改进计算流程对计算资源的管理和调度能力
\end{itemize}
这三个层次的算法改进,主要将围绕核心计算软件的并行能力、自动计算流程对核心计算软件的调度能力、自动流程软件与系统作业管理系统的交互能力三个方向实现,从而整体上提高高通量材料计算自动流程处理复杂模型的尺度与规模。

\subsection{提升核心计算软件的并行规模和能力}
核心计算软件是材料计算最重要的部分,所有的材料模拟和物性计算都要通过核心计算软件完成,传统材料模拟处理复杂体系的策略主要依赖操作系统支持的\textrm{MPI}并行接口,完成大规模的自洽迭代和矩阵对角化,但这一模式的并行扩展性一般不高,而且由于不同核心软件的\textrm{MPI}并行性能差异极大,当自动流程需要组织多个核心计算软件完成跨尺度计算时,有可能造成计算资源的巨大浪费。第一原理计算软件中,\textrm{VASP}的并行和扩展性都是最优秀的代表,根本原因是\textrm{VASP}软件在\textrm{MPI}并行的基础上,引入轮询调度算法\textrm{(Round-Robin Scheduling)},实现计算任务在计算资源(节点或核)分配上的负载均衡。针对材料模拟计算过程中,自洽迭代求解偏微分方程\textrm{(Partial Differential Equations,~PDE)}是计算资源消耗的主要部分,由数据结构出发,构建支撑数据到计算资源的均衡负载的软件框架,将有望系统地提升材料计算核心软件的并行能力和扩展性,从而提升软件处理更大规模计算模型的能力。针对\ch{Ni}-基单晶高温合金材料的模拟,考虑到模型的复杂度(过渡金属原子多、原子成分变化大),软件框架的开发将重点解决软件的并行扩展性问题,通过建立以密度矩阵理论为基础的线性标度算法及软件框架,原理性改变哈密顿矩阵结构,优化作业并行效率。

\subsection{提高计算流程的作业并发与负载均衡能力}
当前主要的高通量材料计算自动流程软件大部分应用\textrm{Python}语言开发,有较好的灵活性,但是简单的顺序式计算任务组织和作业提交模式也限制了流程软件的并发任务能力。参照\textrm{VASP}软件的并行扩展度提高策略,通过重新设计计算流程,引入均衡负载算法,将顺序计算流程中可独立并发的计算任务,按照计算资源均衡分配,提升计算流程的水平扩展\textrm{(Scale Out)}能力。当前的高通量材料计算流程软件中,\textrm{Mater Projects}的计算流程组织、管理和参数传递,都基于数据库实现,该模式大大方便了复杂计算流程中的子任务的有序组装、分配。\ch{Ni}-基单晶高温合金材料的模拟,涉及跨尺度计算的复杂流程,已经远远超过现有自动流程软件的支持范围,如图\ref{CH4_comp_BCC}所示,复杂体系的材料模拟生成海量的初始结构,结构优化子过程的并发将有助于提升计算效率。后续\textrm{DFT-MD}耦合计算中,也存在大量类似可并发的计算子过程(或组装的子过程)。利用数据库技术,针对单晶高温合金材料的计算模拟过程,开发新型自动流程软件,将每个子过程与传递参数都分解为数据库元素,组织并优化成并发度高的计算流程,将会提升计算流程对计算资源的利用率,克服因计算流程设计不合理而导致的资源浪费,最终实现合理、有效地调度和分配计算资源的目的。
\begin{figure}[h!]
\centering
\vskip -2pt
\includegraphics[height=2.25in]{CH4_complex_machine.png}
\caption{面向复杂体系材料模拟顺序流程的子过程并发化示意图}%
\label{CH4_comp_BCC}
\end{figure}

\subsection{改进计算流程对计算资源的管理和调度能力}
高通量材料计算流程与作业管理系统都会涉及到对计算资源的调度、分配、管理,但现有的计算流程在生成计算任务后,将计算资源和任务提交作业管理系统后,直到作业管理系统完成计算任务管理过程,将结果和计算资源返还计算流程,这一过程中,计算流程一般不再参与计算资源的分配和调度。这一计算方式在简单材料模型计算中可以保持较高的计算资源利用率,但是考虑到\ch{Ni}-基单晶高温合金材料的模拟计算流程的复杂性,即使考虑独立子进程的并发,计算过程中仍会存在诸多计算资源空置和等待的情形。以计算模型的结构优化为例,同时并发的多个计算模型,由于元素和组分的不同,并发任务完成时间的差别可能会很大。如果计算流程能在在此过程中动态地监控各计算资源上计算任务的负载情况,将后续队列中的计算任务及时地分发到空载节点上,有望大大加速计算模型的优化效率。在此基础上,如果能适当引入数据挖掘或机器学习算法,也有助于提升材料模拟计算流程的计算速度。以数据库方式管理计算流程,极大地方便计算流程的子过程和参数的分解,同时也为动态分发子过程提供了便利。针对高温合金材料模拟过程,开发新型的材料计算流程软件,加强计算流程对计算资源的管理和调度,对于提升复杂材料模拟的高通量计算有着重要的现实需求。

\section{小结}
材料模拟与计算在第四代\ch{Ni}-基单晶高温合金的研发过程中,将起到非常重要的探索和牵引作用,如何有效地管理、分配和利用计算资源,是应用高通量自动流程方式计算需要重点考虑的问题,现有的材料计算流程软件无法支持具有如此复杂度材料的高效模拟过程。通过分析材料计算过程自洽迭代的特点,从核心计算软件、计算流程与负载均衡和计算流程对计算资源的控制与调度三个层面讨论提升计算效率的可行算法,重点围绕单晶高温合金材料计算模拟的复杂度,结合现有软件基础和计算资源,提出开发对应计算流程软件的可行技术路线。
