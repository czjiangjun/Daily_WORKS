%---------------------- TEMPLATE FOR REPORT ------------------------------------------------------------------------------------------------------%

%\thispagestyle{fancy}   % 插入页眉页脚                                        %
%%%%%%%%%%%%%%%%%%%%%%%%%%%%% 用 authblk 包 支持作者和E-mail %%%%%%%%%%%%%%%%%%%%%%%%%%%%%%%%%
%\title{More than one Author with different Affiliations}				     %
%\title{\rm{VASP}的电荷密度存储文件\rm{CHGCAR}}
%\title{面向高温合金材料设计的计算模拟软件中的几个主要问题}
\title{计算材料团队本年度工作规划:~\\经营计划与科研计划}
\author[ ]{}   %
%\author[ ]{姜~骏\thanks{jiangjun@bcc.ac.cn}}   %
%\affil[ ]{北京市计算中心}    %
%\author[a]{Author A}									     %
%\author[a]{Author B}									     %
%\author[a]{Author C \thanks{Corresponding author: email@mail.com}}			     %
%%\author[a]{Author/通讯作者 C \thanks{Corresponding author: cores-email@mail.com}}     	     %
%\author[b]{Author D}									     %
%\author[b]{Author/作者 D}								     %
%\author[b]{Author E}									     %
%\affil[a]{Department of Computer Science, \LaTeX\ University}				     %
%\affil[b]{Department of Mechanical Engineering, \LaTeX\ University}			     %
%\affil[b]{作者单位-2}			    						     %
											     %
%%% 使用 \thanks 定义通讯作者								     %
											     %
\renewcommand*{\Authfont}{\small\rm} % 修改作者的字体与大小				     %
\renewcommand*{\Affilfont}{\small\it} % 修改机构名称的字体与大小			     %
\renewcommand\Authands{ and } % 去掉 and 前的逗号					     %
\renewcommand\Authands{ , } % 将 and 换成逗号					     %
\date{} % 去掉日期									     %
%\date{2020-12-30}									     %

%%%%%%%%%%%%%%%%%%%%%%%%%%%%%%%%%%%%%%%%%%  不使用 authblk 包制作标题  %%%%%%%%%%%%%%%%%%%%%%%%%%%%%%%%%%%%%%%%%%%%%%
%-------------------------------The Title of The Report-----------------------------------------%
%\title{报告标题:~}   %
%-----------------------------------------------------------------------------

%----------------------The Authors and the address of The Paper--------------------------------%
%\author{
%\small
%Author1, Author2, Author3\footnote{Communication author's E-mail} \\    %Authors' Names	       %
%\small
%(The Address,City Post code)						%Address	       %
%}
%\affil[$\dagger$]{清华大学~材料加工研究所~A213}
%\affil{清华大学~材料加工研究所~A213}
%\date{}					%if necessary					       %
%----------------------------------------------------------------------------------------------%
%%%%%%%%%%%%%%%%%%%%%%%%%%%%%%%%%%%%%%%%%%%%%%%%%%%%%%%%%%%%%%%%%%%%%%%%%%%%%%%%%%%%%%%%%%%%%%%%%%%%%%%%%%%%%%%%%%%%%
\maketitle
%\thispagestyle{fancy}   % 首页插入页眉页脚 
\section{团队概况}
北京市计算中心有限公司计算材料团队当前成员为姜骏、赵琉涛、李鸿飞、王彩群,共计有博士二人,硕士二人。主要工作围绕第一原理和分子动力学软件与方法从事微观尺度材料物性模拟研究、材料微观晶体和物理性质料数据库和\textrm{AI for Science}的相关软件与应用。

\section{工作基础}
近年来团队承担的研究工作主要包括:
\begin{itemize}
	\item 国家重点研发计划项目``高通量并发式材料计算算法和软件''(编号:~\textrm{2017YFB0701500})和``产学研用协同的高通量材料计算融合服务平台''(编号:~\textrm{2018YFB0704300})
%	\item 北科院青年骨干计划项目``基于甲烷催化燃烧机理的材料计算自动流程设计''(编号:~\textrm{YC201820})
	\item 中科合成油技术有限公司外协任务``化学化工知识图谱项目开发''
	\item 国家重大专项``基于人工智能技术的高性能多尺度分子动力学模拟平台''子课题任务,本单位课题经费共计50万元
	\item 北京科技大学外协任务``氢化锆中氢原子扩散及辐照行为模拟'',本单位课题经费共计9.8万元
%	\item 二氧化碳-甲烷催化还原机理研究
\end{itemize}
截止到2025年年底,团队2025年度收入为:~纵向课题经费已到账33.4万元,横向经费已到账3.43万元。

\section{本年度的工作规划}
团队本年度工作继续认真完成现有研究任务,主要包括(1)确保北京科技大学横向任务顺利结题,(2)完成重大专项子课题任务软件著作权申报及经费处理。在完成现有科研工作基础上,结合公司的发展方向,将工作重心转向到以下两个方向:
\begin{itemize}
	\item 横向课题:~与常州电池协会合作,完成电池行业知识图谱的构建;
	\item 纵向课题:~与北京航空航天大学合作,承担国家纵向研究任务。
\end{itemize}

根据前期与常州电池协会的对接,对方的目标是实现电池数据智能化治理,具体的实现方案是建立涵盖科研、生产、质检、回收的电池价值链的数据采集、标注与合成标准,依据\textrm{FAIR}法则提升数据质量与人机可读性,构建行业知识图谱与大模型。综合评估甲方的需求,团队的能力和公司的发展战略,这个项目将是团队依托自身能力,靠拢并汇入公司发展方向,实现团队工作模式转型的重要节点。2023-2024年,团队曾参与中科合成油技术有限公司的横向课题``化学-化工知识图谱项目'',对知识图谱相关内容有所积累。作为\textrm{AI}的重要技术形式,知识图谱是数据关联和可视化的重要方式,将团队的工作重点向知识图谱倾斜,既与公司的发展战略一致,也是强化团队具体实现\textrm{AI4Sci}落地的具体方案。根据目前了解的情况,甲方掌握较为完整的电池行业全链条的数据,有较为系统的图数据库后台,因此协同开发和完成知识图谱项目的基础较好。该项目得到江苏省和常州市一级项目支持的可能性较大,积极参与,因此会力争一定量的经费到位,初步确定为15-20万元。

与北京航空航天大学的对接,争取国家/纵向研究课题任务,主要是面向''\textrm{AI+}新材料''相关的研究课题,甲方是2026年广州首届``\textrm{AI+}新材料''大会第一分会``\textrm{AI}智能材料设计''的分会主席,团队成员姜骏是分会组委会成员,本次大会后,预计将会发布一批研究课题与任务,主要围绕\textrm{AI}加持的智能材料设计,这个方向也是团队自成立以来一直关注和努力的方向,也是\textrm{AI4Sci}实现的具体方向,积极参与该学术大会、申请该方向课题,有助于团队贯彻院刘晖书记关于``要积极拓展和加强与高校、研究机构的交流、合作''的指示,积累学术资源,形成团队和公司的研究特色,培养人才梯队的重要方式,团队将力争与甲方共同参与相关研究任务的申报,力争纵向经费额度10-20万元/年。

综合考虑团队人员与研究能力,\textrm{2026}年度团队的项目预计经费收入情况如表\ref{Table-Income}。

\begin{table}[!h]
\tabcolsep 0pt 
%\vspace*{-12pt}
\caption{\textrm{2026年度~项目与课题经费预计收入.}}
\label{Table-Income}
\begin{minipage}{0.95\textwidth}
%\begin{center}
\centering
\def\temptablewidth{0.88\textwidth}
%\rule{\temptablewidth}{2.0pt}
%\begin{tabular*}{\temptablewidth}{|@{\extracolsep{\fill}}c|@{\extracolsep{\fill}}c|@{\extracolsep{\fill}}c|}
\begin{tabular*}{\temptablewidth}{@{\extracolsep{\fill}}c@{\extracolsep{\fill}}c@{\extracolsep{\fill}}c@{\extracolsep{\fill}}c}
	\toprule[2pt]        %% 绘制三线表顶端线,粗细设置
	\multirow{2}{*}{\fontsize{10.2pt}{6.2pt}\selectfont{项目名称}}  & {\fontsize{7.2pt}{6.2pt}\selectfont{纵向课题}} & {\fontsize{7.2pt}{6.2pt}\selectfont{横向课题}} & \multirow{2}{*}{\fontsize{10.2pt}{6.2pt}\selectfont{备注}}\\
	& {\fontsize{7.2pt}{6.2pt}\selectfont{经费~(万元)}} & {\fontsize{7.2pt}{6.2pt}\selectfont{经费~(万元)}} & \\
%-------------------------------------------------------------------------------------------------------------------------
%&Peak (eV)  & {$\vec k$}-point            &Band{$_v$} to Band{$_c$}  &Transition Orbital
%Components\footnote{波函数主要成分后的括号中,$5s$、$5p$和$5p$、$4f$、$5d$分别指碲和铕的原子轨道。} &Gap (eV)   \\ \hline
%-------------------------------------------------------------------------------------------------------------------------
%&2.35       &(0,0,0)         &33$\rightarrow$34    &$4f$(31.58)$5p$(38.69)$\rightarrow$$5p$      &2.142   \\% \cline{3-7}
%&       &(0,0,0)         &33$\rightarrow$34    &$4f$(31.58)$5p$(38.69)$\rightarrow$$5p$      &2.142   \\% \cline{3-7}
%-------------------------------------------------------------------------------------------------------------------------
	\midrule[1pt]        %% 绘制三线表中间线,粗细设置
	{\fontsize{9.8pt}{6.2pt}\selectfont{北京科技大学~外协课题}} &  & 6.37 &\multirow{2}{*}{\fontsize{9.8pt}{6.2pt}\selectfont{已确定2026年度收入}} \\
	{\fontsize{9.8pt}{6.2pt}\selectfont{重大专项任务}}  &16.6 & & \\
	\midrule[1pt]        %% 绘制三线表中间线,粗细设置
	{\fontsize{9.8pt}{6.2pt}\selectfont{常州电池协会~联合项目}} &  & 35-40 & \multirow{2}{*}{\fontsize{9.8pt}{6.2pt}\selectfont{2026年度外联~(争取)}}\\
	{\fontsize{9.8pt}{6.2pt}\selectfont{北航物理学院~外协或联合申报}} & 10-20 & & \\
	\midrule[1pt]        %% 绘制三线表中间线,粗细设置
	合计 & 约25-35 & 约 43 & \\
	\bottomrule[1.8pt]     %%绘制三线表底部线,粗细设置
\end{tabular*}
%\rule{\temptablewidth}{1.8pt}
%\end{center}
\\\vskip 5pt
%{\heiti 注:~}{\fontsize{9.8pt}{6.2pt}\selectfont{2026年度,预计由纵向经费转横向经费约20-30万元,团队的年度收入定为~43万元。}}
\end{minipage}
\end{table}

\textrm{2026}年度预计提交研究成果如下:
\begin{itemize}
	\item \underline{软件著作权}:~申报软件著作权2项,完成重大专项中本年度的研究任务;%,为重大专项结题作准备;
	\item \underline{研究论文}:~发表高质量学术论文1-2篇。
%	\item \underline{发明专利}:~如条件允许,提交申报发明专利1项
\end{itemize}

年度详细计划:
\begin{itemize}
	\item $1\sim2$月:~制定年度计划、联系常州电池协会与北京航空航天大学物理学院,准备申报材料;~继续完成北科大外协项目
	\item $3\sim4$月:~首届``\textrm{AI+}新材料''大会第一分会:~(1)~跟踪北航物理学院合作意向课题,(2)~收集可能的高校、科研单位合作信息;~北科大外协项目收尾,提交本年度重大专项研究任务成果软著(第二份)申报
	\item $5\sim6$月:~北科大外协项目结题,常州电池协会合作,推动知识图谱相关合作进展
	\item $7\sim9$月:~通过国自然研究任务和申报课题,保持与北航物理学院的合作关系,开展外协任务合作;备选:~如北航外协任务落空,启动化学所外协合作,争取外协项目落地
	\item $10\sim12$月:~着手重大专项任务收尾工作,准备结题材料,根据知识图谱项目进展,评估与常州电池协会的合作程度,确定后续合作与否;~总结年度成果,评估可能合作的高校、科研单位,并与相关老师沟通,准备下一年度的课题申报和相关工作
\end{itemize}

\section{困境与展望}
总的来看,团队年度计划存在不小的变数和不确定性,相关研究工作和任务,对计算资源的需求缺口也比较大,团队成员对于转型可能遇到的问题和困难的预期可能不足,对\textrm{AI}工具的掌握仍不够熟练,作为团队骨干成员,必须努力接洽和争取到相关任务和研究课题,力争解决好这些问题。

全面分析团队现状,2026年对于团队发展非常关键,一方面,依托现有经费预算,团队还能有限开展研究工作和对外争取相关研究任务的接洽;另一方面,团队现有研究工作已经进入收尾和下半场,任务非常繁重。作为团队骨干成员,%本人必须积极做好相关经费和(横向、纵向)任务的争取工作。特别是通过
在年度计划中的工作重心转移过程中,必须竭尽全力,才能实现团队发展和研究方式的转型,确保团队的存在和发展与公司的发展方向更契合。

%\section{异构计算材料软件:~面向多尺度计算与催化机理}
%国家重点专项``高通量并发式材料计算算法和软件''主要基于材料基因组基本理念及其核心科学内涵,以发展高通量多尺度并发式集成计算算法和相应的软件系统为基础,研发国家重大需求的资源化高温合金:~多尺度科学以表述物质跨空间或时间尺度的参数传递或耦合现象为特点,化学多元复杂结构合金的物性局域性或广延性规律内禀于其多尺度关联或跨越之中。本团队在参与该专项时,主要承担的任务是协同清华大学、中科院数学所发展用于高温合金的高通量并发式计算算法及基础软件自动流程,负责高精度设计、软件及对称性分析及高通量并发式计算中$\geqslant5000$中的部分作业建模及效率分析;~与上海交通大学协同提升单线程与整体性能方面的算法优化、调度算法等方面的工作;~注释\textrm{VASP}软件中微动弹性能带\textrm{(NEB)}部分的代码。完成该项目,本单位发表论文3篇,获得软件著作权1项,专利1项。

%\section{院青年骨干项目}
%院青年骨干项目``基于甲烷催化燃烧机理的材料计算自动流程设计''主要针对$\mathrm{CH}_4$燃烧机理研究:~天然气的主要成分$\mathrm{CH}_4$催化燃烧过程复杂性的制约,催化剂与$\mathrm{CH}_4$相互作用的微观机理一直并不清楚,开发适合天然气在$377\sim877^{\circ}\mathrm{C}$范围燃烧的催化剂,多年来一直是研究的重点和难点。%在没有催化剂的条件下,甲烷在空气中直接燃烧,温度高达1600$^{\circ}\mathrm{C}$左右,并生成氮氧化物($\mathrm{NO}_x$)等污染物质。使用催化剂,不仅可以降低$\mathrm{CH}_4$ 的起燃温度和燃烧峰值温度,减少污染物生成;并且燃烧利用率可以达到99.9\%,接近完全氧化,基本不会形成\textrm{CO}和碳氢化合物,因此$\mathrm{CH}_4$ 催化燃烧可以达到近零污染排放。
%开发适合催化材料微观机理模拟的\textrm{DFT-MD}自动流程软件,研究动力学约化耦合算法、高通量基元反应\textrm{Kohn-Sham}方程筛选与调度平衡算法。以$\mathrm{CH}_4$催化燃烧反应研究为牵引,面向微观尺度下异相界面催化燃烧反应动力学机理模拟计算的高效率实现。通过基元反应活化能确定影响进程的决速步,筛选对反应机理干扰的大量自由基反应;优化决速步反应得到的势能面(\textrm{DFT-MD}耦合的关键)提升分子反应动力学计算的迭代稳定性为共性特征,形成含有多决速步或非关联第一原理计算流程作业框架,结合具体的燃烧反应动力学求解流程实例化,得到适用于催化反应动力学研究的流程版本,经优化的并发\textrm{DFT}求解流程可将电子步计算自动流程的计算效率提高20\%左右。研制适应多决速步化学反应的复杂反应动力学机理研究的自动流程软件,并在商业计算机和国产高性能超级计算机上成功测试运行,成为可跨平台典型催化燃烧反应动力学模拟的自动流程软件。完成该项目,本单位发表论文1篇,获得软件著作权1项。

%\section{并行计算架构与资源}
%团队长期从事云计算、大数据等领域的系统研发工作。%、\textrm{Apache~Spark}是一个用于大规模数据处理的开源分布式计算引擎,旨在通过内存计算提高大数据处理的效率。随着数据量和计算复杂度的增加,\textrm{Spark}的性能优化变得至关重要。
%\textrm{Spark}作为一种通用数据处理框架,默认设置在各种场景下可能并不总是最优的,因此在处理特定的工作负载时,进行优化可以大幅提高系统的性能。团队对执行计划优化\textrm{(Execution Plan Optimization)}、数据分区与调度优化\textrm{(Data Partitioning and Task Scheduling Optimization)}以及内存管理与缓存优化\textrm{(Memory Management and Caching Optimization)}等方向对\textrm{Spark}应用以及\textrm{Spark}系统进行优化等方向进行研究,并在工程实践上获得了良好效果。此外,团队还参与北科智算平台的设计和维护,该平台是基于四个``材料基因工程''的成果凝练而成,构建了一套高效、安全的材料研发生态系统。团队还负责科学计算\textrm{(GPUMD,VASP,Lammps,CP2K等)}、工程仿真、图形图像等高性能计算领域的行业软件的技术咨询、安装、编译、维护管理工作。%提供技术咨询或其他技术支持,调研平台用户的使用需求。处理平台用户在使用中出现的问题。%协助同事实现网页的基本框架和交互功能,优化网页内容。
%围绕基于\textrm{FP-LAPW}高精度第一原理计算软件\textrm{WIEN2k}的总体重构:\\
%\textrm{WIEN2k}代码陈旧、数据结构零散、数据读写效率低下,大大制约了该软件在实际应用中的影响力,但\textrm{FP-LAPW}方法本身具有极高的精度,而且不依赖原子赝势。本人对\textrm{WIEN2k}代码进行全面梳理,提出针对\textrm{WIEN2k}代码重构和效率提升有总体的方案,并就核心代码(自洽迭代部分)加速提出具体的实现路线。

%围绕基于\textrm{PAW}方法的原子数据集(赝势)的构造:\\
%\textrm{VASP}的原子数据集集文件\textrm{POTCAR}性能优异,但仅有原子信息文件,而无生成方案;开源软件\textrm{QE}、\textrm{PWSCF}、\textrm{ABINIT}等的原子数据集生成方案不及\textrm{VASP}。本人通过对相关原子数据集和\textrm{PAW}方法的深入研究,探索了支持\textrm{VASP}原子数据集生成的可行方案

%此间完成专著一部(与人合著)。
%\section{稀土和半导体材料微观晶体和物理性质数据库建设}
%通过国家重点专项和院骨干项目的实施,团队完成支持合金与多相催化的异构化材料计算自动流程的开发,对现有国际常用材料计算流程与数据库及其代码实现作了系统剖析,掌握了包括\textrm{Material Projects}、\textrm{ASE}在内的多种计算软件自动流程与数据库实现方案,以该自动流程为基础,搭建了稀土和半导体材料微观晶体和物理性质数据库。该数据库涵盖稀土金属硼化物、过渡金属氧化物、镍基单晶高温合金模型的晶体结构和电子结构(能带、态密度)和磁学、光学性质等,总数据量约为1000余条,有效可读数据1G。数据以``北京-稀土与碱土金属化合物、半导体材料电子结构数据''形式在北京国际大数据交易有限公司完成了数据资产登记,并以``金属与半导体微观尺度晶体和物理性质数据集''形式完成了数据知识产权登记。
 
%\section{二氧化碳-甲烷催化还原机理研究}
%自2022年秋,在完成$\mathrm{CH}_4$催化燃烧机理的基础上,与北京科技大学大学、北京航空航天大学和北京低碳清洁能源研究院相关课题组合作,基于第一原理和分子动力学的理论和方法,研究$\mathrm{CO}_2$催化还原的催化机理,特别是针对金属氧化物$\mathrm{CeO2}_2$担载金属$\mathrm{Ni}$的界面,对$\mathrm{CO}_2$经$\cdot\mathrm{COOH}$还原为$\mathrm{CO}$,研究可能的催化动力学过程;$\mathrm{CO}$在金属$\mathrm{Ni}$表面与$\mathrm{H}$作用,形成$\mathrm{CH}_4$的可能动力学机理。团队的计算表明,$\mathrm{CO}_2$在过渡金属氧化物表面的吸附和缺陷作用是$\mathrm{CO}_2\rightarrow\mathrm{CO}$的重要催化过程,而过渡金属$\mathrm{Ni}$的主要作用是分解还原剂$\mathrm{H}_2$,为还原反应提供足够的活化$\mathrm{H}$原子。因此较好地说明了金属氧化物担载金属催化剂的耦合作用。相关研究结果正在整理成研究论文。

%完成上述工作的同时,团队成员完成专著《计算材料科学理论与实践》的编写,并于2021年由人民邮电出版社出版。


%电池数据智能化治理年度工作计划
%
%一、工作背景与意义
%
%当前,新能源电池产业正处于高速发展的战略机遇期,科研创新迭代加速、生产规模持续扩大、质检要求不断提高、回收体系逐步完善,全价值链的高效协同已成为产业高质量发展的核心驱动力。而数据作为贯穿电池科研、生产、质检、回收全生命周期的关键要素,其价值挖掘与高效利用是提升产业核心竞争力的重要支撑。
%
%然而,当前电池行业数据治理存在诸多痛点:各环节数据采集标准不统一,数据格式杂乱、口径不一,导致数据互通共享困难;数据标注缺乏规范流程,标签体系混乱,影响数据的精准性与可用性;数据合成技术应用不足,难以满足科研创新与场景模拟的需求;数据质量参差不齐,人机可读性差,无法有效支撑智能化应用与决策。在此背景下,开展电池数据智能化治理工作,建立全价值链数据治理标准,依据FAIR法则(可查找性、可访问性、互操作性、可重用性)提升数据质量,构建行业知识图谱与大模型,对于打破数据壁垒、激活数据价值、推动产业数字化转型、保障产业安全高效发展具有重要的现实意义与战略价值。
%
%二、总体工作目标
%
%本年度围绕电池数据智能化治理核心任务,聚焦科研、生产、质检、回收全价值链,完成数据采集、标注与合成标准体系构建;全面提升数据质量,实现数据符合FAIR法则要求,提升人机可读性;初步建成电池行业知识图谱与基础大模型,为产业科研创新、生产优化、质量提升、回收高效开展提供数据支撑与智能化服务,推动电池行业数据治理规范化、智能化水平显著提升。
%
%三、主要工作任务及实施细则
%
%(一)构建全价值链数据采集、标注与合成标准体系
%
%1.  开展全价值链数据调研与梳理。组建专业调研团队,联合行业协会、头部企业、科研院所,对电池科研、生产、质检、回收各环节的数据现状进行全面调研。重点梳理各环节数据类型(如科研阶段的材料配方数据、性能测试数据,生产阶段的工艺参数数据、设备运行数据,质检阶段的外观检测数据、性能检测数据,回收阶段的电池残值评估数据、拆解流程数据等)、数据来源、数据格式、现有采集方式及存在的问题。形成《电池全价值链数据现状调研报告》,明确各环节数据治理的重点与难点,为标准体系构建奠定基础。调研工作于本年度3月底前完成。
%
%2.  制定数据采集标准。基于调研结果,依据科学性、实用性、兼容性原则,分环节制定数据采集标准。明确各环节数据采集的范围、指标、精度要求、采集频率、采集方式(如自动采集、手动录入)及数据存储格式。例如,科研阶段重点规范材料成分、实验条件、性能测试结果等数据的采集指标与格式;生产阶段明确极片制作、电芯装配、化成检测等关键工序工艺参数的采集精度与实时性要求;质检阶段统一外观缺陷、容量、循环寿命等检测数据的采集标准与判定口径;回收阶段规范电池型号、使用年限、健康状态、拆解产物等数据的采集内容。同时,制定数据采集设备与系统的技术要求,确保数据采集的规范性与一致性。标准草案于4月底前完成,经行业专家评审、试点验证修改后,6月底前正式发布实施。
%
%3.  制定数据标注标准。建立统一的电池数据标注体系,明确标注原则、标注流程、标注规范及质量评估标准。根据数据类型划分标注类别,如科研数据标注分为材料类型标注、实验场景标注、性能等级标注等;生产数据标注分为工艺环节标注、设备状态标注、质量隐患标注等;质检数据标注分为缺陷类型标注、缺陷等级标注、合格状态标注等;回收数据标注分为电池状态标注、回收方式标注、资源可利用性标注等。制定标注操作手册,规范标注人员的操作流程,明确标注工具的技术要求。同时,建立标注质量审核机制,确保标注数据的准确性与完整性。标准草案于5月底前完成,7月底前正式发布实施。
%
%4.  制定数据合成标准。结合电池行业应用需求,制定数据合成标准,明确数据合成的目标、原则、方法、技术要求及质量评估指标。规范数据合成的流程,包括原始数据筛选、合成模型选择、参数设置、结果验证等环节。针对不同应用场景(如科研创新中的材料性能预测、生产工艺优化中的场景模拟、回收体系中的资源评估等),制定相应的合成数据格式与输出要求。明确合成数据的安全性与合规性要求,确保合成数据不泄露原始敏感信息,同时符合行业数据使用规范。标准草案于6月底前完成,8月底前正式发布实施。
%
%(二)依据FAIR法则提升数据质量与人机可读性
%
%1.  数据可查找性提升。建立统一的电池数据元数据标准,明确元数据的核心要素(如数据名称、数据来源、数据类型、创建时间、负责人、关键字等),为每一条数据赋予唯一的标识符(DOI),确保数据可精准查找。构建数据目录管理系统,对全价值链数据进行分类编目,实现数据目录的动态更新与检索功能。用户可通过关键字、数据类型、应用场景等多维度快速查找所需数据。此项工作于9月底前完成系统搭建与数据目录初始化。
%
%2.  数据可访问性提升。搭建安全可靠的数据共享访问平台,制定数据访问权限管理规范,明确不同用户(科研人员、企业生产人员、质检人员、回收企业人员等)的访问权限与数据使用范围。建立数据访问接口标准,支持不同系统间的数据互联互通,确保授权用户可便捷访问所需数据。同时,建立数据访问日志管理机制,对数据访问行为进行全程记录,保障数据安全。平台搭建与权限配置工作于10月底前完成。
%
%3.  数据互操作性提升。开展数据格式标准化转换工作,对现有不同格式的历史数据进行梳理与转换,使其符合制定的数据采集标准格式。建立数据语义映射规则,规范数据术语定义,消除不同环节、不同企业间的数据语义差异。开发数据互操作工具,支持不同系统、不同格式数据的高效对接与交互,实现全价值链数据的顺畅流转。此项工作于11月底前完成历史数据转换与互操作工具开发。
%
%4.  数据可重用性提升。建立数据质量评估体系,从准确性、完整性、一致性、时效性、可用性等维度制定数据质量评估指标与方法。组建数据质量审核团队,对采集的原始数据、标注数据、合成数据进行全面审核,对不合格数据进行清洗、修正或剔除。同时,为数据添加详细的使用说明、授权信息等元数据,明确数据的适用范围与重用条件,提升数据的可重用性。建立数据质量持续改进机制,定期对数据质量进行监测与评估,不断优化数据治理流程。数据质量评估体系于7月底前建立,全年开展不少于4次数据质量全面审核工作。
%
%5.  人机可读性优化。优化数据呈现形式,采用直观的图表、结构化的文档等形式展示数据,提升数据的人类可读性。同时,规范数据的机器可读格式(如JSON、XML等),确保数据可被计算机程序高效解析与处理。开发数据可视化工具,支持用户自定义数据展示方式,实现数据的多维度、可视化分析。此项工作于12月底前完成工具开发与应用部署。
%
%(三)构建电池行业知识图谱与大模型
%
%1.  行业知识图谱构建。开展电池行业知识体系梳理,明确知识图谱的核心领域与本体结构,包括材料知识、工艺知识、产品知识、质检知识、回收知识、标准规范知识等。基于制定的数据采集与标注标准,采集整理行业内的文献资料、企业生产数据、科研数据、标准规范等多源数据,作为知识图谱构建的数据源。运用自然语言处理、知识抽取等技术,从数据源中抽取实体、关系、属性等知识要素,建立知识三元组。构建知识图谱数据库,实现知识的存储、管理与更新。开发知识图谱可视化与查询系统,支持用户对行业知识的检索、浏览与分析。本年度完成知识图谱核心本体构建与部分关键领域知识的抽取与入库,初步实现知识查询与可视化功能,于12月底前完成系统部署。
%
%2.  行业基础大模型研发。开展电池行业大模型需求分析,明确模型的应用场景(如科研辅助、生产工艺优化、质检缺陷识别、回收方案推荐等)与核心功能。基于构建的知识图谱与标准化数据,搭建大模型训练数据集,对数据集进行清洗、预处理与增强,提升数据集的质量与规模。选择合适的基础模型架构,结合电池行业特点进行模型微调与优化,开发具备行业特色的基础大模型。建立模型评估指标体系,从准确性、效率、泛化能力等维度对模型进行评估与迭代优化。本年度完成大模型训练数据集的搭建与基础模型的微调优化,实现部分核心应用场景的功能验证,于12月底前完成模型原型开发。
%
%四、实施步骤与时间安排
%
%(一)准备阶段(1-2月):成立专项工作小组,明确各成员职责与分工;制定详细的工作方案与进度计划;开展行业调研前期准备工作,确定调研对象与调研提纲;搭建工作沟通机制与协作平台。
%
%(二)标准体系构建阶段(3-8月):完成全价值链数据现状调研,形成调研报告;分批次完成数据采集、标注、合成标准的制定、评审与发布;开展标准宣贯培训前期准备工作。
%
%(三)数据质量提升阶段(7-12月):建立数据元数据标准与数据目录管理系统;搭建数据共享访问平台,完成权限配置;开展历史数据格式转换与数据语义映射;建立数据质量评估体系,开展数据质量审核与优化;开发数据可视化工具,优化人机可读性。
%
%(四)知识图谱与大模型构建阶段(9-12月):完成电池行业知识图谱核心本体构建;采集整理知识图谱数据源,完成部分关键领域知识抽取与入库,开发知识图谱可视化与查询系统;开展大模型需求分析,搭建训练数据集,完成基础模型微调优化与原型开发。
%
%(五)总结评估阶段(12月):对本年度各项工作任务完成情况进行全面梳理与总结;开展工作成效评估,分析存在的问题与不足;制定下一年度工作改进计划与重点任务。
%
%五、保障措施
%
%(一)组织保障:成立由行业专家、企业技术骨干、科研人员组成的专项工作小组,明确牵头单位与协作单位的职责分工,建立定期会商机制,及时协调解决工作推进过程中存在的问题。邀请行业协会、监管部门参与工作指导,确保工作方向符合行业发展与政策要求。
%
%(二)资源保障:合理调配人力、物力、财力资源,保障工作顺利开展。配备专业的技术团队,负责标准制定、系统开发、数据治理等核心工作;加大资金投入,保障调研、标准制定、系统搭建、模型研发等工作的经费需求;搭建专用的软硬件环境,为数据存储、处理、模型训练提供技术支撑。
%
%(三)技术保障:加强与高校、科研院所的技术合作,引进先进的数据治理、人工智能、大数据等技术,提升工作的技术水平。建立技术攻关机制,针对工作推进过程中遇到的技术难题,组织开展联合攻关。加强技术人员培训,提升团队的专业能力与技术素养。
%
%(四)制度保障:建立健全工作管理制度,包括项目管理制度、数据安全管理制度、质量控制管理制度、知识产权保护制度等,规范工作流程,保障工作质量与数据安全。建立考核评价机制,对工作任务完成情况进行定期考核,确保各项工作落到实处。
%
%(五)协同保障:加强行业内企业、科研院所、行业协会的协同合作,建立数据共享与利益协调机制,推动多源数据的汇聚与共享。开展标准试点应用工作,选取部分代表性企业进行标准落地验证,广泛收集反馈意见,不断优化完善标准体系。加强行业宣传与交流,提升行业对数据智能化治理的认知与参与度。
%
%六、预期成果与效益
%
%(一)预期成果:形成1套涵盖电池科研、生产、质检、回收全价值链的数据采集、标注与合成标准体系;建成1个电池数据目录管理系统与1个数据共享访问平台;完成现有核心历史数据的标准化转换与质量优化;构建1个电池行业核心知识图谱及可视化查询系统;研发1个电池行业基础大模型原型;形成《电池数据智能化治理年度工作报告》及相关技术文档。
%
%(二)预期效益:通过标准体系的实施,实现电池全价值链数据的规范化采集与管理,打破数据壁垒,提升数据互通共享能力;通过数据质量提升,增强数据的可查找性、可访问性、互操作性与可重用性,提升人机可读性,为企业决策与智能化应用提供高质量数据支撑;通过行业知识图谱与大模型的构建,整合行业知识资源,提升科研创新效率,优化生产工艺,提高质检准确性与效率,推动回收体系的规范化与高效化,助力电池行业数字化转型与高质量发展,提升行业核心竞争力。
%
