%\newpage	        % 每个新的/newpage 即可有新的\thispagestyle 引领      %
%\thispagestyle{fancy}   % 插入页眉页脚                                        %

%%%%---%%%%%%---- The Main Body OF THE PAPER ----%%%%%----%%%%%
%\chapter{第一章} %%\Doucmentclass{book}时可以用
\section{正文章节}

参考文献的引用方式1\upcite{QCQC_2014}

%------------%%%%%%%%%%%%%%%%%%%%%%%%%%%%%%%%%%%%%------------%

%---%%%%%%%%%%%-----The Figure Of The Paper-----%%%%%%%%%----
%\begin{figure}[h!]
%\centering
%\includegraphics[height=3.35in,width=2.85in,viewport=0 0 400 475,clip]{PbTe_Band_SO.eps}
%\hspace{0.5in}
%\includegraphics[height=3.35in,width=2.85in,viewport=0 0 400 475,clip]{EuTe_Band_SO.eps}
%\caption{\small Band Structure of PbTe (a) and EuTe (b).}%(与文献\cite{EPJB33-47_2003}图1对比)
%\label{Pb:EuTe-Band_struct}
%\end{figure}
%------------%%%%%%%%%%%%%%%%%%%%%%%%%%%%%%%%%%%%%------------%

%--%%%%%%%%%%%------The Equation Of The Paper-----%%%%%%%%%---%
%\begin{equation}
%\varepsilon_1(\omega)=1+\frac2{\pi}\mathscr P\int_0^{+\infty}\frac{\omega'\varepsilon_2(\omega')}{\omega'^2-\omega^2}d\omega'
%\label{eq:magno-1}
%\end{equation}

%\begin{equation} 
%\begin{split}
%\varepsilon_2(\omega)&=\frac{e^2}{2\pi m^2\omega^2}\sum_{c,v}\int_{BZ}d{\vec k}\left|\vec e\cdot\vec M_{cv}(\vec k)\right|^2\delta [E_{cv}(\vec k)-\hbar\omega] \\
% &= \frac{e^2}{2\pi m^2\omega^2}\sum_{c,v}\int_{E_{cv}(\vec k=\hbar\omega)}\left|\vec e\cdot\vec M_{cv}(\vec k)\right|^2\dfrac{dS}{\nabla_{\vec k}E_{cv}(\vec k)}
% \end{split}
%\label{eq:magno-2}
%\end{equation}
%------------%%%%%%%%%%%%%%%%%%%%%%%%%%%%%%%%%%%%%------------%

%---%%%%%%%%%%%-----The Table Of The Paper----%%%%%%%%%%%%%---%
%\begin{table}[!h]
%\tabcolsep 0pt \vspace*{-12pt}
%%\caption{The representative $\vec k$ points contributing to $\sigma_2^{xy}$ of interband transition in EuTe around 2.5 eV.}
%\label{Table-EuTe_Sigma}
%\begin{minipage}{\textwidth}
%%\begin{center}
%\centering
%\def\temptablewidth{0.84\textwidth}
%\rule{\temptablewidth}{1pt}
%\begin{tabular*} {\temptablewidth}{|@{\extracolsep{\fill}}c|@{\extracolsep{\fill}}c|@{\extracolsep{\fill}}l|}

%-------------------------------------------------------------------------------------------------------------------------
%&Peak (eV)  & {$\vec k$}-point            &Band{$_v$} to Band{$_c$}  &Transition Orbital
%Components\footnote{波函数主要成分后的括号中,$5s$、$5p$和$5p$、$4f$、$5d$分别指碲和铕的原子轨道。} &Gap (eV)   \\ \hline
%-------------------------------------------------------------------------------------------------------------------------
%&2.35       &(0,0,0)         &33$\rightarrow$34    &$4f$(31.58)$5p$(38.69)$\rightarrow$$5p$      &2.142   \\% \cline{3-7}
%&       &(0,0,0)         &33$\rightarrow$34    &$4f$(31.58)$5p$(38.69)$\rightarrow$$5p$      &2.142   \\% \cline{3-7}
%-------------------------------------------------------------------------------------------------------------------------
%\end{tabular*}
%\rule{\temptablewidth}{1pt}
%\end{minipage}{\textwidth}
%\end{table}
%------------%%%%%%%%%%%%%%%%%%%%%%%%%%%%%%%%%%%%%------------%

%---%%%%%%%%%%%-----The Long Table Of The Paper---%%%%%%%%%%%%----%
%\begin{small}
%%\begin{minipage}{\textwidth}
%%\begin{longtable}[l]{|c|c|cc|c|c|} %[c]指定长表格对齐方式
%\begin{longtable}[c]{|c|c|p{1.9cm}p{4.6cm}|c|c|}
%\caption{Assignment for the peaks of EuB$_6$}
%\label{tab:EuB6-1}\\ %\\长表格的caption中换行不可少
%\hline
%%
%--------------------------------------------------------------------------------------------------------------------------------
%\multicolumn{2}{|c|}{\bfseries$\sigma_1(\omega)$谱峰}&\multicolumn{4}{c|}{\bfseries部分重要能带间电子跃迁\footnotemark}\\ \hline
%\endfirsthead
%--------------------------------------------------------------------------------------------------------------------------------
%%
%\multicolumn{6}{r}{\it 续表}\\
%\hline
%--------------------------------------------------------------------------------------------------------------------------------
%标记 &峰位(eV) &\multicolumn{2}{c|}{有关电子跃迁} &gap(eV)  &\multicolumn{1}{c|}{经验指认} \\ \hline
%\endhead
%--------------------------------------------------------------------------------------------------------------------------------
%%
%\multicolumn{6}{r}{\it 续下页}\\
%\endfoot
%\hline
%--------------------------------------------------------------------------------------------------------------------------------
%%
%%\hlinewd{0.5$p$t}
%\endlastfoot
%--------------------------------------------------------------------------------------------------------------------------------
%%
%% Stuff from here to \endlastfoot goes at bottom of last page.
%%
%--------------------------------------------------------------------------------------------------------------------------------
%标记 &峰位(eV)\footnotetext{见正文说明。} &\multicolumn{2}{c|}{有关电子跃迁\footnotemark} &gap(eV) &\multicolumn{1}{c|}{经验指认\upcite{PRB46-12196_1992}}\\ \hline
%--------------------------------------------------------------------------------------------------------------------------------
%
%     &0.07 &\multicolumn{2}{c|}{电子群体激发$\uparrow$} &--- &电子群\\ \cline{2-5}
%\raisebox{2.3ex}[0pt]{$\omega_f$} &0.1 &\multicolumn{2}{c|}{电子群体激发$\downarrow$} &--- &体激发\\ \hline
%--------------------------------------------------------------------------------------------------------------------------------
%
%     &1.50 &\raisebox{-2ex}[0pt][0pt]{20-22(0,1,4)} &2$p$(10.4)4$f$(74.9)$\rightarrow$ &\raisebox{-2ex}[0pt][0pt]{1.47} &\\%\cline{3-5}
%     &1.50$^\ast$ & &2$p$(17.5)5$d_{\mathrm E}$(14.0)$\uparrow$ & &4$f$$\rightarrow$5$d_{\mathrm E}$\\ \cline{3-5}
%     \raisebox{2.3ex}[0pt][0pt]{$a$} &(1.0$^\dagger$) &\raisebox{-2ex}[0pt][0pt]{20-22(1,2,6)} &\raisebox{-2ex}[0pt][0pt]{4$f$(89.9)$\rightarrow$2$p$(18.7)5$d_{\mathrm E}$(13.9)$\uparrow$}\footnotetext{波函数主要成分后的括号中,2$s$、2$p$和5$p$、4$f$、5$d$、6$s$分别指硼和铕的原子轨道;5$d_{\mathrm E}$、5$d_{\mathrm T}$分别指铕的(5$d_{z^2}$,5$d_{x^2-y^2}$和5$d_{xy}$,5$d_{xz}$,5$d_{yz}$)轨道,5$d_{\mathrm{ET}}$(或5$d_{\mathrm{TE}}$)则指5个5$d$轨道成分都有,成分大的用脚标的第一个字母标示;2$ps$(或2$sp$)表示同时含有硼2$s$、2$p$轨道成分,成分大的用第一个字母标示。$\uparrow$和$\downarrow$分别标示$\alpha$和$\beta$自旋电子跃迁。} &\raisebox{-2ex}[0pt][0pt]{1.56} &激子跃迁。 \\%\cline{3-5}
%     &(1.3$^\dagger$) & & & &\\ \hline
%--------------------------------------------------------------------------------------------------------------------------------

%     & &\raisebox{-2ex}[0pt][0pt]{19-22(0,0,1)} &2$p$(37.6)5$d_{\mathrm T}$(4.5)4$f$(6.7)$\rightarrow$ & & \\\nopagebreak %\cline{3-5}
%     & & &2$p$(24.2)5$d_{\mathrm E}$(10.8)4$f$(5.1)$\uparrow$ &\raisebox{2ex}[0pt][0pt]{2.78} &a、b、c峰可能 \\ \cline{3-5}
%     & &\raisebox{-2ex}[0pt][0pt]{20-29(0,1,1)} &2$p$(35.7)5$d_{\mathrm T}$(4.8)4$f$(10.0)$\rightarrow$ & &包含有复杂的\\ \nopagebreak%\cline{3-5}
%     &2.90 & &2$p$(23.2)5$d_{\mathrm E}$(13.2)4$f$(3.8)$\uparrow$ &\raisebox{2ex}[0pt][0pt]{2.92} &强激子峰。$^{\ast\ast}$\\ \cline{3-5}
%$b$  &2.90$^\ast$ &\raisebox{-2ex}[0pt][0pt]{19-22(0,1,1)} &2$p$(33.9)4$f$(15.5)$\rightarrow$ & &B2$s$-2$p$的价带 \\ \nopagebreak%\cline{3-5}
%     &3.0 & &2$p$(23.2)5$d_{\mathrm E}$(13.2)4$f$(4.8)$\uparrow$ &\raisebox{2ex}[0pt][0pt]{2.94} &顶$\rightarrow$B2$s$-2$p$导\\ \cline{3-5}
%     & &12-15(0,1,2) &2$p$(39.3)$\rightarrow$2$p$(25.2)5$d_{\mathrm E}$(8.6)$\downarrow$ &2.83 &带底跃迁。\\ \cline{3-5}
%     & &14-15(1,1,1) &2$p$(42.5)$\rightarrow$2$p$(29.1)5$d_{\mathrm E}$(7.0)$\downarrow$ &2.96 & \\\cline{3-5}
%     & &13-15(0,1,1) &2$p$(40.4)$\rightarrow$2$p$(28.9)5$d_{\mathrm E}$(6.6)$\downarrow$ &2.98 & \\ \hline
%--------------------------------------------------------------------------------------------------------------------------------
%%\hline
%%\hlinewd{0.5$p$t}
%\end{longtable}
%%\end{minipage}{\textwidth}
%%\setlength{\unitlength}{1cm}
%%\begin{picture}(0.5,2.0)
%%  \put(-0.02,1.93){$^{1)}$}
%%  \put(-0.02,1.43){$^{2)}$}
%%\put(0.25,1.0){\parbox[h]{14.2cm}{\small{\\}}
%%\put(-0.25,2.3){\line(1,0){15}}
%%\end{picture}
%\end{small}
%-------------%%%%%%%%%%%%%%%%%%%%%%%%%%%%%%%%%%%%%-------------%


%%%%%%%%%%%%%%%%%%%%%%%%%%%%%%%%%%%%%%%%  绘制流程图  %%%%%%%%%%%%%%%%%%%%%%%%%%%%%%%%%%%%%%%%%%%%%%%%%%%%%%%%%%%%%%%%%
%\tikzstyle{startstop} = [rectangle,rounded corners, minimum width=3cm,minimum height=1cm,text centered,text width =3cm, draw=black,fill=red!30]
%\tikzstyle{io} = [trapezium, trapezium left angle = 70,trapezium right angle=110,minimum width=3cm,minimum height=1cm,text centered,text width =3cm,draw=black,fill=blue!30]
%\tikzstyle{process} = [rectangle,minimum width=3cm,minimum height=1cm,text centered,text width =3cm,draw=black,fill=orange!30]
%\tikzstyle{decision} = [diamond,aspect = 3,text centered,draw=black,fill=green!30]
%\tikzstyle{arrow} = [thick,->,>=stealth]
%\tikzstyle{straightline} = [line width = 1pt,-]
%\tikzstyle{point}=[coordinate]
%
%\begin{tikzpicture}[node distance=2cm]
%\node (start) [startstop] {开始};
%\node (input1) [io,below of=start] {输入聚类的个数 $k$ 和最大迭代次数 $n$ };
%\node (process1) [process,below of=input1] {初始化 $k$ 个聚类中心};
%\node (process2) [process,below of=process1] {分配各数据对象到距离最近的类中};
%\node (decision1) [decision,below of=process2,yshift=-0.5cm] {是否收敛或迭代次数达到 $n$ };
%\node (stop) [startstop,below of=decision1,node distance=3cm] {输出聚类结果};
%\node(point1)[point,left of=input1,node distance=5cm]{};
%
%\draw [arrow] (start) -- (input1);
%\draw [arrow] (input1) -- (process1);
%\draw [arrow] (process1) -- (process2);
%\draw [arrow] (process2) -- (decision1);
%\draw [arrow] (decision1) -- node[anchor=east] {是} (stop);
%\draw [straightline] (decision1) -|  (point1);
%\draw [arrow] (point1) -- node[anchor=south] {否} (input1);
%\end{tikzpicture}

%%%%%%%%%%%%%%%%%%%%%%%%%%%%%%%%%%%%%%%%  绘制流程图  %%%%%%%%%%%%%%%%%%%%%%%%%%%%%%%%%%%%%%%%%%%%%%%%%%%%%%%%%%%%%%%%%
%\begin{figure}
%\scriptsize
%\tikzstyle{format}=[rectangle,draw,thin,fill=white]
%%定义语句块的颜色,形状和边
%\tikzstyle{test}=[diamond,aspect=2,draw,thin]
%%定义条件块的形状,颜色
%\tikzstyle{point}=[coordinate,on grid,]
%%像素点,用于连接转移线
%\begin{tikzpicture}%[node distance=10mm,auto,>=late',thin,start chain=going below,every join/.style={norm},]
%%start chain=going below指明了流程图的默认方向,node distance=8mm则指明了默认的node距离。这些可以在定义node的时候更改,比如说
%%\node[point,right of=n3,node distance=10mm] (p0){};
%%这里声明了node p0,它在node n3 的右边,距离是10mm。
%\node[format] (start){Start};
%\node[format,below of=start,node distance=7mm] (define){Some defines};
%\node[format,below of=define,node distance=7mm] (PCFinit){PCF8563 Initialize};
%\node[format,below of=PCFinit,node distance=7mm] (DS18init){DS18 Initialize};
%\node[format,below of=DS18init,node distance=7mm] (LCDinit){LCD Initialize};
%\node[format,below of=LCDinit,node distance=7mm] (processtime){Processtime};
%\node[format,below of=processtime,node distance=7mm] (keyinit){Key Initialize};
%\node[test,below of=keyinit,node distance=15mm](setkeycheck){Check Set Key};
%\node[point,left of=setkeycheck,node distance=18mm](point3){};
%\node[format,below of=setkeycheck,node distance=15mm](readtime){Read Time};
%\node[point,right of=readtime,node distance=15mm](point4){};
%\node[format,below of=readtime](processtime1){Processtime};
%\node[format,below of=processtime1](gettemp){Get Temperature};
%\node[format,below of=gettemp](display){Display All Data};
%\node[format,right of=setkeycheck,node distance=40mm](setsetflag){Set SetFlag=1};
%\node[format,below of=setsetflag](setinit){Set Mode Initialize};
%\node[format,below of=setinit](checksetting){Checksetting()};
%\node[test,below of=checksetting,node distance=15mm](savecheck){Check Save Key};
%\node[format,below of=savecheck,node distance=15mm](clearsetflag){Clear SetFlag=0};
%\node[format,below of=clearsetflag](settime){Set Time};
%\node[point,below of=display,node distance=7mm](point1){};
%\node[point,below of=settime,node distance=7mm](point2){};
%%\node[format] (n0) at(4,4){A}; 直接指定位置
%%定义完node之后进行连线,
%%\draw[->] (n0.south) -- (n1); 带箭头实线
%%\draw[-] (n0.south) -- (n1); 不带箭头实线
%%\draw[&lt;->] (n0.south) -- (n1.north);   双箭头
%%\draw[&lt;-,dashed] (n1.south) -- (n2.north); 带箭头虚线 
%%\draw[&lt;-] (n0.south) to node{Yes} (n1.north);  带字,字在箭头方向右边
%%\draw[->] (n1.north) to node{Yes} (n0.south);  带字,字在箭头方向左边
%%\draw[->] (n1.north) to[out=60,in=300] node{Yes} (n0.south);  曲线
%%\draw[->,draw=red](n2)--(n1);  带颜色的线
%\draw[->] (start)--(define);
%\draw[->] (define)--(PCFinit);
%\draw[->](PCFinit)--(DS18init);
%\draw[->](DS18init)--(LCDinit);
%\draw[->](LCDinit)--(processtime);
%\draw[->](processtime)--(keyinit);
%\draw[->](keyinit)--(setkeycheck);
%\draw[->](setkeycheck)--node[above]{Yes}(setsetflag);
%\draw[->](setkeycheck) --node[left]{No} (readtime);
%\draw[->](readtime)--(processtime1);
%\draw[->](processtime1)--(gettemp);
%\draw[->](gettemp)--(display);
%\draw[-](display)--(point1);
%\draw[-](point1)-|(point3);
%\draw[->](point3)--(setkeycheck.west);
%\draw[->](setsetflag)--(setinit);
%\draw[->](setinit)--(checksetting);
%\draw[->](checksetting)--(savecheck);
%\draw[->](savecheck)--node[left]{Yes}(clearsetflag);
%\draw[->](savecheck.west)|-node[left]{No}(checksetting);
%\draw[->](clearsetflag)--(settime);
%\draw[-](settime)--(point2);
%\draw[-](point2)-|(point4);
%\draw[->](point4)--(readtime.east);
%\end{tikzpicture}
%\end{figure}

%%%%%%%%%%%%%%%%%%%%%%%%%%%%%%%%%%%%%%%%  绘制流程图(跨页)  %%%%%%%%%%%%%%%%%%%%%%%%%%%%%%%%%%%%%%%%%%%%%%%%%%%%%%%%%%%%%%%%%
%\begin{lrbox}{\mysavebox}%
%\begin{tikzpicture}[red,thick]
% \draw (0,0) rectangle (-.9\textwidth,-2.8\textheight);
% \draw (0,0) -- (-.9\textwidth,-2.8\textheight);
% \draw (-.9\textwidth,0) -- (0,-2.8\textheight);
% \path (-1mm,-1mm);
% \path (current bounding box.north east) +(1mm,1mm);
%\end{tikzpicture}%
%\end{lrbox}%
%%
%\ifdim\ht\mysavebox>\textheight
%    \setlength{\myrest}{\ht\mysavebox}%
%    \loop\ifdim\myrest>\textheight
%        \newpage\par\noindent
%        \clipbox{0 {\myrest-\textheight} 0 {\ht\mysavebox-\myrest}}{\usebox{\mysavebox}}%
%        \addtolength{\myrest}{-\textheight}%
%    \repeat
%    \newpage\par\noindent
%    \clipbox{0 0 0 {\ht\mysavebox-\myrest}}{\usebox{\mysavebox}}%
%\else
%    \usebox{\mysavebox}%
%\fi

%-----%%%%%%%%%%%%%%%%%%%%%%%%%%%%%%%%%%%%%%%%%%%%-----直-接-插-入-文-件-----%%%%%%%%%%%%%%%%%%%%%%%%%%%%%%%%%%%%%%%%%%%%%%%%%%%%------%
%\textcolor{red}{\textbf{直接插入文件}}:\verbatiminput{/home/jun_jiang/Documents/Latex_art_beamer/Daily_WORKS/Report-2020_model.tex} %为保险:~选用文件名绝对路径
%\textcolor{red}{\textbf{备忘录}}:\verbatiminput{/home/jun_jiang/Documents/备忘录.txt}
%--------------------------------------%%%%%%%%%%%%%%%%%%%%%%%%%%%%%%%%%%%%%%%%%%%%%%%%%%%%%%%%%%%%%%----------------------------------%


%-----------------------------------------------The Bibliography of The Papart------------------------------------%
%%%%%% 没有 \chapterbib 时仍然可以每个章节都出现参考文献~(\thebibliography 模式)~,但意义不大
%\phantomsection\addcontentsline{toc}{subsection}{Bibliography} %直接调用\addcontentsline命令可能导致超链指向不准确,一般需要在之前调用一次\phantomsection命令加以修正%


%\begin{thebibliography}{99}		            %
%\input{Content-4-Bibliography}                     %
%\end{thebibliography}			            %

%%%%%%%%%%%%%%%%%%%%%%%%%%%%%%%%%%%%%%%%%%%%%%%
%----------------------------------------------------------------------------------------------------------------------------------------------------%

%%%%%%%%%%%%%%%%%%%%%%%%%%%%%%%%%%%%%%%%%%%%%%%%%%%%%%%%%%%%%%%%%%%%%%%%%%%%%%%%%%%%%%%%%%%%%%%%%%%%%%%%%%%%%%%%%%%%%%%%%%%%%%%%%%%%%%%%%%%%%%%
