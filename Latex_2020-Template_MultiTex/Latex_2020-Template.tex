\documentclass[10pt, oneside, a4paper]{article}      % Specifies the document class
%\documentclass[10pt, oneside, AutoFakeBold, a4paper]{article}      % Specifies the document class
%\documentclass[10pt, twoside, a4paper]{article}      % Specifies the document class
%\documentclass[10pt, oneside, a4paper]{book}      % Specifies the document class
%AutoFakeBold 克服汉字加粗问题

%%%%%%%%%%%%%%%%% CJK 中文版面控制  %%%%%%%%%%%%%%%%%%%%%%%%%%%%%%
%\usepackage{CJK} % CTEX-CJK 中文支持                            %
\usepackage{xeCJK} % seperate the english and chinese		 %
\usepackage{CJKutf8} % Texlive 中文支持                         %
\usepackage{CJKnumb} %中文序号                                   %
\usepackage{indentfirst} % 中文段落首行缩进                      %
%\setlength\parindent{22pt}       % 段落起始缩进量               %
\renewcommand{\baselinestretch}{1.2} % 中文行间距调整            %
\setlength{\textwidth}{16cm}                                     %
\setlength{\textheight}{24cm}                                    %
\setlength{\topmargin}{-1cm}                                     %
\setlength{\oddsidemargin}{0.1cm}                                %
\setlength{\evensidemargin}{\oddsidemargin}                      %
%%%%%%%%%%%%%%%%%%%%%%%%%%%%%%%%%%%%%%%%%%%%%%%%%%%%%%%%%%%%%%%%%%

\usepackage{authblk}					 %作者地址和E-mail
\usepackage{amsmath,amsthm,amsfonts,amssymb,bm}          %数学公式
\usepackage{mathrsfs}                                    %英文花体
\usepackage{tikz}					 %绘制平面图形
%\usepackage[dvipdfmx]{movie15_dvipdfmx} %插入视频
\usepackage{xcolor}                                        %使用默认允许使用颜色
%\usepackage{hyperref} 
\usepackage{graphicx}
\usepackage{subfigure}           %图片跨页
\usepackage{animate}		 %插入动画
\usepackage{caption}
\captionsetup{font=footnotesize}

%\usepackage[version=3]{mhchem}		%化学公式
\usepackage{chemformula}
\usepackage{chemfig}		%化学公式

\usepackage{fontspec} % use to set font
\setCJKmainfont{SimSun}
\XeTeXlinebreaklocale "zh"  % Auto linebreak for chinese
\XeTeXlinebreakskip = 0pt plus 1pt % Auto linebreak for chinese

\usepackage{longtable}                                   %使用长表格
\usepackage{multirow}
\usepackage{makecell}		%允许单元格内换行

\usepackage{arydshln}
\newcommand{\adots}{\mathinner{\mkern2mu%
\raisebox{0.1em}{.}\mkern2mu\raisebox{0.4em}{.}%
\mkern2mu\raisebox{0.7em}{.}\mkern1mu}}

%%%%%%%%%%%%%%%%%%%%%%%%%  参考文献引用 %%%%%%%%%%%%%%%%%%%%%%%%%%%
%%尽量使用 BibTeX(含有超链接,数据库的条目URL即可)                %
%%%%%%%%%%%%%%%%%%%%%%%%%%%%%%%%%%%%%%%%%%%%%%%%%%%%%%%%%%%%%%%%%%%

\usepackage[numbers,sort&compress]{natbib} %紧密排列              %
%\usepackage[square,super,numbers,sort&compress]{natbib} %紧密排列%
%\usepackage{footbib}			   %脚注列出参考文献     %
%                                                                 %
% !!!!!!!!!!!!!!!!!!!!!!!!!!!!!!!!!!!!!!!!!!!!!!!!!!!!!!!!!!!!!!!!!!!!!!!!!!!!!!!!!!!!!!!!!!!! %
%\usepackage[sectionbib]{chapterbib}        %每章节单独参考文献   %
% !!! chapterbib 只有当 Book 状态下有效/ Article 状态下全文不会有错,但如果分割正文就会报错!!! %
\usepackage{hypernat}                                                                          %
%\usepackage[dvipdfm,bookmarksopen=true,pdfstartview=FitH,CJKbookmarks]{hyperref}              %
\usepackage[bookmarksopen=true,pdfstartview=FitH,CJKbookmarks]{hyperref}                       %
\hypersetup{bookmarksnumbered,colorlinks,linkcolor=green,citecolor=blue,urlcolor=red}          %
%参考文献含有超链接引用时需要下列宏包,注意与natbib有冲突        %
%\usepackage[dvipdfm]{hyperref}                                  %
%\usepackage{hypernat}                                           %
\newcommand{\upcite}[1]{\hspace{0ex}\textsuperscript{\cite{#1}}} %
%%%%%%%%%%%%%%%%%%%%%%%%%%%%%%%%%%%%%%%%%%%%%%%%%%%%%%%%%%%%%%%%%%%%%%%%%%%%%%%%%%%%%%%%%%%%%%%
%\AtBeginDvi{\special{pdf:tounicode GBK-EUC-UCS2}} %CTEX用dvipdfmx的话,用该命令可以解决      %
%						   %pdf书签的中文乱码问题		      %
%%%%%%%%%%%%%%%%%%%%%%%%%%%%%%%%%%%%%%%%%%%%%%%%%%%%%%%%%%%%%%%%%%%%%%%%%%%%%%%%%%%%%%%%%%%%%%%
%%%%%%%%%%%%%%%%-- Biblatex --%%%%%%%%%%%%%%%%%%%%%%%%%%%%%%%%%%%%%%%%%%%%%%%%%%%%%%%%%%%%%%%%%
%%%%            新版本的BibLaTeX默认使用Biber而非BibTeX进行处理参考文献。                     %
%\usepackage[backend=bibtex]{biblatex}                                                        %
%\usepackage[style=authoryear,backend=biber]{biblatex}                                        %
%%%%%%%%%%%%%%%%%%%%%%%%%%%%%%%%%%%%%%%%%%%%%%%%%%%%%%%%%%%%%%%%%%%%%%%%%%%%%%%%%%%%%%%%%%%%%%%

%---------%%%%%%%%%%%%%%%%--------xeCJK下设置中文字体-----------%%%%%%%%%%%%%%%%%%-----------%  
\include{Template_Fonts}                                                                     %
%%%%%%%%%%%%%%%%%%%%%%%%%%%%%%%%%%%%%%%%%%%%%%%%%%%%%%%%%%%%%%%%%%%%%%%%%%%%%%%%%%%%%%%%%%%%%%

%%%%%%%%%%%%%%%%%%%%%  % 插图使用位置  %%%%%%%%%%%%%%%%%%%%%%%%%%%
\graphicspath{{Presentation_Beamer/}}                            %
%%%%%%%%%%%%%%%%%%%%%%%%%%%%%%%%%%%%%%%%%%%%%%%%%%%%%%%%%%%%%%%%%%

\usepackage{verbatim}			%Verbatim 宏包重新实现了 Verbatim 环境,并且提供一个命令可以导入一个 ASCII 文件到文档中
%\verbatiminput{filename}

%在beamer里面使用verbatim环境,可以通过在frame的参数里面添加 containsverbatim / fragile来解决,不过 containsverbatim 会导致pause失效
%\begin{frame}[containsverbatim] %也可以用 \begin{frame}[fragile]
%	\begin{verbatim}
%	\usepackage{xcolor}
%	TEST
%	\end{verbatim}
%\end{frame}

%%%%%%%%%%%%%%%%%%%%%%%%%  % 段落缩进   %%%%%%%%%%%%%%%%%%%%%%%%%%%
%%%%%%       首行缩进   %%%%%%%%%%%
%\setlength{\parindent}{2em}{     %
%首行缩进2em的段落(可跨段落)      %
%}                                %
%%\CJKindent    %中文首行缩进     %
%---------------------------------%
%%%%%%%%%%% 取消多段首行缩进%%%%%%%%%%%%%%%
%\setlength{\parindent}{0pt}{             %
% 取消缩进段落(可跨段落)                  %
%}                                        %
%-----------------------------------------------------------------%
%%%%%%%%%%% 悬挂缩进 %%%%%%%%%%%%%%%%%%%%%%%%%%%%%%%%%%%%%%%%%%%%%% 
%\noindent                      %1. 取消⾸⾏缩进                  %
%\hangafter=1                   %2. 设置从第1⾏之后开始悬挂缩进  %
%\setlength{\hangindent}{2em}   %3. 设置悬挂缩进量                %
%{ 悬挂缩进段落(不可跨段落)                                       %
%}                                                                %
%-----------------------------------------------------------------%
%%%%%%%%%%%%%%%%%%%%%%%%%%%%%%%%%%%%%%%%%%%%%%%%%%%%%%%%%%%%%%%%%%%

%%%%%%%%%%%%%%%%%%%%%  % 页眉-页脚设计  %%%%%%%%%%%%%%%%%%%%%%%%%%%

%---------------------- TEMPLATE FOR GENERAL -------------------------------------------------------%

%%%%%%%%%%%%%%%%%%%%%  % 页眉-页脚设计  %%%%%%%%%%%%%%%%%%%%%%%%%%%
\usepackage{fancyhdr}           %使用页眉-页脚                   %
%%\renewcommand{\headrulewidth}{3pt} %页眉(单)线宽(默认黑色),设为0可以去页眉线
\renewcommand{\headrulewidth}{0pt} %页眉(单)线宽(默认黑色),设为0可以去页眉线
%\makeatletter % 双线页眉
%\def\headrule{\color{blue}{\if@fancyplain\let\headrulewidth\plainheadrulewidth\fi%
%\hrule\@height 0.5pt \@width\headwidth\vskip1pt %上面线为0.5pt粗
%\hrule\@height 3.0pt\@width\headwidth  %下面3pt粗
%\vskip-2\headrulewidth\vskip-1pt}      %两条线的距离1pt
%  \vspace{6mm}}     %双线与下面正文之间的垂直间距
%\makeatother

%%\renewcommand{\footrulewidth}{3pt} %页脚线宽(默认黑色),设为0可以去页脚线
%\makeatletter % 双线页眉
%\def\footrule{{\color{blue}{\if@fancyplain\let\footrulewidth\plainfootrulewidth\fi%
%\hrule\@height 3.0pt \@width\headwidth}}
%  \vspace{2mm}}
%\makeatother

\pagestyle{fancy}    %与文献引用超链接style有冲突
%\lhead{\bfseries Result} %页眉左边位置内容,并加粗 
\lhead{} %页眉左边位置 清空
\chead{} % 页眉中间位置内容
%\rhead{\includegraphics[scale=0.20]{Figures/BCC_logo-1.png}}%在此处插入logo.pdf图片 图片靠右
\rhead{}%页眉右侧位置 清空

%\lfoot{}  %页脚
%\rule{\temptablewidth}{1pt}
%\cfoot{}
%\rfoot{}
%\fancyfoot[C]{} %去掉页码

%\usepackage{lastpage}
%\fancyfoot[C]{\fontsize{8.5pt}{6.2pt}\selectfont{\textrm{第~\thepage~页/共~\pageref{LastPage}~页}}}      %设置 第x页/共y页 格式
%\fancyfoot[C]{\fontsize{8.5pt}{6.2pt}\selectfont{\textrm{No.\thepage of \pageref{Lastpage}}}}      %设置 第x页/共y页 格式
%% 如果不用 \usepage{lastpage} 可采用如下方案
%\fancyfoot[C]{\kaishu 第\thepage 页共\pageref{unknown}页}      %设置 第x页/共y页 格式
%\label{unknown}                                                %\label{unknown}置于文末


%%%%%%%%%%%%%%%%%  % pagestyler常用格式  %%%%%%%%%%%%%%%%%%%%%%%%%
%% empty 无页眉页脚
%% plain 无页眉,页脚为居中页码
%% headings 页眉为章节标题,无页脚
%% myheadings 页眉内容可自定义,无页脚
%%%%%%%%%%%%%%%%%%%%%%%%%%%%%%%%%%%%%%%%%%%%%%%%%%%%%%%%%%%%%%%%%%

%---------------------- TEMPLATE FOR BCC -----------------------------------------------------------%

%%%%%%%%%%%%%%%%%%%%%  % 页眉-页脚设计  %%%%%%%%%%%%%%%%%%%%%%%%%%%
%\usepackage{fancyhdr}           %使用页眉-页脚                   %
%%\renewcommand{\headrulewidth}{3pt} %页眉(单)线宽(默认黑色),设为0可以去页眉线
%\makeatletter % 双线页眉
%\def\headrule{\color{blue}{\if@fancyplain\let\headrulewidth\plainheadrulewidth\fi%
%\hrule\@height 0.5pt \@width\headwidth\vskip1pt %上面线为0.5pt粗
%\hrule\@height 3.0pt\@width\headwidth  %下面3pt粗
%\vskip-2\headrulewidth\vskip-1pt}      %两条线的距离1pt
%  \vspace{6mm}}     %双线与下面正文之间的垂直间距
%\makeatother
%
%%\renewcommand{\footrulewidth}{3pt} %页脚线宽(默认黑色),设为0可以去页脚线
%\makeatletter % 双线页眉
%\def\footrule{{\color{blue}{\if@fancyplain\let\footrulewidth\plainfootrulewidth\fi%
%\hrule\@height 3.0pt \@width\headwidth}}
%  \vspace{2mm}}
%\makeatother
%
%\pagestyle{fancy}    %与文献引用超链接style有冲突
%%\lhead{\bfseries Result} %页眉左边位置内容,并加粗 
%\chead{} % 页眉中间位置内容
%\lhead{\includegraphics[scale=0.35]{BCC_logo-1.png}}%在此处插入logo.pdf图片 图片靠右
%\lfoot{\hei{地址:~~北京市海淀区丰贤中路7号北科产业3号楼 ~~~~ 邮编:~~100094\\
%电话:~~010-59341997/98 ~~~~~~~ 传真:~~010-59341888 ~~~~~~\!网址:~~www.bcc.ac.cn}}  %页脚
%%\rule{\temptablewidth}{1pt}
%%\cfoot{}
%%\rfoot{}
%\fancyfoot[C]{} %保留页脚时去掉页码
%%%%%%%%%%%%%%%%%  % pagestyleR常用格式  %%%%%%%%%%%%%%%%%%%%%%%%%
%% empty 无页眉页脚
%% plain 无页眉,页脚为居中页码
%% headings 页眉为章节标题,无页脚
%% myheadings 页眉内容可自定义,无页脚
%%%%%%%%%%%%%%%%%%%%%%%%%%%%%%%%%%%%%%%%%%%%%%%%%%%%%%%%%%%%%%%%%%


%%%%%%%%%%%%%%%%%%%%%%%%%%%%%%%%%%%%%%%%%%%%%%%%%%%%%%%%%%%%%%%%%%%

\begin{document}

%\begin{CJK}{UTF8}{gbsn} %针对文字编码为unix %CJK自带的utf-8简体字体有gbsn(宋体)和gkai(楷体)
%\begin{CJK}{GBK}{hei}	%针对文字编码为doc
%\begin{CJK}{GBK}{hei}	 %针对文字编码为doc
%\CJKindent     %在CJK环境中,中文段落起始缩进2个中文字符
%\indent
%
\renewcommand{\abstractname}{\small{\CJKfamily{hei} 摘\quad 要}} %\CJKfamily{hei} 设置中文字体,字号用\large \small来设
%\renewcommand{\contentsname}{\centering\CJKfamily{hei} 目~~~录}
\renewcommand{\refname}{\centering\CJKfamily{hei} 参考文献}
%\renewcommand{\figurename}{\CJKfamily{hei} 图.}
\renewcommand{\figurename}{{\bf Fig}.}
%\renewcommand{\tablename}{\CJKfamily{hei} 表.}
\renewcommand{\tablename}{{\bf Tab}.}
%\renewcommand{\thesubfigure}{\roman{subfigure}}  \makeatletter %子图标记罗马字母
%\renewcommand{\thesubfigure}{\tiny(\alph{subfigure})}  \makeatletter %子图标记英文字母
%\renewcommand{\thesubfigure}{}  \makeatletter %子图无标记

%将图表的Caption写成 图(表) Num. 格式
\makeatletter
\long\def\@makecaption#1#2{%
  \vskip\abovecaptionskip
  \sbox\@tempboxa{#1. #2}%
  \ifdim \wd\@tempboxa >\hsize
    #1. #2\par
  \else
    \global \@minipagefalse
    \hb@xt@\hsize{\hfil\box\@tempboxa\hfil}%
  \fi
  \vskip\belowcaptionskip}
\makeatother

\newcommand{\keywords}[1]{{\hspace{0pt}\small{\CJKfamily{hei} 关键词:}{\hspace{2ex}{#1}}\bigskip}}

%----------------------------------------------------------------------------------------------------------------------------------------------------%
%
%%%%%%%%%%%%%%%%%%%%%%%%%%%%% 用 authblk 包 支持作者和E-mail %%%%%%%%%%%%%%%%%%%%%%%%%%%%%%%%%
\include{Template_Author}									     %
%%%%%%%%%%%%%%%%%%%%%%%%%%%%%%%%%%%%%%%%%%%%%%%%%%%%%%%%%%%%%%%%%%%%%%%%%%%%%%%%%%%%%%%%%%%%%%

\maketitle
%\thispagestyle{fancy}   % 首页插入页眉页脚 
%
%----------------------------------------------------------------------------------------------------------------------------------------------------%
%
%%%%------The Abstract and the keywords of The Paper-------%%%%
%------%%%%%%%%%%%%%%------The Abstract and the keywords of The Paper-------%%%%%%%%%%%%%--------%

\begin{abstract}
The content of the abstract
\end{abstract}

\keywords{Keyword1; Keyword2; Keyword3}

%%%%%%%%%%%%%%%%%%%%%%%%%%%%%%%%%%%%%%%%%%%%%%%%%%%%%%%%%%%%%%%%%%%%%%%%%%%%%%%%%%%%%%%%%%%%%%%%%%
                         %
%%%%%%%%%%%%%%%%%%%%%%%%%%%%%%%%%%%%%%%%%%%%%%%%%%%%%%%%%%%%%%%
%
%---------------The Content of The Paper----------------------%
\tableofcontents %% 制作目录(目录是根据标题自动生成的)        %
%--------%%%%%%%%%%%%%%%%%%%%%%%%%%%%%%%%%%%%%%%%%%%%%--------%
%
%----------------------------------------------------------------------------------------------------------------------------------------------------%
%
%----------------The Body Of The Paper------------------------%
%
%-----%%%%%%%------- Introduction ---------%%%%%%%%%%---------%
\include{Template_Content-1-Introduction}                     %
%--------%%%%%%%%%%%%%%%%%%%%%%%%%%%%%%%%%%%%%%%%%%%%%--------%
%
%-----%%%%%%%----Main Body of The Paper----%%%%%%%%%%---------%
\include{Template_Content-2-Main}                             %
%--------%%%%%%%%%%%%%%%%%%%%%%%%%%%%%%%%%%%%%%%%%%%%%--------%
%
%----------------------------------------------------------------------------------------------------------------------------------------------------%
%
%----%%%%%%%%%%%%%%%%----Acknowledge----%%%%%%%%%%%%%%%%%-----%
\include{Template_Content-3-Acknowledge}                      %
%--------%%%%%%%%%%%%%%%%%%%%%%%%%%%%%%%%%%%%%%%%%%%%%--------%
%
%-----------------------------------------------The Bibliography of The Paper-------------------------------------%
%
%----%%%%%%%%%%%%%%%%%%%%%--------The Biblography of The Paper--------%%%%%%%%%%%%%%%%%%%-----%

%%\bibitem{PRL58-65_1987}H.Feil, C. Haas, {\it Phys. Rev. Lett.} {\bf 58}, 65 (1987).         %
%\bibitem{kp-method} \textrm{Zhenxi Pan, Yong Pan, Jun Jiang$^{\ast}$, Liutao Zhao}, \textrm{High-Throughput Electronic Band Structure Calculations for Hexaborides}, \textit{Intelligent Computing}, \textbf{Springer}, \textbf{P.386-395}, (2019).              %
\bibitem{QCQC_2014} \textrm{姜骏},\textrm{PAW原子数据集的构造与检验}, \textit{中国化学会第十二届全国量子化学会议论文摘要集},\textbf{太原},(2014).
%											      %
%---%%%%%%%%%%%%%%%%%%%%%-------%%%%%%%%%%%%%%%%%%%%%%%%%%%%%--------%%%%%%%%%%%%%%%%%%%------%

                     %
%
%----------------------------------------------------------------------------------------------------------------------------------------------------%
%
\clearpage     %\end{CJK} 前加上\clearpage是CJK的要求
%\end{CJK*}
\end{document}
