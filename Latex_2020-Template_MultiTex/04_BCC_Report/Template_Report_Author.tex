%------------------------------------------------------------------------------ main Body------------------------------------------------------------------------------------
\newpage	        % 每个新的/newpage 即可有新的\thispagestyle 引领      %
\thispagestyle{fancy}   % 插入页眉页脚                                        %
%----------------------------------------------------------------------------------------The Body Of The Report----------------------------------------------------------------------------------------%
%Introduction
%\section{Introduction}
%报告人简介
%1.~个人简历
\vskip 40pt
\begin{minipage}[b]{0.69\textwidth}
%{\hei{\large 报告人个人简历}}
{\hspace*{120pt} \hei{\huge 个人简历}}
%{\hspace*{69pt} \hei{\large 报告人简历}}
\vskip 25pt
\hspace*{-10pt} {\fontsize{15.0pt}{10.0pt}\selectfont{姜骏,北京市计算中心~~副研究员}}
\vskip 4pt
\hspace*{-10pt} 2001.06~中国纺织大学~(现~东华大学)~染整专业毕业.\\
\hspace*{-10pt} 2008.01~北京大学物理化学专业~博士研究生毕业.\\
\hspace*{-10pt} 2008.04-2012.03~北京大学~博士后.\\
\hspace*{-10pt} 2012.03-2013.03~北京宏剑公司~高级技术支持.\\
\hspace*{-10pt} 2013.04-2016.03~中物院高性能数值模拟软件中心~助理研究员.\\
\hspace*{-10pt} 2016.04-至今~北京市计算中心~副研究员.
\end{minipage}
\hskip 15pt
%2.~个人近照
\begin{minipage}[b]{0.19\textwidth}
\vspace{17pt}
\includegraphics[height=1.5in]{Figures/Person_Photo.JPG}
\end{minipage}
\vskip 30pt

%3.~个人研究内容与兴趣
{\fontsize{12.0pt}{10.0pt}\selectfont{
姜骏长年从事第一原理计算方法、算法和软件研究,熟悉密度泛函理论~(\textrm{Density Functional Theory, DFT})~和第一原理计算方法,特别是对赝势~(\textrm{Pseudo-Potential, PP})~方法、线性缀加平面波~(\textrm{Linearised Augmeneted Plane-Wave, LAPW})~方法和投影子缀加波~(\textrm{Projected~Augmented~Wave, PAW})~方法及相关软件有系统深入理解,对分子动力学~(\textrm{Molecular Dynamics, MD})方法有所涉猎。
\vskip 10pt
{\fontsize{20.0pt}{10.0pt}\selectfont{\hei{\noindent 
2016年起,开始尝试将数据挖掘、知识图谱、人工智能相关的机器学习算法,特别是人工神经网络算法(人工智能的主要核心算法)应用到第一原理和分子动力学材料研究的软件开发和模拟计算中,对相关方法、算法有全面的认识。
}}}
\vskip 10pt
\begin{itemize}
	\item 2017年7月起,作为骨干研究人员承担科技部“材料基因工程关键技术与支撑平台”重点专项课题“研发多尺度集成化高通量计算方法与计算软件”研究任务,从事高通量跨尺度并发式集成计算主体算法与软件开发。
	\item 2024年10月起,作为骨干研究人员承担科技部“新材料研发及应用”国家重大专项(科技创新2030重大项目)课题“基于人工智能技术的高性能多尺度分子动力学模拟平台”研究任务,从事支持CPU+GPU异构并行计算多尺度分子动力学模拟软件开发。
	\item 2025年1月起,作为共同申请人承担国家自然基金面上项目“低维材料等离和激子极化激元的第一性原理研究”的研究任务,结合第一原理计算和理论模型,研究各种低维材料的等离激元和激子性质,揭示新型等离和激子极化激元的形成机制。
\end{itemize}
\vskip 10pt
{\fontsize{15.0pt}{10.0pt}\selectfont{\noindent 发表论文6篇,获得发明专利1项、软著2项、参与编著专著1本,论文6篇。}}
%近年来的主要工作包括第一原理-分子动力学材料计算软件平台的设计与开发、晶体空间群软件开发和~\textrm{VASP}~软件的赝势重构等。
%	姜骏长年从事第一原理计算方法和软件研究,熟悉密度泛函理论~(\textrm{Density Functional Theory, DFT})~和第一原理计算方法,特别是对赝势~(\textrm{Pseudo-Potential, PP})~方法、线性缀加平面波~(\textrm{Linearised Augmeneted Plane-Wave, LAPW})~方法和投影子缀加波~(\textrm{Projected~Augmented~Wave, PAW})~方法及相关软件有系统深入的理解,对分子动力学~(\textrm{Molecular Dynamics, MD})方法也有所涉猎。
%
%2017年7月起,作为骨干研究人员承担科技部“材料基因工程关键技术与支撑平台”重点专项课题1“研发多尺度集成化高通量计算方法与计算软件”研究任务,从事高通量跨尺度并发式集成计算主体算法与软件开发。
%
%近年来的主要工作包括第一原理-分子动力学材料计算软件平台的设计与开发、晶体空间群软件开发和~\textrm{VASP}~软件的赝势重构等。参与编著《计算材料科学理论与实践》一书。
}}

%4. 个人名章、E-mail
%\hskip 0.75\textwidth
%\begin{minipage}[b]{0.20\textwidth}
%		\includegraphics[scale=0.07]{signature-seal_Jiang-1.png}
%\end{minipage}

%\section{正文章节}参考文献的引用方式1\upcite{QCQC_2014}
