%-------%%%%%%%% PREPARE FOR THE ABSTRACT BIBLIOGRAPH FIG. AND TAB. %%%%%%%%-----------%
%
%\begin{CJK}{UTF8}{gbsn} %针对文字编码为unix %CJK自带的utf-8简体字体有gbsn(宋体)和gkai(楷体)
%\begin{CJK}{GBK}{hei}	%针对文字编码为doc
%\begin{CJK}{GBK}{hei}	 %针对文字编码为doc
%\CJKindent     %在CJK环境中,中文段落起始缩进2个中文字符
%\indent
%
\renewcommand{\abstractname}{\small{\CJKfamily{hei} 到\quad 会\quad 人\quad 员}} %\CJKfamily{hei} 设置中文字体,字号用\large \small来设
%\renewcommand{\contentsname}{\centering\CJKfamily{hei} 目~~~录}
\renewcommand{\refname}{\centering\CJKfamily{hei} 参考文献}
%\renewcommand{\figurename}{\CJKfamily{hei} 图.}
\renewcommand{\figurename}{{\bf Fig}.}
%\renewcommand{\tablename}{\CJKfamily{hei} 表.}
\renewcommand{\tablename}{{\bf Tab}.}
%\renewcommand{\thesubfigure}{\roman{subfigure}}  \makeatletter %子图标记罗马字母
%\renewcommand{\thesubfigure}{\tiny(\alph{subfigure})}  \makeatletter %子图标记英文字母
%\renewcommand{\thesubfigure}{}  \makeatletter %子图无标记

%将图表的Caption写成 图(表) Num. 格式
\makeatletter
\long\def\@makecaption#1#2{%
  \vskip\abovecaptionskip
  \sbox\@tempboxa{#1. #2}%
  \ifdim \wd\@tempboxa >\hsize
    #1. #2\par
  \else
    \global \@minipagefalse
    \hb@xt@\hsize{\hfil\box\@tempboxa\hfil}%
  \fi
  \vskip\belowcaptionskip}
\makeatother

\newcommand{\keywords}[1]{{\hspace{0pt}\small{\CJKfamily{hei} 关键词:}{\hspace{2ex}{#1}}\bigskip}}
\newcommand{\peopinfo}[3]{\small\CJKfamily{hei} #1:~#2,~~#3.~~~}
%
%----------------------------------------------------------------------------------------------------------------------------------------------------%
%
%------%%%%%%%%%%%%%%------The Abstract and the keywords of The Paper-------%%%%%%%%%%%%%--------%

\thispagestyle{fancy}   % 插入页眉页脚                                        %
\begin{abstract}
%	\begin{itemize}
%		\item \peopinfo{姜骏}{北京市计算中心}{副研究员}   %到会人员信息(姓名、单位、职称)
	\noindent
		%		\item \peopinfo{姜骏}{北京市计算中心}{副研究员}   %到会人员信息(姓名、单位、职称)
%		\peopinfo{王崇愚}{清华大学}{科学院院士}\peopinfo{毛勇}{云南大学}{教授}\peopinfo{毛勇-博士生}{云南大学}{}\\
%		\peopinfo{吴健}{清华大学}{教授}\peopinfo{于涛}{钢铁研究总院}{教授}\peopinfo{姜骏}{北京市计算中心}{副研究员} 
%\peopsimp{聂淼}{}\peopsimp{陶应龙}{}\peopsimp{王彩群}{}\peopsimp{高朋林}{}\peopsimp{姜骏}{}
\peopsimp{姚洁}{}\peopsimp{姜骏}{}

%	\end{itemize}
\end{abstract}

%\keywords{Keyword1; Keyword2; Keyword3}

%%%%%%%%%%%%%%%%%%%%%%%%%%%%%%%%%%%%%%%%%%%%%%%%%%%%%%%%%%%%%%%%%%%%%%%%%%%%%%%%%%%%%%%%%%%%%%%%%%
%
%---------------The Content of The Paper----------------------%
%\newpage
%\pagestyle{plain}   % 删除页眉                               %
%\addcontentsline{toc}{subsection}{\CJKfamily{hei} 目~录}
%\tableofcontents %% 制作目录(目录是根据标题自动生成的)
%--------%%%%%%%%%%%%%%%%%%%%%%%%%%%%%%%%%%%%%%%%%%%%%--------%
%
%----------------------------------------------------------------------------------------------------------------------------------------------------%
%
