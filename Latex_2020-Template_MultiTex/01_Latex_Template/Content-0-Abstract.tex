%-------%%%%%%%% PREPARE FOR THE ABSTRACT BIBLIOGRAPH FIG. AND TAB. %%%%%%%%-----------%
%
%\begin{CJK}{UTF8}{gbsn} %针对文字编码为unix %CJK自带的utf-8简体字体有gbsn(宋体)和gkai(楷体)
%\begin{CJK}{GBK}{hei}	%针对文字编码为doc
%\begin{CJK}{GBK}{hei}	 %针对文字编码为doc
%\CJKindent     %在CJK环境中,中文段落起始缩进2个中文字符
%\indent
%
\renewcommand{\abstractname}{\small{\CJKfamily{hei} 摘\quad 要}} %\CJKfamily{hei} 设置中文字体,字号用\large \small来设
%\renewcommand{\contentsname}{\centering\CJKfamily{hei} 目~~~录}
\renewcommand{\refname}{\centering\CJKfamily{hei} 参考文献}
%\renewcommand{\figurename}{\CJKfamily{hei} 图.}
\renewcommand{\figurename}{{\bf Fig}.}
%\renewcommand{\tablename}{\CJKfamily{hei} 表.}
\renewcommand{\tablename}{{\bf Tab}.}
%\renewcommand{\thesubfigure}{\roman{subfigure}}  \makeatletter %子图标记罗马字母
%\renewcommand{\thesubfigure}{\tiny(\alph{subfigure})}  \makeatletter %子图标记英文字母
%\renewcommand{\thesubfigure}{}  \makeatletter %子图无标记

%将图表的Caption写成 图(表) Num. 格式
\makeatletter
\long\def\@makecaption#1#2{%
  \vskip\abovecaptionskip
  \sbox\@tempboxa{#1. #2}%
  \ifdim \wd\@tempboxa >\hsize
    #1. #2\par
  \else
    \global \@minipagefalse
    \hb@xt@\hsize{\hfil\box\@tempboxa\hfil}%
  \fi
  \vskip\belowcaptionskip}
\makeatother

\newcommand{\keywords}[1]{{\hspace{0pt}\small{\CJKfamily{hei} 关键词:}{\hspace{2ex}{#1}}\bigskip}}
%
%----------------------------------------------------------------------------------------------------------------------------------------------------%
%
%------%%%%%%%%%%%%%%------The Abstract and the keywords of The Paper-------%%%%%%%%%%%%%--------%

\begin{abstract}
%The content of the abstract
	异相界面催化的微观动力学模拟涉及跨尺度高通量计算,我们通过对比各种的材料计算自动流程软件实现方案,提出了适应模拟异质界面催化材料的高通量材料计算自动流程软件的基本结构,并面向异相催化反应的特点,完成了高通量材料计算自动流程的软件调度、组织与优化,实现跨尺度计算的流程化。引入$\vec k\cdot\vec p$微扰,将周期体系\textrm{DFT}计算的效率提升一个数量级,满足异质界面催化反应机理研究涉及反应多、模拟动力学过程计算耗时长的问题。在自动流程形成计算材料数据库的基础上,构建碳催化材料知识图谱,为数据驱动的异质界面催化反应提供软件支持。
\end{abstract}

%\keywords{Keyword1; Keyword2; Keyword3}
\keywords{异质界面催化,高通量计算,$\vec k\cdot\vec p$微扰,知识图谱}

\maketitle
%%%%%%%%%%%%%%%%%%%%%%%%%%%%%%%%%%%%%%%%%%%%%%%%%%%%%%%%%%%%%%%%%%%%%%%%%%%%%%%%%%%%%%%%%%%%%%%%%%
%
%---------------The Content of The Paper----------------------%
%\newpage
%\pagestyle{plain}   % 删除页眉                               %
%\addcontentsline{toc}{subsection}{\CJKfamily{hei} 目~录}
%\tableofcontents %% 制作目录(目录是根据标题自动生成的)
%--------%%%%%%%%%%%%%%%%%%%%%%%%%%%%%%%%%%%%%%%%%%%%%--------%
%
%----------------------------------------------------------------------------------------------------------------------------------------------------%
%
